\documentclass[12pt]{article}
\setlength\parindent{0pt}
\usepackage{fullpage}
\usepackage{epsf}
\usepackage{amsmath}
\usepackage{graphicx}
\setlength{\parskip}{4mm}
\def\LL{\left\langle}   % left angle bracket
\def\RR{\right\rangle}  % right angle bracket
\def\LP{\left(}         % left parenthesis
\def\RP{\right)}        % right parenthesis
\def\LB{\left\{}        % left curly bracket
\def\RB{\right\}}       % right curly bracket
\def\PAR#1#2{ {{\partial #1}\over{\partial #2}} }
\def\PARTWO#1#2{ {{\partial^2 #1}\over{\partial #2}^2} }
\def\PARTWOMIX#1#2#3{ {{\partial^2 #1}\over{\partial #2 \partial #3}} }
\newcommand{\BE}{\begin{displaymath}}
\newcommand{\EE}{\end{displaymath}}
\newcommand{\BNE}{\begin{equation}}
\newcommand{\ENE}{\end{equation}}
\newcommand{\BEA}{\begin{eqnarray}}
\newcommand{\EEA}{\nonumber\end{eqnarray}}
\newcommand{\EL}{\nonumber\\}
\newcommand{\la}[1]{\label{#1}}
\newcommand{\ie}{{\em i.e.\ }}
\newcommand{\eg}{{\em e.\,g.\ }}
\newcommand{\cf}{cf.\ }
\newcommand{\etc}{etc.\ }
\newcommand{\Tr}{{\rm tr}}
\newcommand{\etal}{{\it et al.}}
\newcommand{\OL}[1]{\overline{#1}\ } % overline
\newcommand{\OLL}[1]{\overline{\overline{#1}}\ } % double overline
\newcommand{\OON}{\frac{1}{N}} % "one over N"
\newcommand{\OOX}[1]{\frac{1}{#1}} % "one over X"

\pagenumbering{gobble}

\begin{document}


\bigskip
\bigskip
\bigskip
\bigskip

\Large \centerline{\sc{Physics 211, Exam 1 (makeup)}}


\vspace{2in}


\hspace{1in} Name: \underline{\hspace{3in}}

\bigskip
\bigskip


\normalsize

\vspace{1.4in}

\begin{itemize}
  \item{There are six questions worth 25 points each, and a possible 20 points extra credit.}
  \item{``Graph'' means to make a precise graph on a separate sheet of graph paper. The locations of intercepts, inflection points, maxima and minima, slopes, and the sign of concavity need to be accurate.}
  \item{``Sketch'' means to make a casual graph which does not need to be on a separate page. These sketches need to illustrate only the essential features of the motion; they do not
    need to be perfectly to scale.}
  \item{{\bf You must show your reasoning to receive credit}. A numerical answer with no logic shown will be treated as no answer.}
  \item{If you run out of room, continue your work on the back of the page.}
  \item{Remember, show your reasoning as thoroughly as possible for partial credit.}
  \item{Use $g=10\, \rm m/\rm s^2$ throughout to minimize arithmetic.}
\end{itemize}
\newpage
\small





\begin{flushright}
    Name: \underline{\hspace{3in}}
    \end{flushright}
  
    \Large \centerline{\sc{Question 1}}
    \normalsize
  \rm

  An object has an acceleration of 4 $\frac{\rm m}{\rm s^2}$ eastward for 5 s, followed by an acceleration of 12 $\frac{\rm m}{\rm s^2}$ westward for 3s. The object then 
  decelerates uniformly over a period of 4s until it comes to rest.

  \it \bigskip

  Graph (on separate graph paper) the object's acceleration vs. time (5 points), velocity vs. time (10 points), and position vs. time (10 points). Remember to label your axes (how many meters per box?)
  Ensure that the coordinates of maxima, minima, starting and ending points, and inflection points are accurate.

\newpage
\includegraphics[width=\textwidth]{graphpaper.png}
\newpage
\includegraphics[width=\textwidth]{graphpaper.png}
\newpage
\includegraphics[width=\textwidth]{graphpaper.png}
\newpage

\begin{flushright}
Name: \underline{\hspace{3in}}
        \end{flushright}

        \Large \centerline{\sc{Question 2}}
        \normalsize
        \rm

A ball is thrown from ground level at a speed $v_0$ at an angle $\theta$ above the horizontal.

\bigskip

Give your answers to the following in terms of $g$, $v_0$, and $\theta$. If you run out of room, use the back of the page.

\it \bigskip

a) What are the horizontal and vertical components of the ball's initial velocity? (5 points)

\vspace{1in}

b) What functions $x(t)$ and $y(t)$ give its position as a function of time? (5 points)

\vspace{1in}

c) How long after it is thrown does the ball land back on the ground? (5 points)

\vspace{1in}

d) How far away from the starting point does the ball land? (5 points)

\vspace{1in}

e) What is the maximum height achieved by the ball? (5 points)

\newpage


\begin{flushright}
Name: \underline{\hspace{3in}}
        \end{flushright}

        \Large \centerline{\sc{Question 3}}
        \normalsize
        \rm

A frog jumps horizontally off of a table 1.5 m high. It lands on the floor 2 m away from the bottom of the table.

\it \bigskip

a) How long was the frog in the air? (5 points)

\vspace{1.2in}

b) What initial velocity did the frog jump with? (10 points)

\vspace{1.2in}

c) What are the $x-$ and $y-$components of the frog's velocity when it lands on the ground? (5 points)

\vspace{1.2in}

d) What is the frog's speed when it lands on the ground? (2 points)

\vspace{1in}

e) What is the angle between the frog's velocity vector and the $x$-axis when it lands? (3 points)

\bigskip

\newpage
\begin{flushright}
  Name: \underline{\hspace{3in}}
\end{flushright}




\Large \centerline{\sc{Question 4}}
\normalsize

A bucket hangs from a rope, 13m below the edge of a cliff. This rope is connected to a motor that accelerates it upward at $3 \frac{\rm m}{\rm s^2}$.

\it{a) How long does the bucket take to rise to the cliff edge after the motor is switched on? (15 points)}

\vspace{1in}

\it{b) What is its speed when it arrives there? (10 points})

\vspace{1in}

\rm \centerline{(The remaining parts of this question are extra credit)}

At the same instant as the motor is switched on, a stone is dropped from the edge of the cliff, causing the bucket to accelerate upward. (Both objects start from rest.)

\bigskip

\it{c) Sketch the position vs. time of the bucket and of the stone on the same set of axes. (5 points extra credit)}

\vspace{1in}

\it{d) How long after the objects begin moving does the stone land in the bucket? (10 points extra credit)}

\vspace{1in}

\it{e) How far does the stone fall before it lands in the bucket? (5 points extra credit)}

\newpage

\begin{flushright}
    Name: \underline{\hspace{3in}}
  \end{flushright}

  \Large \centerline{\sc{Question 5}}
  \normalsize
\rm
Alice is standing on the street with a baseball. Bob is watching above from a window at a height of 6m above her. She throws the ball straight upward to him at
a speed of 13 $\frac{\rm m}{\rm s}$, and he catches it as it flies upward.

\bigskip

\it{a) How long is the ball in the air before he catches it? (15 points)}

\vspace{3in}

\it{b) The quadratic formula gives you two roots. One of these will help you answer the preceding question. What is the physical interpretation of the other root? (5 points)}
\vspace{1in}

\it{c) What is the ball's velocity when he catches it? (5 points)}


  \newpage

\begin{flushright}
    Name: \underline{\hspace{3in}}
  \end{flushright}

  \Large \centerline{\sc{Question 6}}
  \normalsize
\rm

A hiker on flat ground walks at a constant speed. She walks north for one hour, walks for two hours at an angle $35^o$ south of west, then walks for three hours at an angle
$40^o$ north of east. (In the Cartesian plane: she walks in the positive $y$-direction for one hour, walks for two hours at an angle $35^o$ below the negative $x$-axis, then walks 
for three hours at an angle $40^o$ above the positive $x$-axis.)

\bigskip
  
\it

a) How long must she walk to return to her starting point? (15 points)

\vspace{3in}

b) In what direction must she walk to return to her starting point? (10 points)




\end{document}
