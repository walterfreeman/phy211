\documentclass[12pt]{article}
\setlength\parindent{0pt}
%\usepackage{fullpage}
\usepackage{amsmath}
\usepackage{hyperref}
\usepackage[margin=2cm]{geometry}
\usepackage{graphicx}
\setlength{\parskip}{4mm}
\def\LL{\left\langle}   % left angle bracket
\def\RR{\right\rangle}  % right angle bracket
\def\LP{\left(}         % left parenthesis
\def\RP{\right)}        % right parenthesis
\def\LB{\left\{}        % left curly bracket
\def\RB{\right\}}       % right curly bracket
\def\PAR#1#2{ {{\partial #1}\over{\partial #2}} }
\def\PARTWO#1#2{ {{\partial^2 #1}\over{\partial #2}^2} }
\def\PARTWOMIX#1#2#3{ {{\partial^2 #1}\over{\partial #2 \partial #3}} }
\newcommand{\BE}{\begin{displaymath}}
\newcommand{\EE}{\end{displaymath}}
\newcommand{\BNE}{\begin{equation}}
\newcommand{\ENE}{\end{equation}}
\newcommand{\BEA}{\begin{eqnarray}}
\newcommand{\EEA}{\nonumber\end{eqnarray}}
\newcommand{\EL}{\nonumber\\}
\newcommand{\la}[1]{\label{#1}}
\newcommand{\ie}{{\em i.e.\ }}
\newcommand{\eg}{{\em e.\,g.\ }}
\newcommand{\cf}{cf.\ }
\newcommand{\etc}{etc.\ }
\newcommand{\Tr}{{\rm tr}}
\newcommand{\etal}{{\it et al.}}
\newcommand{\OL}[1]{\overline{#1}\ } % overline
\newcommand{\OLL}[1]{\overline{\overline{#1}}\ } % double overline
\newcommand{\OON}{\frac{1}{N}} % "one over N"
\newcommand{\OOX}[1]{\frac{1}{#1}} % "one over X"



\begin{document}
\begin{center}
\Large
\sc Second Chance Exercise - Techniques in Combination / Dog and Boat Problem \rm

\pagenumbering{gobble}




\normalsize
This second chance exercise is for people who want to review the use of multiple techniques in combination, knowing what principle to use where.

If one of your lowest three exam problem grades was on the ``dog and boat problem'' from Exam 3, you may complete it and turn it in to earn an extra homework grade, which we will help you earn 100\% on. This will help your end-of-term average.

If you do this, it is due on the last day of recitation.

\end{center}

\vspace{1.5in}

This exam problem tested your ability to look at a motion involving multiple aspects and determine which technique you should use to analyze which part.

\begin{enumerate}
	
		\item How did you know which technique to use to relate the dog's forward velocity to the boat's backward velocity in part (c)?
		
		\item How did you know which technique to use to relate the dog's forward velocity off the boat (which you found in parts (a) and (c)) to the landing position of the dog? Why can you {\it not} use energy methods to find this?
	
		\item Compare this exam problem to homework 5 question 2 and homework 7 question 3. How is each of these problems the same as the ``dog and boat problem'', and how is it different?


	\item Identify a problem in Practice Exam 3 (at \url{https://walterfreeman.github.io/phy211/practice-exam-3-2023.pdf}) where you need to use ideas related to energy, momentum, and kinematics in combination. How is that problem the same as the ``dog and boat'' problem, and how did you know which techniques to use in which aspects of its motion? How is it different?
	

	
    \item Write a similar problem where you have to use the conservation of momentum, energy methods, and/or kinematics in combination to analyze a multi-step motion. Then solve it. {\it (Particularly good problems here will earn extra credit!)}
\end{enumerate}



\end{document}
