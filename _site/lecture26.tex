% DPF 09 talk on strangeness in nucleon
\documentclass[10pt]{beamer}
\usefonttheme{professionalfonts} % using non standard fonts for beamer
\usefonttheme{serif} % default family is serif
\usepackage{amsmath}
\usepackage{mathtools}
\usepackage{mwe}
%\documentclass[12pt]{beamerthemeSam.sty}
\usepackage{epsf}
%\usepackage{pstricks}
%\usepackage[orientation=portrait,size=A4]{beamerposter}
\geometry{paperwidth=160mm,paperheight=120mm}
%DT favorite definitions
\def\LL{\left\langle}	% left angle bracket
\def\RR{\right\rangle}	% right angle bracket
\def\LP{\left(}		% left parenthesis
\def\RP{\right)}	% right parenthesis
\def\LB{\left\{}	% left curly bracket
\def\RB{\right\}}	% right curly bracket
\def\PAR#1#2{ {{\partial #1}\over{\partial #2}} }
\def\PARTWO#1#2{ {{\partial^2 #1}\over{\partial #2}^2} }
\def\PARTWOMIX#1#2#3{ {{\partial^2 #1}\over{\partial #2 \partial #3}} }

\def\rightpartial{{\overrightarrow\partial}}
\def\leftpartial{{\overleftarrow\partial}}
\def\diffpartial{\buildrel\leftrightarrow\over\partial}

\def\BS{\bigskip}
\def\BC{\begin{center}}
\def\EC{\end{center}}
\def\BI{\begin{itemize}}
\def\EI{\end{itemize}}
\def\BE{\begin{displaymath}}
\def\EE{\end{displaymath}}
\def\BEA{\begin{eqnarray*}}
\def\EEA{\end{eqnarray*}}
\def\BNEA{\begin{eqnarray}}
\def\ENEA{\end{eqnarray}}
\def\EL{\nonumber\\}


\newcommand{\map}[1]{\frame{\frametitle{\textbf{Course map}}
\centerline{\includegraphics[height=0.86\paperheight]{../../map/#1.png}}}}
\newcommand{\wmap}[1]{\frame{\frametitle{\textbf{Course map}}
\centerline{\includegraphics[width=0.96\paperwidth]{../../map/#1.png}}}}

\newcommand{\etal}{{\it et al.}}
\newcommand{\gbeta}{6/g^2}
\newcommand{\la}[1]{\label{#1}}
\newcommand{\ie}{{\em i.e.\ }}
\newcommand{\eg}{{\em e.\,g.\ }}
\newcommand{\cf}{cf.\ }
\newcommand{\etc}{etc.\ }
\newcommand{\atantwo}{{\rm atan2}}
\newcommand{\Tr}{{\rm Tr}}
\newcommand{\dt}{\Delta t}
\newcommand{\op}{{\cal O}}
\newcommand{\msbar}{{\overline{\rm MS}}}
\def\chpt{\raise0.4ex\hbox{$\chi$}PT}
\def\schpt{S\raise0.4ex\hbox{$\chi$}PT}
\def\MeV{{\rm Me\!V}}
\def\GeV{{\rm Ge\!V}}

%AB: my color definitions
%\definecolor{mygarnet}{rgb}{0.445,0.184,0.215}
%\definecolor{mygold}{rgb}{0.848,0.848,0.098}
%\definecolor{myg2g}{rgb}{0.647,0.316,0.157}
\definecolor{abtitlecolor}{rgb}{0.0,0.255,0.494}
\definecolor{absecondarycolor}{rgb}{0.0,0.416,0.804}
\definecolor{abprimarycolor}{rgb}{1.0,0.686,0.0}
\definecolor{Red}           {cmyk}{0,1,1,0}
\definecolor{Grey}           {cmyk}{.7,.7,.7,0}
\definecolor{Lg}           {cmyk}{.4,.4,.4,0}
\definecolor{Blue}          {cmyk}{1,1,0,0}
\definecolor{Green}         {cmyk}{1,0,1,0}
\definecolor{Brown}         {cmyk}{0,0.81,1,0.60}
\definecolor{Black}         {cmyk}{0,0,0,1}

\usetheme{Madrid}
\newcommand{\vcenteredinclude}[1]{\begingroup
  \setbox0=\hbox{\includegraphics[width=3in]{#1}}%
\parbox{\wd0}{\box0}\endgroup}

%AB: redefinition of beamer colors
%\setbeamercolor{palette tertiary}{fg=white,bg=mygarnet}
%\setbeamercolor{palette secondary}{fg=white,bg=myg2g}
%\setbeamercolor{palette primary}{fg=black,bg=mygold}
\setbeamercolor{title}{fg=abtitlecolor}
\setbeamercolor{frametitle}{fg=abtitlecolor}
\setbeamercolor{palette tertiary}{fg=white,bg=abtitlecolor}
\setbeamercolor{palette secondary}{fg=white,bg=absecondarycolor}
\setbeamercolor{palette primary}{fg=black,bg=abprimarycolor}
\setbeamercolor{structure}{fg=abtitlecolor}

\setbeamerfont{section in toc}{series=\bfseries}

%AB: remove navigation icons
\beamertemplatenavigationsymbolsempty
\title{
  \textbf {To atoms and beyond!}\\
%\centerline{}
%\centering
%\vspace{-0.0in}
%\includegraphics[width=0.3\textwidth]{propvalues_0093.pdf}
%\vspace{-0.3in}\\
%\label{intrograph}
}

\author[W. Freeman] {Physics 211\\Syracuse University, Physics 211 Spring 2021\\Walter Freeman}

\date{\today}

\begin{document}

\frame{\titlepage}

\frame{\frametitle{\textbf{Announcements}}
  \large
  
Extra office hours and reviews ahead of the exam, held on the class Zoom or in Physics Building basement room B126

\BS

\BI
\item Friday, 8:45 AM - 10:45 AM
\item Next Monday, 9AM - 1PM
\item Next Tuesday, 9AM - 1PM 
\EI

\BS

Homework 10 is short and posted now; it is due Sunday end of day.

\BS
\pause
I have posted some sample ``practice exams'' with answers on the website -- 24 questions that span most of the course. Some are things you have seen before; some are 
things you have not. Recitation tomorrow will pertain to these.

\BS\BS

\pause
The paper:

\BI
\item Due at the latest May 21 (no extensions past this)
\item Earlier submissions will get a little extra credit

\item We are not picky about title, format, etc. -- whatever you do, do it well
\EI

}



\frame{\frametitle{\textbf{Final exam format}}
	
	\large 
	
	
  The final exam will be mostly or all multiple choice.
  
  \pause\BS 
  
  This is, simply, because our TA's are worn out and exhausted, like you are and like I am. We want them to focus on grading your papers.
  \pause\BS
  
  It is likely that each person will have a customized exam; I will figure out the details over the weekend.
  \pause\BS
  
  There are two times when people will be able to take the exam: May 19 (Wednesday) 8AM-10AM and May 20 (Thursday) 3PM-5PM. If you need a time different from these for an exceptional emergency, please let me know well in advance; we may be able to accommodate this.
  
  \pause\BS
  
  I would dearly love to give you all a 24-hour window to take the exam. But I cannot do that because of academic integrity concerns, sadly. If you need an individual different time, we can probably accommodate that.
  
  \pause\BS
  
  I have gotten some emails about things like this and will hopefully catch up tonight.
  
}

\frame{\frametitle{\textbf{Resonance}}

\Large

We've learned that a cavity -- like a violin string or organ pipe -- can contain {\it normal modes of vibration}, with specific frequencies.

\BS

Let's set stuff on fire again and explore this...

}
%
%
%
%\frame{\frametitle{\textbf{Kinematics concepts}}
%  \large
%  \BI
%\item{First derivative of position is velocity; second derivative is acceleration}
%\item{Kinematics lets us connect acceleration, velocity, position, and time}
%\item{If $\vec a$ is constant:}
%
%  \begin{align*}
%    s(t) =& s_0 + v_0 t + \frac{1}{2} a t^2 \\
%    v(t) =& v_0 + at \\
%    v_f^2 - v_0^2 =& 2a\Delta x
%  \end{align*}
%
%\item{These relations hold separately and independently in $x$ and $y$}
%\item{Acceleration is $g$ downwards {\bf if and only if} an object is in freefall}
%\EI
%}
%
%\frame{\frametitle{\textbf{Kinematics sample problem: the ball-and-table problem}}
%  \Large
%
%A ball rolls off of a table of height $h$ at speed $v$. How far does it go?
%
%}
%
%\frame{\frametitle{\textbf{Use kinematics when:}}
%    \Large
%    \BI
%  \item{You need to connect some combination of position, velocity, acceleration, and time}
%  \EI
%
%}
%
%\frame{\frametitle{\textbf{Force concepts and Newton's second law}}
%    \large
%    \BI
%  \item{Newton's second law relates the net force $\sum \vec F$ to the acceleration $\vec a$ of the center of mass of an object}
%  \item{If an object both rotates and moves, $\vec F=m\vec a$ gives you $\vec a$ of the center of mass}
%  \item{Newton's third law: forces come in pairs}
%  \item{Some forces you should know about:}
%    \BI
%  \item{Normal forces: as big as they need to be}
%  \item{Friction: $F_{\rm{fric, static, max}} = \mu_s F_N$, $F_{\rm{fric, kinetic}}=\mu_k F_N$}
%  \item{Traction: a type of static friction, points in direction chosen by the
%vehicle}
%  \item{Elastic: $F=-k\Delta x$}
%  \item{Gravity (Earth): $F=mg$ downward}
%  \item{Gravity (general): $F=\frac{Gm_1m_2}{r^2}$}
%  \item{Tension: A rope pulls on both ends}
%    \EI
%    \EI
%  }
%
%  \frame{\frametitle{\textbf{Force diagrams}}
%    \Large
%    \BI
%  \item{Draw all forces acting on the object, as vectors}
%  \item{If you're going to care about torque, draw the whole object and draw the forces where they act}
%  \item{Gravity acts at the center of mass}
%  \item{Draw these diagrams big enough that you can read them clearly and do trig}
%    \EI
%  }
%
%\frame{\frametitle{\textbf{Uniform circular motion}}
%  \Large
%  \BI
%\item{If an object is traveling in a circle, you know its acceleration is $a_c = \omega^2 r = \frac{v_T^2}{r}$ toward the center}
%\item{Often this will ``give you'' the right side of $F=ma$, and let you conclude something about the left}
%  \EI
%}
%  \frame{\frametitle{\textbf{Use Newton's second law when:}}
%    \Large
%    \BI
%  \item{You need to connect the forces on an object to its acceleration}
%  \item{If you don't need $\vec a$ directly, and don't care about time, maybe use energy methods instead?}
%    \EI
%  }
%
%
%  \frame{\frametitle{\textbf{Sample problem: the eraser in the tube}}
%    \Large
%What angular frequency is required to make the eraser not fall?
%  }
%
%  \frame{\frametitle{\textbf{The work-energy theorem and conservation of energy}}
%    \large
%    \BI
%  \item{Work-energy theorem comes from the third kinematics relation}
%  \item{Two formulations, one with potential energy and one without:}
%    \BI
%  \item{$KE_i + W_{\rm{all}} = KE_f$}
%  \item{$KE_i + PE_i + W_{\rm{other}} = KE_f + PE_f$}
%    \EI
%  \item{Draw {\it clear} before and after snapshots}
%  \item{Figure out work done in going from one to the other}
%  \item{Work = $\vec F \cdot d\vec s$}
%    \EI
%  }
%
% \frame{\frametitle{\textbf{Use energy methods when:}}
%      \Large
%      \BI
%    \item{You don't know and don't care about time}
%    \item{You can account for the work done by all forces involved}
%    \item{This is {\bf not} true at the instant of a collision -- use momentum instead}
%      \EI
%    }
%
%    \frame{\frametitle{\textbf{Sample problem: energy}}
%      \Large
%A ball rolls down a hill of height $h$ and across a table. 
%How fast is it moving at the 
%edge of the table?    
%}
%
%  \frame{\frametitle{\textbf{Conservation of momentum}}
%      \BI
%    \item{In the absence of external forces, $\vec p = m\vec v$ is conserved}
%    \item{This is a consequence of Newton's third law}
%    \item{Collisions and explosions are short enough that external forces are small}
%    \item{Momentum is a vector and is conserved separately in $x$ and $y$}
%      \EI
%    }
%
%  \frame{\frametitle{\textbf{Use conservation of momentum when:}}
%        \Large
%        \BI
%      \item{You have a collision or explosion and need to connect the velocities before to the velocities after}
%        \EI
%      }
%
%   \frame{\frametitle{\textbf{Rotation}}
%          \Large
%          Many ideas here, most analogous to translational motion:
%\large
%\BI
%\item{Torque plays the role of force: $\tau = F_\perp r = F r_\perp$}
%\item{Moment of inertia plays the role of mass: $I = \lambda mr^2$}
%\item{$\vec F = m \vec a \rightarrow \tau = I \alpha$: ``Newton's second law for rotation''}
%\item{Rolling motion is translation plus rotation: $v = \pm \omega r$, $a = \pm \alpha r$}
%\item{\bf You must think about the signs here}
%\item{Rotational kinetic energy: $KE_{\rm{rot}} = \frac{1}{2} I \omega^2$}
%\item{Angular momentum: $L = I \omega$}
%  \EI
%}
%
%\frame{\frametitle{\textbf{Static equilibrium problems}}
%  \Large
%  \BI
%\item{Net torque is zero about any pivot}
%\item{Net force is zero (you may not need this)}
%\item{Torque due to any force applied {\bf at} the pivot is zero}
%\EI
%}
%
%\frame{\frametitle{\textbf{The process of science}}
%\Large
%
%Properties of scientific thought:
%\normalsize
%\BI
%\item Empiricism: science relies on the natural world itself as the only true authority
%\item Self-skepticism: people making scientific claims should search for and engage with potentially refuting evidence
%\item Universality: the laws of nature apply everywhere and everywhen, and to all things equally
%\item Objectivity: scientific ideas are independent of any particular human perspective
%\EI
%
%\BS
%\Large
%Ways that this can go wrong:
%\normalsize
%\BI
%\item Cherry-picking
%\item Arguments ad hominem (``they're wrong because they're ugly'') or from authority (``they're right because they have a fancy degree'')
%\item Bad statistics / publication bias
%\item Ignoring refuting evidence 
%\item Manufactured controversy
%\item Arguments from sensationalism (``XYZ is true {\it because} it's exciting'')
%\item ... and others
%\EI
%}
%
%\frame{\frametitle{\textbf{Final reminders}}
%  \large
%  \BI
%\item{Huge amounts of extra review available; use it}
%\item{Get some rest during finals week and take care of yourselves}
%\item{If you're affected by the Calc/Physics exam scheduling nonsense, tell SU!}
%\EI
%}

\frame{\frametitle{\textbf{The power of mechanics}}
\large
The things we've studied in this class are more powerful than you think. 

If you call up a chemist, she'll tell you the approximate force law between two noble gas atoms:

$$ F(r) = \frac{\alpha}{r^{12}} - \frac{\beta}{r^6} $$

Put this into a computer and let it go:

\pause\bigskip

\centerline{\color{Red}We can understand freezing, melting, and boiling just with $\vec F = m \vec a!$}

\centerline{\color{Red}... we can even get the ideal gas law for free along the way!}
}


\frame{\frametitle{\textbf{The rest of physics}}
\large
The other disciplines of physics are variants on what you've learned already:

\BI
\item Electromagnetism (PHY 212) introduces a new force -- just another $\vec F$
\item All you'll do in that class is apply the work-energy theorem and so on to this new force
\BI
\item Light is just a particular manifestation of that force
\EI
\item Statistical mechanics uses statistics to understand 
$\vec F = m\vec a$ acting on a great many particles at once
\item Relativity mixes up space and time, changing the coordinates on us
\item Quantum mechanics mixes up ``particle'' and ``wave''
\EI

Each of these disciplines is supported by a ``three-legged stool'':

\BI
\item Theory: understanding principles and using pen and paper to study them in simple situations (this class)
\item Experiment: designing tests for these principles and building machines to carry them out (221)
\item Computation: using computers to simulate those principles in more complicated situations and study their consequences (my field and class in the fall)
\EI
}
\frame{\frametitle{\textbf{Two invitations}}
\Large
\BC
Like what you've done here? We have multiple options for you to study more physics!
\EC

You could get a {\color{Red}physics minor}. This involves:

\BI
\item Physics 211 (you have this now!)
\item Physics 212 (you will probably take this next semester!)
\item Four more classes at the 300 level of your choice. For instance:
\BI
\item Biophysics: the physics of living things -- how do cells do what they do?
\item Cosmology: the history and future of the Universe!
\item Astrophysics
\item Computational physics (my class in the fall -- all of you are qualified!)
\item Modern physics (quantum mechanics, relativity, atoms)
\item Waves and vibrations: light and sound
\item Advanced laboratory 
\item ... and others I'm forgetting!
\EI
\EI
}


\frame{\frametitle{\textbf{Two invitations}}
\Large
\BC
... or maybe you want to be a physics major! (Come to the dark side -- we have both cookies and the cheat-codes to the Universe!)


\EC


\begin{columns}

\column{0.5\textwidth}
\color{Green}
\Large
\BC Bachelor of Arts\EC
\small      

This degree program prepares you for jobs in industry,
and is also a great double major option with engineering,
computer science, education, and all sorts of things:

\BI
\color{Green}
\item Physics 211/212
\item 300-level class on modern physics (quantum mechanics, relativity, atoms -- the good stuff!)
\item 300-level lab class
\item 5 more elective classes (astrophysics, computational physics, biophysics, cosmology... lots of stuff)
\item 30 physics credits total (you have four, plus four if you took AST101)
\EI


\column{0.5\textwidth}
\color{Blue}
\Large
\BC Bachelor of Science\EC
\small     
\color{Blue}

This degree program prepares you for the most technically demanding 
industry jobs, as well as graduate study in physics or related fields.

It is also a good double major option for other STEM fields, in particular
engineering (there are overlaps in the required classes)

\BI
\color{Blue}
\item Physics 211/212
\item 300-level class on modern physics (quantum mechanics, relativity, atoms -- the good stuff!)
\item 300-level lab class
\item Rigorous courses in computational physics, electromagnetism, quantum mechanics, thermodynamics, and others
\item 39 physics credits total (you have four now!)
\EI

\end{columns}
}

\frame{\frametitle{\textbf{Two invitations}}
\Large

If you've done reasonably well in this course, and have strong communication skills, Physics 211 wants to offer you a job!

\BS

We're always looking for good people to work for us as coaches in future years. Want to help next year's class, have fun,
earn some money, and {\color{Red}get a job that looks great on your resume?}

\BS\pause

Come talk to us!

}


\frame{\frametitle{\textbf{And, finally...}}
  \Large
... thank you all.

\pause
\bigskip
\bigskip
\bigskip
\bigskip

\centerline{``Science is the belief in the ignorance of experts.'' (R. Feynman)}

\pause

\bigskip
\bigskip
\bigskip
\bigskip



\centerline{``All science is either physics or stamp collecting.'' (E. Rutherford)}
}
\frame{\frametitle{\textbf{I leave you with two quotes...}}
\pause
``Nature uses only the longest threads to weave her patterns, so that each small piece of her fabric reveals the organization of the entire tapestry.''

\bigskip
\bigskip

\pause

``Poets say science takes away from the beauty of the stars -- mere globs of gas atoms. Nothing is "mere". I too can see the stars on a desert night, and feel them. But do I see less or more? The vastness of the heavens stretches my imagination -- stuck on this carousel my little eye can catch one-million-year-old light. A vast pattern -- of which I am a part... What is the pattern, or the meaning, or the why? It does not do harm to the mystery to know a little about it. For far more marvelous is the truth than any artists of the past imagined!''

\bigskip

--Richard Feynman, from {\em Lectures on Physics}
}

\frame{
\BC
\Huge
Thanks for a wonderful semester!
\EC
}

\end{document}
