
\documentclass[12pt]{article}
\setlength\parindent{0pt}
\usepackage{fullpage}
\usepackage{amsmath}
\usepackage{graphicx}
\setlength{\parskip}{4mm}
\def\LL{\left\langle}   % left angle bracket
\def\RR{\right\rangle}  % right angle bracket
\def\LP{\left(}         % left parenthesis
\def\RP{\right)}        % right parenthesis
\def\LB{\left\{}        % left curly bracket
\def\RB{\right\}}       % right curly bracket
\def\PAR#1#2{ {{\partial #1}\over{\partial #2}} }
\def\PARTWO#1#2{ {{\partial^2 #1}\over{\partial #2}^2} }
\def\PARTWOMIX#1#2#3{ {{\partial^2 #1}\over{\partial #2 \partial #3}} }
\newcommand{\BE}{\begin{displaymath}}
\newcommand{\EE}{\end{displaymath}}
\newcommand{\BNE}{\begin{equation}}
\newcommand{\ENE}{\end{equation}}
\newcommand{\BEA}{\begin{eqnarray}}
\newcommand{\EEA}{\nonumber\end{eqnarray}}
\newcommand{\EL}{\nonumber\\}
\newcommand{\la}[1]{\label{#1}}
\newcommand{\ie}{{\em i.e.\ }}
\newcommand{\eg}{{\em e.\,g.\ }}
\newcommand{\cf}{cf.\ }
\newcommand{\etc}{etc.\ }
\newcommand{\Tr}{{\rm tr}}
\newcommand{\etal}{{\it et al.}}
\newcommand{\OL}[1]{\overline{#1}\ } % overline
\newcommand{\OLL}[1]{\overline{\overline{#1}}\ } % double overline
\newcommand{\OON}{\frac{1}{N}} % "one over N"
\newcommand{\OOX}[1]{\frac{1}{#1}} % "one over X"



\begin{document}
\Large
\centerline{\sc{Recitation Questions}}
\normalsize
\centerline{\sc{Week of 16 March}}

\begin{enumerate}
\item{Someone drops a penny of mass 2.5g off of the Empire State Building (height 380 m). It strikes the ground traveling at 50 m/s, having been slowed somewhat by air resistance.}
\begin{enumerate}
\item{With what velocity would it have struck the ground if there were no air resistance?}
\item{What was the work done by the drag force?}
\item{This penny strikes the sidewalk and penetrates the surface, digging a hole 2 cm deep. What was the upward force exerted on the penny by the pavement?}
\end{enumerate}


  \item{A police officer sets up a speed trap to catch cars driving over the speed limit coming around a curve. A car comes around the curve and sees the officer, and immediately slams on his brakes to slow down before the officer can take a speed reading. 
By the time the officer measures the car's speed, the car is traveling 25 m/s, in an area where the speed limit is 30 m/s. However, the officer pulls over the driver anyway, saying ``I saw you slam on your brakes. You must have been speeding!''

The car's driver protests the ticket in court. She says to the magistrate, ``Your Honor, I can prove that I never exceeded the speed limit. It's true that I slammed on my brakes out of reflex as soon as I saw the officer. But I went back and measured the marks
my tires left on the ground. Those marks are only 10.6 meters long, and by braking for that distance there's no way I could have decelerated from over the speed limit down to the 25 m/s that your officer measured.''

Should the magistrate believe the driver? Could the car have been speeding when she first applied her brakes? Note that you will need to figure out the frictional force applied by the car's brakes, and to do that you will need to estimate the coefficient of 
friction between the tires and the pavement.}

\item{A rock climber of mass 70 kg is climbing a cliff face when she slips and falls. There is 4m of slack in her climbing rope, so she undergoes free fall for 4 meters before the rope begins to arrest her fall. If the stiffness in her rope is 1400 N/m, then:}
  \begin{enumerate}
    \item{How far will she fall in total?}
    \item{What is the maximum force that her rope will exert on her as it arrests her fall?}
   \end{enumerate}

 \item{A laptop battery says it has a capacity of 51 ``watt-hours''.}
   \begin{enumerate}
     \item{What are the dimensions of this odd unit ``watt-hour'', and what does it measure? What is 51 watt-hours in more familiar units?}
     \item{If this battery were used to power an electric motor, how high could it lift the battery? Assume the battery has a mass of 200 grams.}
   \end{enumerate}

 \item{A ball of mass $m$ on a cord of length $L$ is held at an angle $\theta$ to the left of the vertical and released. A very strong wind blows from left to right, exerting a constant horizontal force $F$. }

     \begin{enumerate}
       \item{Find the speed of the ball at the bottom of its swing.}
       \item{Find an equation for the maximum angle that the ball reaches when it swings to the right. You do not need to actually solve it, since it's messy and involves a lot of trig identities; just write it down.}
       \item{When the ball swings back to the left, find the height that it reaches. Will it come
         back to the same point where it was released?}
     \end{enumerate}
 \end{enumerate}
 \end{document}
