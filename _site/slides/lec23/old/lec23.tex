% DPF 09 talk on strangeness in nucleon

\documentclass[10pt]{beamer}
\usepackage{amsmath}
\usefonttheme{professionalfonts} % using non standard fonts for beamer
\usefonttheme{serif} % default family is serif\
\usepackage{mathtools}
%\documentclass[12pt]{beamerthemeSam.sty}
\usepackage{epsf}
\usepackage{ulem}
\usepackage{array}
%\usepackage{pstricks}
%\usepackage[orientation=portrait,size=A4]{beamerposter}
\geometry{paperwidth=160mm,paperheight=120mm}
%DT favorite definitions
\def\LL{\left\langle}	% left angle bracket
\def\RR{\right\rangle}	% right angle bracket
\def\LP{\left(}		% left parenthesis
\def\RP{\right)}	% right parenthesis
\def\LB{\left\{}	% left curly bracket
\def\RB{\right\}}	% right curly bracket
\def\PAR#1#2{ {{\partial #1}\over{\partial #2}} }
\def\PARTWO#1#2{ {{\partial^2 #1}\over{\partial #2}^2} }
\def\PARTWOMIX#1#2#3{ {{\partial^2 #1}\over{\partial #2 \partial #3}} }

\def\rightpartial{{\overrightarrow\partial}}
\def\leftpartial{{\overleftarrow\partial}}
\def\diffpartial{\buildrel\leftrightarrow\over\partial}

\def\BI{\begin{itemize}}
\def\EI{\end{itemize}}
\def\BE{\begin{displaymath}}
\def\EE{\end{displaymath}}
\def\BEA{\begin{eqnarray*}}
\def\EEA{\end{eqnarray*}}
\def\BNEA{\begin{eqnarray}}
\def\ENEA{\end{eqnarray}}
\def\EL{\nonumber\\}
\def\BS{\bigskip}
\def\BC{\begin{center}}
\def\EC{\end{center}}
\def\BCC{\begin{columns}}
\def\ECC{\end{columns}}
\def\HC{\column{0.5\textwidth}}
\newcommand{\map}[1]{\frame{\frametitle{\textbf{Course map}}
\centerline{\includegraphics[height=0.86\paperheight]{../../map/#1.png}}}}
\newcommand{\wmap}[1]{\frame{\frametitle{\textbf{Course map}}
\centerline{\includegraphics[width=0.96\paperwidth]{../../map/#1.png}}}}

\newcommand{\etal}{{\it et al.}}
\newcommand{\gbeta}{6/g^2}
\newcommand{\la}[1]{\label{#1}}
\newcommand{\ie}{{\em i.e.\ }}
\newcommand{\eg}{{\em e.\,g.\ }}
\newcommand{\cf}{cf.\ }
\newcommand{\etc}{etc.\ }
\newcommand{\atantwo}{{\rm atan2}}
\newcommand{\Tr}{{\rm Tr}}
\newcommand{\dt}{\Delta t}
\newcommand{\op}{{\cal O}}
\newcommand{\msbar}{{\overline{\rm MS}}}
\def\chpt{\raise0.4ex\hbox{$\chi$}PT}
\def\schpt{S\raise0.4ex\hbox{$\chi$}PT}
\def\MeV{{\rm Me\!V}}
\def\GeV{{\rm Ge\!V}}

%AB: my color definitions
%\definecolor{mygarnet}{rgb}{0.445,0.184,0.215}
%\definecolor{mygold}{rgb}{0.848,0.848,0.098}
%\definecolor{myg2g}{rgb}{0.647,0.316,0.157}

\definecolor{A}{rgb}{0.8,0.0,0.0}
\definecolor{B}{rgb}{0.0,0.6,0.0}
\definecolor{C}{rgb}{0.4,0.4,0.0}
\definecolor{D}{rgb}{0.0,0.0,0.5}
\definecolor{E}{rgb}{0.4,0.4,0.4}


\definecolor{abtitlecolor}{rgb}{0.0,0.255,0.494}
\definecolor{absecondarycolor}{rgb}{0.0,0.416,0.804}
\definecolor{abprimarycolor}{rgb}{1.0,0.686,0.0}
\definecolor{Red}           {cmyk}{0,1,1,0}
\definecolor{Grey}           {cmyk}{.7,.7,.7,0}
\definecolor{Lg}           {cmyk}{.4,.4,.4,0}
\definecolor{Blue}          {cmyk}{1,1,0,0}
\definecolor{Green}         {cmyk}{1,0,1,0}
\definecolor{Brown}         {cmyk}{0,0.81,1,0.60}
\definecolor{Black}         {cmyk}{0,0,0,1}

\usetheme{Madrid}
\newcommand{\vcenteredinclude}[1]{\begingroup
  \setbox0=\hbox{\includegraphics[width=3in]{#1}}%
\parbox{\wd0}{\box0}\endgroup}

%AB: redefinition of beamer colors
%\setbeamercolor{palette tertiary}{fg=white,bg=mygarnet}
%\setbeamercolor{palette secondary}{fg=white,bg=myg2g}
%\setbeamercolor{palette primary}{fg=black,bg=mygold}
\setbeamercolor{title}{fg=abtitlecolor}
\setbeamercolor{frametitle}{fg=abtitlecolor}
\setbeamercolor{palette tertiary}{fg=white,bg=abtitlecolor}
\setbeamercolor{palette secondary}{fg=white,bg=absecondarycolor}
\setbeamercolor{palette primary}{fg=black,bg=abprimarycolor}
\setbeamercolor{structure}{fg=abtitlecolor}

\setbeamerfont{section in toc}{series=\bfseries}

%AB: remove navigation icons
\beamertemplatenavigationsymbolsempty
\title{
  \textbf {Waves}\\
%\centerline{}
%\centering
%\vspace{-0.0in}
%\includegraphics[width=0.3\textwidth]{propvalues_0093.pdf}
%\vspace{-0.3in}\\
%\label{intrograph}
}

\author[W. Freeman] {Physics 211\\Syracuse University, Physics 211 Spring 2017\\Walter Freeman}

\date{\today}

\begin{document}

\frame{\titlepage}

\frame{\frametitle{\textbf{Announcements}}
\large
\BI
\item{HW9 posted tomorrow, due next Friday}
\item{Extra credit homework assignment posted Friday, due by 5PM on May 2}
\BI
\large
\item Difficult analytical problems, like you've been doing
\item Conceptual applications to engineering -- interpretation problems
\item Choose one of the two
\item Up to 2 points on your final course grade
\EI
\EI
\BS
\BS
\BC
\pause
Next Tuesday we're talking about the physics of musical instruments.

Want to demonstrate your instrument and study how it works? Come talk to me!
\EC
}

\frame{\frametitle{\textbf{Exam 3 comments}}
}

\frame{\frametitle{\textbf{Preparation for the final}}
\large
\BI
\item You can drop one exam
\item You {\bf can't} drop the final
\item The final will involve more, easier problems
\item You can expect more conceptual things and less algebra
\item There will be lots of review sessions, etc.
\EI
}


\frame{\frametitle{\textbf{Waves, an overview}}
 \large
  \BI
\item{The next few classes are going to focus on the physics of waves}
\item{We'll use strings and tubes -- musical instruments -- as examples}
\item{... but all waves behave the same!}
  \BI
\Large
\item{Light waves}
\item{Radio waves: an antenna is just like waves on a string!}
\item{Sound waves}
\item{Water waves}
  \pause
\item{Matter waves in quantum mechanics: $s, p, d, f$ orbitals!}
  \EI
  \EI
}

\frame{\frametitle{\textbf{Waves in 1D -- modeling}}
\large
  \BI
\item{Start with something empirical: can we model a vibrating string based on what we know so far?}
  \EI

\Large
\bigskip

Which equation that you've learned could be used to understand a vibrating string?

\bigskip

\BI
\item A: $\vec x_f = \vec x_i + \vec v_0 t + \frac{1}{2}\vec a t^2$
\item B: $\vec p_i = \vec p_f$
\item C: $F = -k(x - x_0)$
\item D: $F_c = m\omega^2 r$
\EI

}


\frame{\frametitle{\textbf{Waves in 1D -- modeling}}
\large
\BI
\item{Hooke's law describes elasticity, right?}
\item{Connect some Hooke's law springs between two points (simple3.c)}
  \pause
\item{This isn't very flexible, is it?}
\EI

\bigskip
\Large

How could we make this more accurate using the physics we know?

\bigskip

\BI
\item Make the springs curved
\item Use a smaller amount of time between ``steps''
\item Use more individual springs
\item Use a larger spring constant
\EI
}




\frame{\frametitle{\textbf{Waves in 1D -- modeling}}
\Large
Use more springs and masses (simple10.c):
  \pause

\BS
If we use very many of them, we should get ``real'' behavior


  \BI
\item How much math is our computer doing here?
\BI
\large
\item 10 segments
\item X and Y directions
\item Position, velocity, Hooke's-law force
\item Calculating $r$ requires a square root -- computer has to sum a power series
\item Even drawing those little arrows requires trig, which means more power series
\item This is a {\bf lot} of math
\pause
\item Computers can do a few hundred million operations a second! This is cake.
\EI
\bigskip
\item{Like pixels on a digital display: we forget that they're there!}
\item{Now, what can we learn from how this behaves?}
\EI
}

\frame{\frametitle{\textbf{Waves in 1D -- learning from our model}}
\large
  Some important properties: (pulse.c: width/stiffness/tension) 
  \BI
\item{Pulses (regardless of their size or shape) go at a constant speed}
\item{{\bf The wave speed $c$} refers to how fast pulses travel down the string}
\item Empirically, we see that the wave speed depends on the {\bf tension}
(one of the inputs to my model)

\bigskip

\item{The property of {\bf linearity:} (twopulse.c)}
  \BI
\item{Multiple pulses can pass through each other without interference}
\item{We will take this as absolutely true for our study here}
\item{Often not quite true for real waves -- very interesting behavior!}
  \EI
\item{Does a real string do this?}
  \BI
  \pause
\item{Wave speed $c$ goes up with more tension!}
  \EI
  \EI
}

\frame{\frametitle{\textbf{Sine waves}}
  \BI
\item{We're particularly concerned with waves that look like sines and cosines (sines.c: wavelength/c/A1/A2/xlabel)}
\item{These waves have two new properties: {\bf wavelength $\lambda$} and {\bf frequency $f$}}
\BI
\item{Wavelength: distance from crest to crest}
\item{Frequency: how many crests go by per second, equal to $1/T$ ($T$ = period)}
  \pause
\item{Speed = distance $\times$ time}
\EI
\EI
\Huge
\BS
  $$c=\lambda f$$
}

\frame{\frametitle{\textbf{Sine waves}}
\Large
Suppose I have this speaker here beeping at 500 Hz.

The speed of sound in air is about 340 m/s. What is the wavelength of the sound?

\BS\BS

\BI
\item A: About a meter
\item B: About 60 cm
\item C: About 1.5 m
\item D: About 2 m
\item E: About 0.5 m
\EI
}

\frame{\frametitle{\textbf{Sine waves}}
\Large
Suppose I have this speaker here beeping at 500 Hz.

What happens if I put it underwater ($c \approx 1500$ m/s)
instead of air ($c \approx 340$ m/s)?

\BS
\BS
\BI
\item A: The frequency will go up
\item B: The frequency will go down
\item C: The wavelength will go down
\item D: The wavelength will go up
\pause
\item E: Sam will be mad at me, since I broke his speaker
\EI
}


\frame{\frametitle{\textbf{Standing waves}}
\Large
What kind of sine and cosine waves can we put on our string?

  \BI
\item{Not any wavelengths will do, since the ends have to be fixed}
\item I clearly can't do this with just one sine wave
\pause
\item I need two, one going in each direction!
\EI

Are there other wavelengths of standing waves that will work?

\BS

\BI
\item A: Twice the wavelength
\item B: Half the wavelength
\item C: Three times the wavelength
\item D: One-third the wavelength
\EI
}


\frame{\frametitle{\textbf{Standing waves, in more detail}}
  \Large
  \vcenteredinclude{mode1-crop.pdf} Fundamental: $\lambda = \frac{2L}{1}$\\
 \bigskip
  \vcenteredinclude{mode2-crop.pdf} 2nd harmonic: $\lambda = \frac{2L}{2}$\\
 \bigskip

  \vcenteredinclude{mode3-crop.pdf} 3rd harmonic: $\lambda = \frac{2L}{3}$\\
 \bigskip

  \vcenteredinclude{mode4-crop.pdf} 4th harmonic: $\lambda = \frac{2L}{4}$\\
 \bigskip

  \bigskip
  \centerline{Can we write these wavelengths in terms of $f$ using $c=f\lambda$?}
}

\frame{\frametitle{\textbf{Standing waves, in more detail}}
  \Large
  \vcenteredinclude{mode1-crop.pdf} Fundamental: $f_1 = \frac{c}{2L}$\\
 \bigskip
  \vcenteredinclude{mode2-crop.pdf} 2nd harmonic: $f_2 = 2 f_1 $\\
 \bigskip
  \vcenteredinclude{mode3-crop.pdf} 3rd harmonic: $f_3 = 3 f_1$ \\
 \bigskip
  \vcenteredinclude{mode4-crop.pdf} 4th harmonic: $f_4 = 4 f_1$ \\

 \bigskip

}
\frame{\frametitle{\textbf{Standing waves, in more detail}}
\Large
A simulation: harm.c and resonances.c

\BS
\BS

}

\frame{
\BC
\Huge
Why do I have this blowtorch?
\EC
}

\end{document}
