\documentclass[12pt]{article}
\setlength\parindent{0pt}
\usepackage{fullpage}
\usepackage{amsmath}
\usepackage{graphicx}
\setlength{\parskip}{4mm}
\def\LL{\left\langle}   % left angle bracket
\def\RR{\right\rangle}  % right angle bracket
\def\LP{\left(}         % left parenthesis
\def\RP{\right)}        % right parenthesis
\def\LB{\left\{}        % left curly bracket
\def\RB{\right\}}       % right curly bracket
\def\PAR#1#2{ {{\partial #1}\over{\partial #2}} }
\def\PARTWO#1#2{ {{\partial^2 #1}\over{\partial #2}^2} }
\def\PARTWOMIX#1#2#3{ {{\partial^2 #1}\over{\partial #2 \partial #3}} }
\newcommand{\BE}{\begin{displaymath}}
\newcommand{\EE}{\end{displaymath}}
\newcommand{\BNE}{\begin{equation}}
\newcommand{\ENE}{\end{equation}}
\newcommand{\BEA}{\begin{eqnarray}}
\newcommand{\EEA}{\nonumber\end{eqnarray}}
\newcommand{\EL}{\nonumber\\}
\newcommand{\la}[1]{\label{#1}}
\newcommand{\ie}{{\em i.e.\ }}
\newcommand{\eg}{{\em e.\,g.\ }}
\newcommand{\cf}{cf.\ }
\newcommand{\etc}{etc.\ }
\newcommand{\Tr}{{\rm tr}}
\newcommand{\etal}{{\it et al.}}
\newcommand{\OL}[1]{\overline{#1}\ } % overline
\newcommand{\OLL}[1]{\overline{\overline{#1}}\ } % double overline
\newcommand{\OON}{\frac{1}{N}} % "one over N"
\newcommand{\OOX}[1]{\frac{1}{#1}} % "one over X"



\begin{document}
\Large
\centerline{\sc{Recitation Questions}}
\normalsize
\centerline{\sc{Week of April 10}}
\small

\medskip

\centerline{\large Question 1: on rotational energy (Wednesday)} 

\medskip

Consider a person of mass $M$ standing on a rotating platform and holding two dumbbells of 
mass $m$
in her hands, as we saw in class. Model the person as a cylinder of radius $r$ (thus, she has
a moment of inertia $\frac{1}{2}Mr^2$). Recall the lecture demo:

\begin{itemize}
\item{She held the dumbbells at arm's length, and someone spun her and the platform around
its axis at angular velocity $\omega_0$}
\item{She then pulled the dumbbells inward}
\item{... and started rotating faster, with a new angular velocity $\omega_f$}
\end{itemize}

Suppose that her arms are length $L$, and that she pulls the dumbbells all the way to the 
center (thus, their radius becomes zero).

a) As a warmup, calculate her new angular velocity in terms of $M$, $m$, $r$, $L$, and $\omega_0$. 

Make sensible assumptions for the values of $M$, $m$, $r$, and $L$. By what fraction does
her angular velocity increase?

\newpage

Now, let's think about the energy involved. Write an expression for her rotational kinetic
energy ${\rm KE}_{\rm rot}=\frac{1}{2}I\omega^2$ before and after she pulls the weights in. You'll find that they are not equal, and
that she has {\it more} kinetic energy after she pulls the weights inward.

How do you explain this? How is it possible for a system to increase its total energy
simply by rearranging its mass?

According to the work-energy theorem, in order for the rotational kinetic energy to increase,
something must do positive work on the system. What force is that?

(Discuss this with your peers, your coach, and your TA. The answer here is somewhat subtle.)

\newpage
\centerline{\large Question 2: on the work-energy theorem (Wednesday)} 

A ball of mass $m$ on a cord of length $L$ is held at an angle $\theta$ to the left of the
vertical and released. A very strong wind blows from left to right, exerting a constant horizontal
force $F$.

a) Find the speed of the ball at the bottom of its swing.

\vspace{2in}

b) Find an equation for the maximum angle that the ball reaches when it swings to the 
right. You don't need to actually solve it, since it's messy and involves a lot of trig
identities; just write it down.

\vspace{3in}

c) When the ball swings back to the left, find the height that it reaches. Will it 
come back to the same point at which it was released? (If you're clever this part requires
no algebra at all, just words.)

\newpage
\centerline{\large Question 3: on torque} 

\vspace{1in}

  A light cable is wound around a cylindrical spool fixed in place of radius 50 cm and mass 10 kg. One end of the cable is attached to a motor, which pulls with a constant force of 20 N on the cable. When the motor is switched on, the force exerted by the cable causes the spool to rotate faster and faster.
\begin{enumerate}
      \item{What is the moment of inertia of the spool?}
\vspace{0.7in}
      \item{What is the torque applied to the spool by the motor?}
\vspace{0.7in}
      \item{What is the angular acceleration of the spool?}
\vspace{0.7in}
      \item{How long will it take for the spool to make a full revolution?}
\vspace{0.7in}
      \item{After five seconds, how fast is the cable moving?}
\vspace{0.7in}
      \item{After five seconds, what is the kinetic energy of the spool?}
\vspace{0.7in}
      \item{What is the work done by the motor in five seconds?}
     \end{enumerate}
\newpage

\centerline{\large Question 4: on torque}

 A unicyclist rides at a constant speed of 5 m/s; she and her unicycle have a combined mass of 70 kg. The wheel of her unicycle has a radius of 50 cm. At this speed, air resistance exerts a force of 80 N on her.

       1) What is the angular velocity of the wheel?
\vspace{1.2in}

       2) As you know, the force that wheeled vehicles use to propel themselves forward is static friction. What is the size of this force?
\vspace {1.2in}

       3) What torque must she apply to the wheel to maintain her speed?
\vspace{2in}

\newpage

       4) Suppose the pedals are attached to a crank with a radius of 25 cm. What force must she apply to the pedals to maintain her speed?
\vspace{3in}

       5) What power does she apply to the pedals? What power does the force of static friction apply to the unicycle? What power does the air resistance apply?

 


\end{document}
