
\documentclass[12pt]{article}
\setlength\parindent{0pt}
\usepackage{fullpage}
\usepackage{amsmath}
\usepackage{graphicx}
\setlength{\parskip}{4mm}
\def\LL{\left\langle}   % left angle bracket
\def\RR{\right\rangle}  % right angle bracket
\def\LP{\left(}         % left parenthesis
\def\RP{\right)}        % right parenthesis
\def\LB{\left\{}        % left curly bracket
\def\RB{\right\}}       % right curly bracket
\def\PAR#1#2{ {{\partial #1}\over{\partial #2}} }
\def\PARTWO#1#2{ {{\partial^2 #1}\over{\partial #2}^2} }
\def\PARTWOMIX#1#2#3{ {{\partial^2 #1}\over{\partial #2 \partial #3}} }
\newcommand{\BE}{\begin{displaymath}}
\newcommand{\EE}{\end{displaymath}}
\newcommand{\BNE}{\begin{equation}}
\newcommand{\ENE}{\end{equation}}
\newcommand{\BEA}{\begin{eqnarray}}
\newcommand{\EEA}{\nonumber\end{eqnarray}}
\newcommand{\EL}{\nonumber\\}
\newcommand{\la}[1]{\label{#1}}
\newcommand{\ie}{{\em i.e.\ }}
\newcommand{\eg}{{\em e.\,g.\ }}
\newcommand{\cf}{cf.\ }
\newcommand{\etc}{etc.\ }
\newcommand{\Tr}{{\rm tr}}
\newcommand{\etal}{{\it et al.}}
\newcommand{\OL}[1]{\overline{#1}\ } % overline
\newcommand{\OLL}[1]{\overline{\overline{#1}}\ } % double overline
\newcommand{\OON}{\frac{1}{N}} % "one over N"
\newcommand{\OOX}[1]{\frac{1}{#1}} % "one over X"



\begin{document}
\Large
\centerline{\sc{Recitation Questions}}
\normalsize
\centerline{\sc{Week of 30 March}}

\begin{enumerate}
  \item{A light cable is wound around a cylindrical spool fixed in place of radius 50 cm and mass 10 kg. One end of the cable is attached to a motor, which pulls with a constant force of 20 N on the cable. When the motor is switched on, the force exerted by the cable causes the spool to rotate faster and faster.}
    \begin{enumerate}
      \item{What is the moment of inertia of the spool?}
      \item{What is the torque applied to the spool by the motor?}
      \item{What is the angular acceleration of the spool?}
      \item{How long will it take for the spool to make a full revolution?}
      \item{After five seconds, how fast is the cable moving?}
      \item{After five seconds, what is the kinetic energy of the spool?}
      \item{What is the work done by the motor in five seconds?}
     \end{enumerate}
 
   \item{A hoop, a disk, and a ball roll down a hill. If the hoop is traveling at speed $v$ when it reaches the bottom, how fast are the other two traveling?

       {\em (While it may not seem so, you do in fact have enough information to solve this problem. I'll leave it to you and your peers to think about what technique to use!)}
     }


   \item{A unicyclist rides at a constant speed of 5 m/s; she and her unicycle have a combined mass of 70 kg. The wheel of her unicycle has a radius of 50 cm. At this speed, air resistance exerts a force of 80 N on her.}
     \begin{enumerate}
       \item{What is the angular velocity of the wheel?}
       \item{As you know, the force that wheeled vehicles use to propel themselves forward is static friction. What is the size of this force?}
       \item{What torque must she apply to the wheel to maintain her speed?}
       \item{Suppose the pedals are attached to a crank with a radius of 25 cm. What force must she apply to the pedals to maintain her speed?}
     \end{enumerate}

   \item{A Yo-Yo consists of a cylinder of mass $m$ and radius $r$. A slot is cut in the middle of the cylinder such that the inner radius is only $0.4r$, and a string is wound around the middle. (If you don't know what a Yo-Yo is, there is an animation on Wikipedia.)
     A person holds the string and allows the Yo-Yo to fall. As it falls, it has both a linear acceleration (moving downward) and an angular acceleration (spinning faster and faster).}
     \begin{enumerate}
       \item{What is the relation between the linear velocity $v$ of the Yo-Yo (moving downward) and its angular velocity $\omega$?}
       \item{What is the relation between the linear acceleration $a$ of the Yo-Yo (moving downward) and its angular velocity $\alpha$? (Note: this is trivial once you've done the previous part...)}
       \item{Draw a force diagram for the Yo-Yo, indicating the location where all forces act as well as their magnitude.}
       \item{Write down Newton's second law ($F=ma$) for the Yo-Yo, putting in expressions for the various forces.}
       \item{Write down ``Newton's second law for rotation'' $\tau = I \alpha$, putting in expressions for the net torque and the moment of inertia.}
       \item{What is the acceleration of the Yo-Yo and the tension in the string?}
       \item{Will the Yo-Yo accelerate faster or slower if the inner radius is changed to $0.2r$?}
     \end{enumerate}

   \item{A horizontal disk of mass 2 kg and radius 30 cm is covered with sand; the coefficient of friction between the sand and the turntable is 0.5. A motor applies a constant torque $\tau = 2 \rm N \cdot \rm m$. Once the motor is switched on, how long will it 
     take before some of the sand flies off the turntable?}

   \item{An untrained bowler throws a bowling ball down a lane without applying any spin to it. The ball contacts the wood with an initial velocity $v$ and no angular velocity. The coefficient of kinetic friction between the ball and the lane is $\mu_k$. 
     Gradually, the ball begins to spin as it slides down the lane. How far will the ball travel before it begins to roll without slipping?
     
     Here are some hints for the problem:
     \begin{enumerate}
       \item{What is the net force on the ball? What acceleration does it cause?}
       \item{What is the net torque on the ball? What angular acceleration does it cause?}
       \item{What is the condition on $v$ and $\omega$ that must be satisfied for the ball to roll without slipping?}
       \item{At what time $t$ will this condition be satisfied?}
       \item{How far will the ball move in this time?}
     \end{enumerate}

   }

\newpage

\centerline{\sc{Reference Material - Rotational Motion}}


Moments of Inertia:
\begin{itemize}
  \item{Disk or cylinder, rotating about center: $I = \frac{1}{2}MR^2$}
  \item{Sphere, rotating about center: $I = \frac{2}{5}MR^2$}
  \item{Ring or hollow cylinder, rotating about center: $I = MR^2$}
\end{itemize}

\bigskip
\bigskip
\bigskip

Correspondence between linear dynamics and rotational dynamics: 
  \scriptsize

\begin{tabular}{| c | c | c | c |}
  \hline
  Position & $s$ & Angle & $\theta$  \\
  Velocity & $\vec v$ & Angular velocity & $\omega$  \\
  Acceleration & $\vec a$ & Angular acceleration & $\alpha$  \\
  \hline
                                   & $v(t) = v_0 + at$ & & $\omega(t) = \omega_0 + \alpha t$ \\
                                   & $x(t) = x_0 + v_0 t + \frac{1}{2} at^2$ & & $\theta(t) = \theta_0 + \omega_0 t + \frac{1}{2} \alpha t^2$ \\
                                   & $v_f^2 - v_0^2 = 2a \Delta x$ & & $\omega_f^2 - \omega_0^2 = 2 \alpha \Delta \theta$ \\
  \hline
  Mass & $m$ & Moment of inertia & $I$ \\
  \hline
  Force & $F$ & Torque & $\tau = F_\perp r = F r_\perp$ \\
  \hline
  Newton's second law & $\vec F = m \vec a$ & ``Newton's second law for rotation'' & $\tau = I \alpha$ \\
  \hline
  Kinetic energy & $\frac{1}{2} mv^2$ & Kinetic energy & $\frac{1}{2}I\omega^2$ \\
  \hline
  Momentum & $\vec p = m \vec v$ & Angular momentum & $L = I \omega$ \\
  \hline
\end{tabular}

\bigskip
\bigskip
\bigskip

Arc length $s=\theta r$ \\
Tangential velocity $v=\omega r$ \\
Tangential acceleration $a=\alpha r$




 \end{enumerate}
 \end{document}
