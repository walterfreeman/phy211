
\documentclass[12pt]{article}
\setlength\parindent{0pt}
\usepackage[margin=1cm]{geometry}
\usepackage{fullpage}
\usepackage{amsmath}
\usepackage{graphicx}
\setlength{\parskip}{4mm}
\def\LL{\left\langle}   % left angle bracket
\def\RR{\right\rangle}  % right angle bracket
\def\LP{\left(}         % left parenthesis
\def\RP{\right)}        % right parenthesis
\def\LB{\left\{}        % left curly bracket
\def\RB{\right\}}       % right curly bracket
\def\PAR#1#2{ {{\partial #1}\over{\partial #2}} }
\def\PARTWO#1#2{ {{\partial^2 #1}\over{\partial #2}^2} }
\def\PARTWOMIX#1#2#3{ {{\partial^2 #1}\over{\partial #2 \partial #3}} }
\newcommand{\BE}{\begin{displaymath}}
\newcommand{\EE}{\end{displaymath}}
\newcommand{\BNE}{\begin{equation}}
\newcommand{\ENE}{\end{equation}}
\newcommand{\BEA}{\begin{eqnarray}}
\newcommand{\EEA}{\nonumber\end{eqnarray}}
\newcommand{\EL}{\nonumber\\}
\newcommand{\la}[1]{\label{#1}}
\newcommand{\ie}{{\em i.e.\ }}
\newcommand{\eg}{{\em e.\,g.\ }}
\newcommand{\cf}{cf.\ }
\newcommand{\etc}{etc.\ }
\newcommand{\Tr}{{\rm tr}}
\newcommand{\etal}{{\it et al.}}
\newcommand{\OL}[1]{\overline{#1}\ } % overline
\newcommand{\OLL}[1]{\overline{\overline{#1}}\ } % double overline
\newcommand{\OON}{\frac{1}{N}} % "one over N"
\newcommand{\OOX}[1]{\frac{1}{#1}} % "one over X"



\begin{document}
\Large
\centerline{\sc{Recitation Questions}}
\normalsize
\centerline{\sc{Tech Practice Recitation}}

\bigskip

\it \begin{center} \footnotesize Please hang around on Collaborate after you are done with the problems, so that a coach or instructor can talk to you and get feedback about how things went, and record your name as having attended so we can give you your extra credit. We will use this feedback to make things better for you next week. \bf THANKS! \end{center}
\rm 
\pagenumbering{gobble}
\begin{enumerate}

  \item{A car of mass 1500 kg is driving at a speed of 40 m/s (89 mph; 144 km/hr) when the driver sees a sharp curve ahead in the road, and applies their brakes suddenly so they can safely
make it around the curve. They want to slow down to 15 m/s.

This car has ``anti-lock'' brakes, ensuring that friction between
the tires and the ground is the maximum possible with static friction.} 

\begin{enumerate}
\item Estimate the coefficient of static friction between the tires and the pavement, and determine the frictional force on the car.

\vspace{3in}

\item What distance will the car travel as it decelerates from 40 m/s to 15 m/s?
\vspace{3in}
\item As you will learn next week, energy is not created or destroyed; any force that does negative work on an object simply transforms that energy into another form. 



After the car applies its brakes, what form will that energy be converted to? 
\vspace{2in}

\item Determine the amount of energy converted into this other form by the car's brakes.

\vspace{4in}
\newpage
\end{enumerate} 

 \item{A ball of mass $m$ on a cord of length $L$ is held at an angle $\theta$ to the left of the vertical and released. A very strong wind blows from left to right, exerting a constant horizontal force $F$. }

     \begin{enumerate}
     	\item The key to working with the work-energy theorem (or conservation of momentum) is to draw clear pictures, then clearly identify the ``before'' and ``after'' states for the work-energy theorem.
     	     	In the space below, draw three pictures: 
     	\begin{enumerate}
     		\item The ball before it is released, held to the left at an angle $\theta$ from the vertical
     		\item The ball when it is at the bottom of its swing, traveling at a speed $v$
     		\item The ball at its highest point on the right, at an angle $\phi$ from the vertical
     	\end{enumerate}
     	
     	\newpage
       \item{Find the speed of the ball at the bottom of its swing. Which two pictures are you looking at here?}
       
      \vspace{4in}
       \item{Find an equation for the maximum angle $\phi$ that the ball reaches when it swings to the right. You do not need to actually solve it, since it's messy and involves a lot of trig identities; just write it down. Which two pictures are you looking at here?}
       
       \newpage
       
       \vspace{3in}
 
       \item{When the ball swings back to the left, find the height that it reaches. Will it come
         back to the same point where it was released?}
     
     \vspace{1in}
     \end{enumerate}
 \end{enumerate}
 \end{document}
