\documentclass[12pt]{article}
\setlength\parindent{0pt}
\usepackage{fullpage}
\usepackage{amsmath}
\usepackage{graphicx}
\setlength{\parskip}{4mm}
\def\LL{\left\langle}   % left angle bracket
\def\RR{\right\rangle}  % right angle bracket
\def\LP{\left(}         % left parenthesis
\def\RP{\right)}        % right parenthesis
\def\LB{\left\{}        % left curly bracket
\def\RB{\right\}}       % right curly bracket
\def\PAR#1#2{ {{\partial #1}\over{\partial #2}} }
\def\PARTWO#1#2{ {{\partial^2 #1}\over{\partial #2}^2} }
\def\PARTWOMIX#1#2#3{ {{\partial^2 #1}\over{\partial #2 \partial #3}} }
\newcommand{\BE}{\begin{displaymath}}
\newcommand{\EE}{\end{displaymath}}
\newcommand{\BNE}{\begin{equation}}
\newcommand{\ENE}{\end{equation}}
\newcommand{\BEA}{\begin{eqnarray}}
\newcommand{\EEA}{\nonumber\end{eqnarray}}
\newcommand{\EL}{\nonumber\\}
\newcommand{\la}[1]{\label{#1}}
\newcommand{\ie}{{\em i.e.\ }}
\newcommand{\eg}{{\em e.\,g.\ }}
\newcommand{\cf}{cf.\ }
\newcommand{\etc}{etc.\ }
\newcommand{\Tr}{{\rm tr}}
\newcommand{\etal}{{\it et al.}}
\newcommand{\OL}[1]{\overline{#1}\ } % overline
\newcommand{\OLL}[1]{\overline{\overline{#1}}\ } % double overline
\newcommand{\OON}{\frac{1}{N}} % "one over N"
\newcommand{\OOX}[1]{\frac{1}{#1}} % "one over X"



\begin{document}
\Large
\centerline{\sc{Recitation Questions}}
\normalsize
\centerline{\sc{April 29}}

\begin{enumerate}

\item A ball of mass $m$ is connected to one end of a rubber band and swung in a circle. The rubber band has spring constant $k$, and
its unstretched length is $r_0$.

If it is swung at an angular velocity $\omega$, to what length will it stretch the rubber band?

\vspace{3in}


\item A mad scientist has built a rocket-powered sled, and wants to show off by using it and a ramp made out of
snow to jump through the air, much like a skier. She sets her sled a distance $d$ in front of the ramp
and fires the rocket. The rocket accelerates forward toward the ramp, ascends the ramp, then flies through
the air before landing back on the ground.

Suppose that:

\begin{itemize}
\item The sled and rider together have mass $m$
\item The coefficient of friction between the snow and the sled is $\mu_k$
\item The thrust force from the rocket is $F_T$
\item The overall (diagonal) length of the ramp is $L$
\item The ramp is inclined at an angle $\theta$ above the horizontal
\end{itemize}

\begin{enumerate}

\item Draw a cartoon of the situation, labeling interesting things (i.e. the trigonometry related to the ramp).

\vspace{2in}

\item Using energy methods, calculate how fast she is traveling when she leaves the top of the ramp.

\vspace{2in}

\item Using energy methods, calculate how fast she is traveling when she lands back on the ground. Think 
carefully about what your ``initial'' and ``final'' states are; there's a hard way and an easy way to do this.

\vspace{2in}

\item Can you use energy methods to figure out the horizontal distance she travels before landing back
on the ground? If so, write down an equation you can solve for that distance. If not, explain what other
techniques you need to use.
\end{enumerate}
\newpage




\item On a hot day in southern Arizona (air temperature $98^\circ$ F/$37^\circ$ C), a hiker decides to climb Mt. Wrightson, ascending 1200m from the start of the trail to the summit.
Suppose that she and her equipment have a mass of 100 kg.

\begin{enumerate}

\item How much mechanical work must her muscles do on her in order to climb the mountain?

\vspace{1in}

\item Suppose that human muscle is 15\% efficient -- that is, of the chemical energy provided to it, 15\% is converted into
mechanical work, and the other 85\% is converted to heat. We typically measure the chemical energy of food in kilocalories
(often called just ``calories'', frustrating the scientists everywhere). 1 kilocalorie is 4180 J. How many kilocalories of food
must our hiker consume to fuel her climb?

\vspace{2in}

\item If 15\% of her food energy gets converted into useful mechanical work, the other 85\% is converted into heat. Since the air temperature is
equal to her body temperature, the only way she can cool herself is by sweating.

The evaporation of one kilogram (one liter) of water carries with it 2.3 MJ of heat energy. How many liters of water must the
hiker drink as she climbs the mountain?

\vspace{2in}

\item If you have any experience hiking or climbing mountains, compare these numbers to your experience. Are they roughly correct? (One
Clif Bar or similar contains about 250 kcal.)

\end{enumerate}
\end{enumerate}
\end{document}
