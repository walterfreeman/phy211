\documentclass[12pt]{article}
\setlength\parindent{0pt}
\usepackage{fullpage}
\usepackage{amsmath}
\usepackage{graphicx}
\setlength{\parskip}{4mm}
\def\LL{\left\langle}   % left angle bracket
\def\RR{\right\rangle}  % right angle bracket
\def\LP{\left(}         % left parenthesis
\def\RP{\right)}        % right parenthesis
\def\LB{\left\{}        % left curly bracket
\def\RB{\right\}}       % right curly bracket
\def\PAR#1#2{ {{\partial #1}\over{\partial #2}} }
\def\PARTWO#1#2{ {{\partial^2 #1}\over{\partial #2}^2} }
\def\PARTWOMIX#1#2#3{ {{\partial^2 #1}\over{\partial #2 \partial #3}} }
\newcommand{\BE}{\begin{displaymath}}
\newcommand{\EE}{\end{displaymath}}
\newcommand{\BNE}{\begin{equation}}
\newcommand{\ENE}{\end{equation}}
\newcommand{\BEA}{\begin{eqnarray}}
\newcommand{\EEA}{\nonumber\end{eqnarray}}
\newcommand{\EL}{\nonumber\\}
\newcommand{\la}[1]{\label{#1}}
\newcommand{\ie}{{\em i.e.\ }}
\newcommand{\eg}{{\em e.\,g.\ }}
\newcommand{\cf}{cf.\ }
\newcommand{\etc}{etc.\ }
\newcommand{\Tr}{{\rm tr}}
\newcommand{\etal}{{\it et al.}}
\newcommand{\OL}[1]{\overline{#1}\ } % overline
\newcommand{\OLL}[1]{\overline{\overline{#1}}\ } % double overline
\newcommand{\OON}{\frac{1}{N}} % "one over N"
\newcommand{\OOX}[1]{\frac{1}{#1}} % "one over X"



\begin{document}
\Large
\centerline{\sc{Recitation Questions}}
\normalsize
\centerline{\sc{17 February}}

\bf

Note: \rm Remember that all Newton's-second-law problems can and should be solved by this pattern:

\begin{enumerate}

\item Draw force diagrams of the object or objects whose motion you care about
\begin{enumerate}
\item Decide on a coordinate system for the problem
\item Resolve the forces into x- and y-components
\end{enumerate}
\item Write down $\sum F = ma$ for the object(s), in both x- and y-directions if necessary
\item Ask a question concerning the algebraic quantities (for instance, ``If static friction has its maximum value of $\mu_s F_N$, for what value of $\theta$ does $a=0$?'')
\item Do the algebra, solving a system of equations if need be
\end{enumerate}

\begin{enumerate}

\newpage

\item Wheeled vehicles accelerate themselves forwards or backwards by turning their wheels using an engine or a brake. This force is called {\it traction}, and is 
a special case of static friction (if the wheels do not slip on the pavement. Just like the force of static friction has a maximum value $F_{f,stat,max} = \mu_s F_N$, the maximum force of traction $F_{tr, max} = \mu_s F_N$. This means that, no matter how powerful the engine or the brakes in a car, its acceleration (forward or backward)
is limited by the maximum traction of the wheels.

Consider a vehicle with the following parameters:

\begin{itemize}
\item The total mass of the vehicle is $m$
\item The engine is in the front, making that end heavier. The normal force on the front two wheels is $F_{N,f} = 2/3\, mg$, while the normal force on the rear two wheels is
$F_{N,r} = 1/3\, mg$.
\item The coefficient of static friction between the tires and the pavement is $\mu_s$. On dry pavement the value is 1, while on snow the value is 0.3. 
\end{itemize}

Find:

\begin{enumerate}
\item the maximum forward acceleration of the vehicle, if it is front-wheel drive, driving on dry pavement
\newpage
\item the maximum forward acceleration of the vehicle, if it is front-wheel drive, driving on snow
\vspace{1in}
\item the maximum forward acceleration of the vehicle, if it is rear-wheel drive, driving on snow
\vspace{1in}
\item the maximum backward acceleration of the vehicle, if brakes are attached to all four wheels, and it is on snow
\vspace{1in}
\item Two-wheel-drive pickup trucks are often rear-wheel-drive. Drivers will often load the bed of these trucks with snow or sand when driving on mud or snow. Why 
do they do this?
\end{enumerate}
\newpage
\item A book rests on a table that can be tilted. The coefficent of static friction between the book and the table $\mu_s$ is 0.5, while the coefficient of kinetic friction $\mu_k$ is 0.4.

You observe the following:
\begin{itemize}

\item If the tilt angle (between the table and the horizontal) is less than some critical angle $\phi$, then the book will not fall. If it is pushed, then it will move a small distance and come to rest again.
\item If the tilt angle is greater than $\phi$, but less than another angle $\psi$, then the book will not fall on its own. However, if it is pushed downward just a bit, then it will continue to slide all the way down the table.
\item If the tilt angle is greater than $\psi$, it will slide down on its own without being pushed.

\end{itemize}

\begin{enumerate}

\item Describe in words why you think this happens. Can you reproduce this with things around your classroom? (desks, books, objects placed on books, objects placed on desks)? 
\vspace{2in}

\item Find $\phi$ and $\psi$.
\end{enumerate}
\end{enumerate}
\end{document} 
