\documentclass[10pt]{beamer}
\usefonttheme{professionalfonts} % using non standard fonts for beamer
\usefonttheme{serif} % default family is serif
\usepackage{amsmath}
\usepackage{mathtools}
\usepackage{mwe}
%\documentclass[12pt]{beamerthemeSam.sty}
\usepackage{epsf}
%\usepackage{pstricks}
%\usepackage[orientation=portrait,size=A4]{beamerposter}
\geometry{paperwidth=160mm,paperheight=120mm}
%DT favorite definitions
\def\LL{\left\langle}	% left angle bracket
\def\RR{\right\rangle}	% right angle bracket
\def\LP{\left(}		% left parenthesis
\def\RP{\right)}	% right parenthesis
\def\LB{\left\{}	% left curly bracket
\def\RB{\right\}}	% right curly bracket
\def\PAR#1#2{ {{\partial #1}\over{\partial #2}} }
\def\PARTWO#1#2{ {{\partial^2 #1}\over{\partial #2}^2} }
\def\PARTWOMIX#1#2#3{ {{\partial^2 #1}\over{\partial #2 \partial #3}} }

\def\rightpartial{{\overrightarrow\partial}}
\def\leftpartial{{\overleftarrow\partial}}
\def\diffpartial{\buildrel\leftrightarrow\over\partial}

\def\BS{\bigskip}
\def\BC{\begin{center}}
\def\EC{\end{center}}
\def\BI{\begin{itemize}}
\def\EI{\end{itemize}}
\def\BE{\begin{displaymath}}
\def\EE{\end{displaymath}}
\def\BEA{\begin{eqnarray*}}
\def\EEA{\end{eqnarray*}}
\def\BNEA{\begin{eqnarray}}
\def\ENEA{\end{eqnarray}}
\def\EL{\nonumber\\}


\newcommand{\map}[1]{\frame{\frametitle{\textbf{Course map}}
\centerline{\includegraphics[height=0.86\paperheight]{../../map/#1.png}}}}
\newcommand{\wmap}[1]{\frame{\frametitle{\textbf{Course map}}
\centerline{\includegraphics[width=0.96\paperwidth]{../../map/#1.png}}}}

\newcommand{\etal}{{\it et al.}}
\newcommand{\gbeta}{6/g^2}
\newcommand{\la}[1]{\label{#1}}
\newcommand{\ie}{{\em i.e.\ }}
\newcommand{\eg}{{\em e.\,g.\ }}
\newcommand{\cf}{cf.\ }
\newcommand{\etc}{etc.\ }
\newcommand{\atantwo}{{\rm atan2}}
\newcommand{\Tr}{{\rm Tr}}
\newcommand{\dt}{\Delta t}
\newcommand{\op}{{\cal O}}
\newcommand{\msbar}{{\overline{\rm MS}}}
\def\chpt{\raise0.4ex\hbox{$\chi$}PT}
\def\schpt{S\raise0.4ex\hbox{$\chi$}PT}
\def\MeV{{\rm Me\!V}}
\def\GeV{{\rm Ge\!V}}

%AB: my color definitions
%\definecolor{mygarnet}{rgb}{0.445,0.184,0.215}
%\definecolor{mygold}{rgb}{0.848,0.848,0.098}
%\definecolor{myg2g}{rgb}{0.647,0.316,0.157}
\definecolor{abtitlecolor}{rgb}{0.0,0.255,0.494}
\definecolor{absecondarycolor}{rgb}{0.0,0.416,0.804}
\definecolor{abprimarycolor}{rgb}{1.0,0.686,0.0}
\definecolor{Red}           {cmyk}{0,1,1,0}
\definecolor{Orange}           {rgb}{1,0.4,0}
\definecolor{Grey}           {cmyk}{.7,.7,.7,0}
\definecolor{Lg}           {cmyk}{.4,.4,.4,0}
\definecolor{Blue}          {cmyk}{1,1,0,0}
\definecolor{Green}         {cmyk}{1,0,1,0}
\definecolor{Brown}         {cmyk}{0,0.81,1,0.60}
\definecolor{Black}         {cmyk}{0,0,0,1}

\usetheme{Madrid}
\newcommand{\vcenteredinclude}[1]{\begingroup
  \setbox0=\hbox{\includegraphics[width=3in]{#1}}%
\parbox{\wd0}{\box0}\endgroup}

%AB: redefinition of beamer colors
%\setbeamercolor{palette tertiary}{fg=white,bg=mygarnet}
%\setbeamercolor{palette secondary}{fg=white,bg=myg2g}
%\setbeamercolor{palette primary}{fg=black,bg=mygold}
\setbeamercolor{title}{fg=abtitlecolor}
\setbeamercolor{frametitle}{fg=abtitlecolor}
\setbeamercolor{palette tertiary}{fg=white,bg=abtitlecolor}
\setbeamercolor{palette secondary}{fg=white,bg=absecondarycolor}
\setbeamercolor{palette primary}{fg=black,bg=abprimarycolor}
\setbeamercolor{structure}{fg=abtitlecolor}

\setbeamerfont{section in toc}{series=\bfseries}

%AB: remove navigation icons
\beamertemplatenavigationsymbolsempty
\title{
  \textbf {Review and recap}\\
%\centerline{}
%\centering
%\vspace{-0.0in}
%\includegraphics[width=0.3\textwidth]{propvalues_0093.pdf}
%\vspace{-0.3in}\\
%\label{intrograph}
}

\author[W. Freeman] {Physics 211\\Syracuse University, Physics 211 Spring 2022\\Walter Freeman}

\date{\today}

\begin{document}

\frame{\titlepage}

\frame{\frametitle{\textbf{Announcements}}
  \Large
Extra office hours and reviews ahead of the exam:
\BI
\item Thursday, 9:00-2:00, room to be announced
\item Saturday, 12:00-5:00, room to be announced
\item Monday, 10-3, room to be announced
\EI
\BS
\BS
Homework 9 is due tomorrow -- if you need an extension, ask your TA.

\BS


}

\frame{\frametitle{\textbf{On grades -- a final reminder}}
	
	\Large
	
\BI
\item I've asked you to do challenging things this semester
\BI
\item This helps you learn more
\item ... but I know they're challenging and the grading scale takes that into account
\EI
\BS\BS



\item Most grades will be A's and B's
\item If you've been keeping up (going to recitation, doing homework) it's overwhelmingly likely you'll pass

\EI
}


\frame{\frametitle{\textbf{Final exam format}}
  \Large
\BI
\item Covers the whole semester
\item No messy mathematics -- we want you to think more and calculate less
\item 8 questions plus a short set of multiple choice questions about the basic nature and dimensions of what we've studied
\EI
}

\frame{\frametitle{\textbf{Unit 1: Kinematics}}
  \large
  \BI
\item{First derivative of position is velocity; second derivative is acceleration}
\item{Kinematics lets us connect acceleration, velocity, position, and time}
\BS\pause
\item Slope of position graph is velocity; slope of velocity graph is acceleration
\item Area under acceleration graph is change in velocity; area under velocity graph is change in position
\BS
\item{If $\vec a$ is constant:}

  \begin{align*}
    \vec s(t) =& \vec s_0 + \vec v_0 t + \frac{1}{2} \vec a t^2 \\
    \vec v(t) =& \vec v_0 + \vec at \\
    v_f^2 - v_0^2 =& 2a\Delta x
  \end{align*}

\item{These relations hold separately and independently in $x$ and $y$}
\item{Acceleration is $g$ downwards {\bf if and only if} an object is in freefall}
\EI
}

\frame{\frametitle{\textbf{Problem-solving guide: kinematics}}
	\large
	
	\BI
	\item Draw a clear diagram and choose your coordinate system (where is the origin?)
	\BS
	\item Write down the kinematics relations for $x(t)$, $y(t)$, $v_x(t)$, $v_y(t)$
	\BI
	\item You'll need to think about what $a$, $v_0$, and $x_0$, etc., are
	\item You may need to decompose a vector into x and y components
	\EI
	\BS
	\item Determine what instant in time you care about, and write down a sentence like:
	
\begin{center}
	\color{Red}``What is the value of $x$ at the time that $y=-h$?''
\end{center}
\BS
\item Substitute in what you know and do the algebra your sentence directs you to do
\EI
}

\frame{\frametitle{\textbf{Kinematics sample problem: the ball-and-table problem}}
  \Large

A ball rolls off of a table of height $h$ at speed $v$. How far does it go?

}

\frame{\frametitle{\textbf{Use kinematics when:}}
    \Large
    \BI
  \item{You need to connect some combination of position, velocity, acceleration, and time}
  \EI

}

\frame{\frametitle{\textbf{Unit 2: Force concepts and Newton's second law}}
    \large
    \BI
  \item{Newton's second law relates the net force $\sum \vec F$ to the acceleration $\vec a$ of the center of mass of an object}
  \BS
  \item{If an object both rotates and moves, $\vec F=m\vec a$ gives you $\vec a$ of the center of mass}
  \item{Newton's third law: forces come in pairs}
  \BS\pause
  \item{Some forces you should know about:}
    \BI
  \item{Normal forces: as big as they need to be}
  \item{Friction: $F_{\rm{fric, static, max}} = \mu_s F_N$, $F_{\rm{fric, kinetic}}=\mu_k F_N$}
  \item{Traction: a type of static friction, points in direction chosen by the
vehicle}
  \item{Elastic: $F=-k\Delta x$}
  \item{Gravity (Earth): $F=mg$ downward}
  \item{Gravity (general): $F=\frac{Gm_1m_2}{r^2}$}
  \item{Tension: A rope pulls on both ends}
    \EI
    \BS
    \pause
    
    \item Often tension and normal forces are {\color{Red}unknowns you solve for along the way}
    \EI
  }

  \frame{\frametitle{\textbf{Force diagrams}}
    \Large
    \BI
  \item{Draw all forces acting on the object, as vectors}
  \item{If you're going to care about torque, draw the whole object and draw the forces where they act}
  \item{Gravity acts at the center of mass}
  \item{Draw these diagrams big enough that you can read them clearly and do trig}
    \EI
  }

\frame{\frametitle{\textbf{Uniform circular motion}}
  \Large
  \BI
\item{If an object is traveling in a circle, you know its acceleration is $a_c = \omega^2 r = \frac{v_T^2}{r}$ toward the center}
\item{Often this will ``give you'' the right side of $F=ma$, and let you conclude something about the left}
  \EI
}
  \frame{\frametitle{\textbf{Use Newton's second law when:}}
    \Large
    \BI
  \item{You need to connect the forces on an object to its acceleration}
  \item{If you don't need $\vec a$ directly, and don't care about time, maybe use energy methods instead?}
    \EI
  }


  \frame{\frametitle{\textbf{Sample problem: Car driving around a curve}}
    \Large
How fast can a car drive around a curve with a radius of curvature $R$?
  }

  \frame{\frametitle{\textbf{Unit 3a: The work-energy theorem and conservation of energy}}
    \large
    \BI
  \item{Work-energy theorem comes from the third kinematics relation}
  \item{Two formulations, one with potential energy and one without:}
    \BI
  \item{$KE_i + W_{\rm{all}} = KE_f$}
  \item{$KE_i + PE_i + W_{\rm{other}} = KE_f + PE_f$}
    \EI
  \item{Draw {\it clear} before and after snapshots}
  \item{Figure out work done in going from one to the other}
  \item {\color{Green} Add in rotational kinetic energy if things rotate}
  \item{Work = $\vec F \cdot d\vec s$}
    \EI
  }

 \frame{\frametitle{\textbf{Use energy methods when:}}
      \Large
      \BI
    \item{You don't know and don't care about time}
    \item{You can account for the work done by all forces involved}
    \item{This is {\bf not} true at the instant of a collision -- use momentum instead}
      \EI
    }

    \frame{\frametitle{\textbf{Sample problem: energy}}
      \Large
A ball rolls down a hill of height $h$ and across a table. 
How fast is it moving at the 
edge of the table?    
}

  \frame{\frametitle{\textbf{Unit 3b: Conservation of momentum}}
      \BI
    \item{In the absence of external forces, $\vec p = m\vec v$ is conserved}
    \item{This is a consequence of Newton's third law}
    \item{Collisions and explosions are short enough that external forces are small}
    \item{Momentum is a vector and is conserved separately in $x$ and $y$}
      \EI
    }

  \frame{\frametitle{\textbf{Use conservation of momentum when:}}
        \Large
        \BI
      \item{You have a collision or explosion and need to connect the velocities before to the velocities after}
        \EI
      }



   \frame{\frametitle{\textbf{Rotation}}
          \Large
          Many ideas here, most analogous to translational motion:
\large
\BI
\item{Torque plays the role of force: $\tau = F_\perp r = F r_\perp$}
\item{Moment of inertia plays the role of mass: $I = \lambda mr^2$}
\item{$\vec F = m \vec a \rightarrow \tau = I \alpha$: ``Newton's second law for rotation''}
\item{Rolling motion is translation plus rotation: $v = \pm \omega r$, $a = \pm \alpha r$}
\item{\bf You must think about the signs here}
\item{Rotational kinetic energy: $KE_{\rm{rot}} = \frac{1}{2} I \omega^2$}
\item{Angular momentum: $L = I \omega$}
  \EI
}

\frame{\frametitle{\textbf{Static equilibrium problems}}
  \Large
  
  Often we need to know the conditions for something to be balanced, and neither rotate nor translate:
  
  \BI
\item{Net torque is zero about any pivot}
\item{Net force is zero (you may not need this)}
\item{Torque due to any force applied {\bf at} the pivot is zero}
\EI
}
%
%\frame{\frametitle{\textbf{The process of science}}
%\Large
%
%Properties of scientific thought:
%\normalsize
%\BI
%\item Empiricism: science relies on the natural world itself as the only true authority
%\item Self-skepticism: people making scientific claims should search for and engage with potentially refuting evidence
%\item Universality: the laws of nature apply everywhere and everywhen, and to all things equally
%\item Objectivity: scientific ideas are independent of any particular human perspective
%\EI
%
%\BS
%\Large
%Ways that this can go wrong:
%\normalsize
%\BI
%\item Cherry-picking
%\item Arguments ad hominem (``they're wrong because they're ugly'') or from authority (``they're right because they have a fancy degree'')
%\item Bad statistics / publication bias
%\item Ignoring refuting evidence 
%\item Manufactured controversy
%\item Arguments from sensationalism (``XYZ is true {\it because} it's exciting'')
%\item ... and others
%\EI
%}

\frame{\frametitle{\textbf{Final reminders}}
  \large
  \BI
\item{Huge amounts of extra review available; use it}
\item{Get some rest during finals week and take care of yourselves}
\item{If you're affected by the Calc/Physics exam scheduling nonsense, tell SU!}
\EI
}

\frame{\frametitle{\textbf{The power of mechanics}}
\large
The things we've studied in this class are more powerful than you think: mechanics is not just blocks on ramps!

\BS

Let's apply them to atoms in a box and see how much we can understand.

\BS\BS

\pause

This computer program does the following:

\BS\BS
\color{Red}
Repeat forever:
\begin{itemize}
	\color{Red}
	\item Calculate the net force on each atom
	\item Determine the acceleration on it from $\vec F=m\vec a$
	\item Use kinematics to calculate where it is some small amount of time later
	\item After some amount of time, calculate pressure from the average force on the walls
	\item The temperature is just the average kinetic energy of the particles
\end{itemize}
}


\frame{
	
\Large
Forces involved:
\BI
\item The walls repel atoms that touch them (normal force) \BS \pause
\item Atoms that get very close repel each other \BS  \pause
\item {\color{Red}\bf Atoms that are somewhat close attract!}
\EI
}

\frame{\frametitle{\textbf{The ideal gas law}}
	
	\Large
	
	The ``ideal gas law'' describes the pressure from a diffuse, hot gas on the container it's in:
	\pause
	\begin{align*}
		{\color{Red}\text{(Pressure)}} &= \frac{{\color{Orange}\text{(Amount of gas)}} \times {\color{Blue}\text{(Temperature)}}}{\color{Green}\text{(Volume of box)}}\\
		{\color{Red}P}{\color{Green}V} &= {\color{Orange}N} {\color{Blue}kT}
	\end{align*}
\pause
Chemists have observed it, but we can show why it's true!

\BS\pause

What if the gas isn't diffuse or ``hot''?
}


\frame{\frametitle{\textbf{Cooling things down}}
	
	\Large
	
	What happens if we change the temperature?
	
	
}




\frame{\frametitle{\textbf{The rest of physics}}
\large
The other disciplines of physics are variants on what you've learned already:

\BS

\BI
\item Electromagnetism (PHY 212) introduces a new force -- just another $\vec F$
\item All you'll do in that class is apply the work-energy theorem and so on to this new force
\BI
\item Light is just a particular manifestation of that force
\EI
\BS

\item Statistical mechanics uses statistics to understand 
$\vec F = m\vec a$ acting on a great many particles at once {\color{Red} (We just did this!)}

\BS

\item Relativity mixes up space and time, changing the coordinates on us
\item Quantum mechanics mixes up ``particle'' and ``wave'' and gives a new lens on mechanics
\EI
}

\frame{\frametitle{\textbf{Research at Syracuse Physics}}

We apply the ideas of physics to all kinds of things:

\BI
\item Astrophysics
\BI
\item How do black holes and neutron stars shake the fabric of spacetime when they collide?
\item How do extreme tidal forces rip apart objects that get too close to stars?
\item How can we reconcile quantum mechanics with gravity to understand the Big Bang? (my work)
\EI
\BS

\item Biophysics
\BI
\item What are the dynamics of the microtubules inside our cells?
\item Can we build better sensors to detect minute amounts of things in blood?
\item How does cognition in primitive brains work?
\EI

\BS

\item Soft and active matter 
\BI
\item Can we model people in a mosh pit? (really!)
\item How do grains of sand ``jam up''?
\item Much more stuff I don't quite understand/know about :)
\EI

\BS

\item Particle physics
\BI
\item What are the elementary particles and forces in nature?
\item How do we make the LHC and other detectors work even better?
\item How do we detect ``ghost particles'' (neutrinos)?
\EI
\BS
\item Quantum computing, nuclear physics, advanced imaging...
\EI
}

\frame{\frametitle{\textbf{Approaches}}
	
	\large
	
Each of these fields is supported by a ``three-legged stool'':

\BI
\item Theory: understanding principles and using pen and paper to study them in simple situations (this class)
\item Experiment: designing tests for these principles and building machines to carry them out (221)
\item Computation: using computers to simulate those principles in more complicated situations and study their consequences (my field and class in the fall)
\EI
}
\frame{\frametitle{\textbf{Two invitations}}
\Large
\BC
Like what you've done here? We have multiple options for you to study more physics!
\EC

You could get a {\color{Red}physics minor}. This involves:

\BI
\item Physics 211 (you have this now!)
\item Physics 212 (you will probably take this next semester!)
\item Four more classes at the 300 level of your choice. For instance:
\BI
\item Biophysics: the physics of living things -- how do cells do what they do?
\item Cosmology: the history and future of the Universe!
\item Astrophysics
\item Computational physics (my class in the fall -- all of you are qualified!)
\item Modern physics (quantum mechanics, relativity, atoms)
\item Waves and vibrations: light and sound
\item Advanced laboratory 
\item Physics education (how do we teach physics?)
\item ... and others I'm forgetting!
\EI
\EI
}


\frame{\frametitle{\textbf{Two invitations}}
\Large
\BC
... or maybe you want to be a physics major! (Come to the dark side -- we have both cookies and the cheat-codes to the Universe!)


\EC


\begin{columns}

\column{0.5\textwidth}
\color{Green}
\Large
\BC Bachelor of Arts\EC
\small      

This degree program prepares you for jobs in industry,
and is also a great double major option with engineering,
computer science, education, and all sorts of things:

\BI
\color{Green}
\item Physics 211/212
\item 300-level class on modern physics (quantum mechanics, relativity, atoms -- the good stuff!)
\item 300-level lab class
\item 5 more elective classes (astrophysics, computational physics, biophysics, cosmology... lots of stuff)
\item 30 physics credits total (you have four, plus four if you took AST101)
\EI


\column{0.5\textwidth}
\color{Blue}
\Large
\BC Bachelor of Science\EC
\small     
\color{Blue}

This degree program prepares you for the most technically demanding 
industry jobs, as well as graduate study in physics or related fields.

It is also a good double major option for other STEM fields, in particular
engineering (there are overlaps in the required classes and you can count
ECS classes toward a physics degree)

\BI
\color{Blue}
\item Physics 211/212
\item 300-level class on modern physics (quantum mechanics, relativity, atoms -- the good stuff!)
\item 300-level lab class
\item Rigorous courses in computational physics, electromagnetism, quantum mechanics, thermodynamics, and others
\item 39 physics credits total (you have four now!)
\EI

\end{columns}
}

\frame{\frametitle{\textbf{Two invitations}}
\Large

If you've done reasonably well in this course, and have strong communication skills, Physics 211 wants to offer you a job!

\BS

We're always looking for good people to work for us as coaches in future years. Want to help next year's class, have fun,
earn some money, and {\color{Red}get a job that looks great on your resume?}

\BS\pause

Come talk to us! (I'll be contacting many of you over the summer with a job offer, but if you're interested, get a head start and talk to me!)

}


\frame{\frametitle{\textbf{And, finally...}}
  \Large
... thank you all; you have been an absolutely wonderful class. We're just getting back from the pandemic; we have hit the ground running. 
You've done challenging things, learned a vast amount, and I'm proud of all of you. 

\BS

Think about how far you've come since January!

\pause
\bigskip
\bigskip
\bigskip
\bigskip

\centerline{``Science is the belief in the ignorance of experts.'' (R. Feynman)}

\pause

\bigskip
\bigskip
\bigskip
\bigskip



\centerline{``All science is either physics or stamp collecting.'' (E. Rutherford)}
}
\frame{\frametitle{\textbf{Final words}}
%\pause
%``A poet once said, `The whole universe is in a glass of wine.'...  [I]f we look at a glass of wine closely enough we see the entire universe. 
%There are the things of physics: the twisting liquid which evaporates depending on the wind and weather, the reflection in the glass; and our imagination adds atoms. 
%The glass is a distillation of the earth's rocks, and in its composition we see the secrets of the universe's age, and the evolution of stars. 
%What strange array of chemicals are in the wine? 
%How did they come to be?... 
%If our small minds, for some convenience, divide this glass of wine, this universe, into parts -- physics, biology, geology, astronomy, psychology, 
%and so on -- remember that nature does not know it! So let us put it all back together, not forgetting ultimately what it is for. Let it give us one more final pleasure; drink it and forget it all!''

\bigskip
\bigskip

\pause

``Poets say science takes away from the beauty of the stars -- mere globs of gas atoms. Nothing is ``mere". I too can see the stars on a desert night, and feel them. But do I see less or more? The vastness of the heavens stretches my imagination -- stuck on this carousel my little eye can catch one-million-year-old light. A vast pattern -- of which I am a part... What is the pattern, or the meaning, or the why? It does not do harm to the mystery to know a little about it. For far more marvelous is the truth than any artists of the past imagined!''

\bigskip

--Richard Feynman, from {\em Lectures on Physics}
}

\frame{
\BC
\Huge
Thanks for a wonderful semester!
\EC
}

\end{document}
