
\documentclass[12pt]{article}
\setlength\parindent{0pt}
\usepackage{fullpage}
\usepackage{amsmath}
\usepackage{graphicx}
\setlength{\parskip}{4mm}
\def\LL{\left\langle}   % left angle bracket
\def\RR{\right\rangle}  % right angle bracket
\def\LP{\left(}         % left parenthesis
\def\RP{\right)}        % right parenthesis
\def\LB{\left\{}        % left curly bracket
\def\RB{\right\}}       % right curly bracket
\def\PAR#1#2{ {{\partial #1}\over{\partial #2}} }
\def\PARTWO#1#2{ {{\partial^2 #1}\over{\partial #2}^2} }
\def\PARTWOMIX#1#2#3{ {{\partial^2 #1}\over{\partial #2 \partial #3}} }
\newcommand{\BI}{\begin{itemize}}
\newcommand{\EI}{\end{itemize}}
\newcommand{\BNE}{\begin{equation}}
\newcommand{\ENE}{\end{equation}}
\newcommand{\BEA}{\begin{eqnarray}}
\newcommand{\EEA}{\nonumber\end{eqnarray}}
\newcommand{\EL}{\nonumber\\}
\newcommand{\la}[1]{\label{#1}}
\newcommand{\ie}{{\em i.e.\ }}
\newcommand{\eg}{{\em e.\,g.\ }}
\newcommand{\cf}{cf.\ }
\newcommand{\etc}{etc.\ }
\newcommand{\Tr}{{\rm tr}}
\newcommand{\etal}{{\it et al.}}
\newcommand{\OL}[1]{\overline{#1}\ } % overline
\newcommand{\OLL}[1]{\overline{\overline{#1}}\ } % double overline
\newcommand{\OON}{\frac{1}{N}} % "one over N"
\newcommand{\OOX}[1]{\frac{1}{#1}} % "one over X"



\begin{document}
\pagenumbering{gobble}
\Large
\centerline{\sc{Recitation -- Units and Measurement}}

\normalsize
\centerline{\sc{January 15}}


This material is designed to get you familiar with the way that physicists think about {\it units of measure}
and how to do mathematics with them. There are three principles you need to know:

\BI
\item Some quantities bear {\it dimensions} -- various types of things that you measure. One example, for instance, is {\it time},
which can be measured in many different units (seconds, minutes, hours...)

\item Any numeric value that describes something with dimensions must always be attached in the units that it is measured in.

\item {\bf You may multiply and divide by units just like any other variable when doing algebra.}
For instance, we know that

$$
1 \,{\rm minute} = 60 \,{\rm seconds}.
$$
\EI

\begin{enumerate}

\item What do you get if you divide both sides of this equation by ``1 minute''?



\vspace{1in}

\item How far is it from Syracuse to Albany along I-90? Give your answer in as many different ways as you can think of
(at least three). It's okay to give an informal answer; for instance, how do drivers talk about distances between cities?
If you're having trouble thinking of different ways to answer the question, you might ask:

\BI
\item an American student (if you are international)
\item an international student (if you are American)
\item someone who has driven from Syracuse to Albany who knows how long it takes
\EI

\vspace{3in}


\newpage

\item Drivers measure distances in ``hours'' frequently. For instance, I might say that it is six hours' drive 
from Syracuse to Baltimore. However, this is not a dimensionally-correct way to describe distance, since an hour
is a measure of time, not distance. What other piece of information do you need for this to make sense?

\vspace{2in}

\item What are the dimensions of this piece of information, and what units would you measure it in? Write your answer both in
words and as a fraction. (What mathematical operation does ``per'' suggest?)

\vspace{2in}
\newpage

\item Use a satellite navigation app on a smartphone to get the distance to Albany, plus the the time that it will take.
From this, calculate how fast the app thinks you will be driving on average. Show your work to a TA or coach when you are done.

\vspace{2in}

\item In physics we generally measure this quantity in meters per second instead. 
Convert your result to meters per second from the units you have measured it in. As you do your conversion, write out all of the 
steps; the point is to ensure that you get comfortable doing algebra with units and numbers together. Show your work to a TA
or coach when you are done.

\vspace{3in}

\newpage


\item People who make fast cars often brag about how quickly they can accelerate. For instance, Tesla Motors claims that
their upcoming Roadster can accelerate from 0 to 60 miles per hour in 1.9 seconds. 

You will study acceleration in more detail soon, but for now, just know that

$$
{\rm acceleration} = \frac{\rm change\, in\, velocity}{\rm time}.
$$

What is the average acceleration of this (ludicrously fast) car as it goes from zero to 60 miles per hour? Write out all of the
steps in your calculation, making sure to treat units (like hours, seconds, and miles) like variables when manipulating fractions.
What are the units of your answer? Do they have the right dimensions?

\vspace{2in}

\item In physics, we like to measure distances in meters and time in seconds. Convert your answer to the previous into these
units, and simplify as much as possible. Call a TA or coach over to check your work when you are done.

\vspace{2in}

\item What does it mean that acceleration can be measured in ``meters per second squared''? I've never seen a ``squared second'';
does this make sense?

\vspace{1in}

\item The volume of a cube 10 cm on a side is equal to one liter. How many cubic centimeters (${\rm cm}^3$) are in a liter?
Make sure you doublecheck your answer with your common sense!

\newpage

\end{enumerate}
\end{document}

