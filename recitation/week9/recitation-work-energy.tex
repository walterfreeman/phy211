
\documentclass[12pt]{article}
\setlength\parindent{0pt}
\usepackage{fullpage}
\usepackage[margin=0.5in, paperwidth=13.5in, paperheight=8.4in]{geometry}
\usepackage{amsmath}
\usepackage{graphicx}
\setlength{\parskip}{4mm}
\def\LL{\left\langle}   % left angle bracket
\def\RR{\right\rangle}  % right angle bracket
\def\LP{\left(}         % left parenthesis
\def\RP{\right)}        % right parenthesis
\def\LB{\left\{}        % left curly bracket
\def\RB{\right\}}       % right curly bracket
\def\PAR#1#2{ {{\partial #1}\over{\partial #2}} }
\def\PARTWO#1#2{ {{\partial^2 #1}\over{\partial #2}^2} }
\def\PARTWOMIX#1#2#3{ {{\partial^2 #1}\over{\partial #2 \partial #3}} }
\newcommand{\BE}{\begin{displaymath}}
\newcommand{\EE}{\end{displaymath}}
\newcommand{\BNE}{\begin{equation}}
\newcommand{\ENE}{\end{equation}}
\newcommand{\BEA}{\begin{eqnarray}}
\newcommand{\EEA}{\nonumber\end{eqnarray}}
\newcommand{\EL}{\nonumber\\}
\newcommand{\la}[1]{\label{#1}}
\newcommand{\ie}{{\em i.e.\ }}
\newcommand{\eg}{{\em e.\,g.\ }}
\newcommand{\cf}{cf.\ }
\newcommand{\etc}{etc.\ }
\newcommand{\Tr}{{\rm tr}}
\newcommand{\etal}{{\it et al.}}
\newcommand{\OL}[1]{\overline{#1}\ } % overline
\newcommand{\OLL}[1]{\overline{\overline{#1}}\ } % double overline
\newcommand{\OON}{\frac{1}{N}} % "one over N"
\newcommand{\OOX}[1]{\frac{1}{#1}} % "one over X"



\begin{document}
\pagenumbering{gobble}
\Large
\centerline{\sc{Recitation Questions -- Work and Energy}}

\normalsize
\centerline{\sc{April 13}}

In this recitation, you will discover for yourself one of the most useful laws of mechanics -- the {\it work-energy theorem}.

Our starting point is the ``third kinematics relation''

$$ v_f^2 - v_i^2 = 2 a \Delta x$$

which you worked out in recitation a while back and have found useful in this most recent segment, for instance where you needed to determine the velocity of the puck that Alice slid to Bob once Bob picked it up.

Don't be intimidated by the fact that this is new material: everything that you will do follows logically from things that you have already learned.

In this class, usually you spend most of your practice time applying the laws of physics that you have learned already. This time, you will see how the work-energy theorem is just a {\it mathematical consequence} of what you know already. 

In the process, you will learn some new mathematics, too.


\newpage

{\Large Section 1: Work and Energy in One Dimension}

Consider a constant force $F$ acting on an object of mass $m$ moving only in the $x-$direction as it moves; its change in position during some interval (from ``before'' to ``after'') is $\Delta x$. (We call $\Delta x$ -- the change in the x-position during some 
motion -- the {\it x-displacement}.)

\begin{enumerate}
	\item Our goal here is to rewrite the ``third kinematics equation'' so that the right-hand side references only the {\it forces} that act on the object, rather than its acceleration. You know that $F=ma$. Multiply both sides of that equation by some factor so that the right-hand side becomes only the force multiplied by the x-displacement.  {\it (Hint: If you have a numerical factor you need to get rid of, you can multiply by its inverse.)} What do you get? 
	
	
	\vspace{1in}
	
	\item This equation will become a good friend in the next few weeks, so we give the terms that appear in it special meanings. There are two terms on the left, representing the final and initial {\it kinetic energy} during the motion. (Kinetic energy is a type of energy associated with motion.)
	
	What is the formula for the kinetic energy of something? {\it (Give only the general form -- one term, either initial or final.)}
	
	\vspace{1in}
	
	\item The right-hand side of this equation -- force times displacement -- is also new thing we have not met before, called {\it work}. 
	
	In one dimension, we have two choices for our coordinate system: we can choose either right or left to be positive. Does the value of {\it work} depend on which coordinate system you choose? {\it (If you want, come up with a hypothetical situation: choose a direction for the force and the displacement, and then think about the result in each coordinate system.)}
	
	\vspace{1in}
	\newpage
	\item Under what conditions is work positive? Under what conditions is it negative?
	\vspace{2in}
	
	\item Does the {\it kinetic energy} that you found above depend on your choice of coordinate system? Why or why not?
	
	\vspace{2in}
	
	\item The result you got above is called the {\it work-energy theorem}. It says in words:
	
	$$\text{(change in kinetic energy due to a force)} = \text{(work done by that force)}.$$
	
	\vspace{1in}
	
	Write the mathematical expressions corresponding to these ideas below the words.
\end{enumerate}

\newpage


{\Large Section 2: Work and Energy in Two Dimensions}

 Like the rest of kinematics, the ``third kinematics equation'' applies separately for $x$ and $y$ directions. Since the work-energy theorem is based on the third kinematics equation, let's see how we can modify the above result for motion in more than one dimension.

\begin{enumerate}
\item Begin by writing down the ``third kinematics equation'' in both $x$ and $y$. {\it (You will need double subscripts, such as $v_{f_y}$; that is okay.)}

\vspace{1in}

\item You know that Newton's second law applies separately in $x$ and $y$; that is $\vec F = m \vec a$ implies both $F_x = m a_x$ and $F_y = m a_y$. As you did before, multiply both sides of the equations by $\frac {1}{2}m$, then substitute the force in for the right-hand side.

\vspace{1in}

\item In the past, we emphasized the fact that $x-$ and $y-$directions were independent, and we could not combine them without great care. However, we'll do that now. {\it Add the two equations} above together. (Yes, it is gross, but it won't be gross for long. It will simplify greatly in two steps!)
\vspace{1in}
\newpage

\item Group the terms on the left-hand side so that all the ``initial'' terms are together, and all the ``final'' terms are together. (In doing this, we are trying to figure out what the initial and final kinetic energy in two dimensions looks like.)

\vspace{1.5in}

\item You may see right away how to simplify the left-hand side. If you don't, think about the Pythagorean theorem; if you are still stuck, ask a GTA or coach for help. {\it (You may use the simple symbol $v$, without an arrow, for ``the magnitude of the velocity.'')} Write your simplified form of the work-energy theorem in two dimensions below. 

\vspace{1.5in}

\item In two dimensions, what is an expression for the kinetic energy? Does it depend on which coordinate system you choose, or which direction the velocity vector is pointing?

\newpage

\item Now, let's think about the right-hand side -- the ``work side''. You have found that 
$\text{(work in two dimensions)} = F_x \Delta x + F_y \Delta y.$

A moment ago, you found that the kinetic energy doesn't have a direction or depend on the coordinate system. If that is true for the left-hand side of an equation, it must also be true for the right-hand side! Here we will see why.

Imagine that you have a force $\vec F$ acting 30 degrees south of east on an object that moves a distance $d$ due east. Draw arrows representing the force vector and the displacement vector below.
\vspace{1in}


\item Using the definition of work $W = F_x \Delta x + F_y \Delta y$, calculate the work that the force does on the object during the motion using two different coordinate systems:

\begin{itemize}
\item A coordinate system where the positive $x-$direction is pointing in the direction of the displacement

\item A coordinate system where the positive $x-$direction is pointing in the direction of the force
\end{itemize}

Do you think that the work depends on the coordinate system you choose? The result you got here is true for arbitrary coordinate systems as well; you can prove that using sum and difference trig identities if you want to, but we won't make you do that -- we have better things to do and so do you!

\newpage

\item Note that $\Delta x$ and $\Delta y$ are components of the displacement vector $\Delta \vec s = (\Delta x, \Delta y)$ Our form for the work $W = F_x \Delta x + F_y \Delta y$ has the form $A_x B_x + A_y B_y$. This combination occurs quite a bit in physics, so we give it a special name: the {\bf dot product.} Specifically,

$$A_x B_x + A_y B_y = \vec A \cdot \vec B.$$

Rewrite your version of the work-energy theorem so that the work is written in terms of a dot product between force and displacement. Then label the three terms:

\begin{itemize}
\item work
\item initial kinetic energy
\item final kinetic energy
\end{itemize}

Note that {\it none} of them depend on coordinate system!

\end{enumerate}

\newpage

{\Large Summary}

In the previous sections, you found that

\begin{align*}
\text{(change in kinetic energy during a motion)} &= \text {(work done by all forces during that motion)} \\
\frac{1}{2}mv_f^2 - \frac{1}{2}mv_i^2 \hspace{0.5in} &= \hspace{1.2in}\sum \vec F \cdot \Delta \vec s
\end{align*}

Here $\vec F \cdot \Delta \vec s$ means whichever of the following is easiest to figure out:

\begin{itemize}
\item The size of the force, multiplied by the distance moved in the direction of that force
\item The distance moved, multiplied by the size of the force in the direction of motion
\item The distance moved, multiplied by the size of the force, multiplied by the cosine of the angle between them
\end{itemize}

You may calculate the work done by different forces using different methods if you choose.
\newpage

{\Large Section 3: An Application}

Suppose I gave you the following question: A person is standing in front of a window that is a height $h$ above the ground. They throw a rock out of the window at an angle $\theta$ above the horizontal. It flies through the air and hits the ground. How fast is it traveling when it hits the ground? {\it (``How fast'' refers to the magnitude of the velocity.)}

Using prior methods, you'd have to do a lot of trigonometry to do this. Now, however:

\begin{enumerate}
	\item Determine the work done by gravity on the rock as it flies through the air.
	\vspace{1.5in}
	
	\item Write down the work-energy theorem for its motion: $\text{(change in kinetic energy)}=\text{(work done by gravity)}$
	\vspace{1.5in}
	
	\item Do a bit of algebra to solve for $v_f$.
\end{enumerate}

\vfill

This is much easier than our previous approach! Compare it to the previous approach with your group, and include your TA/coach in the discussion.

\end{document}
