\documentclass[12pt]{article}
\setlength\parindent{0pt}
\usepackage{fullpage}
\usepackage[margin=0.5in]{geometry}
\usepackage{amsmath}
\usepackage{graphicx}
\setlength{\parskip}{1.5em}
\def\LL{\left\langle}   % left angle bracket
\def\RR{\right\rangle}  % right angle bracket
\def\LP{\left(}         % left parenthesis
\def\RP{\right)}        % right parenthesis
\def\LB{\left\{}        % left curly bracket
\def\RB{\right\}}       % right curly bracket
\def\PAR#1#2{ {{\partial #1}\over{\partial #2}} }
\def\PARTWO#1#2{ {{\partial^2 #1}\over{\partial #2}^2} }
\def\PARTWOMIX#1#2#3{ {{\partial^2 #1}\over{\partial #2 \partial #3}} }
\newcommand{\BI}{\begin{itemize}}
\newcommand{\EI}{\end{itemize}}
\newcommand{\BE}{\begin{displaymath}}
\newcommand{\EE}{\end{displaymath}}
\newcommand{\BNE}{\begin{equation}}
\newcommand{\ENE}{\end{equation}}
\newcommand{\BEA}{\begin{eqnarray}}
\newcommand{\EEA}{\nonumber\end{eqnarray}}
\newcommand{\EL}{\nonumber\\}
\newcommand{\la}[1]{\label{#1}}
\newcommand{\ie}{{\em i.e.\ }}
\newcommand{\eg}{{\em e.\,g.\ }}
\newcommand{\cf}{cf.\ }
\newcommand{\etc}{etc.\ }
\newcommand{\Tr}{{\rm tr}}
\newcommand{\etal}{{\it et al.}}
\newcommand{\OL}[1]{\overline{#1}\ } % overline
\newcommand{\OLL}[1]{\overline{\overline{#1}}\ } % double overline
\newcommand{\OON}{\frac{1}{N}} % "one over N"
\newcommand{\OOX}[1]{\frac{1}{#1}} % "one over X"



\begin{document}
\Large
\begin{center}
	Recitation Exercises - Rotational Kinetic Energy \\
	Week 12, Day 1 -- Replacement First Page
\end{center}
\normalsize




Formula One race cars use a ``kinetic energy recovery system'' that stores energy in rapidly spinning wheels called flywheels. When the car slows down to make a turn, the car couples the flywheel to the transmission; it begins to spin rapidly, storing some of the car's translational kinetic energy in its rotation. One type of these systems uses a flywheel of mass $m=5$ kg and radius $r=12$ cm that rotates at up to 60,000 revolutions/minute.

You can model the flywheel as a uniform cylinder ($I=\frac{1}{2}mr^2$).


a) What is its angular velocity in radians per second?

\vspace{2in}

b) How much energy does it store when rotating at full speed?

\vspace{2in}

c) Suppose the wheel is only spinning at half speed (30,000 rpm). What fraction of its maximum energy does it store? \textit{(You should be able to answer this without a calculator. If you are not sure how, ask your instructors for advice.)}

\vspace{0.5in}

d) When the car accelerates again, this system uses the rotational energy stored in the flywheel to supplement the engine power. If the system provides 100 kW of extra power, how many seconds of ``boost power'' can the system deliver before it is out of energy?




\end{document}
