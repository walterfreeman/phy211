\documentclass[12pt]{article}
\setlength\parindent{0pt}
\usepackage{fullpage}
\usepackage{amsmath}
\usepackage{graphicx}
\setlength{\parskip}{4mm}
\def\LL{\left\langle}   % left angle bracket
\def\RR{\right\rangle}  % right angle bracket
\def\LP{\left(}         % left parenthesis
\def\RP{\right)}        % right parenthesis
\def\LB{\left\{}        % left curly bracket
\def\RB{\right\}}       % right curly bracket
\def\PAR#1#2{ {{\partial #1}\over{\partial #2}} }
\def\PARTWO#1#2{ {{\partial^2 #1}\over{\partial #2}^2} }
\def\PARTWOMIX#1#2#3{ {{\partial^2 #1}\over{\partial #2 \partial #3}} }
\newcommand{\BE}{\begin{displaymath}}
\newcommand{\EE}{\end{displaymath}}
\newcommand{\BNE}{\begin{equation}}
\newcommand{\ENE}{\end{equation}}
\newcommand{\BEA}{\begin{eqnarray}}
\newcommand{\EEA}{\nonumber\end{eqnarray}}
\newcommand{\EL}{\nonumber\\}
\newcommand{\la}[1]{\label{#1}}
\newcommand{\ie}{{\em i.e.\ }}
\newcommand{\eg}{{\em e.\,g.\ }}
\newcommand{\cf}{cf.\ }
\newcommand{\etc}{etc.\ }
\newcommand{\Tr}{{\rm tr}}
\newcommand{\etal}{{\it et al.}}
\newcommand{\OL}[1]{\overline{#1}\ } % overline
\newcommand{\OLL}[1]{\overline{\overline{#1}}\ } % double overline
\newcommand{\OON}{\frac{1}{N}} % "one over N"
\newcommand{\OOX}[1]{\frac{1}{#1}} % "one over X"



\begin{document}
\Large
\centerline{\sc{Recitation Questions}}
\normalsize
\centerline{\sc{Week of April 4 -- on energy}}
\small

\medskip

\centerline{\large Question 1 -- work and energy in one dimension}

\medskip

Someone drops a penny of mass 2.5g off of the Empire State Building (height 380 m). It strikes the ground traveling at 50 m/s, having been slowed somewhat by air resistance.


\vspace{1in}

a) With what velocity would it have struck the ground if there were no air resistance? 

\vspace{2in}

b) What was the work done by the drag force?

\vspace{2in}

c) This penny strikes the sidewalk and penetrates the surface, digging a hole 2 cm deep. What was the upward force
exerted on the penny by the pavement? 

\newpage

\centerline{\large Question 2 -- on elasticity}

\medskip

A rock climber of mass 70 kg is climbing a cliff face when she slips and falls. 
There is 4m of slack in her climbing rope, so she undergoes free fall for 4 meters before the 
rope begins to arrest her fall. If the spring constant of her rope is 1400 N/m, then:

\vspace{1in}

a) How far will she fall in total?

\vspace{2in}

b) What is the maximum force that her rope will exert on her as it arrests her fall? Convert this force to either pounds or
kilogram-weights, as appropriate for your background. Is this a large force?

\vspace{2in}

c) Under what conditions would a climber want to use a rope with a lower spring constant? What about a larger spring constant?

\newpage

\centerline{\large Questions 3 and 4 -- on elasticity, again}

3) A ball of mass $m$ is connected to one end of a rubber band and swung in a circle. The rubber band has spring constant $k$, and
its unstretched length is $r_0$.

If it is swung at an angular velocity $\omega$, to what length will it stretch the rubber band? 

\vspace{2in}

4) A child uses a slingshot to shoot a rock through the air. Suppose that the slingshot has a spring constant of 200 N/m, and the 
child is strong enough to exert a force of 100 N as he draws the slingshot back. If the rock has a mass of 50g, how fast will the 
rock be traveling as it leaves the slingshot? (Assume the elastic of the slingshot themselves is very light.)

\vspace{1.5in}

Describe in words how you would figure out how far the rock would travel. What other information would you need to know?

\newpage
\centerline{\large Question 5 -- on energy, and work}

A hiker climbs a steep mountain that is 1200 meters tall. Suppose that she and her equipment have a mass of 100 kg.

a) How much mechanical work must her muscles do on her in order to climb the mountain?

\vspace{1in}

b) Suppose that human muscle is 15\% efficient -- that is, of the chemical energy provided to it, 15\% is converted into
mechanical work, and the other 85\% is converted to heat. We typically measure the chemical energy of food in kilocalories
(often called just ``calories'', frustrating the scientists everywhere). 1 kilocalorie is 4180 J. How many kilocalories of food
must our hiker consume to fuel her climb?

\vspace{2in}

c) If 15\% of her food energy gets converted into useful mechanical work, the other 85\% is converted into heat. Suppose it is
a hot day, with an air temperature equal to her body temperature; thus, the only way she can cool herself is by sweating.

The evaporation of one kilogram (one liter) of water carries with it 2.3 MJ of heat energy. How many liters of water must the
hiker drink as she climbs the mountain?

\newpage
\centerline{\large Question 6 -- on power}

A very fast sports car has an engine that can deliver 300 kW of power to the wheels. (This is equivalent to 400 horsepower.) 
Suppose that this car has a mass of 1000 kg. (For the people who know cars: suppose that it has a transmission flexible enough
that it is never torque-limited, so don't worry about that.)

The coefficient of static friction between the tires and the ground is 1, and you may neglect air resistance for this problem.

a) What determines the maximum acceleration of this car at slow speeds? (Hint: It's not the power output of the engine.) If 
you're stumped here, talk to your classmates who drive cars; what happens if you press the accelerator fully at low speed?

Argue that the maximum acceleration of the car at low speed is $g$, and has nothing to do with the engine.

\vspace{2in}

b) What determines the maximum acceleration of the car at high speed? 

\vspace{2in}

c) How much time is required for this car to accelerate from a stop to a speed of 60 m/s? 

\vspace{2in}






\end{document}
