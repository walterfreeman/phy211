---
layout: page
title: Syllabus
category: top
permalink: syllabus.html
use_math: true
---

##### Quick links: <a id="top"></a>

* [Contact Information](#contact)
* [Textbooks](#books)
* [Course philosophy](#philosophy)
* [Learning objectives](#material)
* [Activities](#activities)
  * [Recitations](#homework)
  * [Readings](#readings)
  * [Lectures](#lectures)
  * [Homework](#homework)
  * [Labs](#labs)
  * [Help sessions](#help)
* [Grading](#grading)
  * [Exams](#exams)
  * [Incompletes](#incompletes)
* [Course policies](#policy)
  * [Academic integrity](#integrity)
  * [Students with disabilities](#disability)
  * [Religious observances and excused absences](#excuses)

---

 <a id="contact"></a>
### Contact Information

-   Professors: 
    - Walter Freeman, <wafreema@syr.edu>, Physics Building 215
    - Matt Rudolph, <msrudolp@syr.edu>, Physics Building 325
-   Class meetings: Tuesdays and Thursdays, 9:30-10:50 AM or 11 AM-12:20 PM, Stolkin Auditorium 
-   Help sessions (office hours): 
    - Matt:  
    - Walter: Friday 9:30-11:30 AM, others TBD; held in the Physics Clinic, Physics Building room 112
-   Course website: <https://walterfreeman.github.io/phy211/>, in addition to Blackboard
-   Recitation TA's:
    * Merrill Asp, head TA, <masp01@syr.edu>
    * Vidyesh Rao Anisetti, <vvaniset@syr.edu>
    * Ahmet Bahar, <asbahar@syr.edu>
    * Soumik Banerjee, <sbaner03@syr.edu>
    * Prashali Chauhan, <prchauha@syr.edu>
    * Kevin Ching, <keching@syr.edu>
    * Mingwei Dai, <mdai07@syr.edu>
    * Seth Kelly, <skell101@syr.edu>
    * Mario Olivares, <maolivar@syr.edu>

---

<a id="books"></a>

### Textbooks and materials

- *University Physics Volume 1* (W. Moebs, S. Ling, J. Sanny, et al.), published by OpenStax. This is an open-access textbook available to anyone under the terms of the CC-BY 4.0 license and can be <a href="https://openstax.org/details/books/university-physics-volume-1">downloaded for free</a>. You are free to print this textbook on your own; OpenStax will sell you a printed copy if you like.

- You are encouraged to have a nonprogrammable calculator for use on exams. This calculator should be able to do trigonometry and arithmetic, but should not be capable of graphing functions or solving equations symbolically. Searching "scientific calculator" on Amazon will get you many options available for less than $10. (You may not use a graphing calculator on exams.)

<a id="philosophy"></a>

---

### Course philosophy 

#### 0. Physics is simple!

**This is the simplest class you will take in your university career.** It is rather accurate to say that the content of this class consists of Newton's law of motion $\vec F=m\vec a$, along with a little math (algebra, simple trigonometry, extremely simple calculus). That's it.

This class won't be easy, of course; the difficult aspect of this course will be learning to use these simple tools -- the elementary principles governing forces and motion -- to understand situations that vary from how to drive safely on ice, how to measure the speed of a bullet, how to throw a basketball so that it goes into the hoop, how a bicycle works, and so on.

Physics is a science of simplicity. It is the most reductionist of the sciences; the aim of physics is to reduce the world around us to its simplest parts, understand how they work, and then put them back together to understand the things they make up. Physics is difficult because understanding how these simple pieces combine to determine the behavior of larger systems requires cleverness, ingenuity, and problem-solving skills. The most difficult aspect of this course is learning to solve problems with simple tools. It's like building things out of Legos: you're supposed to build a rocketship, or a statue of Yoda, and all you have are these little bricks!

This is very different than the life sciences, where the difficulty lies in complexity: nature has built very complicated machines called "lizard" and "tree" and "physicist", and it is up to biology to try to make sense of the complexity behind her creations. Biology is hard because you have to understand all of the different pieces that make up lizards and trees and physicists. But physics isn't like that: in this class we have only objects and forces that act on them, and from that foundation you have to build up the solution to many different situations. That's the power of physics: simple laws in combination drive everything around us.

So, if you're stuck on a problem, think simple; that's how physics works. 

#### 1. Reasoning and synthesis, not memorization

This course is emphatically not a class where you will come to lecture, sit there and listen to a presentation of some facts, and then repeat them back to us on exams. The laws of mechanics are very simple, and you could 
memorize them in an hour if you wanted. The challenging aspect of this class is the *application* of those
principles to understand the motion of physical systems -- to take the principles of nature and, using mathematics
as a tool, synthesize them into an understanding of how a particular system behaves. 
You are not going to be learning a list of currently-accepted facts; you are going to be practicing skills and learning to see the universe as scientists see it.

#### 2. Ask for help, early and often

Since the difficult part of the course is the problem-solving aspect, it's only natural that we are going to give you lots of help in solving problems, especially at first. Learning physics is most similar to learning to play
a sport or learning a musical instrument: it requires practice and the guidance of a coach. We do not expect that you can do all of the homework problems on your own; it is crucial that you ask for help in doing your homework. If you're stuck on a homework problem, you can:

- Come to our office hours and ask, or make an appointment, or drop by our office (room 215)
- Go by the Physics Clinic; you are likely to find a TA, other students, or the professors there to help you.
- Ask a question in the lecture: if you're stuck on something your peers probably are too, and will welcome your question. We *always* have time in lecture to answer questions; don't be intimidated by the size of the class.
- Ask your TA or coaches during recitation
- Ask your peers for help (and insist that they help you understand how to think about the problem, not just give you the answer)
- Write me, your coaches, or your TA an email.

Again: it is **intended that you will get stuck**, just like no pianist plays a difficult piece perfectly the first time. The problem-solving skills in this course are things you have to practice, and we expect you to have to practice in order to make progress; come ask us for help, and we'll guide you as you practice.

#### 3. Learn from your work

As you look at problems -- whether you're solving them the first time or reviewing for an exam -- remember: it's not enough to know the answer. You likely won't see the same problems again.

It's also not enough to know how to get the answer. Knowing how to get the answer -- looking through the solution and understanding how each step follows logically from the last -- is also not enough.

Instead, you should make sure you know how you know what to do to solve the problem. After you complete a problem, take just a few minutes to look back over it and ask yourself: what about this problem led me to the solution? Make sure you're aware of what aspects of the problem make it solvable in a particular way. This will help you build a flexible toolkit of problem-solving skills, tools that will serve you well on the exams and in life.

We've chosen to give you only six to ten homework problems per week. We could give you more, and in fact if you want more practice problems there are more available in your textbook. This is because we intend for you to spend more time thinking about what each problem entails, and learning from them; our experience has shown that students who have to slog through 25 homework problems per week are less likely to actually think about the skills they've applied in each one, since homework becomes a grueling endurance challenge rather than an opportunity to learn physics. We'd much rather have you do the latter. 

#### 4. This is not a math class
In this class, you will use mathematics, but it is only a tool. Do not let yourself become a thrall to mathematics; this class is no more about mathematics than a class on Shakespeare is a class about words.

The laws of physics are written in the language of mathematics, but they describe things beyond math: the physical interactions between objects.

If you are stuck, resist the temptation to go leafing through your textbook looking for "the right equation to use". Physics isn't about equations; it's about ideas and the ability to solve problems. Instead, put your pencil down and think: what is going on here? What principles are at work in this problem? How do I expect the system to move (or not move)? What things do I know, and what other things can I figure out from them? What does my intuition tell me should happen? What forces act on the objects? If you still can't figure out how to proceed after thinking for a while and consulting your notes on problem-solving approaches, it's a good time to ask for help.

The mathematics you will need for this class are:

- Algebra:
	- You will need the ability to solve a system of $N$ equations for $N$ unknowns, using substitution
	- You will need to know how to use the quadratic formula to find the roots of a quadratic equation 
	- There is guaranteed to be one problem on the first exam where you will need to use the quadratic formula
- Trigonometry:
	- You will need to know how to compute the legs of a right triangle given knowledge of its hypotenuse and one of its angles
	- You will need to know how to compute the angles of a right triangle and the length of its hypotenuse given the lengths of the legs
- Calculus: 
	- You need to know the concepts of "derivative" (rate of change) and "integral" (cumulative effect / area under curve). If you are just now in Calculus I, don't worry; it is no accident that Newton developed both mechanics and calculus, and we will teach you what you need to know. You won't have to do any difficult derivatives or integrals.

That's it.

#### 5. This is your class, too

As part of this philosophy of inquiry and questioning, we welcome your input. If there is some aspect of physics that inspires or fascinates you, please ask; if you have feedback for us that will help you enjoy the class more, then please let us know.

---

 <a id="material"></a>

### Learning Objectives

After taking this class, you will be able to:

* Unit 1 (Kinematics):
  * Translate between verbal, graphical, algebraic, and numerical descriptions of an object's motion
  * Given a description of an object's acceleration, create a description of how its position and velocity change, or vice versa
  * Use vectors to describe motion in two and three dimensions, and use trigonometry to manipulate them
  * Incorporate physical units (meters, seconds, and so on) into algebraic and arithmetic statements

* Unit 2 (Forces):  
  * Identify the forces that act on the objects present in a variety of physical situations
  * Describe the basic properties of the forces of tension, friction, gravity, and normal forces
  * Using Newton's second law, construct mathematical relations between those forces and the objects' motion
  * Identify constraints on those forces and on objects' motion based on Newton's third law and the geometry of the situations at hand
  * Describe the forces required to cause an object to move in uniform circular motion
  * Use the previous skills to predict how an object will move in any given situation, and the forces involved in its motion

* Unit 3 (Conservation laws)  
  * Use conservation of momentum to solve problems that involve collisions and explosions
  * Use the work-energy theorem and conservation of energy to determine properties of the motion of systems to which they apply, and recognize which systems those are

* Unit 4 (Rotation)
  * Construct analogies between the properties of rotational motion and the properties of translational motion
  * Use conservation of energy to solve problems in which objects rotate as well as translate
  * Describe the relationship between the forces that act on an object and the torque they apply about any given axis
  * Describe the relationship between the torque applied to an object and its angular acceleration
  * Use both $\vec F=m\vec a$ and $\tau = I \alpha$ in tandem to predict the motion of objects that both translate and rotate

* Throughout (Process of Science)
  * Describe the basics of scientific integrity and the properties of both honest scientific arguments and dishonest pseudoscientific claims
  * Critique (pseudo)scientific claims that are made in bad faith, and describe the problems with them
  * Critically examine sound and unsound scientific claims that seek to overturn preexisting consensus. In particular:
    * Describe the primacy of empirical measurements in evaluating scientific claims
    * Critique flawed arguments that fail to address empirical data 
    * Critique flawed arguments that fail to address the broader framework of physical laws of nature
    * Critique flawed arguments that focus on the identities of the people involved, rather than their data and logic
  
<a id="activities"></a>

---

### Course Activities 

<a id="recitations"></a>
#### Recitations

Twice a week, you will have discussion sections led by one of your TA's, assisted by a few dozen students from previous years of Physics 211. *These discussion sections are the most crucial 
part of this class*, since it's there that you will do the hard and crucial work of practicing the skills you learn. Physics takes practice. It's not something you only learn from a lecture;
it's something you practice with a coach. In recitations, you'll practice your skills in groups of three -- learning from your peers, teaching them, and asking questions of the numerous
guides that are there to help you.

Recitation attendance and participation are graded. Before each in-class exam, you'll take a practice group exam in recitation with your group. These practice exams are graded as well.
A further set of guidelines (incorporated into this syllabus) for recitation, and the homework you'll submit during them, may be found at <a href="https://walterfreeman.github.io/phy211/recitation-guidelines.html">https://walterfreeman.github.io/phy211/recitation-guidelines.html</a>.

<a id="readings"></a>
#### Readings

We will post assigned readings on the calendar before each class. (These may change during the semester based on our progress, but we
will let you know at least a week ahead of time.) There will be short (usually 3 question) pre-lecture quizzes
on basic concepts from the reading, taken on Blackboard. 
We don’t expect everyone to get all these correct; the grades will be curved so that any student who does the reading and makes an honest attempt will get full credit.

This portion of the course has two purposes. First, it will be beneficial for you if you're already familiar with the basics of the concepts we'll be discussing ahead of time. Second, we will be looking at your responses before class to better understand 
if there are any concepts that are particularly confusing for you, so we know what we should spend extra time on.

<a id="lectures"></a>
#### Lectures

In the auditorium, we will alternate between presentation and practice. We will first introduce you to the new ideas we are studying, asking questions and getting your answers using colored cards. (These take the place of clickers.) If you have done the reading ahead of class, these presentations will serve as review and enrichment. Questions during the presentation are encouraged and welcome! We will also demonstrate for you the analytic processes involved in solving problems. 

**At any time during class, feel free to interrupt us and ask questions**. If you do not understand something,
ask. We don't care how many students are in the auditorium -- we almost certainly have time for your question.

<a id="homework"></a>
#### Homework

Homework in this class is designed as a tool to help you develop the problem-solving skills needed to understand physical situations on your own.

You will have an assignment due each week (more or less), which you will hand in to your recitation TA. We do not intend for you to work on these problems by yourself without help. The Physics Clinic is a great place to come to do your homework; you will likely find many of your peers there as well. You are also welcome to come to our office hours and sit and work, asking questions as they arise. Doing the homework thoughtfully and with an eye toward understanding "So how did I know what to do here?", and asking for help is the single best thing you can do in this class.

When writing your homework solutions, you must describe what you are doing in words, even if these descriptions are brief; your solutions should not consist only of equations. Show us what you are thinking and why you are doing what you're doing; this will both help you learn and help us give you more partial credit if you understand what you're doing but mess up the math. **If you do not describe what you are doing and why, you may not get full
credit for a solution, even if it is correct.**

You must submit each problem on a separate side of a page; this is to help us grade your work more easily, and will give the TA's more time to help you learn physics -- which is what we'd all prefer!

Two problems from each set will be graded fully (out of ten points); the rest will be quickly graded for completeness out of two. Your lowest homework set grade will be dropped.

<a id="labs"></a>
#### Labs

You are enrolled in a lab. It is a separate course, with separate grades and TA's. However, Walter Freeman is helping with its design this year, so you can talk to Walter about issues related to lab. 
You may also ask your lab TA or Sam Sampere (smsamper@syr.edu). (You may, of course, ask 
any of us questions about the
*physics* of things you do in lab.)

<a id="help"></a>
#### Help Sessions

These help sessions are opportunities for you to interact with us and the rest of the teaching team in small groups or individually. (Some folks call them "office hours".) If you have questions or suggestions, need help with your homework or with studying, or just want to chat, this is a great opportunity. They will be held in the Physics Clinic, room 112, or elsewhere as announced.

<a id="grading"></a>

---

### Grading and Exams 


| Item                         |            Date           | Points           |
|------------------------------|:-------------------------:|-----------------:|
| Homework                     | Due throughout            |              25  |
| Exam 1                       |    4 February             |              15  |
| Exam 2                       |    3 March                |              15  |
| Exam 3                       |    7 April                |              15  |
| Final Exam                   |    4 May, 3PM-5PM         |              30  |
| Recitation participation |   Throughout the semester | 10               |
| Group practice exams         | The Friday prior to each exam                          | 15               |
| Paper on the nature of science | 27 April                | 15               |
| Pre-lecture quizzes          | Throughout the semester   | 5                |

The lowest of your exam grades, the lowest homework set grade, and your four lowest recitation participation grades will be dropped. If your final exam grade is lower 
than any of your three midterm exam grades, then the final exam will instead only count for 15 points. 
Note however that students with an unusually low recitation participation grade for any given unit will not be eligible to drop the
corresponding exam without prior permission, at our discretion. 

This will result in a total of 130 possible points. This value will then be converted to a percentage (by dividing by 1.3), and grades will be assigned as follows:

- A : >88
- A-: 80-88
- B+: 75-80
- B : 70-75
- B-: 65-70
- C+: 62-64
- C : 58-62
- C-: 55-58
- D : 50-55
- F : less than 50

<a id="exams"></a>
#### Exams

There will be three exams and a final on the dates shown on the course schedule. You may bring a double-sided page of 
handwritten notes, a nonprogrammable calculator, and writing implements. (Calculators that graph functions
or solve equations symbolically are not allowed.) Note that your notes must be *handwritten* by you; photocopies of 
someone else's notes are not allowed. (If you have a disability that impairs your ability to write, then you may 
prepare your notes yourself using means accessible to you.)
 **Cellphones,
smartwatches, and the like may not be used during exams for any reason. Using these devices is presumptive 
evidence of academic dishonesty. If, due to an emergency situation, you require an exception to this, notify
me or a proctor before the exam starts.**

Makeup exams will not be given except in extreme circumstances involving serious disabling illness (not just a cold),
family emergency, or events of singular
importance to your personal life that occur on inflexible dates (e.g. your sibling is getting married). If you
must miss an exam for such a reason, notify one of the professors or the head TA as far in advance as possible. We may ask for
documentation. We may either:

- assign a time for a makeup exam, written or oral, which will likely be on the following weekend
- replace your missed exam grade with your grade on the portion of the final corresponding to the same material
- drop the grade for the missed exam

<a id="incompletes"></a>

#### Incompletes

A grade of "incomplete" may be given to any student who is unable to complete the course material by to the end of the semester due to
unavoidable problems outside his or her control. This is a "grade pending" status that allows you to finish up the course in the future and then
receive a grade. In general, any student who is unable to meaningfully participate in class for a period of two weeks or more due to

* serious illness or injury, physical or mental;
* caregiving for the serious illness of a family member;
* legal involvement or proceedings;
* or international issues

is eligible to take an incomplete in the course. If you think that you may need to take an incomplete, please contact us as soon as possible. 

In general, students may *only* take an incomplete if they have finished a substantial portion of the course (two units) with a grade of C+ or better. 

<a id="policy"></a>

---
<a id="integrity"></a>
### Academic integrity

While you are encouraged to discuss your homework with your peers and collaborate with them on solving our problems, 
all work you submit must reflect your own understanding and be a product of your own work. Submitting any work that you do not understand and cannot explain, or that
is a result of wholesale copying, will be considered academic dishonesty. Please don't let this discourage you from working on your homework with your peers.
That is exactly what you *should* do! But *copying someone's work* is different than *working together with them*. 

Additionally, you are not allowed to post solutions to the homework on the Internet. In particular, posting solutions to Chegg, CourseHero, or any other websites that charge students a fee or otherwise monetize access to that material
is an extremely serious breach of the Academic Integrity Policy and may result in your suspension or expulsion from SU.

We reserve the right to seek a sanction of course failure for any violation of the Academic Integrity Policy.

*(The following is boilerplate from the University)*

Syracuse University’s Academic Integrity Policy reflects the high value that we, as
a university community, place on honesty in academic work. The policy defines
our expectations for academic honesty and holds students accountable for the
integrity of all work they submit. Students should understand that it is their
responsibility to learn about course-specific expectations, as well as about
university-wide academic integrity expectations. The policy governs appropriate
citation and use of sources, the integrity of work submitted in exams and
assignments, and the veracity of signatures on attendance sheets and other
verification of participation in class activities. The policy also prohibits students
from submitting the same work in more than one class without receiving written
authorization in advance from both instructors. Under the policy, students found in
violation are subject to grade sanctions determined by the course instructor and
non-grade sanctions determined by the School or College where the course is
offered as described in the Violation and Sanction Classification Rubric. SU
students are required to read an online summary of the University’s academic
integrity expectations and provide an electronic signature agreeing to abide by
them twice a year during pre-term check-in on MySlice.


<a id="disability"></a>

---

### Students with disabilities

Syracuse University values diversity and inclusion; we are committed to a climate of mutual respect and full participation.  There may be aspects of the instruction or design of this course that result in barriers to your inclusion and full participation in this course.  We invite any student to meet with us to discuss strategies and/or accommodations (academic adjustments) that may be essential to your success and to collaborate with the Office of Disability Services (ODS) in this process.

If you would like to discuss disability-related accommodations with ODS, please visit their website 
at <a href="http://disabilityservices.syr.edu">disabilityservices.syr.edu</a>, visit them in person in Room 309 of 8047 University Avenue, or call (315) 443-4498, TDD: (315) 443-1371 for an appointment to discuss your needs and the process for requesting accommodations. ODS is responsible for coordinating disability-related accommodations. Since accommodations may require early planning and generally are not provided retroactively, please contact ODS as soon as possible.

More generally, if there is anything we can do to help you, whether it is related to a disability, a medical condition, or something else, please let us know. We have an excellent working relationship with ODS and will do anything in our power to make your experience in our class a good one.

<a id="excuses"></a>

---

### Religious observances and excused absences

We anticipate that students will occasionally need to miss class for events occurring on inflexible dates that are of singular 
importance to their education, their families, their health or the health of others, their careers, their
religious faith or life stance, or their participation in the 
democratic process. These absences will be excused (i.e. missing recitation or class for these reasons will not count against your 
participation grade).

These excused absences are generally: 

* Religious observances (declared in advance on MySlice during the first two weeks of class)

* Personal events of equal solemnity to major religious observances occurring on inflexible dates. This includes weddings and commitment ceremonies of close family members, funerals, and the like.

* Caregiving duties for sick family members, other family emergencies, or singularly-important events

* Illness or injury that jeopardizes your health or the health of others if you come to class

* Attendance at academic events (for instance, travel to an academic conference, participation in a seminar, performance in the marching band or a student ensemble, etc.).

* Attendance at a professional development opportunity (a career fair, a job interview, a ROTC event) or work emergencies

* Participation in the political process (canvassing for a campaign, attendance at a scheduled demonstration or a spontaneous one in response to developing events)

If you must miss recitation for such a reason, please notify your recitation instructor in advance if possible. If you must miss an
*exam* for such a reason, please talk to one of the professors to make arrangements.


Absence from class for recreational activities unrelated to academics, professional development, or solemn family life will not be excused; these absences will count against your four dropped recitation participation grades. Unexcused absences include:

* Any Greek Life event (any activity put on by a Greek-letter organization, with the exception of purely academic honor societies that accept all students regardless of gender and do not maintain a house)

* Any athletic event (as an athlete or a spectator), whether NCAA-sanctioned or not. (If you must miss an *exam* for a NCAA-sponsored athletic event, come talk to us; we may be able to make arrangements for you to take it on the road.)

* Weather (unless you are a commuter student who cannot get to campus safely or campus is closed)
