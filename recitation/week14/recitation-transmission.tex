\documentclass[12pt]{article}
\setlength\parindent{0pt}
\usepackage{fullpage}
\usepackage{amsmath}
\usepackage{lscape}
\usepackage[table,xcdraw]{xcolor}
\usepackage{graphicx}
\usepackage[margin=0.5in]{geometry}
\setlength{\parskip}{4mm}
\def\LL{\left\langle}   % left angle bracket
\def\RR{\right\rangle}  % right angle bracket
\def\LP{\left(}         % left parenthesis
\def\RP{\right)}        % right parenthesis
\def\LB{\left\{}        % left curly bracket
\def\RB{\right\}}       % right curly bracket
\def\PAR#1#2{ {{\partial #1}\over{\partial #2}} }
\def\PARTWO#1#2{ {{\partial^2 #1}\over{\partial #2}^2} }
\def\PARTWOMIX#1#2#3{ {{\partial^2 #1}\over{\partial #2 \partial #3}} }
\newcommand{\BI}{\begin{itemize}}
\newcommand{\EI}{\end{itemize}}
\newcommand{\BE}{\begin{displaymath}}
\newcommand{\EE}{\end{displaymath}}
\newcommand{\BNE}{\begin{equation}}
\newcommand{\ENE}{\end{equation}}
\newcommand{\BEA}{\begin{eqnarray}}
\newcommand{\EEA}{\nonumber\end{eqnarray}}
\newcommand{\EL}{\nonumber\\}
\newcommand{\la}[1]{\label{#1}}
\newcommand{\ie}{{\em i.e.\ }}
\newcommand{\eg}{{\em e.\,g.\ }}
\newcommand{\cf}{cf.\ }
\newcommand{\etc}{etc.\ }
\newcommand{\Tr}{{\rm tr}}
\newcommand{\etal}{{\it et al.}}
\newcommand{\OL}[1]{\overline{#1}\ } % overline
\newcommand{\OLL}[1]{\overline{\overline{#1}}\ } % double overline
\newcommand{\OON}{\frac{1}{N}} % "one over N"
\newcommand{\OOX}[1]{\frac{1}{#1}} % "one over X"

\pagenumbering{gobble}

\begin{document}


\Large
\centerline{\sc{Recitation Exercises}}
\normalsize
\centerline{\sc{Week 14, Day 1}}

In this penultimate exercise of the semester, you will study {\it transmissions} -- the assemblages of gears that transmit torque from a motor to the machine that it turns. For instance, the motor might be the engine of a car or the legs of a cyclist; the ``machine'' is the drive wheel applying traction to the ground. This works the same for all sorts of machines that use rotary motion to transmit power.

This exercise is a little different than others: it is designed to let you explore a very useful sort of device we see around us. 
We intend for you to work actively with your coaches and TA's here, so ask us questions!

In all of these cases, the motor applies a torque to the driveshaft, which is connected to a machine that applies an equal and opposite torque. Thus, the motor delivers power to the machine. For this problem, the motor will always be spinning at a constant angular velocity, so $\sum \tau = 0$.

Motors have two limitations: 

\begin{itemize}
	\item They are limited in the torque that they can apply. For instance, for a human riding a bicycle, there is only so much force they can apply to the pedals.
	\item They are limited in their angular velocity. For instance, a person riding a bicycle can only spin their legs so fast.
\end{itemize}

		We will see how we can partially overcome these limitations using gears. Let's think about this in an idealized case of an electric motor spinning a machine, and then apply it to a person riding a bike. (Perhaps the machine is a circular saw used to cut wood, or the motor spinning the propeller
		on a quadcopter drone, or a water pump...)

Suppose our motor's limits are as follows:

		\begin{itemize}
			\item Maximum torque: $\tau_{\rm {max}}=100~\rm N \cdot \rm m$ to the driveshaft
			\item Maximum angular velocity: $\omega_{\rm max} = 50$ rad/sec
		\end{itemize}

Let's imagine a situation where the motor is always applying maximum torque, and see what happens to the machine the motor is driving.

\bigskip
\newpage


\begin{enumerate}

\item What is the maximum power that the motor can deliver to the machine? \textit{(Hint: For translational motion, $P = \vec F \cdot \vec v$. What is the analogous formula for rotation?)} 

Fill in the first row of the table on the back; note that since the motor and machine are connected directly, the torque and angular velocity of the motor are the same as those of the machine.
	
	
\vspace{0.8in}
\item However, in general, machines need to run at different speeds; for instance, cars can drive at many different speeds, and 
	the driver may want high power at any speed.

		Suppose that the operator of the machine wants to run it at low speed -- say, at 25 rad/s. Can the motor still deliver the same power in this case? If not, how much power can it deliver? Complete the second row of the table on the back.
	
\vspace{1.5in}
\newpage



The motor is simply {\it unable} to rotate any faster than 50 rad/sec. For instance, running the machine at $\omega = 100$ rad/s is impossible with the motor alone. Since the machine operator may want to run the machine at any speed, and will likely want the most power from the motor at {\it any} speed, they construct a transmission out of gears.
\medskip

\begin{minipage}{0.4\textwidth}
	
	In this figure, the motor is connected to the red gear with $r_{in}=10$ cm; the machine is connected to the blue gear. We will first think about how this transmission works using
	a single blue gear with a radius of $R_{out}=20$~cm in order to understand the principles at work here; then, we will think about the advantages of {\it shifting} gears,
	as in a bicycle or car transmission.
	
\end{minipage}
\begin{minipage}{0.49\textwidth}
	\begin{center}
		\includegraphics[width=0.8\textwidth]{Gears.png}
	\end{center}
\end{minipage}
\medskip

Here everything is rotating at a constant angular velocity, although the angular velocities of the two gears aren't necessarily the same. This means that the motor applies a clockwise torque to the red gear; the blue gear applies an equal and opposite counterclockwise torque to it.

\item Newton's third law applies to the forces that the two gears exert on each other: the two gears push on each other with equal and opposite forces. Given this, determine the relationship between $\tau_{\rm {motor}}$ (the torque the motor applies to the red gear) and $\tau_{\rm {machine}}$ (the torque the blue gear applies to the machine) in terms of $r_{in}$ and $R_{out}$. Record this formula on the back page. Note that the ratio between $r_{in}$ and $R_{out}$ appears in this formula: this is called the {\it gear ratio}, and is critically important here.

\vspace{1.8in}

\item The velocities of the gears' teeth must also be the same as they turn. Given this, determine the relationship between $\omega_{\rm {motor}}$ (the speed at which the motor and red gear turn) and $\omega_{\rm {machine}}$ (the speed at which the blue gear and the machine turn). Again, you should have a result that depends on the gear ratio. Record this formula on the back page; call your coach or TA over to check your result.
\newpage

\vspace{2in}

\item  Suppose that the motor is running at its maximum angular velocity and torque, and the output (blue) gear has $R=20$ cm. Calculate the angular velocity of the machine and the torque and power supplied to the machine. Enter those in your data table.

%{\large
%\begin{itemize}
%	\item Angular velocity of machine (blue gear) = \underline{\hspace{2in}} rad/s
%	\item Torque applied to machine = \underline{\hspace{2in}} $\rm N \cdot \rm m$
%	\item Power applied to machine = \underline{\hspace{2in}} W
%\end{itemize}
%}

\item How does this power compare to the maximum power that the motor could deliver to the machine at this angular velocity without the transmission?


\vspace{1.5in}


\item Suppose that the engineer designing the machine wants to be able to run the machine at an angular velocity of $\omega = 100$ rad/s. (Perhaps
	it is a quadcopter rotor that needs to spin very quickly.) This would be simply impossible without a transmission, since the motor can only turn at 50 rad/s. What radius should the output gear have so the machine can spin at $\omega = 100$ rad/s, and how much torque could be delivered to the output
	gear in this case? Record these parameters in the next row of your data table.


\vspace{2.5in}


\item Suppose that you now need to use the same motor to generate an extremely large torque to run a different machine. (Perhaps you are trying to lift something extremely heavy: you are not terribly concerned with how {\it quickly} you lift it, you just need to generate a very large force.) If you need an output torque of $\tau_{\rm{machine}} = 1000 \rm N \cdot \rm m$, what radius should the output gear have? Enter this in your data table.


\newpage


\item A bicycle uses a chain to connect the input and output gears, but the principle is the same: the rider's legs are limited in both their torque and their angular velocity. They want to be able to apply maximum power to the output gear (connected to the rear wheel) for a range of angular velocities -- whether this is to climb a steep hill at low speed, or to go as fast as possible on flat ground. 

On many bicycles the rider can change the radius of both input and output gears. Discuss how this allows the rider to deliver maximum power to the bicycle's wheel at any speed. What combination of gears does a rider want when they are climbing a steep hill at low speed? What combination do they want when they are going very fast on flat ground?

\vspace{3in}


\item As you've seen, with appropriate choice of gear sizes, a transmission lets a motor (or a human) produce either extremely large torque or extremely high angular velocity. Is a transmission able to increase the amount of {\it power} that a motor (or cyclist) can produce? Why or why not?

\end{enumerate}

\newpage

\begin{landscape}
% Please add the following required packages to your document preamble:
% \usepackage[table,xcdraw]{xcolor}
% If you use beamer only pass "xcolor=table" option, i.e. \documentclass[xcolor=table]{beamer}
% Please add the following required packages to your document preamble:
% \usepackage[table,xcdraw]{xcolor}
% If you use beamer only pass "xcolor=table" option, i.e. \documentclass[xcolor=table]{beamer}

\begin{table}[]
	\begin{tabular}{l|c|c|cc|c|c|c|}
		\cline{2-8}
		& \textbf{\begin{tabular}[c]{@{}c@{}}Torque from motor\\ (100 $\rm N \cdot \rm m$ max)\end{tabular}} & \textbf{\begin{tabular}[c]{@{}c@{}}$\omega$ of motor\\ (50 rad/s max)\end{tabular}} & \multicolumn{1}{c|}{\textbf{\begin{tabular}[c]{@{}c@{}}Radius of input gear\\ (connected to motor)\end{tabular}}} & \textbf{\begin{tabular}[c]{@{}c@{}}Radius of output gear\\ (connected to machine)\end{tabular}} & \textbf{\begin{tabular}[c]{@{}c@{}}$\tau$ to \\ machine\\ ($\rm N \cdot \rm m$)\end{tabular}} & \textbf{\begin{tabular}[c]{@{}c@{}}$\omega$ of \\ machine\\ (rad/s)\end{tabular}} & \textbf{\begin{tabular}[c]{@{}c@{}}Power \\ to machine\\ (watts)\end{tabular}} \\ \cline{2-8} 
		\begin{tabular}[c]{@{}l@{}}.\\ .\\ .\end{tabular} & 100                                                                                                &                                                                                     & \multicolumn{2}{c|}{\cellcolor[HTML]{C0C0C0}--- No gears, motor connected directly ---}                                                                                                                             &                                                                                               & 50                                                                                &                                                                                \\ \cline{2-8} 
		\begin{tabular}[c]{@{}l@{}}.\\ .\\ .\end{tabular} & 100                                                                                                &                                                                                     & \multicolumn{2}{c|}{\cellcolor[HTML]{C0C0C0}--- No gears, motor connected directly ---}                                                                                                                             &                                                                                               & 25                                                                                &                                                                                \\ \cline{2-8} 
		\begin{tabular}[c]{@{}l@{}}.\\ .\\ .\end{tabular} & 100                                                                                                &                                                                                     & \multicolumn{2}{c|}{\cellcolor[HTML]{C0C0C0}--- No gears, motor connected directly ---}                                                                                                                             &                                                                                               & 100                                                                               &                                                                                \\ \cline{2-8} 
		\begin{tabular}[c]{@{}l@{}}.\\ .\\ .\end{tabular} & 100                                                                                                &                                                                                     & \multicolumn{1}{c|}{10 cm}                                                                                        & 20 cm                                                                                           &                                                                                               &                                                                                   &                                                                                \\ \cline{2-8} 
		\begin{tabular}[c]{@{}l@{}}.\\ .\\ .\end{tabular} & 100                                                                                                &                                                                                     & \multicolumn{1}{c|}{10 cm}                                                                                        &                                                                                                 &                                                                                               & 100                                                                               &                                                                                \\ \cline{2-8} 
		\begin{tabular}[c]{@{}l@{}}.\\ .\\ .\end{tabular} & 100                                                                                                &                                                                                     & \multicolumn{1}{c|}{}                                                                                             &                                                                                                 & 1000                                                                                          &                                                                                   &                                                                                \\ \cline{2-8} 
	\end{tabular}
\end{table}
\null


\hspace{1in}\begin{minipage}{0.5\textheight}
	\begin{center}
		\large
		Relationship between \\ $\tau_{\rm{motor}}$ (red gear) and \\$\tau_{\rm{machine}}$ (blue gear)
		
		
		\vspace{1.5in}
		
		\underline{\hspace{3in}}
		
	\end{center}
\end{minipage}
\begin{minipage}{0.5\textheight}
	\begin{center}
		\large
		Relationship between \\ $\omega_{\rm{motor}}$ (red gear) and \\$\omega_{\rm{machine}}$ (blue gear)
		
		\vspace{1.5in}
		
		\underline{\hspace{3in}}
		
	\end{center}
\end{minipage}
\end{landscape}

\vspace{2in}



%Consider a Ping-Pong ball resting on a smooth table. (Since this is a hollow shell, its moment of inertia is 
%$I=\frac{2}{3}mr^2$.) The coefficient of static friction between the ball and the table is $\mu_s$, 
%the coefficient of kinetic friction is $\mu_k$, and the 
%ball has a mass $m$ and radius $r$.
%
%A gentle breeze begins to blow, exerting a force $F_w$ on the ball that is much less than $mg$. (Since this force is spread out uniformly over the ball,
%you may treat it as acting at the center of the ball.)
%
%\begin{enumerate}
%
%\item Draw an extended force diagram for the ball.
%
%\vspace{4in}
%
%\newpage
%
%\item How large will the frictional force between the ball and the table be? (Note that $\mu_s F_N$ is only the 
%{\it maximum} force of static friction.)
%
%\vspace{3in}
%
%\item Now, suppose that a gust of wind strikes the ball, so that $F_w$ becomes larger. How large can the wind force
%be so that the ball rolls without slipping, instead of skidding?
%
%\vspace{2in}
%
%\item Suppose that the wind force is larger than this, so that the ball {\it skids}. Calculate both its angular
%acceleration and its translational acceleration.
%\end{enumerate}
%
%\newpage
%
%\begin{minipage}{0.5\textwidth}
%A table has a mass $m$, and its height is $2/3$ of its width. The legs of the table are very light; all of the mass is in the top.
%The legs of the table are located at the ends, and a rope is tied to one side; a tension $T$ is applied to the rope. Assume that the coefficient of friction between
%the legs and the ground is very large, so that the table does not slide.
%\end{minipage}
%\begin{minipage}{0.5\textwidth}
%\begin{center}
%\includegraphics[width=0.7\textwidth]{table-crop.pdf}
%\end{center}
%\end{minipage}
%
%In this problem, you will calculate the required tension $T$ to tip the table.
%
%\begin{enumerate}
%
%\item Draw a force diagram for the table. Indicate your choice of pivot.
%
%\newpage
%
%\item What tension force $T$ is required to tip the table (so that the back legs come off the ground)?
%(Hint: What is true about the normal forces on the table when it begins to tip?)
%
%\vspace {4in}
%
%\item What coefficient of static friction between the legs and the floor is
%required so that the table tips, rather than sliding?
%
%\end{enumerate}
 \end{document}
