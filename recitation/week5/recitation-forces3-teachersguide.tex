\documentclass[12pt]{article}
\setlength\parindent{0pt}
\usepackage{fullpage}
\usepackage{amsmath}
\usepackage{graphicx}
\usepackage{color}
\usepackage[margin=2cm]{geometry}
\definecolor{Red}           {rgb}{1,0.3,0.0}
\setlength{\parskip}{4mm}
\def\LL{\left\langle}   % left angle bracket
\def\RR{\right\rangle}  % right angle bracket
\def\LP{\left(}         % left parenthesis
\def\RP{\right)}        % right parenthesis
\def\LB{\left\{}        % left curly bracket
\def\RB{\right\}}       % right curly bracket
\def\PAR#1#2{ {{\partial #1}\over{\partial #2}} }
\def\PARTWO#1#2{ {{\partial^2 #1}\over{\partial #2}^2} }
\def\PARTWOMIX#1#2#3{ {{\partial^2 #1}\over{\partial #2 \partial #3}} }
\newcommand{\BE}{\begin{displaymath}}
	\newcommand{\EE}{\end{displaymath}}
\newcommand{\BNE}{\begin{equation}}
	\newcommand{\ENE}{\end{equation}}
\newcommand{\BEA}{\begin{eqnarray}}
	\newcommand{\EEA}{\nonumber\end{eqnarray}}
\newcommand{\EL}{\nonumber\\}
\newcommand{\la}[1]{\label{#1}}
\newcommand{\ie}{{\em i.e.\ }}
\newcommand{\eg}{{\em e.\,g.\ }}
\newcommand{\cf}{cf.\ }
\newcommand{\etc}{etc.\ }
\newcommand{\Tr}{{\rm tr}}
\newcommand{\etal}{{\it et al.}}
\newcommand{\OL}[1]{\overline{#1}\ } % overline
\newcommand{\OLL}[1]{\overline{\overline{#1}}\ } % double overline
\newcommand{\OON}{\frac{1}{N}} % "one over N"
\newcommand{\OOX}[1]{\frac{1}{#1}} % "one over X"

\begin{document}
		\pagenumbering{gobble}
		\Large
		\centerline{\sc{Recitation Questions}}
		\normalsize
		\centerline{\sc{25 February}}
		
		




		
		A stack of two books, each of mass 1 kg, sits on a table. The coefficients of static friction between the books, and between the bottom book and the table, are 0.4; the coefficients of kinetic friction are 0.2. A person exerts a sudden force on the bottom book. Intuitively, we know what happens:
		\begin{itemize}
			\item{I. If this force is moderate, then the two books accelerate together at the same rate.}
			\item{II. If this force is very large, then the bottom book is yanked out from beneath the top one, and they accelerate at different rates.}
		\end{itemize}
		
		a) In case I, what type of friction exists between the bottom book and the table? What about between the two books?
		
		\vspace{0.7in}
		
		b) In case II, what type of friction exists between the bottom book and the table? What about between the two books?
		
		\vspace{0.7in}
		
		c) Draw force diagrams for the situation when the force is almost large enough to pull the bottom book out from underneath the top one.
		
		\vspace{1.5in}
		
		d) Calculate how much force is required to do this.
		
		\newpage
		
		
		Two very small cats, Fifi and Tali, are sitting on a smooth table when the table begins to tip. Fifi has a mass of $m_f$ kg and Tali has a mass of $m_t$ kg.\footnote{This problem was inspired by the joke: ``Q: Two kittens are sitting on a roof. Which one slides off first? A: The one with the smallest mew.''}
		
		\begin{minipage}{0.6\textwidth}
			The coefficients of friction between the kitties and the table are the following (Tali is slightly fuzzier). Their masses are also
			given (measured by their owner a few days ago); it's up to you to determine if they matter
		\end{minipage}\hspace{0.1\textwidth}
		\begin{minipage}{0.3\textwidth}
			\begin{tabular}{|l|l|l|}
				\hline
				& Fifi & Tali \\ \hline
				$\mu_k$ & 0.4  & 0.3  \\ \hline
				$\mu_s$ & 0.5  & 0.4  \\ \hline
				mass (kg) & 3.4 & 3.6 \\ \hline
			\end{tabular}
		\end{minipage}
		
		As the angle $\theta$ between the table and the horizontal becomes larger and larger, eventually the cats will slide off the 
		table.\footnote{They will land on their feet, since they are graceful cats. Their brother Pierre is a klutz and would land on his head, which is why we're not using him for this problem. But he's cute.}
		
		Remember two things about friction for this problem:
		
		\begin{enumerate}
			\item If two things are {\it already sliding} past one another, the force of kinetic friction between them is equal to $\mu_k F_N$ in whatever direction opposes that motion;
			\item If two things are {\it not sliding}, the force of static friction is {\it however big it needs to be} in order to stop
			them from sliding, up to a {\it maximum} of $\mu_s F_N$.
		\end{enumerate}
		
		a) Draw a cartoon of the problem, and choose a coordinate system. Recall what you learned last recitation about choosing
		coordinate systems that make your life easy.
		\vfill
		
		\vspace{2in}
		\newpage
		
		b) Draw a force diagram for the cat. Make it nice and large, since you'll need to do trigonometry to decompose the 
		weight force into components.
		
		\vspace{3in}
		
		
		c) Decompose the weight force into components. Do this as always: draw a right triangle with the weight force as its 
		hypotenuse, and with its legs aligned with your coordinate system. Then, figure out which angle in the right triangle
		is the same as $\theta$. (Do this on your diagram above.)
		
		d) Write down Newton's second law $\sum F = ma$ in both $x-$ and $y-$directions. 
		
		\vspace{2in}
		
		e) Right before the cat begins to slide off the table, what is true about the frictional force on them? Use this 
		mathematical condition to solve for the angle $\theta$ at which each cat begins to slip off the table.
		
		\vspace{2in}
		
		f) Right after Fifi begins to slide, what will her acceleration be? What will Tali's be?
		
		\vspace{3in}
		
		g) Here is a graph of the frictional force (whether static or kinetic) vs. tilt angle. Interpret as many of its features as you
		can; call your TA and/or coach over to join your conversation.
		
		\begin{center}
			\includegraphics[width=.9\textwidth]{tilt.pdf}
		\end{center}
		
		
		
	\end{document}
