\documentclass[12pt]{article}
\setlength\parindent{0pt}
\usepackage{fullpage}
\usepackage{epsf}
\usepackage{amsmath}
\usepackage{graphicx}
\setlength{\parskip}{4mm}
\def\LL{\left\langle}   % left angle bracket
\def\RR{\right\rangle}  % right angle bracket
\def\LP{\left(}         % left parenthesis
\def\RP{\right)}        % right parenthesis
\def\LB{\left\{}        % left curly bracket
\def\RB{\right\}}       % right curly bracket
\def\PAR#1#2{ {{\partial #1}\over{\partial #2}} }
\def\PARTWO#1#2{ {{\partial^2 #1}\over{\partial #2}^2} }
\def\PARTWOMIX#1#2#3{ {{\partial^2 #1}\over{\partial #2 \partial #3}} }
\newcommand{\BE}{\begin{displaymath}}
\newcommand{\EE}{\end{displaymath}}
\newcommand{\BNE}{\begin{equation}}
\newcommand{\ENE}{\end{equation}}
\newcommand{\BEA}{\begin{eqnarray}}
\newcommand{\EEA}{\nonumber\end{eqnarray}}
\newcommand{\EL}{\nonumber\\}
\newcommand{\la}[1]{\label{#1}}
\newcommand{\ie}{{\em i.e.\ }}
\newcommand{\eg}{{\em e.\,g.\ }}
\newcommand{\cf}{cf.\ }
\newcommand{\etc}{etc.\ }
\newcommand{\Tr}{{\rm tr}}
\newcommand{\etal}{{\it et al.}}
\newcommand{\OL}[1]{\overline{#1}\ } % overline
\newcommand{\OLL}[1]{\overline{\overline{#1}}\ } % double overline
\newcommand{\OON}{\frac{1}{N}} % "one over N"
\newcommand{\OOX}[1]{\frac{1}{#1}} % "one over X"

\pagenumbering{gobble}

\begin{document}


\bigskip
\bigskip
\bigskip
\bigskip

\Large \centerline{\sc{Physics 211 Practice Exam 2}}
\normalsize

{\bf Question 1:}

A train consists of a locomotive with a mass $m_L$ and two cars with mass $m_c$. The rear of the locomotive is coupled to the front of the first car, and the front of the second car is coupled to the rear of the first car.

The engineer of the train wants to accelerate from a stop to a speed of 20 m/s in 25 seconds.

\begin{enumerate}
  \item{Draw force diagrams for the locomotive and the two cars.}
  \item{If the locomotive has a mass of 10 tonnes and each car has a mass of 5 tonnes, compute the size of the two tension forces in the couplings. (A tonne is 1000 kg.)}
  \item{What coefficient of static friction between the locomotive and the rails is required for this to happen?}
\end{enumerate}

\bigskip
\bigskip
\bigskip
\bigskip


{\bf Question 2:}

A flat road travels in a curve with a radius of curvature of 80 m.

\begin{enumerate}
  \item{What is the fastest that a car can drive around this curve if the coefficient of static friction between the tires and the pavement is 0.5?}
  \item{A stone hangs from the roof of the car by a string. If the car travels around the curve at the maximum speed you calculated in part 1, what happens to the stone? (Give both the angle and the direction that the
    string makes with the vertical.)}
\end{enumerate}

\bigskip
\bigskip
\bigskip
\bigskip


{\bf Question 3:}

A chandelier hangs from two ropes. One makes a 30 degree angle with the vertical, while the other makes a 40 degree angle with the vertical. If the chandelier has a mass of 200 kg, what is the tension in each rope?


\bigskip
\bigskip
\bigskip
\bigskip


{\bf Question 4:}

A carnival ride consists of a horizontal platform attached to a vertical wheel of radius 8m. The wheel spins with angular velocity $\omega$.
A person stands on a scale on the platform. At the bottom of the wheel, the scale reads 500N; at the top of the wheel, it reads
450N.

\begin{enumerate}
  \item{Draw a force diagram for the person.}
  \item{How does the scale reading relate to the forces in your diagram?}
  \item{What is the person's weight?}
  \item{What is the angular velocity of the wheel?}
\end{enumerate}

\bigskip
\bigskip
\bigskip
\bigskip


{\bf Question 5:}

A baseball is pitched toward a batter. The batter hits it, knocking the ball back over the pitcher's head.
Just before the batter hits it, it is traveling at 40 m/s parallel to the ground. After the batter hits it, it is traveling at 60 m/s at an angle 30 degrees above the horizontal.

The bat is in contact with the ball for 10 milliseconds; assume that the bat exerts a constant force on the ball during this time. If the ball has a mass of 145 grams, what is the size of this force,
and in what direction does it point?

\bigskip
\bigskip
\bigskip
\bigskip


{\bf Question 6:}

The coefficient of kinetic friction between car tires and a section of pavement is 0.6, while the coefficient of static friction is 0.8. If a car is traveling at 30 m/s, what distance is required for the car to stop if

\begin{enumerate}
  \item{... the wheels stick to the pavement?}
  \item{... the wheels lock and the car slides?}
\end{enumerate}

\bigskip
\bigskip
\bigskip
\bigskip

{\bf Question 7:}

Satellites in geostationary orbit seem to remain at a fixed point in the sky. They do this by revolving around the earth in its equatorial plane with the same angular velocity that the earth itself rotates around its axis.

Note that their altitude is high enough that you can't use $F_{\rm{grav}} = gm$; instead, you need to take into account their distance from the center of the earth using Newton's law of universal gravitation, $F_{\rm{grav}} = \frac{GMm}{r^2}$.

\begin{enumerate}
  \item{Draw a force diagram for such a satellite.}
  \item{What is the angular velocity of the earth's rotation?}
  \item{Compute the altitude of a geostationary orbit.}
\end{enumerate}





{\bf Question 8:}

(This problem is somewhat more involved than those that would appear on the exam.)
The coefficient of kinetic friction between a ramp and a block is $\mu_k$ and the coefficient of static friction is $\mu_s$. A person slides a block up the ramp with initial velocity $v_0$. The ramp is inclined at an angle
$\theta$ above the horizontal.

\begin{enumerate}
  \item{Draw a force diagram for the block as it slides up the ramp.}
  \item{Compute the acceleration for the block as it slides up the ramp.}
  \item{How far up the ramp does it travel before it comes to a stop?}
  \item{If $\mu_s$ is sufficiently large and $\theta$ is sufficiently small, the block will ``stick'' at the top of the ramp and not come back down. What condition must $\theta$ and $\mu_s$ satisfy for this to happen?}
  \item{Suppose that it does not, and slides back down. Draw a force diagram for the block as it slides back down.}
  \item{Compute the acceleration for the block as it slides back down. Is it greater or smaller than it was on the way up? Why?}
\end{enumerate}

\end{document}
