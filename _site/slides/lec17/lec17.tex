\documentclass[10pt]{beamer}
\usefonttheme{professionalfonts}
\usefonttheme{serif}
\usepackage{amsmath}
\usepackage{mathtools}
%\documentclass[12pt]{beamerthemeSam.sty}
\usepackage{epsf}
%\usepackage{pstricks}
%\usepackage[orientation=portrait,size=A4]{beamerposter}
\geometry{paperwidth=160mm,paperheight=120mm}
%DT favorite definitions
\def\LL{\left\langle}	% left angle bracket
\def\RR{\right\rangle}	% right angle bracket
\def\LP{\left(}		% left parenthesis
\def\RP{\right)}	% right parenthesis
\def\LB{\left\{}	% left curly bracket
\def\RB{\right\}}	% right curly bracket
\def\PAR#1#2{ {{\partial #1}\over{\partial #2}} }
\def\PARTWO#1#2{ {{\partial^2 #1}\over{\partial #2}^2} }
\def\PARTWOMIX#1#2#3{ {{\partial^2 #1}\over{\partial #2 \partial #3}} }

\def\rightpartial{{\overrightarrow\partial}}
\def\leftpartial{{\overleftarrow\partial}}
\def\diffpartial{\buildrel\leftrightarrow\over\partial}

\def\BI{\begin{itemize}}
\def\EI{\end{itemize}}
\def\BE{\begin{displaymath}}
\def\EE{\end{displaymath}}
\def\BEA{\begin{eqnarray*}}
\def\EEA{\end{eqnarray*}}
\def\BNEA{\begin{eqnarray}}
\def\ENEA{\end{eqnarray}}
\def\EL{\nonumber\\}

\newcommand{\etal}{{\it et al.}}
\newcommand{\gbeta}{6/g^2}
\newcommand{\la}[1]{\label{#1}}
\newcommand{\ie}{{\em i.e.\ }}
\newcommand{\eg}{{\em e.\,g.\ }}
\newcommand{\cf}{cf.\ }
\newcommand{\etc}{etc.\ }
\newcommand{\atantwo}{{\rm atan2}}
\newcommand{\Tr}{{\rm Tr}}
\newcommand{\dt}{\Delta t}
\newcommand{\op}{{\cal O}}
\newcommand{\msbar}{{\overline{\rm MS}}}
\def\chpt{\raise0.4ex\hbox{$\chi$}PT}
\def\schpt{S\raise0.4ex\hbox{$\chi$}PT}
\def\MeV{{\rm Me\!V}}
\def\GeV{{\rm Ge\!V}}

%AB: my color definitions
%\definecolor{mygarnet}{rgb}{0.445,0.184,0.215}
%\definecolor{mygold}{rgb}{0.848,0.848,0.098}
%\definecolor{myg2g}{rgb}{0.647,0.316,0.157}
\definecolor{abtitlecolor}{rgb}{0.0,0.255,0.494}
\definecolor{absecondarycolor}{rgb}{0.0,0.416,0.804}
\definecolor{abprimarycolor}{rgb}{1.0,0.686,0.0}
\definecolor{Red}           {cmyk}{0,1,1,0}
\definecolor{Grey}           {cmyk}{.7,.7,.7,0}
\definecolor{Blue}          {cmyk}{1,1,0,0}
\definecolor{Green}         {cmyk}{1,0,1,0}
\definecolor{Brown}         {cmyk}{0,0.81,1,0.60}
\definecolor{Black}         {cmyk}{0,0,0,1}

\usetheme{Madrid}


%AB: redefinition of beamer colors
%\setbeamercolor{palette tertiary}{fg=white,bg=mygarnet}
%\setbeamercolor{palette secondary}{fg=white,bg=myg2g}
%\setbeamercolor{palette primary}{fg=black,bg=mygold}
\setbeamercolor{title}{fg=abtitlecolor}
\setbeamercolor{frametitle}{fg=abtitlecolor}
\setbeamercolor{palette tertiary}{fg=white,bg=abtitlecolor}
\setbeamercolor{palette secondary}{fg=white,bg=absecondarycolor}
\setbeamercolor{palette primary}{fg=black,bg=abprimarycolor}
\setbeamercolor{structure}{fg=abtitlecolor}

\setbeamerfont{section in toc}{series=\bfseries}

%AB: remove navigation icons
\beamertemplatenavigationsymbolsempty
\title[Momentum]{
  \textbf {Momentum}\\
%\centerline{}
%\centering
%\vspace{-0.0in}
%\includegraphics[width=0.3\textwidth]{propvalues_0093.pdf}
%\vspace{-0.3in}\\
%\label{intrograph}
}

\author[W. Freeman] {Physics 211\\Syracuse University, Physics 211 Spring 2022\\Walter Freeman}

\date{\today}

\begin{document}

\frame{\titlepage}

\frame{\frametitle{\textbf{Announcements}}
\BI
\large
\item A reminder: The final exam is on {\color{Red}May 11}. You cannot take it early by University rules.
\item Group Exam 3 on Friday; Exam 3 next Tuesday
\item Review notes posted on the website
\item{Extra homework help hours today: 9:45-10:45, 2:30-3:30 PM, 5-7 PM (Physics Clinic -- combination of me and others)}
\item HW7 due Friday
\EI
}


\frame{\frametitle{\textbf{Recitation or homework questions?}}
	
	\pause
	
	A request: ``the one with the penguin'' from last week
}



\frame{\frametitle{\textbf{Sample problems: an excited dog}}
	
	\Large
	
	A person of mass $m$ is sitting in a tire swing with a string of length $L$ when their dog (mass $M$) runs and jumps horizontally into their lap.
	
	\bigskip
	
	If they swing up to an angle $\theta$ above the horizontal, how fast was their dog running?
}

\frame{\frametitle{\textbf{Sample problems: ``springs and momentum''}}
	
	\Large
	This was on 2020's practice exam (on a separate PDF)...
}



\frame{\frametitle{\textbf{What about ``bouncing''?}}
	
	Just knowing that momentum is conserved doesn't fully tell you what happens after a collision.
	
	\bigskip
	
	You also need to know how ``sticky'' or ``bouncy'' the objects are.
	
	\pause\bigskip
	
	One way to describe this: {\it how does the total kinetic energy change?}
	
	\begin{itemize}
		\item Objects stick together: most kinetic energy lost (``inelastic collision'') \pause
		\item Objects bounce a bit: some kinetic energy lost (``partially inelastic'') \pause
		\item Objects bounce ``fully'': no kinetic energy lost (``elastic collision'') \pause
		\item Kinetic energy {\it increases}: only possible with some energy source! \pause
	\end{itemize}
}

\frame{\frametitle{\textbf{The Newton's cradle}}
	\large
	
	Remember:
	
	\begin{itemize}
		\item {\color{Red}{\bf All} collisions conserve momentum}
		\item {\color{Blue}{\bf Elastic} collisions conserve kinetic energy too}
		\item {\color{Green}{\bf These} collisions are mostly elastic}
	\end{itemize}

\pause

\bigskip\bigskip

Let's look at some examples from 2020's practice exam on this. For each one, tell me:


\begin{itemize}
	\item A: Conserves momentum, but loses energy (inelastic collision)
	\item B: Conserves momentum and conserves energy (elastic collision)
	\item C: Conserves momentum, but gains energy (not possible)
	\item D: Does not conserve momentum or energy (not possible)
	\item E: Does not conserve momentum, but does conserve energy (not possible)
\end{itemize}
}
	
\frame{\frametitle{\textbf{An application: neutron moderators}}

The only truly elastic collisions in nature are between particles. If we want {\it totally} elastic collisions, we should look to nuclear physics!

\bigskip
{\color{Blue}
A note: this calculation we are going to do here demonstrates two things:

\BI
\color{Blue}
\item How elastic collisions work
\item ... and how {\it the art of approximation} is used in physics and engineering!
\EI
}

\bigskip

Recall how a nuclear reactor works:

\BI
\item $\null^{235}U$ fissions when struck by neutrons with low energy (600 times more likely at low energy, less than 0.1 eV)
\item When $\null^{235}U$ fissions, it produces neutrons with 2 MeV of kinetic energy ($v \approx$ 20 million m/s)
\EI

\bigskip\bigskip

\begin{center}

\color{Red}

How do we make these neutrons go from 2 million eV to 0.1 eV of kinetic energy so they can produce more fissions?

\end{center}

}

\frame{\frametitle{\textbf{Bounce it off of other atoms!}}

Compared to a neutron moving at $2 \times 10^6$ m/s, other atoms appear to be holding still!

\bigskip

This leads us directly to something we know how to do:


\bigskip\pause

\large

\color{Red}
\begin{center}
A neutron of mass $m$ and kinetic energy $E_0$ strikes another atom of mass $M$. They collide elastically. What is the neutron's energy after the collision?
\end{center}

\pause

$$v_f = -v_0 \frac{M-m}{m+M}$$

\pause

How much kinetic energy is lost?

\pause
\bigskip

The fraction of kinetic energy lost is $$4 \frac{m}{M}.$$

}


\frame{\frametitle{\textbf{So what atoms can we use as moderators?}}
	
	\large
	
	They have to scatter neutrons more readily than they absorb them (hydrogen so-so, oxygen/carbon/heavy hydrogen great)
	
	\bigskip
	
	They have to be lightweight (that's what we just found)
	 
	\bigskip
	
	They have to not be chemically grouchy (no hydrogen or oxygen by themselves!)
	
	\begin{itemize}
		\item $\rm H_2 \rm O$ (light water: most of the world, not the best moderator)
		\item $\rm D_2 \rm O$ (heavy water: Canadians)
		\item $\rm C \rm O_2$ (carbon dioxide: British)
		\item Pure carbon (Graphite: Soviets)
	\end{itemize}


}







\end{document}



