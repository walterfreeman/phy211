
\documentclass[12pt]{article}
\setlength\parindent{0pt}
\usepackage{fullpage}
\usepackage{amsmath}
\usepackage[margin=0.5in, paperwidth=13.5in, paperheight=8.4in]{geometry}
\usepackage{graphicx}
\setlength{\parskip}{4mm}
\def\LL{\left\langle}   % left angle bracket
\def\RR{\right\rangle}  % right angle bracket
\def\LP{\left(}         % left parenthesis
\def\RP{\right)}        % right parenthesis
\def\LB{\left\{}        % left curly bracket
\def\RB{\right\}}       % right curly bracket
\def\PAR#1#2{ {{\partial #1}\over{\partial #2}} }
\def\PARTWO#1#2{ {{\partial^2 #1}\over{\partial #2}^2} }
\def\PARTWOMIX#1#2#3{ {{\partial^2 #1}\over{\partial #2 \partial #3}} }
\newcommand{\BE}{\begin{displaymath}}
\newcommand{\EE}{\end{displaymath}}
\newcommand{\BNE}{\begin{equation}}
\newcommand{\ENE}{\end{equation}}
\newcommand{\BEA}{\begin{eqnarray}}
\newcommand{\EEA}{\nonumber\end{eqnarray}}
\newcommand{\EL}{\nonumber\\}
\newcommand{\la}[1]{\label{#1}}
\newcommand{\ie}{{\em i.e.\ }}
\newcommand{\eg}{{\em e.\,g.\ }}
\newcommand{\cf}{cf.\ }
\newcommand{\etc}{etc.\ }
\newcommand{\Tr}{{\rm tr}}
\newcommand{\etal}{{\it et al.}}
\newcommand{\OL}[1]{\overline{#1}\ } % overline
\newcommand{\OLL}[1]{\overline{\overline{#1}}\ } % double overline
\newcommand{\OON}{\frac{1}{N}} % "one over N"
\newcommand{\OOX}[1]{\frac{1}{#1}} % "one over X"

\pagenumbering{gobble}

\begin{document}
\Large
\centerline{\sc{Recitation Exercises}}
\normalsize
\centerline{\sc{Friday, 14 May}}

We've made it to the end! In this recitation, you will do a final review by looking over the practice exams (from a previous year) posted on the course website.

There are three exams with solutions. Each one has eight problems; in total, there are 24 problems you can use to study and as an exercise. During recitation, you should look at each problem and write down, {\it in words}, a narrative describing the solution. 

Describe:

\begin{itemize}
	\item What techniques you are using
	\item How you know to use those techniques
	\item What subtleties/nuances you need to think about in applying them
\end{itemize}

Again, you have the solutions here; for recitation, I want you to look at them and convert the mathematical solutions into a narrative. 

For instance, for Exam 1 Problem 1:

\it ``This problem involves addition of vectors in two dimensions, where we use trigonometry to convert from magnitude-and-direction form to components. Then we can add and subtract vectors component-by-component. The only tricky thing here is keeping track of which vectors add up to which other ones; drawing a diagram helps keep that straight.''

\rm And for Exam 2 Problem 1 (which you had this year):

\it ``Since this problem requires us to relate knowledge of an object's motion to the forces on it, we use Newton's second law. It is going in a circle so we know the acceleration is $\omega^2 r$ toward the center. The only force that can provide this acceleration is static friction. The only subtlety here is not adding extra forces, like a fictitious extra centripetal force, that do not exist.''

\rm Exam 3 Problem 1:

\it ``Here we have clear before and after states, have forces where we can figure out the work that they do, and don't care about time. Thus we use the work-energy theorem, relating the initial state to the final state when the truck comes to rest. In calculating the work done by each force, we've got to remember that only the displacement in the direction of the force matters for the work. In the final part, we just remember $P=\vec F \cdot \vec v$.''
	\rm
	
The rest are for you to do; this will be a great study guide for the final. Your group likely won't finish all 24 problems during recitation, but that is okay; you can finish later. Good luck!

-Walter





 \end{document}
