\documentclass[12pt]{article}
\setlength\parindent{0pt}
\usepackage{fullpage}
\usepackage{amsmath}
\usepackage{array}
\usepackage{hyperref}
\usepackage{graphicx}
\setlength{\parskip}{4mm}
\def\LL{\left\langle}   % left angle bracket
\def\RR{\right\rangle}  % right angle bracket
\def\LP{\left(}         % left parenthesis
\def\RP{\right)}        % right parenthesis
\def\LB{\left\{}        % left curly bracket
\def\RB{\right\}}       % right curly bracket
\def\PAR#1#2{ {{\partial #1}\over{\partial #2}} }
\def\PARTWO#1#2{ {{\partial^2 #1}\over{\partial #2}^2} }
\def\PARTWOMIX#1#2#3{ {{\partial^2 #1}\over{\partial #2 \partial #3}} }
\newcommand{\vs}[1]{\vspace{#1}}
\newcommand{\BC}{\begin{center}}
\newcommand{\EC}{\end{center}}
\newcommand{\BI}{\begin{itemize}}
\newcommand{\EI}{\end{itemize}}
\newcommand{\BE}{\begin{enumerate}}
\newcommand{\EE}{\end{enumerate}}
\newcommand{\BNE}{\begin{equation}}
\newcommand{\ENE}{\end{equation}}
\newcommand{\BEA}{\begin{eqnarray}}
\newcommand{\EEA}{\nonumber\end{eqnarray}}
\newcommand{\EL}{\nonumber\\}
\newcommand{\la}[1]{\label{#1}}
\newcommand{\ie}{{\em i.e.\ }}
\newcommand{\eg}{{\em e.\,g.\ }}
\newcommand{\cf}{cf.\ }
\newcommand{\etc}{etc.\ }
\newcommand{\Tr}{{\rm tr}}
\newcommand{\etal}{{\it et al.}}
\newcommand{\OL}[1]{\overline{#1}\ } % overline
\newcommand{\OLL}[1]{\overline{\overline{#1}}\ } % double overline
\newcommand{\OON}{\frac{1}{N}} % "one over N"
\newcommand{\OOX}[1]{\frac{1}{#1}} % "one over X"


\pagenumbering{gobble}
\begin{document}

\large
\BC
\sc{Physics Practice}
\sc
\\Quadratic equations, their graphs, and the quadratic formula
\EC
\normalsize

Throughout, the letters A, B, and C refer to the three
coefficients in the quadratic equation $At^2 + Bt + C = 0$, which has solutions

$$
t = \frac {-b \pm \sqrt{B^2 - 4AC}}{2A}
$$

{\large \bf Problem 1:}

Draw five graphs below, as follows:

\begin{enumerate}
\item One in which $A>0$
\item One in which $A<0$
\item One where $t$ has two real, positive intersections with the $x$-axis (``roots'')
\item One where $t$ has one positive root and one negative root
\item One where the only roots to $t$ are imaginary.
\end{enumerate}

\newpage

{\large \bf Problem 2:}

A bucket is being lowered into a well at a constant speed of 2 m/s. When the 
bucket is 10 meters below the top of the well, a physics student drops a rock
into it from the top of the well. However, the bucket is a bit rusty, and might
break if the rock hits it too hard. How fast will the rock be going when it lands
in the bucket?

Take the following steps:
\BI
\item Write position equations $x(t)$ for both objects
\item Graph these equations, so you have a pictoral representation of what is going on
\item Write a sentence that will let you find the solution (of the form of the 
others we've been writing) 
\item Do the algebra and write down the quadratic formula
\item Do anything else you need to do to find the velocity
\EI


\newpage

{\large \bf Problem 3:}

In the previous problem, you used the quadratic formula to find the point of 
intersection between a parabola and a line. In the next problem,
you'll use the quadratic formula to find out where two parabolas intersect.

However, your math teacher told you that the quadratic formula tells you
the roots of a parabola, i.e. where it intersects the line $x=0$. 
After all, it only solves equations of the form $At^2 + Bt + C = 0$.

How do you reconcile these two claims? How do we use the quadratic formula
to find out where parabolas intersect things other than the line $x=0$?

\newpage

{\large \bf Problem 4:}

Suppose I have two objects which move as follows:

\begin{enumerate}
\item Initial velocity $v_0$, initial position 0, acceleration $-g$
\item Initial velocity $0$, initial position $h$, acceleration $\alpha$ (positive)
\end{enumerate}

For instance, Object 1 is a flea jumping in a box which is accelerating
upward ($v_0$ is the velocity with which the flea can jump; it is in freefall
after its feet leave the ground) and Object 2 is the top of the box.

Can the flea jump high enough to get to the top of the box?

Take the following steps:

\BI
\item Write position equations $x(t)$ for both objects
\item Graph these equations, so you have a pictoral representation of what is going on
\item Write a sentence that will let you find the solution: how do you say ``flea reaches top of box'' in terms of our variables?
\item Do the algebra and write down the quadratic formula
\item How do you interpret the solutions to the quadratic formula?
\EI


\newpage

{\large \bf Problem 5:}

In your groups, write down a problem that must be solved with the quadratic 
formula, and whose solution is either:

\begin{enumerate}
\item The ``plus'' root of the quadratic formula
\item The ``minus'' root of the quadratic formula
\item Both roots
\item ... determined by looking at the sign of the determinant (the thing under the square root)
\end{enumerate}

Then, when everyone has written their problems, swap with another group and
solve theirs!



\end{document}
