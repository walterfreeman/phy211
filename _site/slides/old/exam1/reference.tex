\documentclass[12pt]{article}
\setlength\parindent{0pt}
\usepackage{fullpage}
\usepackage{epsf}
\usepackage{amsmath}
\usepackage{graphicx}
\setlength{\parskip}{4mm}
\def\LL{\left\langle}   % left angle bracket
\def\RR{\right\rangle}  % right angle bracket
\def\LP{\left(}         % left parenthesis
\def\RP{\right)}        % right parenthesis
\def\LB{\left\{}        % left curly bracket
\def\RB{\right\}}       % right curly bracket
\def\PAR#1#2{ {{\partial #1}\over{\partial #2}} }
\def\PARTWO#1#2{ {{\partial^2 #1}\over{\partial #2}^2} }
\def\PARTWOMIX#1#2#3{ {{\partial^2 #1}\over{\partial #2 \partial #3}} }
\newcommand{\BE}{\begin{displaymath}}
\newcommand{\EE}{\end{displaymath}}
\newcommand{\BNE}{\begin{equation}}
\newcommand{\ENE}{\end{equation}}
\newcommand{\BEA}{\begin{eqnarray}}
\newcommand{\EEA}{\nonumber\end{eqnarray}}
\newcommand{\EL}{\nonumber\\}
\newcommand{\la}[1]{\label{#1}}
\newcommand{\ie}{{\em i.e.\ }}
\newcommand{\eg}{{\em e.\,g.\ }}
\newcommand{\cf}{cf.\ }
\newcommand{\etc}{etc.\ }
\newcommand{\Tr}{{\rm tr}}
\newcommand{\etal}{{\it et al.}}
\newcommand{\OL}[1]{\overline{#1}\ } % overline
\newcommand{\OLL}[1]{\overline{\overline{#1}}\ } % double overline
\newcommand{\OON}{\frac{1}{N}} % "one over N"
\newcommand{\OOX}[1]{\frac{1}{#1}} % "one over X"



\begin{document}
\nonumber

\bigskip
\bigskip
\bigskip
\bigskip

\Large \centerline{\sc{Physics 211 Exam 1 Reference Sheet}}

\normalsize

The quadratic formula: if $Ax^2 + Bx + C = 0$, then

\begin{equation*}
x = \frac{-B \pm \sqrt{B^2-4AC}}{2A}
\end{equation*}

Constant-acceleration distance and velocity relations (one dimension):
\begin{align*}
  s(t) =& \frac{1}{2} at^2 + v_0 t + s_0      \\
  v(t) =& at + v_0    
\end{align*}

where the constant acceleration is $a$, the initial velocity is $v_0$, the initial position is $s_0$, and the time elapsed is $t$.


Their rotational equivalents are:
\begin{align*}
  \theta(t) =& \frac{1}{2} \alpha t^2 + \omega_0 t + \theta_0      \\
  \omega(t) =& \alpha t + \omega_0 \\   
\end{align*}

where the angle is $\theta$, the angular velocity is $\omega$, and the angular acceleration is $\alpha$.

\bigskip

The ``third kinematics equation'' and its angular equivalent:

\begin{align*}
v_f^2 - v_0^2 =& 2a(\Delta x)\\
\theta_f^2 - \theta_0^2 =& 2\alpha(\Delta \theta)
\end{align*}

\bigskip

Angle can be measured three ways:

1 complete rotation = $2\pi$ radians = 360 degrees.


\end{document}
