\documentclass[12pt]{article}
\setlength\parindent{0pt}
%\usepackage{fullpage}
\usepackage{amsmath}
\usepackage{hyperref}
\usepackage[margin=2cm]{geometry}
\usepackage{graphicx}
\setlength{\parskip}{4mm}
\def\LL{\left\langle}   % left angle bracket
\def\RR{\right\rangle}  % right angle bracket
\def\LP{\left(}         % left parenthesis
\def\RP{\right)}        % right parenthesis
\def\LB{\left\{}        % left curly bracket
\def\RB{\right\}}       % right curly bracket
\def\PAR#1#2{ {{\partial #1}\over{\partial #2}} }
\def\PARTWO#1#2{ {{\partial^2 #1}\over{\partial #2}^2} }
\def\PARTWOMIX#1#2#3{ {{\partial^2 #1}\over{\partial #2 \partial #3}} }
\newcommand{\BE}{\begin{displaymath}}
\newcommand{\EE}{\end{displaymath}}
\newcommand{\BNE}{\begin{equation}}
\newcommand{\ENE}{\end{equation}}
\newcommand{\BEA}{\begin{eqnarray}}
\newcommand{\EEA}{\nonumber\end{eqnarray}}
\newcommand{\EL}{\nonumber\\}
\newcommand{\la}[1]{\label{#1}}
\newcommand{\ie}{{\em i.e.\ }}
\newcommand{\eg}{{\em e.\,g.\ }}
\newcommand{\cf}{cf.\ }
\newcommand{\etc}{etc.\ }
\newcommand{\Tr}{{\rm tr}}
\newcommand{\etal}{{\it et al.}}
\newcommand{\OL}[1]{\overline{#1}\ } % overline
\newcommand{\OLL}[1]{\overline{\overline{#1}}\ } % double overline
\newcommand{\OON}{\frac{1}{N}} % "one over N"
\newcommand{\OOX}[1]{\frac{1}{#1}} % "one over X"



\begin{document}
\begin{center}
\Large
\sc Second Chance Exercise - 1D Motion / Exam 2 Truck Problem \rm

\pagenumbering{gobble}




\normalsize
This second chance exercise is for people who want to review the relation of force to motion, vector components of force, and analyzing connected objects.

If your lowest grade on Exam 2 was on the ``truck problem'', you may complete it and turn it in to earn an extra homework grade, which we will help you earn 100\% on. This will help your end-of-term average.

If you do this, it is due on the last day of recitation.

\end{center}

\vspace{1.5in}

This exam problem tested your ability to relate forces in two dimensions to an object's motion. In it, you needed to use Newton's law in both X and Y directions for two different objects and solve the resulting system of equations.

\begin{enumerate}
	
		\item There is a general series of steps you can take to analyze any situation in which you need to relate forces to motion using Newton's law. What are those steps?
	
		\item Identify two problems from Homework 4 with two connected objects where you had to construct a system of equations. How are those problems the same as the ``truck problem'', and how are they different?
		
	\item How did you handle the fact that you don't know the tension force between the two objects in these problems?

	\item Identify two problems in Practice Exam 2 (at \url{https://walterfreeman.github.io/phy211/practice-exam-2-2023.pdf}) where you need to analyze the relationship between force and motion for two connected objects. How are those problems the same as the ``truck problem'', and how are they different?
	

	
    \item Write a similar problem where you have to analyze the motion of two connected objects with forces in two dimensions. Then solve it. {\it (Particularly good problems here will earn extra credit!)}
\end{enumerate}



\end{document}
