\documentclass[12pt]{article}
\setlength\parindent{0pt}
\usepackage{fullpage}
\usepackage{enumitem}
\usepackage{amsmath}
%\usepackage[margin=1.5cm]{geometry}
\usepackage[margin=0.4in, paperwidth=13.5in, paperheight=8.4375in]{geometry}
\usepackage{graphicx}
\setlength{\parskip}{4mm}
\def\LL{\left\langle}   % left angle bracket
\def\RR{\right\rangle}  % right angle bracket
\def\LP{\left(}         % left parenthesis
\def\RP{\right)}        % right parenthesis
\def\LB{\left\{}        % left curly bracket
\def\RB{\right\}}       % right curly bracket
\def\PAR#1#2{ {{\partial #1}\over{\partial #2}} }
\def\PARTWO#1#2{ {{\partial^2 #1}\over{\partial #2}^2} }
\def\PARTWOMIX#1#2#3{ {{\partial^2 #1}\over{\partial #2 \partial #3}} }
\newcommand{\BE}{\begin{displaymath}}
\newcommand{\EE}{\end{displaymath}}
\newcommand{\BNE}{\begin{equation}}
\newcommand{\ENE}{\end{equation}}
\newcommand{\BEA}{\begin{eqnarray}}
\newcommand{\EEA}{\nonumber\end{eqnarray}}
\newcommand{\EL}{\nonumber\\}
\newcommand{\la}[1]{\label{#1}}
\newcommand{\ie}{{\em i.e.\ }}
\newcommand{\eg}{{\em e.\,g.\ }}
\newcommand{\cf}{cf.\ }
\newcommand{\etc}{etc.\ }
\newcommand{\Tr}{{\rm tr}}
\newcommand{\etal}{{\it et al.}}
\newcommand{\OL}[1]{\overline{#1}\ } % overline
\newcommand{\OLL}[1]{\overline{\overline{#1}}\ } % double overline
\newcommand{\OON}{\frac{1}{N}} % "one over N"
\newcommand{\OOX}[1]{\frac{1}{#1}} % "one over X"



\begin{document}
\pagenumbering{gobble}
\Large
\centerline{\sc{Recitation Exercises}}
\normalsize
\centerline{\sc{19 March}}

As you will learn next week, the gravity of planets decreases when you move away from them. This means that astronauts traveling between the Earth and the Moon, while far from both objects, experience very little gravity.

Thus, things appear to float around them, and they float in the middle of their spacecraft.

Recall the homework problem involving your apparent weight in an elevator. In this problem, you concluded that it wasn't the {\it strength of the force of gravity} that affected your apparent weight, but the strength of some {\it other} force on you. What force was that?



When Hollywood filmmakers wanted to create a movie telling the story of the Apollo 13 mission (where an explosion crippled the spacecraft and the astronauts, led by Jim Lovell, barely survived), they needed a way to mimic this effect on screen. They couldn't send Tom Hanks (the actor playing Lovell) to the Moon, and there is no way to ``turn off'' gravity without traveling far from Earth. (Currently, humanity has no way to travel this far from Earth; we lost that capability in the 1970's.)

They did this by hiring an aircraft that was capable of flying in a path with an acceleration of 9.8 $\rm m/\rm s^2$ downward, and put the set, the actors, their props, the camera, and the film  crew in this aircraft. When it flew in this way, the actors appeared to ``float'' in the air, and the props appeared to float around them.

https://news.avclub.com/on-apollo-13-s-20th-anniversary-a-look-at-how-they-mad-1798281369

FIXME: https://youtu.be/SdgELXIeY3I -- add correct time


a) Draw a force diagram for Tom Hanks  while he is ``floating''. What is his acceleration?

b) What is the acceleration of an object floating in front of him?

c) What is the acceleration of the movie camera filming both him and the objects around him?

d) Explain how this process simulated the effect of deep-space ``weightlessness'' that the real Jim Lovell 


\end{document}
