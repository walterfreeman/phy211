\documentclass[12pt]{article}
\setlength\parindent{0pt}
\usepackage{fullpage}
\usepackage[margin=0.5in, paperwidth=13.5in, paperheight=8.4375in]{geometry}
\usepackage{amsmath}
\usepackage{graphicx}
\newcommand{\BI}{\begin{itemize}}
\newcommand{\EI}{\end{itemize}}
\newcommand{\BE}{\begin{enumerate}}
\newcommand{\EE}{\end{enumerate}}
\setlength{\parskip}{4mm}

\def\PAR#1#2{ {{\partial #1}\over{\partial #2}} }

\pagenumbering{gobble}

\begin{document}
	\begin{center}\sc \Large Recitation -- February 10 \end{center}
	
	In recitation, you will work in groups of up to three to practice the skills you are learning in our course. 
	
	
	You will be assisted by three types of folks:
	
	\BE
	\item {\bf TA's} are PhD candidates in physics who teach classes as part of their training. They are here to help you learn physics and assist with grading. There may be several TA's in your recitation section, but one in particular will be assigned to primarily grade your work. If you have questions about your grade, that's the person to ask.
	
	\item {\bf Coaches} are students like yourself who took this class in the past and now have jobs helping us teach it. They are here to help you learn physics, and only have a very minor role in determining grades.
	
	\item {\bf Your fellow students} in this course will be an invaluable resource for you. In recitation you will be working very closely with your group of three, but you can also ask questions of anyone else in the recitation section or the broader class. 
	\EE
	
	
	
	
	
\end{document}