\documentclass[12pt]{article}
\setlength\parindent{0pt}
\usepackage{fullpage}
\usepackage{amsmath}
\usepackage{graphicx}
\setlength{\parskip}{4mm}
\def\LL{\left\langle}   % left angle bracket
\def\RR{\right\rangle}  % right angle bracket
\def\LP{\left(}         % left parenthesis
\def\RP{\right)}        % right parenthesis
\def\LB{\left\{}        % left curly bracket
\def\RB{\right\}}       % right curly bracket
\def\PAR#1#2{ {{\partial #1}\over{\partial #2}} }
\def\PARTWO#1#2{ {{\partial^2 #1}\over{\partial #2}^2} }
\def\PARTWOMIX#1#2#3{ {{\partial^2 #1}\over{\partial #2 \partial #3}} }
\newcommand{\BE}{\begin{displaymath}}
\newcommand{\EE}{\end{displaymath}}
\newcommand{\BNE}{\begin{equation}}
\newcommand{\ENE}{\end{equation}}
\newcommand{\BEA}{\begin{eqnarray}}
\newcommand{\EEA}{\nonumber\end{eqnarray}}
\newcommand{\EL}{\nonumber\\}
\newcommand{\la}[1]{\label{#1}}
\newcommand{\ie}{{\em i.e.\ }}
\newcommand{\eg}{{\em e.\,g.\ }}
\newcommand{\cf}{cf.\ }
\newcommand{\etc}{etc.\ }
\newcommand{\Tr}{{\rm tr}}
\newcommand{\etal}{{\it et al.}}
\newcommand{\OL}[1]{\overline{#1}\ } % overline
\newcommand{\OLL}[1]{\overline{\overline{#1}}\ } % double overline
\newcommand{\OON}{\frac{1}{N}} % "one over N"
\newcommand{\OOX}[1]{\frac{1}{#1}} % "one over X"



\begin{document}
\Large
\centerline{\sc{Recitation Questions}}
\normalsize
\centerline{\sc{March 31}}
\small

\medskip

\begin{enumerate}

\item A unicyclist rides at a constant speed of 5 m/s; she and her unicycle have a combined mass of 70 kg. The wheel of her unicycle has a radius of 50 cm. At this speed, air resistance exerts a force of 80 N on her.

\begin{enumerate}
       \item What is the angular velocity of the wheel? 
\vspace{1.2in}

       \item As you know, the force that wheeled vehicles use to propel themselves forward is static friction. What is the size of this force?
\vspace {1.2in}

       \item What torque must she apply to the wheel to maintain her speed?
\vspace{2in}

\newpage
       \item Suppose the pedals are attached to a crank with a radius of 25 cm. What force must she apply to the pedals to maintain her speed?
\vspace{3in}

       \item What power does she apply to the pedals? What power does the force of static friction apply to the unicycle? What power does the air resistance apply? (Remember the correspondence between translational motion and 
rotational motion relations: if $P=Fv$, it follows that $P=\tau \omega$)
\end{enumerate}
 \newpage

\item .

\begin{minipage}[b]{0.4\textwidth}
  \vspace{-2.8in}

A 4m-long pole of mass 80 kg extends from the side of a building, angled at 60 degrees above the horizontal. One meter from the end of the pole, a sign of mass 50 kg is attached. To support the pole,
a horizontal cable runs from the end of the pole to the building. (See the attached figure.)

\bigskip
\bigskip
\bigskip
\bigskip
\bigskip
\bigskip

\end{minipage}
\begin{minipage}[t]{0.6\textwidth}
  \begin{flushright}
  \includegraphics[width=0.9\textwidth]{sign2.jpg}
\end{flushright}
\end{minipage}

\bigskip
\bigskip

\begin{enumerate}
\item Draw a force diagram on the back of this page, showing all of the elements needed to help you compute the tension in the support cable. Indicate
your choice of pivot point.

\bigskip

\item Compute the tension in the cable. 

\vspace{2 in}

\item Suppose now that the store owner wanted to attach the cable to a different point on the building in order to minimize its tension. What angle between the
cable and the horizontal would support the pole with the minimum tension?
\end{enumerate}
\newpage
\newpage


\end{enumerate}


\end{document}
