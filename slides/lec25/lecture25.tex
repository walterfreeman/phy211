% DPF 09 talk on strangeness in nucleon

\documentclass[10pt]{beamer}
\usefonttheme{professionalfonts} % using non standard fonts for beamer
\usefonttheme{serif} % default family is serif
\usepackage{amsmath}
\usepackage{mathtools}
\usepackage{mwe}
%\documentclass[12pt]{beamerthemeSam.sty}
\usepackage{epsf}
%\usepackage{pstricks}
%\usepackage[orientation=portrait,size=A4]{beamerposter}
\geometry{paperwidth=160mm,paperheight=120mm}
%DT favorite definitions
\def\LL{\left\langle}	% left angle bracket
\def\RR{\right\rangle}	% right angle bracket
\def\LP{\left(}		% left parenthesis
\def\RP{\right)}	% right parenthesis
\def\LB{\left\{}	% left curly bracket
\def\RB{\right\}}	% right curly bracket
\def\PAR#1#2{ {{\partial #1}\over{\partial #2}} }
\def\PARTWO#1#2{ {{\partial^2 #1}\over{\partial #2}^2} }
\def\PARTWOMIX#1#2#3{ {{\partial^2 #1}\over{\partial #2 \partial #3}} }

\def\rightpartial{{\overrightarrow\partial}}
\def\leftpartial{{\overleftarrow\partial}}
\def\diffpartial{\buildrel\leftrightarrow\over\partial}

\def\BI{\begin{itemize}}
\def\EI{\end{itemize}}
\def\BE{\begin{displaymath}}
\def\EE{\end{displaymath}}
\def\BEA{\begin{eqnarray*}}
\def\EEA{\end{eqnarray*}}
\def\BNEA{\begin{eqnarray}}
\def\ENEA{\end{eqnarray}}
\def\EL{\nonumber\\}


\newcommand{\map}[1]{\frame{\frametitle{\textbf{Course map}}
\centerline{\includegraphics[height=0.86\paperheight]{../../map/#1.png}}}}
\newcommand{\wmap}[1]{\frame{\frametitle{\textbf{Course map}}
\centerline{\includegraphics[width=0.96\paperwidth]{../../map/#1.png}}}}

\newcommand{\etal}{{\it et al.}}
\newcommand{\gbeta}{6/g^2}
\newcommand{\la}[1]{\label{#1}}
\newcommand{\ie}{{\em i.e.\ }}
\newcommand{\eg}{{\em e.\,g.\ }}
\newcommand{\cf}{cf.\ }
\newcommand{\etc}{etc.\ }
\newcommand{\atantwo}{{\rm atan2}}
\newcommand{\Tr}{{\rm Tr}}
\newcommand{\dt}{\Delta t}
\newcommand{\op}{{\cal O}}
\newcommand{\msbar}{{\overline{\rm MS}}}
\def\chpt{\raise0.4ex\hbox{$\chi$}PT}
\def\schpt{S\raise0.4ex\hbox{$\chi$}PT}
\def\MeV{{\rm Me\!V}}
\def\GeV{{\rm Ge\!V}}

%AB: my color definitions
%\definecolor{mygarnet}{rgb}{0.445,0.184,0.215}
%\definecolor{mygold}{rgb}{0.848,0.848,0.098}
%\definecolor{myg2g}{rgb}{0.647,0.316,0.157}
\definecolor{abtitlecolor}{rgb}{0.0,0.255,0.494}
\definecolor{absecondarycolor}{rgb}{0.0,0.416,0.804}
\definecolor{abprimarycolor}{rgb}{1.0,0.686,0.0}
\definecolor{Red}           {cmyk}{0,1,1,0}
\definecolor{Grey}           {cmyk}{.7,.7,.7,0}
\definecolor{Lg}           {cmyk}{.4,.4,.4,0}
\definecolor{Blue}          {cmyk}{1,1,0,0}
\definecolor{Green}         {cmyk}{1,0,1,0}
\definecolor{Brown}         {cmyk}{0,0.81,1,0.60}
\definecolor{Black}         {cmyk}{0,0,0,1}

\usetheme{Madrid}
\newcommand{\vcenteredinclude}[1]{\begingroup
  \setbox0=\hbox{\includegraphics[width=3in]{#1}}%
\parbox{\wd0}{\box0}\endgroup}

%AB: redefinition of beamer colors
%\setbeamercolor{palette tertiary}{fg=white,bg=mygarnet}
%\setbeamercolor{palette secondary}{fg=white,bg=myg2g}
%\setbeamercolor{palette primary}{fg=black,bg=mygold}
\setbeamercolor{title}{fg=abtitlecolor}
\setbeamercolor{frametitle}{fg=abtitlecolor}
\setbeamercolor{palette tertiary}{fg=white,bg=abtitlecolor}
\setbeamercolor{palette secondary}{fg=white,bg=absecondarycolor}
\setbeamercolor{palette primary}{fg=black,bg=abprimarycolor}
\setbeamercolor{structure}{fg=abtitlecolor}

\setbeamerfont{section in toc}{series=\bfseries}

%AB: remove navigation icons
\beamertemplatenavigationsymbolsempty
\title{
  \textbf {Review and recap}\\
%\centerline{}
%\centering
%\vspace{-0.0in}
%\includegraphics[width=0.3\textwidth]{propvalues_0093.pdf}
%\vspace{-0.3in}\\
%\label{intrograph}
}

\author[W. Freeman] {Physics 211\\Syracuse University, Physics 211 Spring 2017\\Walter Freeman}

\date{\today}

\begin{document}

\frame{\titlepage}

\frame{\frametitle{\textbf{Announcements}}
  \large
\BI
\item{Extra office hours and reviews ahead of the exam:}
\BI
\item Today, 5-7, Physics Clinic
\item Thursday, 10-3:45, Physics Clinic (with a break for lunch)
\item Possibly Friday (will be announced by email)
\item Sunday, 7-9 PM, Stolkin Auditorium
\EI
\item Come see me about exam corrections (if you had an excused absence 
last week) or anything else
\EI
}

\frame{\frametitle{\textbf{Final exam format}}
  \Large
\BI
\item{More questions, but less time-consuming}
\item{Expect questions like:}
  \BI
\item{``What concept could you use to solve this problem?''}
\item{``Write but do not solve a system of two equations that will let you find $a$ and $T$''}
\item{``Which of these equations would be useful here?''}
  \EI
\item{You may make {\bf your own reference sheets}}
\item{Two sides of a standard piece of paper, {\bf handwritten}}
  \EI
}

\frame{\frametitle{\textbf{Kinematics concepts}}
  \large
  \BI
\item{First derivative of position is velocity; second derivative is acceleration}
\item{Kinematics lets us connect acceleration, velocity, position, and time}
\item{If $\vec a$ is constant:}

  \begin{align*}
    s(t) =& s_0 + v_0 t + \frac{1}{2} a t^2 \\
    v(t) =& v_0 + at \\
    v_f^2 - v_0^2 =& 2a\Delta x
  \end{align*}

\item{These relations hold separately and independently in $x$ and $y$}
\item{Acceleration is $g$ downwards {\bf if and only if} an object is in freefall}
\EI
}

\frame{\frametitle{\textbf{Kinematics sample problem: the ball-and-table problem}}
  \Large

A ball rolls off of a table of height $h$ at speed $v$. How far does it go?

}

\frame{\frametitle{\textbf{Use kinematics when:}}
    \Large
    \BI
  \item{You need to connect some combination of position, velocity, acceleration, and time}
  \EI

}

\frame{\frametitle{\textbf{Force concepts and Newton's second law}}
    \large
    \BI
  \item{Newton's second law relates the net force $\sum \vec F$ to the acceleration $\vec a$ of the center of mass of an object}
  \item{If an object both rotates and moves, $\vec F=m\vec a$ gives you $\vec a$ of the center of mass}
  \item{Newton's third law: forces come in pairs}
  \item{Some forces you should know about:}
    \BI
  \item{Normal forces: as big as they need to be}
  \item{Friction: $F_{\rm{fric, static, max}} = \mu_s F_N$, $F_{\rm{fric, kinetic}}=\mu_k F_N$}
  \item{Traction: a type of static friction, points in direction chosen by the
vehicle}
  \item{Elastic: $F=-k\Delta x$}
  \item{Gravity (Earth): $F=mg$ downward}
  \item{Gravity (general): $F=\frac{Gm_1m_2}{r^2}$}
  \item{Tension: A rope pulls on both ends}
    \EI
    \EI
  }

  \frame{\frametitle{\textbf{Force diagrams}}
    \Large
    \BI
  \item{Draw all forces acting on the object, as vectors}
  \item{If you're going to care about torque, draw them where they act}
  \item{Gravity acts at the center of mass}
  \item{Draw these diagrams big enough that you can read them clearly and do trig}
    \EI
  }

\frame{\frametitle{\textbf{Uniform circular motion}}
  \Large
  \BI
\item{If an object is traveling in a circle, you know its acceleration is $a_c = \omega^2 r = \frac{v_T^2}{r}$ toward the center}
\item{Often this will ``give you'' the right side of $F=ma$, and let you conclude something about the left}
  \EI
}
  \frame{\frametitle{\textbf{Use Newton's second law when:}}
    \Large
    \BI
  \item{You need to connect the forces on an object to its acceleration}
  \item{If you don't need $\vec a$ directly, and don't care about time, maybe use energy methods instead?}
    \EI
  }


  \frame{\frametitle{\textbf{Sample problem: the eraser in the tube}}
    \Large
What angular frequency is required to make the eraser not fall?
  }

  \frame{\frametitle{\textbf{The work-energy theorem and conservation of energy}}
    \large
    \BI
  \item{Work-energy theorem comes from the third kinematics relation}
  \item{Two formulations, one with potential energy and one without:}
    \BI
  \item{$KE_i + W_{\rm{all}} = KE_f$}
  \item{$KE_i + PE_i + W_{\rm{other}} = KE_f + PE_f$}
    \EI
  \item{Draw {\it clear} before and after snapshots}
  \item{Figure out work done in going from one to the other}
  \item{Work = $\vec F \cdot d\vec s$}
    \EI
  }

 \frame{\frametitle{\textbf{Use energy methods when:}}
      \Large
      \BI
    \item{You don't know and don't care about time}
    \item{You can account for the work done by all forces involved}
    \item{This is {\bf not} true at the instant of a collision -- use momentum instead}
      \EI
    }

    \frame{\frametitle{\textbf{Sample problem: energy}}
      \Large
A ball rolls down a hill of height $h$ and across a table. 
How fast is it moving at the 
edge of the table?    
}

  \frame{\frametitle{\textbf{Conservation of momentum}}
      \BI
    \item{In the absence of external forces, $\vec p = m\vec v$ is conserved}
    \item{This is a consequence of Newton's third law}
    \item{Collisions and explosions are short enough that external forces are small}
    \item{Momentum is a vector and is conserved separately in $x$ and $y$}
      \EI
    }

  \frame{\frametitle{\textbf{Use conservation of momentum when:}}
        \Large
        \BI
      \item{You have a collision or explosion and need to connect the velocities before to the velocities after}
        \EI
      }

   \frame{\frametitle{\textbf{Rotation}}
          \Large
          Many ideas here, most analogous to translational motion:
\large
\BI
\item{Torque plays the role of force: $\tau = F_\perp r = F r_\perp$}
\item{Moment of inertia plays the role of mass: $I = \lambda mr^2$}
\item{$\vec F = m \vec a \rightarrow \tau = I \alpha$: ``Newton's second law for rotation''}
\item{Rolling motion is translation plus rotation: $v = \pm \omega r$, $a = \pm \alpha r$}
\item{\bf You must think about the signs here}
\item{Rotational kinetic energy: $KE_{\rm{rot}} = \frac{1}{2} I \omega^2$}
\item{Angular momentum: $L = I \omega$}
  \EI
}

\frame{\frametitle{\textbf{Static equilibrium problems}}
  \Large
  \BI
\item{Net torque is zero about any pivot}
\item{Net force is zero (you may not need this)}
\item{Torque due to any force applied {\bf at} the pivot is zero}
\EI
}

\frame{\frametitle{\textbf{Standing waves in strings/tubes:}}
  \Large
  \vcenteredinclude{mode1-crop.pdf} Fundamental: $f_1 = \frac{c}{2L}$\\
 \bigskip
  \vcenteredinclude{mode2-crop.pdf} 2nd harmonic: $f_2 = 2 f_1 $\\
 \bigskip
  \vcenteredinclude{mode3-crop.pdf} 3rd harmonic: $f_3 = 3 f_1$ \\

 \bigskip
  \vcenteredinclude{mode4-crop.pdf} 4th harmonic: $f_4 = 4 f_1$ \\

 \bigskip

\begin{center}
The relation between wavespeed, wavelength, and frequency:

$$c=f\lambda$$
\end{center}

}

\frame{\frametitle{\textbf{Final reminders}}
  \large
  \BI
\item{Huge amounts of extra review available; use it}
\item{Get some rest during finals week and take care of yourselves}
\item{If you're affected by the Calc/Physics exam scheduling nonsense, tell SU!}
\EI
}

\frame{\frametitle{\textbf{The power of mechanics}}
\large
The things we've studied in this class are more powerful than you think. 

If you call up a chemist, she'll tell you the approximate force law between two noble gas atoms:

$$ F(r) = \frac{\alpha}{r^{12}} - \frac{\beta}{r^6} $$

Put this into a computer and let it go:

\pause\bigskip

\centerline{\color{Red}We can understand freezing, melting, and boiling just with $\vec F = m \vec a!$}

\centerline{\color{Red}... we can even get the ideal gas law for free along the way!}
}

\frame{\frametitle{\textbf{The rest of physics}}
\large
The other disciplines of physics are variants on what you've learned already:

\BI
\item Electromagnetism (PHY 211) introduces a new force -- just another $\vec F$
\BI
\item Light is just a particular manifestation of that force
\EI
\item Statistical mechanics uses statistics to understand 
$\vec F = m\vec a$ acting on a great many particles at once
\item Relativity mixes up space and time, changing the coordinates on us
\item Quantum mechanics mixes up ``particle'' and ``wave''
\EI

Each of these disciplines is supported by a ``three-legged stool'':

\BI
\item Theory: understanding principles and using pen and paper to study them in simple situations (this class)
\item Experiment: designing tests for these principles and building machines to carry them out (221)
\item Computation: using computers to simulate those principles in more complicated situations and study their consequences (my field and class in the fall)
\EI
}

\frame{\frametitle{\textbf{And, finally...}}
  \Large
... thank you all; you're the best Physics 211 class yet, in multiple ways,
and you should be proud of yourselves.

\pause
\bigskip
\bigskip
\bigskip
\bigskip

\centerline{``Science is the belief in the ignorance of experts.'' (R. Feynman)}

\pause

\bigskip
\bigskip
\bigskip
\bigskip



\centerline{``All science is either physics or stamp collecting.'' (E. Rutherford)}
}
\frame{\frametitle{\textbf{I leave you with two quotes...}}
\pause
``A poet once said, 'The whole universe is in a glass of wine.'...  [I]f we look at a glass of wine closely enough we see the entire universe. 
There are the things of physics: the twisting liquid which evaporates depending on the wind and weather, the reflection in the glass; and our imagination adds atoms. 
The glass is a distillation of the earth's rocks, and in its composition we see the secrets of the universe's age, and the evolution of stars. 
What strange array of chemicals are in the wine? 
How did they come to be?... 
If our small minds, for some convenience, divide this glass of wine, this universe, into parts -- physics, biology, geology, astronomy, psychology, 
and so on -- remember that nature does not know it! So let us put it all back together, not forgetting ultimately what it is for. Let it give us one more final pleasure; drink it and forget it all!''

\bigskip
\bigskip

\pause

``Poets say science takes away from the beauty of the stars — mere globs of gas atoms. Nothing is "mere". I too can see the stars on a desert night, and feel them. But do I see less or more? The vastness of the heavens stretches my imagination — stuck on this carousel my little eye can catch one-million-year-old light. A vast pattern — of which I am a part... What is the pattern, or the meaning, or the why? It does not do harm to the mystery to know a little about it. For far more marvelous is the truth than any artists of the past imagined!''

\bigskip

--Richard Feynman, from {\em Lectures on Physics}
}
\end{document}
