\documentclass[12pt]{article}
\setlength\parindent{0pt}
\usepackage{fullpage}
\usepackage{amsmath}
\usepackage{graphicx}
\setlength{\parskip}{4mm}
\def\LL{\left\langle}   % left angle bracket
\def\RR{\right\rangle}  % right angle bracket
\def\LP{\left(}         % left parenthesis
\def\RP{\right)}        % right parenthesis
\def\LB{\left\{}        % left curly bracket
\def\RB{\right\}}       % right curly bracket
\def\PAR#1#2{ {{\partial #1}\over{\partial #2}} }
\def\PARTWO#1#2{ {{\partial^2 #1}\over{\partial #2}^2} }
\def\PARTWOMIX#1#2#3{ {{\partial^2 #1}\over{\partial #2 \partial #3}} }
\newcommand{\BI}{\begin{itemize}}
\newcommand{\EI}{\end{itemize}}
\newcommand{\BE}{\begin{displaymath}}
\newcommand{\EE}{\end{displaymath}}
\newcommand{\BNE}{\begin{equation}}
\newcommand{\ENE}{\end{equation}}
\newcommand{\BEA}{\begin{eqnarray}}
\newcommand{\EEA}{\nonumber\end{eqnarray}}
\newcommand{\EL}{\nonumber\\}
\newcommand{\la}[1]{\label{#1}}
\newcommand{\ie}{{\em i.e.\ }}
\newcommand{\eg}{{\em e.\,g.\ }}
\newcommand{\cf}{cf.\ }
\newcommand{\etc}{etc.\ }
\newcommand{\Tr}{{\rm tr}}
\newcommand{\etal}{{\it et al.}}
\newcommand{\OL}[1]{\overline{#1}\ } % overline
\newcommand{\OLL}[1]{\overline{\overline{#1}}\ } % double overline
\newcommand{\OON}{\frac{1}{N}} % "one over N"
\newcommand{\OOX}[1]{\frac{1}{#1}} % "one over X"



\begin{document}
\pagenumbering{gobble}
\Large
\centerline{\sc{Recitation Questions}}
\normalsize
\centerline{\sc{4 March}}

\medskip


\centerline{\Large Question 1: geostationary orbit}

It is sometimes useful to place satellites in orbit so that they stay in a fixed position relative to the
Earth; that is, their orbits are synchronized with the Earth's rotation so that a satellite might stay
above the same point on Earth’s surface all the time.

What is the altitude of such an orbit? Note that it is high enough that you need to use $F_g=\frac{GMm}{r^2}$
rather than just $F_g = mg$.

{\sc Hint 1:} If this orbit is synchronized with Earth's rotation, then you should be able to figure out its
angular velocity.

{\sc Hint 2:} If you do this problem as we have guided you, by waiting to substitute numbers in until the 
very end, you will arrive at an expression relating the radius $R$ of a circular orbit with the mass $M$ of the 
planet being orbited and the angular velocity $\omega$ of the orbit. This question will be on HW5, and is related
to the derivation of Kepler's third law that you will do there.

\newpage

\centerline{\Large Question 2: variation of apparent weight with latitude}

\medskip

{\footnotesize Note about this exercise: This question asks you to consider what $g$ means, and how it might vary on Earth's surface. The effect you are studying here is the reduction in apparent weight because of the Earth's rotation. There is a second effect from the ``non-roundness'' of the Earth, causing a latitude dependence about half as big in the same direction. Ultimately this means that the apparent weight of a 1 kg mass is around 9.83 N at the North (or South) Pole and 9.78 N at the Equator. It is 9.81 N only in latitudes corresponding to Paris, Berlin, or Kyiv; in New York it is closer to 9.80 N, and in the tropical latitudes where many people live (most of Africa, South Asia, most of South America) it is 9.79 N or 9.78 N, to three significant digits. This means that insisting on $g=9.81\,\rm m/\rm s^2$ isn't correct for most of us (unless you are in Europe or Canada).
}


\it For this problem, carry all calculations to five significant digits. Some figures that will be useful:

\rm
\BI
\item Mass of Earth: $5.9722\times 10^{24}$ kg
\item Radius of Earth: $ 6.3710 \times 10^6$ m (assume it is spherical; we don't have the math to deal with its oblateness)
\item Gravitational constant (G): $6.6741 \times 10^{-11} \,\rm N\cdot \rm m^2/{\rm kg}^2$
\item Length of one day: $8.6400 \times 10^4$ s
\EI


a) Using Newton's law of universal gravitation $F_g=\frac{GMm}{r^2}$, determine the force of gravity on a 1 kg mass resting on the surface of the Earth. Are you surprised by this figure?


\vspace{2in}

b) Suppose this mass were resting on a scale sitting on the North Pole owned by Santa Claus. Recall that scales measure
the normal force that they exert. What value would Santa's scale read? What would Santa conclude the value of $g$ is?

\vspace{2in}
\newpage
c) Suppose that an identical 1 kg mass were resting on a scale sitting on the Equator, somewhere in Kenya. What would {\it this} scale read? (Hint: What is the acceleration of the mass?) What would our Kenyan physicist conclude about $g$?

\vspace{3in}

d) This problem shows that your apparent weight depends on your location on Earth. 
Does it make sense to define $g$ as $F_g/m$ 
(the strength of the gravitational force divided by an object's mass) or
$F_N/m$ (the strength of the normal force, and thus the scale reading, divided by mass)? Call your TA/coach over to join your conversation.

\vspace{2in}

e) Is this distinction likely to be relevant to the sort of engineering or science you will do during your
career? (The answer will depend on what you will do, of course!)


%\newpage
%
%\centerline{\Large Question 4: universal gravitation and the Sun's mass}
%
%In this problem, you will compute the mass of the Sun. The Earth’s orbit is very nearly circular,
%and the earth is 150 million km from the Sun.
%
%a) What is the angular velocity of the Earth in its orbit?
%
%\vspace{2in}
%
%b) What is the tangential velocity of the Earth? 
%
%\vspace{1in}
%
%c) What is the radial acceleration of the Earth?
%
%\vspace{1in}
%
%d) What is the mass of the Sun?
%
%\newpage
%
\newpage
\centerline{\Large Question 3: Weightlessness}

Astronauts in orbit around the Earth are not ``so far away that they don't feel Earth's gravity'';
actually, they’re quite close to the surface. However, we’ve all seen the videos of astronauts drifting
around ``weightlessly'' in the International Space Station.

a) Explain how an astronaut can be under the influence of Earth's gravity, and yet exert no normal
force on the surface of the spacecraft they are standing in.

\vspace{2in}

b) Draw a force diagram for the astronaut floating in the middle of the Space Station, not touching
any of the walls or floor. How do you reconcile your diagram with the fact that the astronaut
doesn't seem to fall?

\vspace{2in}

c) Is this astronaut truly ``weightless''? What does ``weightless'' mean? (There are multiple correct answers to both of these questions.)

\newpage

\end{document}
