\documentclass[12pt]{article}
\setlength\parindent{0pt}
%\usepackage{fullpage}
\usepackage{amsmath}
\usepackage{hyperref}
\usepackage[margin=2cm]{geometry}
\usepackage{graphicx}
\setlength{\parskip}{4mm}
\def\LL{\left\langle}   % left angle bracket
\def\RR{\right\rangle}  % right angle bracket
\def\LP{\left(}         % left parenthesis
\def\RP{\right)}        % right parenthesis
\def\LB{\left\{}        % left curly bracket
\def\RB{\right\}}       % right curly bracket
\def\PAR#1#2{ {{\partial #1}\over{\partial #2}} }
\def\PARTWO#1#2{ {{\partial^2 #1}\over{\partial #2}^2} }
\def\PARTWOMIX#1#2#3{ {{\partial^2 #1}\over{\partial #2 \partial #3}} }
\newcommand{\BE}{\begin{displaymath}}
\newcommand{\EE}{\end{displaymath}}
\newcommand{\BNE}{\begin{equation}}
\newcommand{\ENE}{\end{equation}}
\newcommand{\BEA}{\begin{eqnarray}}
\newcommand{\EEA}{\nonumber\end{eqnarray}}
\newcommand{\EL}{\nonumber\\}
\newcommand{\la}[1]{\label{#1}}
\newcommand{\ie}{{\em i.e.\ }}
\newcommand{\eg}{{\em e.\,g.\ }}
\newcommand{\cf}{cf.\ }
\newcommand{\etc}{etc.\ }
\newcommand{\Tr}{{\rm tr}}
\newcommand{\etal}{{\it et al.}}
\newcommand{\OL}[1]{\overline{#1}\ } % overline
\newcommand{\OLL}[1]{\overline{\overline{#1}}\ } % double overline
\newcommand{\OON}{\frac{1}{N}} % "one over N"
\newcommand{\OOX}[1]{\frac{1}{#1}} % "one over X"



\begin{document}
\begin{center}
\Large
\sc Second Chance Exercise - Vectors \rm

\pagenumbering{gobble}




\normalsize
This second chance exercise is for people who want to review vector arithmetic. 

If one of your lowest three exam problem grades was on the ``goose problem'', you may complete it and turn it in to earn an extra homework grade, which we will help you earn 100\% on. This will help your end-of-term average.

If you do this, it is due on the last day of recitation.

\end{center}

\vspace{1.5in}

This exam problem covered vector and subtraction. In it, you needed to determine which pair of vectors added together to make a third, then use vector addition/subtraction to determine the unknown vector.

\begin{enumerate}
	\item You did a very similar problem in Week 2 Day 2's recitation on vectors (\url{https://walterfreeman.github.io/phy211/recitation/week2/recitation-vectors.pdf}). Which one was that?
	\item The first practice exam is here: \url{https://walterfreeman.github.io/phy211/practice-exam-1.pdf}. 
	\begin{enumerate}
		\item Which two problems on the practice exam involve similar ideas?
		\item How is each of those problems similar to the one you had on Exam 1 (with the geese)? How is each of them different?
	\end{enumerate}
    \item Describe in a paragraph how you have used the ideas of vector addition elsewhere in the semester, particularly in Unit 2 (involving forces). 
    \item Write a problem that evaluates students' ability to do vector addition and subtraction, similar to the ``goose problem'' from Exam 1 and the two problems you identified on Practice Exam 1. Then solve it. {\it (Particularly good problems here will earn extra credit!)}
\end{enumerate}



\end{document}
