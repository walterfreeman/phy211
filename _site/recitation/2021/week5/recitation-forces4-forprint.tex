\documentclass[12pt]{article}
\setlength\parindent{0pt}
\usepackage{fullpage}
\usepackage{amsmath}
\usepackage[margin=1.5cm]{geometry}
\usepackage{graphicx}
\setlength{\parskip}{4mm}
\def\LL{\left\langle}   % left angle bracket
\def\RR{\right\rangle}  % right angle bracket
\def\LP{\left(}         % left parenthesis
\def\RP{\right)}        % right parenthesis
\def\LB{\left\{}        % left curly bracket
\def\RB{\right\}}       % right curly bracket
\def\PAR#1#2{ {{\partial #1}\over{\partial #2}} }
\def\PARTWO#1#2{ {{\partial^2 #1}\over{\partial #2}^2} }
\def\PARTWOMIX#1#2#3{ {{\partial^2 #1}\over{\partial #2 \partial #3}} }
\newcommand{\BE}{\begin{displaymath}}
\newcommand{\EE}{\end{displaymath}}
\newcommand{\BNE}{\begin{equation}}
\newcommand{\ENE}{\end{equation}}
\newcommand{\BEA}{\begin{eqnarray}}
\newcommand{\EEA}{\nonumber\end{eqnarray}}
\newcommand{\EL}{\nonumber\\}
\newcommand{\la}[1]{\label{#1}}
\newcommand{\ie}{{\em i.e.\ }}
\newcommand{\eg}{{\em e.\,g.\ }}
\newcommand{\cf}{cf.\ }
\newcommand{\etc}{etc.\ }
\newcommand{\Tr}{{\rm tr}}
\newcommand{\etal}{{\it et al.}}
\newcommand{\OL}[1]{\overline{#1}\ } % overline
\newcommand{\OLL}[1]{\overline{\overline{#1}}\ } % double overline
\newcommand{\OON}{\frac{1}{N}} % "one over N"
\newcommand{\OOX}[1]{\frac{1}{#1}} % "one over X"



\begin{document}
\pagenumbering{gobble}
\Large
\centerline{\sc{Recitation Exercises}}
\normalsize
\centerline{\sc{12 March}}


A person uses a rope to spin a bucket in a vertical circle at a constant speed; the radius of the circle is 80 cm. The bucket goes around the circle once every second. Inside the bucket is a friendly frog of mass 500 grams. 


a) Draw a force diagram for the frog when the bucket is at the top of the circle, and when it is at the bottom.

\vspace{2in}

b) What is the acceleration of the bucket? {\it (Think about both its magnitude and direction.)}

\vspace{1in}


c) As you saw last week in your homework, your ``apparent weight'' is simply the magnitude of the normal force that an object under you exerts on you. What is the frog's apparent weight at the bottom and at the top of the circle?

\vspace{2in}
\newpage
d) Explain why the frog doesn't fall out of the bucket at the top of the swing, despite the fact that the only forces acting on it point downward. This is a pretty subtle but important point -- you should talk about it for a while and call a coach or TA over to join your conversation.

\vspace{4in}

e) Now, imagine that the person swinging the bucket slows down gradually. At some point, the frog will fall out of the bucket. (It's a frog, so it'll land on its feet and not be hurt!) How low can $\omega$ become before the frog falls out of the bucket?

\newpage


A highway curve has a radius of curvature of 500 meters; that is, it is a segment of a circle whose radius is 500 m. It is banked so that traffic moving at 30 m/s can travel
around the curve without needing any help from friction.


a) Draw a force diagram for a car traveling around this curve at a constant speed. Draw the diagram so that you are looking at the rear of the car. {\it Hint:} Do not tilt your coordinate axes for this problem!


\vspace{2in}

b) What is the acceleration of the car in the $x-$direction? What about the $y-$direction?

\vspace{1in}

c) Write down two copies of Newton's second law in the $x-$ and $y-$directions.

\vspace{1.5in}

d) Solve the resulting system of two equations to determine the banking angle of the curve.

\end{document}
