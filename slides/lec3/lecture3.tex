
\documentclass[10pt]{beamer}
\usepackage{amsmath}
\usepackage{tabto}
\usepackage{mathtools}
\usefonttheme{professionalfonts} % using non standard fonts for beamer
\usefonttheme{serif} % default family is serif
%\documentclass[12pt]{beamerthemeSam.sty}
\usepackage{epsf}
%\usepackage{pstricks}
%\usepackage[orientation=portrait,size=A4]{beamerposter}
\geometry{paperwidth=160mm,paperheight=120mm}
%DT favorite definitions
\def\LL{\left\langle}	% left angle bracket
\def\RR{\right\rangle}	% right angle bracket
\def\LP{\left(}		% left parenthesis
\def\RP{\right)}	% right parenthesis
\def\LB{\left\{}	% left curly bracket
\def\RB{\right\}}	% right curly bracket
\def\PAR#1#2{ {{\partial #1}\over{\partial #2}} }
\def\PARTWO#1#2{ {{\partial^2 #1}\over{\partial #2}^2} }
\def\PARTWOMIX#1#2#3{ {{\partial^2 #1}\over{\partial #2 \partial #3}} }

\def\rightpartial{{\overrightarrow\partial}}
\def\leftpartial{{\overleftarrow\partial}}
\def\diffpartial{\buildrel\leftrightarrow\over\partial}

\def\BC{\begin{center}}
\def\EC{\end{center}}
\def\BI{\begin{itemize}}
\def\EI{\end{itemize}}
\def\BE{\begin{displaymath}}
\def\EE{\end{displaymath}}
\def\BEA{\begin{eqnarray*}}
\def\EEA{\end{eqnarray*}}
\def\BNEA{\begin{eqnarray}}
\def\ENEA{\end{eqnarray}}
\def\EL{\nonumber\\}


\newcommand{\etal}{{\it et al.}}
\newcommand{\gbeta}{6/g^2}
\newcommand{\la}[1]{\label{#1}}
\newcommand{\ie}{{\em i.e.\ }}
\newcommand{\eg}{{\em e.\,g.\ }}
\newcommand{\cf}{cf.\ }
\newcommand{\etc}{etc.\ }
\newcommand{\atantwo}{{\rm atan2}}
\newcommand{\Tr}{{\rm Tr}}
\newcommand{\dt}{\Delta t}
\newcommand{\op}{{\cal O}}
\newcommand{\msbar}{{\overline{\rm MS}}}
\def\chpt{\raise0.4ex\hbox{$\chi$}PT}
\def\schpt{S\raise0.4ex\hbox{$\chi$}PT}
\def\MeV{{\rm Me\!V}}
\def\GeV{{\rm Ge\!V}}

%AB: my color definitions
\definecolor{A}{rgb}{0.8,0.0,0.0}
\definecolor{B}{rgb}{0.0,0.6,0.0}
\definecolor{C}{rgb}{0.6,0.6,0.0}
\definecolor{D}{rgb}{0.0,0.0,0.5}
\definecolor{E}{rgb}{0.4,0.4,0.4}

%\definecolor{mygarnet}{rgb}{0.445,0.184,0.215}
%\definecolor{mygold}{rgb}{0.848,0.848,0.098}
%\definecolor{myg2g}{rgb}{0.647,0.316,0.157}
\definecolor{abtitlecolor}{rgb}{0.0,0.255,0.494}
\definecolor{absecondarycolor}{rgb}{0.0,0.416,0.804}
\definecolor{abprimarycolor}{rgb}{1.0,0.686,0.0}
\definecolor{Red}           {cmyk}{0,1,1,0}
\definecolor{Grey}           {cmyk}{.7,.7,.7,0}
\definecolor{Blue}          {cmyk}{1,1,0,0}
\definecolor{Green}         {cmyk}{1,0,1,0}
\definecolor{Brown}         {cmyk}{0,0.81,1,0.60}

\usetheme{Madrid}


%AB: redefinition of beamer colors
%\setbeamercolor{palette tertiary}{fg=white,bg=mygarnet}
%\setbeamercolor{palette secondary}{fg=white,bg=myg2g}
%\setbeamercolor{palette primary}{fg=black,bg=mygold}
\setbeamercolor{title}{fg=abtitlecolor}
\setbeamercolor{frametitle}{fg=abtitlecolor}
\setbeamercolor{palette tertiary}{fg=white,bg=abtitlecolor}
\setbeamercolor{palette secondary}{fg=white,bg=absecondarycolor}
\setbeamercolor{palette primary}{fg=black,bg=abprimarycolor}
\setbeamercolor{structure}{fg=abtitlecolor}

\setbeamerfont{section in toc}{series=\bfseries}

%AB: remove navigation icons
\beamertemplatenavigationsymbolsempty
\title[1D kinematics]{
  \textbf {1D kinematics: solving problems} 
%\centerline{}
%\centering
%\vspace{-0.0in}
%\includegraphics[width=0.3\textwidth]{propvalues_0093.pdf}
%\vspace{-0.3in}\\
%\label{intrograph}
}

\author[W. Freeman] {Physics 211\\Syracuse University, Physics 211 Spring 2022\\Walter Freeman}

\date{\today}

\begin{document}

\frame{\titlepage}

\frame{\frametitle{\textbf{On solving problems}}
\large
\begin{center}
\color{Red}
You can recognize truth by its beauty and simplicity. 
When you get it right, it is obvious that it is right--at least if you have any experience--because 
usually what happens is that more comes out than goes in....
Inexperienced students make guesses that are very complicated, $[$but$]$ the truth always turns out to be simpler than you thought.

\bigskip

\end{center}
\bigskip
--Richard Feynman, quoted by K. C. Cole, in {\it Sympathetic Vibrations: Reflections on Physics as a Way of Life (1985)}

\bigskip
\bigskip
\bigskip
\bigskip

\begin{center}
\color{Red}
Nature uses only the longest threads to weave her patterns, so each small piece of her fabric reveals the organization of the entire tapestry.
\bigskip

\end{center}
\bigskip
--Richard Feynman, {\it The Character of Physical Law} (1965) 
}



\frame{\frametitle{\textbf{Announcements}}
\large
\BI
\item{Homework 1 due Friday; homework 2 assigned tomorrow, due next Friday}
\item If at all possible, finish HW1 early and start HW2 this week
\item I'll be in the Physics Clinic to help people with anything:
\BI
\item 2-4 PM today (Tuesday)
\item 2:15-4 PM Wednesday
\item 1:30-3:30 PM Thursday
\EI
\item You can also ask your classmates and us for advice on Discord
\EI
}




\frame{\frametitle{\textbf{``Ask a Physicist''}}
\Large

Submit questions by email or Discord (or the weekly checkin forms)

}

\frame{\frametitle{\textbf{Today}}
\Large
\BI
\item{Review material from last time}
\item{Position, velocity, and acceleration graphs}
\item{Problem-solving method for kinematics problems}
\item{Sample problems}
\pause
\item{Homework help?}
\EI
}

\frame{\frametitle{\textbf{Last time: Position, velocity, and acceleration}}
\begin{columns}
\column{0.125\textwidth}
\centerline{\Large Position}
\column{0.2\textwidth}
\small
{\color{Red}
\centerline{(derivative of)}
\centerline{rate of change of}
\centerline{$\xleftarrow{\makebox[\textwidth]{}}$}}
{\color{Green}
\color{Green}\centerline{$\xrightarrow{\makebox[\textwidth]{}}$}}
\color{Green}\centerline{area under the curve of}
\color{Green}\centerline{(integral of)}
\column{0.125\textwidth}
\centerline{\Large Velocity}
\pause
\column{0.2\textwidth}
\small
{\color{Red}
\centerline{(derivative of)}
\centerline{rate of change of}
\centerline{$\xleftarrow{\makebox[\textwidth]{}}$}}
{\color{Green}
\color{Green}\centerline{$\xrightarrow{\makebox[\textwidth]{}}$}}
\color{Green}\centerline{area under the curve of}
\color{Green}\centerline{(integral of)}
\column{0.15\textwidth}
\centerline{\Large Acceleration}
\end{columns}
}


%
%\frame{\frametitle{\textbf{Position, velocity, and acceleration}}
%\begin{columns}
%\column{0.125\textwidth}
%\centerline{\Large Position}
%\column{0.2\textwidth}
%\small
%{\color{Red}
%\centerline{(derivative of)}
%\centerline{rate of change of}
%\centerline{$\xleftarrow{\makebox[\textwidth]{}}$}}
%\column{0.125\textwidth}
%\centerline{\Large Velocity}
%\pause
%\column{0.2\textwidth}
%\small
%{\color{Red}
%\centerline{(derivative of)}
%\centerline{rate of change of}
%\centerline{$\xleftarrow{\makebox[\textwidth]{}}$}}
%\column{0.15\textwidth}
%\centerline{\Large Acceleration}
%\end{columns}
%}

\frame{\frametitle{\textbf{Kinematics}}
\Large
\BI
\item{If we know acceleration as a function of time, how do we get from there to position vs. time?}
\pause

\bigskip
\bigskip
\bigskip
\item{\color{A}A. Look at the slope of the acceleration vs. time graph to get velocity, and then look at its slope to get position}
\item{\color{B}B. Look at the area under the curve of the acceleration vs. time graph to get velocity, and then look at the area under {\it} that graph to get position}
\item{\color{C}C. Take two derivatives of the acceleration vs. time graph to get position vs. time}
\item{\color{D}D. Take two integrals of the acceleration vs. time graph to get position vs. time}
\EI
}

\frame{\frametitle{\textbf{The ``kinematics equations''}}

{\color{Red}
\Large
\begin{align*}
v(t) =& at + v_0 \\
x(t) =& \frac{1}{2}at^2 + v_0 t + x_0
\end{align*}}

\bigskip
\bigskip
\bigskip

These equations are valid when...
\BI
\item{\color{A}A. Acceleration is constant}
\item{\color{B}B. Velocity is constant}
\item{\color{C}C. The object moves in only one direction}
\item{\color{D}D. They are always valid}
\pause
\bigskip
\bigskip
\item{A: these are the expressions for $x(t)$ and $v(t)$ when acceleration is constant!}
\EI
}



\frame{\frametitle{\textbf{Example problems}}
\Large
\BC How long does it take an object to fall from a height $h$? \EC
\pause
\bigskip
\bigskip
\bigskip

\large

First: Write the position and velocity equations, substituting in things you know.
(Here, take ground level to be $y=0$, and upward to be the positive direction.)

\bigskip\bigskip

\color{A}A: $x(t) = \hspace{10pt}\frac{1}{2}gt^2 + h$ \tabto{3in} $v(t) = -gt$  \\ \vspace{4pt}
\color{B}B: $x(t) = -\frac{1}{2}gt^2 + v_0 t + h$ \tabto{3in} $v(t) = -gt$  \\ \vspace{4pt}
\color{C}C: $x(t) = -\frac{1}{2}gt^2 + h$ \tabto{3in} $v(t) = gt$  \\ \vspace{4pt}
\color{D}D: $x(t) = -\frac{1}{2}gt^2 + h$ \tabto{3in} $v(t) = -gt$  \\ \vspace{4pt}
\color{E}E: $x(t) = -\frac{1}{2}gt^2 + v_0 t$ \tabto{3in} $v(t) = -gt$  \\ \vspace{4pt}

}

\frame{\frametitle{\textbf{Example problems}}
\Large
\BC How long does it take an object to fall from a height $h$? \EC
\pause
\bigskip
\bigskip
\bigskip

\large

Second: Phrase the question in terms of your algebraic variables.


\bigskip

From the previous: $x(t) = -\frac{1}{2}gt^2 + h$ and $v(t) = -gt$.
 
\bigskip

(Again, take ground level to be $y=0$, and upward to be the positive direction.)

\bigskip\bigskip

\color{A}A: ``What is the value of $t$ when $v=0$?'' \\ \vspace{4pt}
\color{B}B: ``What is the value of $x$ when $t=0$?'' \\ \vspace{4pt}
\color{C}C: ``What is the value of $x$ when $v=0$?'' \\ \vspace{4pt}
\color{D}D: ``What is the value of $t$ when $x=h$?'' \\ \vspace{4pt}
\color{E}E: ``What is the value of $t$ when $x=0$?'' \\ \vspace{4pt}
}

\frame{\frametitle{\textbf{Example problems}}
\Large
\BC How long does it take an object to fall from a height $h$? \EC
\bigskip
\bigskip
\bigskip
\large
Third: Do the algebra your sentence tells you to do: ``What is the value of $t$ when $x=0$?''
 
\bigskip\large

From the previous: $x(t) = -\frac{1}{2}gt^2 + h$ and $v(t) = -gt$.

\bigskip

(Again, take ground level to be $y=0$, and upward to be the positive direction.)

\bigskip
\Large


\color{A}A: $\sqrt{2g/h}$ \\ \vspace{3pt}
\color{B}B: $h/g$\\ \vspace{3pt}
\color{C}C: $\sqrt{2h/g}$ \\\vspace{3pt}
\color{D}D: $2h/g$ \\\vspace{3pt}
\color{E}E: $\sqrt{h/g}$
}

\frame{\frametitle{\textbf{Another example}}
\Large 
You throw an object up with an initial speed of $v_0$. How much time does it take to reach a height $h$?

\pause

\bigskip

\begin{align*}
x(t) = &\hspace{1.25em}\frac{1}{2}at^2 + v_0 t + x_0 \\
h =& -\frac{1}{2}gt^2 + v_0 t \\
0 =& -\frac{1}{2}gt^2 + v_0 t - h
\end{align*}

\large
\BI
\item{$\rightarrow$ {\color{Red}You need the quadratic formula for this -- nonzero $a$, $v_0$, and position}}
\item{The quadratic formula gives you two answers, but there's clearly only one}
\item{In this case, both roots are positive. Do you want (A) the larger one, or (B) the smaller one?}
\pause
\item{The homework asks you to address this idea.}
\item{Hint: graph position vs. time, and interpret the question graphically}
\item{What is the {\it mathematical} interpretation of the quadratic formula?}
\EI
}

\frame{\frametitle{\textbf{Example problems}}
\Large
\BC I am standing at the bottom of a hole of depth $h$. Someone throws a ball down to me at speed $v_0$. 
How fast is it going when it reaches me?\EC
\bigskip
\bigskip
\bigskip
First: Write equations for $x(t)$ and $v(t)$, putting in the things you know. (Here, take 
ground level as zero, and downward to be positive.)
 
\bigskip

\color{A}A: $x(t) = \hspace{10pt}\frac{1}{2}gt^2$ \tabto{3in} $v(t) = -gt$  \\ \vspace{4pt}
\color{B}B: $x(t) = -\frac{1}{2}gt^2 + v_0 t + h$ \tabto{3in} $v(t) = -gt$  \\ \vspace{4pt}
\color{C}C: $x(t) = \hspace{10pt}\frac{1}{2}gt^2 + v_0 t$ \tabto{3in} $v(t) = \hspace{10pt}gt + v_0$  \\ \vspace{4pt}
\color{D}D: $x(t) = \hspace{10pt}\frac{1}{2}gt^2 - v_0 t$ \tabto{3in} $v(t) = -gt$  \\ \vspace{4pt}
\color{E}E: $x(t) = -\frac{1}{2}gt^2 - v_0 t$ \tabto{3in} $v(t) = -gt$  \\ 

}

\frame{\frametitle{\textbf{Example problems}}
\Large
\BC I am standing at the bottom of a hole of depth $h$. Someone throws a ball down to me at speed $v_0$.
How fast is it going when it reaches me?\EC
\bigskip
\bigskip
\bigskip
Second: Ask a question in terms of your algebraic variables. (Here, take
ground level as zero, and downward to be positive.)

\bigskip

\color{A}A: ``What is $v$ at the time when $x$ is $h$?'' \\
\color{B}B: ``What is $v$ at the time when $x$ is 0?''\\
\color{C}C: ``What is $t$ at the time when $x$ is $h$?''\\
\color{D}D: ``What is $v$ at the time when $x$ is $-h$?''\\
\color{E}E: ``What is $x$ at the time when $v$ is 0?''\\
}


\frame{\frametitle{\textbf{Example problems}}
\Large
\BC \Large I am standing at the bottom of a hole of depth $h$. Someone throws a ball down to me at speed $v_0$.
How fast is it going when it reaches me?\EC
\bigskip
\bigskip
\bigskip
Third: Do the algebra. I'll demonstrate this on the document camera. This requires two steps: first find 
the time, then find $v$. 

}

\frame{\frametitle{\textbf{Homework/recitation questions?}}
	
}


\frame{\frametitle{\textbf{Example problems}}
	
	\large Some students get on the elevator in Bray Hall and press the button to go up. The elevator accelerates upward at 0.5 $\rm m/\rm s^2$ for four seconds before it breaks down. 
	
	\bigskip Thankfully, the emergency brake engages; the emergency brake decelerates the elevator at 3 $\rm m/\rm s^2$ until it comes to a stop. 
	
	\bigskip\pause
	
	One of the students has a crowbar and pries the doors open, hoping to crawl out. How far above ground level is the elevator stuck?
\bigskip\bigskip
	
	\begin{itemize}
		\item Can we use $x(t) = \frac{1}{2}at^2 + v_0 t + x_0$ for this? Why or why not? \pause
		\begin{itemize}
			\item \color{Blue} The acceleration isn't constant, but it's {\bf piecewise constant} \pause
			\item \color{C} Use one copy of the constant-acceleration relations for the first phase \\ and another copy for the second phase\pause
		\item \color{Red}Find the position and velocity at the end of the first phase
		\item \color{Green}Those are the position and velocity at the beginning of the first phase 
		\end{itemize}

\end{itemize}	
	\bigskip\bigskip\pause
	
	How long will it take before the administration sends an email out about the elevator?
	
}





\frame{\frametitle{\textbf{Example problems}}
	\Large
	\BI
	\item{A bucket is being lowered from a cliff at a rate of 10 m/s. You drop a rock off the cliff when the bucket is 10 m beneath the top. 
		How long does it take for the rock to land in the bucket?}
	\EI
	\bigskip
	\bigskip
	
	Same idea as before; see example on the document camera.
}



\frame{

\begin{center}
	Those looking at these slides later: we probably won't get to what's after this. It'll come back later in the semester. This is just here if we have extra time and you don't have questions!
\end{center}

}


\frame{\frametitle{\textbf{Rotational kinematics}}
\Large
\BI
\item{Linear motion: care about position as a function of time}
\item{Rotational motion: care about {\color{Red}angle} as a function of time}
\item{\bf Everything we just did translates to rotational kinematics exactly!}
\EI
}

\frame{\frametitle{\textbf{Position, velocity, and acceleration}}
\begin{columns}
\column{0.125\textwidth}
\centerline{\Large Position}
\column{0.2\textwidth}
\small
{\color{Red}
\centerline{(derivative of)}
\centerline{rate of change of}
\centerline{$\xleftarrow{\makebox[\textwidth]{}}$}}
{\color{Green}
\color{Green}\centerline{$\xrightarrow{\makebox[\textwidth]{}}$}}
\color{Green}\centerline{area under the curve of}
\color{Green}\centerline{(integral of)}
\column{0.125\textwidth}
\centerline{\Large Velocity}
\pause
\column{0.2\textwidth}
\small
{\color{Red}
\centerline{(derivative of)}
\centerline{rate of change of}
\centerline{$\xleftarrow{\makebox[\textwidth]{}}$}}
{\color{Green}
\color{Green}\centerline{$\xrightarrow{\makebox[\textwidth]{}}$}}
\color{Green}\centerline{area under the curve of}
\color{Green}\centerline{(integral of)}
\column{0.15\textwidth}
\centerline{\Large Acceleration}
\end{columns}
}

\frame{\frametitle{\textbf{Angle, angular velocity, and angular acceleration}}
\begin{columns}
\column{0.125\textwidth}
\centerline{\Large Angle}
\column{0.2\textwidth}
\small
{\color{Red}
\centerline{(derivative of)}
\centerline{rate of change of}
\centerline{$\xleftarrow{\makebox[\textwidth]{}}$}}
{\color{Green}
\color{Green}\centerline{$\xrightarrow{\makebox[\textwidth]{}}$}}
\color{Green}\centerline{area under the curve of}
\color{Green}\centerline{(integral of)}
\column{0.125\textwidth}
\begin{center}
\Large Angular velocity ($\omega$)
\end{center}
\pause
\column{0.2\textwidth}
\small
{\color{Red}
\centerline{(derivative of)}
\centerline{rate of change of}
\centerline{$\xleftarrow{\makebox[\textwidth]{}}$}}
{\color{Green}
\color{Green}\centerline{$\xrightarrow{\makebox[\textwidth]{}}$}}
\color{Green}\centerline{area under the curve of}
\color{Green}\centerline{(integral of)}
\column{0.15\textwidth}
\begin{center}
\large Angular acceleration ($\alpha$)
\end{center}
\end{columns}

\Large
\bigskip
\bigskip
\bigskip

\pause

$$ x(t) = x_0 + v_0 t + \frac{1}{2}at^2 $$

$$ \theta(t) = \theta_0 + \omega_0 t + \frac{1}{2}\alpha t^2 $$

\pause

\bigskip

$\rightarrow$ Angular kinematics works in exactly the same way as translational kinematics!

}

\frame{\frametitle{\textbf{Angle, angular velocity, and angular acceleration}}
\large
\BI
\item{Angle $\theta$ -- the angle through which something has turned.}
\item{Measured in revolutions, radians, degrees...}
\bigskip
\bigskip
\bigskip
\item{Angular velocity $\omega$ (``omega'', not ``dubya'') -- the rate at which something is turning} 
\item{Measured in revolutions per second, radians per second, degrees per second...}
\bigskip
\bigskip
\bigskip
\item{Angular acceleration $\alpha$ (``alpha'', not ``fish'') -- the rate at which something's rate of turning is changing} 
\item{Measured in $\frac{\rm rev}{\rm s^2}$, $\frac{\rm rad}{\rm s^2}$, $\frac{\rm deg}{\rm s^2}$...}
\EI
}
\end{document}

