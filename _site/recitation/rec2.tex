\documentclass[12pt]{article}
\setlength\parindent{0pt}
\usepackage{fullpage}
\usepackage{amsmath}
\usepackage{array}
\usepackage{hyperref}
\usepackage{graphicx}
\setlength{\parskip}{4mm}
\def\LL{\left\langle}   % left angle bracket
\def\RR{\right\rangle}  % right angle bracket
\def\LP{\left(}         % left parenthesis
\def\RP{\right)}        % right parenthesis
\def\LB{\left\{}        % left curly bracket
\def\RB{\right\}}       % right curly bracket
\def\PAR#1#2{ {{\partial #1}\over{\partial #2}} }
\def\PARTWO#1#2{ {{\partial^2 #1}\over{\partial #2}^2} }
\def\PARTWOMIX#1#2#3{ {{\partial^2 #1}\over{\partial #2 \partial #3}} }
\newcommand{\vs}[1]{\vspace{#1}}
\newcommand{\BC}{\begin{center}}
\newcommand{\EC}{\end{center}}
\newcommand{\BI}{\begin{itemize}}
\newcommand{\EI}{\end{itemize}}
\newcommand{\BE}{\begin{enumerate}}
\newcommand{\EE}{\end{enumerate}}
\newcommand{\BNE}{\begin{equation}}
\newcommand{\ENE}{\end{equation}}
\newcommand{\BEA}{\begin{eqnarray}}
\newcommand{\EEA}{\nonumber\end{eqnarray}}
\newcommand{\EL}{\nonumber\\}
\newcommand{\la}[1]{\label{#1}}
\newcommand{\ie}{{\em i.e.\ }}
\newcommand{\eg}{{\em e.\,g.\ }}
\newcommand{\cf}{cf.\ }
\newcommand{\etc}{etc.\ }
\newcommand{\Tr}{{\rm tr}}
\newcommand{\etal}{{\it et al.}}
\newcommand{\OL}[1]{\overline{#1}\ } % overline
\newcommand{\OLL}[1]{\overline{\overline{#1}}\ } % double overline
\newcommand{\OON}{\frac{1}{N}} % "one over N"
\newcommand{\OOX}[1]{\frac{1}{#1}} % "one over X"


\pagenumbering{gobble}
\begin{document}
\Large
\centerline{\sc{Recitation Problems}}
\large
\BC
\sc
Friday, January 20
\EC
\normalsize


In class, you learned the kinematics relations $x(t) = x_0 + vt + \frac{1}{2}at^2$ and $v(t) = v_0 + at$. 
These relations aren't generally true, however; they are true only in a specific case. What must be true 
for these relations to apply?

\vs{1.5in}


Two runners start at opposite ends of a $L=100m$ long soccer pitch and sprint toward each other. One runs 
at 8 m/s and starts on the east side, while the other runs at 6 m/s and starts at the west side. The slower 
runner has a $\tau=2$ s head start. You'd like to know where on the field they meet.

%\footnote{This means ``The runner has a head start of 2 seconds, and I 
%am suggesting that you use the letter $\tau$ for this quantity.'' This is the lowercase form of the Greek 
%letter tau, which is usually used to mean ``a specific time that appears in the problem''.} head start. You'd like to know where they'll meet.

\BE

\item As with all physics problems, you should work this problem using variables, substituting numbers only
as the very last step. In physics, we often use subscripts to give more information about a variable. 
For instance, $x_1(t)$ means ``the position of object 1 as a function of time''. You can even combine them: ``$x_{0,1}$''
means ``the initial position of object 1''. Choose variables to use for:

\BI
\item the initial position of each runner
\item the velocity of each runner
\item the length of the head start (I suggest $\tau$ for this)
\EI

\item Draw position vs. time graphs for both runners on a single set of axes.

\vs{2in}

\item Write position vs. time equations (of the form $x = x_0 + vt$) for both runners, using the algebraic
variables you chose in part (1) instead of the numbers. 

\vs{1.5in}

\item I asked you to tell me where the runners meet. Write down an English sentence using the physical quantities in the problem (positions, times, etc.)
that can help you answer this question.

\vs{1in}

\item Perform the algebra indicated by the sentence and equations you wrote.

\vs{2in}

\item Finally, substitute in the numerical values to find where they meet. How do you take into account the fact
that they run in opposite directions?

\EE

\vs{2in}

\newpage


\newpage


A car is traveling at 30 m/s and applies its brakes to slow down to 10 m/s. If it is able to decelerate at 5 
$\rm m/\rm s^2$, how far does it travel during the braking period? As always, introduce these numbers as the
very last step, along with their units; do the first steps in terms of variables only.

\BE
\item Write expressions for the car's position and velocity as a function of time.

\vs{1in}

\item How can you translate the question ``How far does it travel during the braking period?'' into an 
algebraic statement? Write a sentence like the ones you have written before.

\vs{1.5in}

\item What intermediate quantity must you find before you find the distance traveled? Find it.

\vs{2in}

\item Finally, how far does the car travel during the braking period?

\vs{2in}

\EE


\newpage

{\bf Challenge question:} Consider a basketball bouncing on the floor.

Draw position vs. time, velocity vs. time, and acceleration vs. time graphs for the ball.
Look at your graphs carefully and make sure they are self-consistent:

\BI 
\item Regions of constant acceleration should correspond to places where the velocity graph is a 
straight line (are there any?)

\item Regions of constant velocity should correspond to places where the position graph is a straight 
line (are there any?)

\item Places where the position graph is flat should correspond to v=0
\EI

Remember, the slope of the position graph is the value of the velocity graph; the slope of the 
velocity graph is the value of the acceleration graph

How would these graphs be different for a bouncing bowling ball?

{\bf Note:} This is a bit tricky! As with all problems in recitation, work together with your peers and ask 
your TA/coach for guidance if you have questions.

\end{document}
