\documentclass[12pt]{article}
\setlength\parindent{0pt}
\usepackage{fullpage}
\usepackage{amsmath}
\usepackage{graphicx}
\setlength{\parskip}{4mm}
\def\LL{\left\langle}   % left angle bracket
\def\RR{\right\rangle}  % right angle bracket
\def\LP{\left(}         % left parenthesis
\def\RP{\right)}        % right parenthesis
\def\LB{\left\{}        % left curly bracket
\def\RB{\right\}}       % right curly bracket
\def\PAR#1#2{ {{\partial #1}\over{\partial #2}} }
\def\PARTWO#1#2{ {{\partial^2 #1}\over{\partial #2}^2} }
\def\PARTWOMIX#1#2#3{ {{\partial^2 #1}\over{\partial #2 \partial #3}} }
\newcommand{\BE}{\begin{displaymath}}
\newcommand{\EE}{\end{displaymath}}
\newcommand{\BNE}{\begin{equation}}
\newcommand{\ENE}{\end{equation}}
\newcommand{\BEA}{\begin{eqnarray}}
\newcommand{\EEA}{\nonumber\end{eqnarray}}
\newcommand{\EL}{\nonumber\\}
\newcommand{\la}[1]{\label{#1}}
\newcommand{\ie}{{\em i.e.\ }}
\newcommand{\eg}{{\em e.\,g.\ }}
\newcommand{\cf}{cf.\ }
\newcommand{\etc}{etc.\ }
\newcommand{\Tr}{{\rm tr}}
\newcommand{\etal}{{\it et al.}}
\newcommand{\OL}[1]{\overline{#1}\ } % overline
\newcommand{\OLL}[1]{\overline{\overline{#1}}\ } % double overline
\newcommand{\OON}{\frac{1}{N}} % "one over N"
\newcommand{\OOX}[1]{\frac{1}{#1}} % "one over X"



\begin{document}
\pagenumbering{gobble}
\Large
\centerline{\sc{Recitation Questions}}
\normalsize
\centerline{\sc{22 and 24 February}}

\medskip

\centerline{\Large Question 1: a curvy road}

Suppose that a car is driving around a flat highway curve with a radius of curvature of $r=100$ meters (that is, it is a segment of a circle whose
radius is 100 m), and that
the coefficient of friction between the car's wheels and the pavement is $\mu_s=0.8$.

a) What force is responsible for the centripetal acceleration of the car, bringing it around the curve?

\vspace{1in}


b) Draw a force diagram for the car.

\vspace{3in}

c) What is the fastest that the car can drive around the curve? How would this change if the highway
was covered in snow?

\vspace{2in}

\newpage

\centerline{\Large Question 2: a banked, curvy road}

As you know, highway curves are ``banked'' inward, so that gravity assists the car's traction in 
carrying it around the curve. Suppose another highway curve has a radius of curvature of 500 meters. 
It is banked so that traffic moving at 30 m/s can travel around the curve without
needing any help from friction.

a) Draw a force diagram for a car traveling around this curve at a constant speed. Draw the
diagram so that you are looking at the rear of the car. Hint: Do not tilt your coordinate axes for
this problem!

\vspace{3in}

b) What is the acceleration of the car in the x-direction? What about the y-direction?

\vspace{1in}

c) Write down two copies of Newton's second law in the $x-$ and $y-$directions.

\vspace{2in}

d) Solve the resulting system of two equations to determine the banking angle of the curve.

\vspace{3in}

e) If the car is driving faster than 30 m/s, which way will traction point on your force diagram?
What if it is driving slower than 30 m/s? 

\newpage

\centerline{\Large Question 3: variation of apparent weight with latitude}

\medskip

\centerline{\it For this problem, use $g=9.8100\, \rm m/\rm s^2$, and carry all calculations to five significant digits.}

a) Assume that the Earth is a sphere of radius 6400.0 km. What is the acceleration of a person
standing on the Equator? (Hint: What is the angular velocity of the Earth?)

\vspace{2in}

b) Suppose a person with a mass of 100 kg stands on a scale. (Remember, scales measure the
normal force that they exert.) What will the scale read at the South Pole?

\vspace{2in}

c) What will the scale read at the Equator?

\vspace{2in}
\newpage

d) This problem shows that your apparent weight depends on your location on Earth. 
Does it make sense to define $g$ as $F_g/m$ 
(the strength of the gravitational force divided by an object's mass) or
$F_N/m$ (the strength of the normal force, and thus the scale reading, divided by mass)?

\vspace{3in}

e) Is this distinction likely to be relevant to the sort of engineering or science you will do during your
career? (The answer will depend on what you will do, of course!)


\newpage

\centerline{\Large Question 4: universal gravitation and the Sun's mass}

In this problem, you will compute the mass of the Sun. The Earth’s orbit is very nearly circular,
and the earth is 150 million km from the Sun.

a) What is the angular velocity of the Earth in its orbit?

\vspace{2in}

b) What is the tangential velocity of the Earth? 

\vspace{1in}

c) What is the radial acceleration of the Earth?

\vspace{1in}

d) What is the mass of the Sun?

\newpage

\centerline{\Large Question 5: universal gravitation and the Sun's mass}

Astronauts in orbit around the Earth are not ``so far away that they don't feel Earth's gravity'';
actually, they’re quite close to the surface. However, we’ve all seen the videos of astronauts drifting
around ``weightlessly'' in the International Space Station.

a) Explain how an astronaut can be under the influence of Earth's gravity, and yet exert no normal
force on the surface of the spacecraft she is standing in.

\vspace{2in}

b) Draw a force diagram for the astronaut floating in the middle of the Space Station, not touching
any of the walls or floor. How do you reconcile your diagram with the fact that the astronaut
doesn't seem to fall?

\vspace{2in}

c) Is this astronaut truly ``weightless''? What does ``weightless'' mean?

\newpage

\centerline{\Large Question 6: geostationary orbit}

It is sometimes useful to place satellites in orbit so that they stay in a fixed position relative to the
Earth; that is, their orbits are synchronized with the Earth's rotation so that a satellite might stay
above the same point on Earth’s surface all the time.

What is the altitude of such an orbit? Note that it is high enough that you need to use $F_g=\frac{GMm}{r^2}$
rather than just $F_g = mg$.

{\sc Hint 1:} If this orbit is synchronized with Earth's rotation, then you should be able to figure out its
angular velocity.

{\sc Hint 2:} If you do this problem as we have guided you, by waiting to substitute numbers in until the 
very end, you will arrive at an expression relating the radius $R$ of a circular orbit with the mass $M$ of the 
planet being orbited and the angular velocity $\omega$ of the orbit. This question will be on HW5, and is related
to the derivation of Kepler's third law that you will do there.

\end{document}
