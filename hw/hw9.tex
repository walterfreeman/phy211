\documentclass[12pt]{article}
\setlength\parindent{0pt}
\usepackage{fullpage}
\usepackage{graphicx}
\usepackage{amsmath}
\setlength{\parskip}{4mm}
\def\LL{\left\langle}   % left angle bracket
\def\RR{\right\rangle}  % right angle bracket
\def\LP{\left(}         % left parenthesis
\def\RP{\right)}        % right parenthesis
\def\LB{\left\{}        % left curly bracket
\def\RB{\right\}}       % right curly bracket
\def\PAR#1#2{ {{\partial #1}\over{\partial #2}} }
\def\PARTWO#1#2{ {{\partial^2 #1}\over{\partial #2}^2} }
\def\PARTWOMIX#1#2#3{ {{\partial^2 #1}\over{\partial #2 \partial #3}} }
\newcommand{\BE}{\begin{displaymath}}
\newcommand{\EE}{\end{displaymath}}
\newcommand{\BNE}{\begin{equation}}
\newcommand{\ENE}{\end{equation}}
\newcommand{\BEA}{\begin{eqnarray}}
\newcommand{\EEA}{\nonumber\end{eqnarray}}
\newcommand{\EL}{\nonumber\\}
\newcommand{\la}[1]{\label{#1}}
\newcommand{\ie}{{\em i.e.\ }}
\newcommand{\eg}{{\em e.\,g.\ }}
\newcommand{\cf}{cf.\ }
\newcommand{\etc}{etc.\ }
\newcommand{\Tr}{{\rm tr}}
\newcommand{\etal}{{\it et al.}}
\newcommand{\OL}[1]{\overline{#1}\ } % overline
\newcommand{\OLL}[1]{\overline{\overline{#1}}\ } % double overline
\newcommand{\OON}{\frac{1}{N}} % "one over N"
\newcommand{\OOX}[1]{\frac{1}{#1}} % "one over X"



\begin{document}
\Large
\centerline{\sc{Homework 9}}
\normalsize
\centerline{\sc{Due Tuesday, May 2, to your TA's mailbox}}


\begin{enumerate}

\item Suppose that the largest pipe in a pipe organ is 16 feet long, and the smallest pipe is 0.25 feet long. The speed of sound in air is 1125 feet per second. What 
are the fundamental frequencies of these two pipes?\footnote{Organists, even outside the US, still measure pipes in feet.} These figures are accurate (I believe) for the
organ in Setnor Auditorium at Syracuse University.

\item You made observations of the frequencies of the first six resonant modes of the flame-tube in class. Its resonant frequencies do not agree with those that you will
calculate in the previous problem, however, because the tube was filled with methane, not air. Using the observations you made, estimate the speed of sound in methane.
({\bf Note:} The tube is 91 cm long. 

\item The wave speed in a stretched string is given by $c=\sqrt{\frac{T}{\mu}}$, where $T$ is the tension the string is under, and $\mu$ is the linear mass 
density of the string in kilograms per meter. 

Suppose that the vibrating part of the highest-frequency string on a guitar has a length of 65 cm and a linear mass density of 0.3 grams per meter. This string
is usually tuned to a fundamental frequency of 293 Hz. What tension must the body of the guitar apply to the string in order to do this?

\item Draw pictures of the first six resonant modes of a vibrating string. (For a guide to the first four, see the slides from 20 or 25 April.) Using the parameters
of the previous problem (fundamental frequency of 293 Hz and vibrating length of 65 cm), calculate the frequencies and wavelengths of these six modes. {\bf Note: 
} You should use an entire sheet of paper for these drawings.

\item The other strings in a guitar have the same length and about the same tension. Why do they have lower fundamental frequencies? (Hint: What is different 
about the strings? Look up a photograph of a guitar, and remember that $c=\sqrt{\frac{T}{\mu}}$.) 


\end{enumerate}
\end{document}
