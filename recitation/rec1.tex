\documentclass[12pt]{article}
\setlength\parindent{0pt}
\usepackage{fullpage}
\usepackage{amsmath}
\usepackage{array}
\usepackage{hyperref}
\usepackage{graphicx}
\setlength{\parskip}{4mm}
\def\LL{\left\langle}   % left angle bracket
\def\RR{\right\rangle}  % right angle bracket
\def\LP{\left(}         % left parenthesis
\def\RP{\right)}        % right parenthesis
\def\LB{\left\{}        % left curly bracket
\def\RB{\right\}}       % right curly bracket
\def\PAR#1#2{ {{\partial #1}\over{\partial #2}} }
\def\PARTWO#1#2{ {{\partial^2 #1}\over{\partial #2}^2} }
\def\PARTWOMIX#1#2#3{ {{\partial^2 #1}\over{\partial #2 \partial #3}} }
\newcommand{\vs}[1]{\vspace{#1}}
\newcommand{\BC}{\begin{center}}
\newcommand{\EC}{\end{center}}
\newcommand{\BI}{\begin{itemize}}
\newcommand{\EI}{\end{itemize}}
\newcommand{\BE}{\begin{enumerate}}
\newcommand{\EE}{\end{enumerate}}
\newcommand{\BNE}{\begin{equation}}
\newcommand{\ENE}{\end{equation}}
\newcommand{\BEA}{\begin{eqnarray}}
\newcommand{\EEA}{\nonumber\end{eqnarray}}
\newcommand{\EL}{\nonumber\\}
\newcommand{\la}[1]{\label{#1}}
\newcommand{\ie}{{\em i.e.\ }}
\newcommand{\eg}{{\em e.\,g.\ }}
\newcommand{\cf}{cf.\ }
\newcommand{\etc}{etc.\ }
\newcommand{\Tr}{{\rm tr}}
\newcommand{\etal}{{\it et al.}}
\newcommand{\OL}[1]{\overline{#1}\ } % overline
\newcommand{\OLL}[1]{\overline{\overline{#1}}\ } % double overline
\newcommand{\OON}{\frac{1}{N}} % "one over N"
\newcommand{\OOX}[1]{\frac{1}{#1}} % "one over X"


\pagenumbering{gobble}
\begin{document}
\Large
\centerline{\sc{Recitation Problems}}
\large
\BC
\sc
Wednesday, January 18
\EC
\normalsize


\BC Fill out the following for your group. One member of your group should turn this page in at the end
of recitation, so your TA knows who you are. \EC


\BC

\hspace{-1in}

\begin{tabular}{|m{4cm}|m{4cm}|m{4cm}|m{4cm}|}
\hline
Name & & &  \\[1.5cm] \hline
Academic year  & & & \\[1.5cm] \hline
Major & & &  \\[1.5cm] \hline
High school physics course (AP, IB, regular, or none)?  & & & \\[1.5cm]\hline
Highest math class completed? & &  & \\[1.5cm]\hline

\end{tabular}
\EC
\newpage

\BE

\item One cubic centimeter of water has a mass of one gram. What is the mass of a cubic meter of water?

This problem is not that difficult, but I want you to establish good habits in working with units right 
off the bat. In this problem, {\it carry the units along with every dimensionful quantity}. This may seem
nitpicky, but it will avoid so many errors later in this course. Show your TA your work when you are done.

\vs{2in}



\newpage


\item The only interstate signed in metric in the US is Interstate 19, running between Nogales on the Mexican
border (kilometer marker 0) and Tucson, AZ (kilometer marker 101).

\BE

\item A car leaves Nogales driving north to Tucson at 2PM. How fast must the driver travel to reach 
Tucson by 3PM? What about 3:10PM?

\vs{1.5in}

\item Draw on one graph the car's position as a function of time in both cases. You may measure distance 
and time in any units you choose, but label your axes!

\vs{3in}

\item Suppose instead that the car drives at a constant 120 km/hr. How many hours will it take to reach 
Tucson? As you solve this problem, remember to {\it carry the units along with every dimensionful quantity}.
The distance from Nogales to Tucson is not 101, and the speed is not 120; 
they are ``101 km'' and ``120 km/hr'', respectively.
\EE
\newpage

\item A bus leaves Tucson headed for Mexico. One of the passengers realizes that he forgot his passport, and 
calls his friend to bring it to him. She finds it and sets off after him, entering I-19 ten minutes after the 
bus. If the bus drives at 80 km/hr, how fast must she drive to catch the bus before the Mexican border?

\BE

\item On a single set of axes, sketch graphs of the position of the bus and the position of our forgetful 
passenger's friend if she drives just fast enough to catch him as he gets to Mexico. (You can make a 
rough sketch before doing any mathematics; this is a good way to get an idea of how to attack the 
problem.)

\vs{3in}

\item What choice did you have to make to do this problem? Would your answer to the previous question 
change if the "zero-km" marker were on the Tucson end of the highway?

\vs{1in}

\EE
\EE

\end{document}
