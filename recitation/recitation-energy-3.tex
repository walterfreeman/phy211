\documentclass[12pt]{article}
\setlength\parindent{0pt}
\usepackage{fullpage}
\usepackage[margin=0.5in]{geometry}
\usepackage{amsmath}
\usepackage{graphicx}
\setlength{\parskip}{4mm}
\def\LL{\left\langle}   % left angle bracket
\def\RR{\right\rangle}  % right angle bracket
\def\LP{\left(}         % left parenthesis
\def\RP{\right)}        % right parenthesis
\def\LB{\left\{}        % left curly bracket
\def\RB{\right\}}       % right curly bracket
\def\PAR#1#2{ {{\partial #1}\over{\partial #2}} }
\def\PARTWO#1#2{ {{\partial^2 #1}\over{\partial #2}^2} }
\def\PARTWOMIX#1#2#3{ {{\partial^2 #1}\over{\partial #2 \partial #3}} }
\newcommand{\BI}{\begin{itemize}}
\newcommand{\EI}{\end{itemize}}
\newcommand{\BE}{\begin{displaymath}}
\newcommand{\EE}{\end{displaymath}}
\newcommand{\BNE}{\begin{equation}}
\newcommand{\ENE}{\end{equation}}
\newcommand{\BEA}{\begin{eqnarray}}
\newcommand{\EEA}{\nonumber\end{eqnarray}}
\newcommand{\EL}{\nonumber\\}
\newcommand{\la}[1]{\label{#1}}
\newcommand{\ie}{{\em i.e.\ }}
\newcommand{\eg}{{\em e.\,g.\ }}
\newcommand{\cf}{cf.\ }
\newcommand{\etc}{etc.\ }
\newcommand{\Tr}{{\rm tr}}
\newcommand{\etal}{{\it et al.}}
\newcommand{\OL}[1]{\overline{#1}\ } % overline
\newcommand{\OLL}[1]{\overline{\overline{#1}}\ } % double overline
\newcommand{\OON}{\frac{1}{N}} % "one over N"
\newcommand{\OOX}[1]{\frac{1}{#1}} % "one over X"



\begin{document}
\Large
\centerline{\sc{Recitation Questions}}
\normalsize
\centerline{\sc{March 27}}

\begin{enumerate}


\item A mad scientist has built a rocket-powered sled, and wants to show off by using it and a ramp made out of
snow to jump through the air, much like a skier. She sets her sled a distance $d$ in front of the ramp
and fires the rocket. The rocket accelerates forward toward the ramp, ascends the ramp, then flies through
the air before landing back on the ground. She turns off her rocket when she reaches the ramp.

Suppose that:

\begin{itemize}
\item The sled and rider together have mass $m$
\item The coefficient of friction between the snow and the sled is $\mu_k$
\item The thrust force from the rocket is $F_T$
\item The overall (diagonal) length of the ramp is $L$
\item The ramp is inclined at an angle $\theta$ above the horizontal
\end{itemize}

\begin{enumerate}

\item Draw a cartoon of the situation, labeling interesting things (i.e. the trigonometry related to the ramp).

\vspace{2in}

\item Using energy methods, calculate how fast she is traveling when she leaves the top of the ramp.

\vspace{2in}
\newpage
\item Using energy methods, calculate how fast she is traveling when she lands back on the ground. Think 
carefully about what your ``initial'' and ``final'' states are; there's a hard way and an easy way to do this.

\vspace{3in}

\item Can you use energy methods to figure out the horizontal distance she travels before landing back
on the ground? If so, write down an equation you can solve for that distance. If not, explain what other
techniques you need to use.
\end{enumerate}
\newpage



\item A pinball machine uses a spring-loaded launcher to launch a solid steel ball up a ramp. The ball rolls up the ramp without slipping. (Remember that if something rolls without slipping, its translational
velocity and angular velocity are related by $v=\omega r$.) You'd like to find the speed of the ball at the top of the ramp. Suppose that:

\BI
\item The ramp is inclined at an angle $\theta$ above the horizontal, and has a total length $L$
\item The ball has mass $m$ and radius $r$, and thus has moment of inertia $I=\frac{2}{5}mr^2$.
\item The spring has spring constant $k$, and is compressed a distance $d$
\EI

\begin{enumerate}

\item Draw a cartoon of the situation, showing clearly the initial and final states.

\vspace{3in}

\item Write down an expression of conservation of energy for the ball. Underneath each term, label it -- something like ``final rotational kinetic energy'' or ``initial spring potential energy''.

\vspace{3in}
\newpage
\item Show that the speed of the ball at the top of the ramp is $$v_f = \sqrt {\frac{10}{7} \frac {\frac{1}{2}kd^2 - mgL \sin \theta}{m}}\hspace{0.2in}\text{which is equal to}\hspace{0.2in} \sqrt{\frac{5}{7} kd^2 - \frac{10}{7}gL \sin \theta}.$$
\vspace{4in}
\end{enumerate}

\item A ball of mass $m$ is connected to one end of a rubber band and swung in a circle. The rubber band has spring constant $k$, and
its unstretched length is $r_0$.

If it is swung at an angular velocity $\omega$, to what length will it stretch the rubber band?



\end{enumerate}
\end{document}
