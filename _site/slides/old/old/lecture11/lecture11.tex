% DPF 09 talk on strangeness in nucleon

\documentclass[10pt]{beamer}
\usepackage{amsmath}
\usepackage{mathtools}
%\documentclass[12pt]{beamerthemeSam.sty}
\usepackage{epsf}
%\usepackage{pstricks}
%\usepackage[orientation=portrait,size=A4]{beamerposter}
\geometry{paperwidth=160mm,paperheight=120mm}
%DT favorite definitions
\def\LL{\left\langle}	% left angle bracket
\def\RR{\right\rangle}	% right angle bracket
\def\LP{\left(}		% left parenthesis
\def\RP{\right)}	% right parenthesis
\def\LB{\left\{}	% left curly bracket
\def\RB{\right\}}	% right curly bracket
\def\PAR#1#2{ {{\partial #1}\over{\partial #2}} }
\def\PARTWO#1#2{ {{\partial^2 #1}\over{\partial #2}^2} }
\def\PARTWOMIX#1#2#3{ {{\partial^2 #1}\over{\partial #2 \partial #3}} }

\def\rightpartial{{\overrightarrow\partial}}
\def\leftpartial{{\overleftarrow\partial}}
\def\diffpartial{\buildrel\leftrightarrow\over\partial}

\def\BI{\begin{itemize}}
\def\EI{\end{itemize}}
\def\BE{\begin{displaymath}}
\def\EE{\end{displaymath}}
\def\BEA{\begin{eqnarray*}}
\def\EEA{\end{eqnarray*}}
\def\BNEA{\begin{eqnarray}}
\def\ENEA{\end{eqnarray}}
\def\EL{\nonumber\\}


\newcommand{\map}[1]{\frame{\frametitle{\textbf{Course map}}
\centerline{\includegraphics[height=0.86\paperheight]{../../map/#1.png}}}}
\newcommand{\wmap}[1]{\frame{\frametitle{\textbf{Course map}}
\centerline{\includegraphics[width=0.96\paperwidth]{../../map/#1.png}}}}

\newcommand{\etal}{{\it et al.}}
\newcommand{\gbeta}{6/g^2}
\newcommand{\la}[1]{\label{#1}}
\newcommand{\ie}{{\em i.e.\ }}
\newcommand{\eg}{{\em e.\,g.\ }}
\newcommand{\cf}{cf.\ }
\newcommand{\etc}{etc.\ }
\newcommand{\atantwo}{{\rm atan2}}
\newcommand{\Tr}{{\rm Tr}}
\newcommand{\dt}{\Delta t}
\newcommand{\op}{{\cal O}}
\newcommand{\msbar}{{\overline{\rm MS}}}
\def\chpt{\raise0.4ex\hbox{$\chi$}PT}
\def\schpt{S\raise0.4ex\hbox{$\chi$}PT}
\def\MeV{{\rm Me\!V}}
\def\GeV{{\rm Ge\!V}}

%AB: my color definitions
%\definecolor{mygarnet}{rgb}{0.445,0.184,0.215}
%\definecolor{mygold}{rgb}{0.848,0.848,0.098}
%\definecolor{myg2g}{rgb}{0.647,0.316,0.157}
\definecolor{abtitlecolor}{rgb}{0.0,0.255,0.494}
\definecolor{absecondarycolor}{rgb}{0.0,0.416,0.804}
\definecolor{abprimarycolor}{rgb}{1.0,0.686,0.0}
\definecolor{Red}           {cmyk}{0,1,1,0}
\definecolor{Grey}           {cmyk}{.7,.7,.7,0}
\definecolor{Blue}          {cmyk}{1,1,0,0}
\definecolor{Green}         {cmyk}{1,0,1,0}
\definecolor{Brown}         {cmyk}{0,0.81,1,0.60}
\definecolor{Black}         {cmyk}{0,0,0,1}

\usetheme{Madrid}


%AB: redefinition of beamer colors
%\setbeamercolor{palette tertiary}{fg=white,bg=mygarnet}
%\setbeamercolor{palette secondary}{fg=white,bg=myg2g}
%\setbeamercolor{palette primary}{fg=black,bg=mygold}
\setbeamercolor{title}{fg=abtitlecolor}
\setbeamercolor{frametitle}{fg=abtitlecolor}
\setbeamercolor{palette tertiary}{fg=white,bg=abtitlecolor}
\setbeamercolor{palette secondary}{fg=white,bg=absecondarycolor}
\setbeamercolor{palette primary}{fg=black,bg=abprimarycolor}
\setbeamercolor{structure}{fg=abtitlecolor}

\setbeamerfont{section in toc}{series=\bfseries}

%AB: remove navigation icons
\beamertemplatenavigationsymbolsempty
\title[Exam 2 Review]{
  \textbf {Exam 2 Review}\\
%\centerline{}
%\centering
%\vspace{-0.0in}
%\includegraphics[width=0.3\textwidth]{propvalues_0093.pdf}
%\vspace{-0.3in}\\
%\label{intrograph}
}

\author[W. Freeman] {Physics 211\\Syracuse University, Physics 211 Spring 2015\\Walter Freeman}

\date{\today}

\begin{document}

\frame{\titlepage}

\frame{\frametitle{\textbf{Announcements}}
\BI
\large
\item{Exam 2 next Tuesday}
\item{Homework 5 due Friday}
\item{No recitation next Wednesday (TA's will be grading)}
\item{Recitation next Friday: going over exams}
\item{Exam regrades still in progress (I've been teaching nonstop, sorry!)}
\EI

}

\frame{\frametitle{\textbf{Exam details}}
  \BI
  \large
\item{4 questions (+ possible extra credit)}
\item{No detailed graphs}
\item{More symbolic problems (no numbers) than before}
\item{Please arrive a bit early if at all possible}
\item{\bf{Please know your recitation section number} to get your exam back on time}
\item{Bring a calculator and your pencil}
  \EI
}

\frame{\frametitle{\textbf{Extra exam preparation}}
  \large
  \BI
\item{Practice exam posted}
\BI
\item{Solutions posted Friday after recitations}
  \EI
\item{Review session {\color{Red}today: 3:30-6:00, Physics Building B126}}
\item{Review session {\color{Red}tomorrow: 10AM-4PM, Heroy Geology Building room 013}}
\item{I'm available to answer questions by email, Facebook chat, etc. all weekend}
  \BI
\item{Lots of people have been Facebook-messaging me cellphone pictures of work with questions; please do this!}
  \EI
\item{There is a huge amount of help available to you: use it!}
  \EI
}

\frame{\frametitle{\textbf{Midterm evaluations}}
  \BI
  \large
\item{There are midterm course evaluations at Syracuse -- you should have gotten emails about this}
\item{Go to http://aaf-ratings.syr.edu/ and enter a passcode you were sent by mail}
\item{I take your feedback extremely seriously}
\item{At the end of class if you like I can give you time to complete this if you'd like}
  \EI
}

\frame{\frametitle{\textbf{Review: overview}}
  \large
  \BI
\item{Newton's second law: $\sum \vec F = m \vec a$}
  \BI
  \normalsize
\item{Forces (left hand side) cause accelerations (right hand side)}
\item{Acceleration is not a force; it {\it results} from forces}
  \EI

  \bigskip
  \bigskip
  \bigskip

\item{Newton's third law: Forces come in pairs. If A pushes on B, B pushes back on A}


  \bigskip
  \bigskip
  \bigskip

\item{Forces are things you can feel:}
  \BI
  \normalsize
\item{Normal forces: one thing pushes on another}
\item{Gravity}
\item{Tension: a rope pulls on something}
\item{Friction: opposes things sliding}
\item{\color{Red}Acceleration is not a force: forces {\it cause} acceleration}
\item{``Centripetal force'' is not a separate force: it describes one of the above}
  \EI
  \EI
}

\frame{\frametitle{\textbf{A few things about these forces: gravity}}
  \large
  \BI
\item{On Earth: always acts downward with $F_g = mg$}
\item{The acceleration of an object is {\it only} $g$ if there are no other forces}
  
  \bigskip
  
  \centerline{$-mg = ma_y$ : only if $\sum F_y = -mg$!}
  \pause

  \bigskip
  \bigskip
\item{This is only true on Earth. Elsewhere: all objects attract each other}
\bigskip

\centerline{$F_g = \frac{Gm_1 m_2}{r^2}$}
\item{$m_1$ and $m_2$ are the masses of the two objects; $r$ is the distance between their centers.}
\item{$G = 6.67 \times 10^{-11} \rm N \cdot \rm m^2/\rm{kg}^2$}
\item{This distance is measured between their centers (for planets)}
\item{On Earth: $F_g = m_1 g = \frac{GM_e m_1}{r_e^2}$, so $g = \frac{GM_e}{r_e^2}$}
  \EI
}

\frame{\frametitle{\textbf{A few things about these forces: tension}}
\large
\BI
\item{Just the force exerted by a rope}
\item{Always goes in the direction of the rope, and is the same throughout}
\item{Can only pull; can never push}
\item{Force is the same on both ends (Newton's 3rd law)}
  \EI
}

\frame{\frametitle{\textbf{A few things about these forces: normal forces}}
\large
\BI
\item{Stops two things from moving through each other}
\item{Always directed normal (perpendicular) to a surface}
\item{Magnitude is as large as it needs to be to stop objects from ``crossing'' ($a_\perp = 0$)}
\item{Newton's third law: if A pushes on B, B pushes back on A (the book problem)}
\item{Can only push; can never pull (the water-in-bucket problem)}
    \EI
}

\frame{\frametitle{\textbf{A few things about these forces: friction}}


Friction depends on a property of the surfaces called the {\color{Red}coefficient of friction} $\mu$
  \BI
\item{Roughly: ``how sticky things are''.}
\item{Force of kinetic friction = $\mu_k F_N$}
\item{Max force of static friction = $\mu_s F_N$}
\item{Friction points in whatever direction opposes the tendency to slide}
\item{Static friction {\it can} make objects move (cars, people walking)}
  \EI
}

\frame{\frametitle{\textbf{Rotational motion}}
  \BI
  \large
\item{``Uniform circular motion'': object steadily moving in a circle}
\item{Angular velocity: how fast does the thing turn? (RPM's, degrees per second, {\bf radians per second})}
\item{Constant speed does {\it not} mean constant velocity or zero acceleration!}
  
  \bigskip
\bigskip
 
\centerline{\Large $a = \omega^2 r = \frac{v^2}{r}$ toward the center of the circle}

  \bigskip
  \bigskip

\pause

\item{``How many force problems and how many circular motion problems will we have?''}
  \pause
\item{They're the same: circular motion just tells you that $a=\omega^2 r$. You do these problems in {\it exactly the same way}.}
  \EI
}

\frame{\frametitle{\textbf{Problem solving strategies (the important thing!)}}
    \BI
  \item{\Large 1. Force diagrams (``Accounting'')}
\BI
\item{Draw all forces and only forces (things you can feel)}
\item{Choose a pair of axes (tilted axes are sometimes helpful, like for things on ramps)}
\item{Break forces into components along these axes, if needed}
  \EI

  \bigskip\pause

\item{\Large 2. Newton's laws (``Physics'')}
  \BI
\item{Write down $\sum F = ma$ for {\color{Red}each object} in {\color{Red}each direction. You can read this off your diagram. For instance:}}

  \begin{align*}
     T_1 \cos \theta - T_2 =& ma_x \\
    T_1 \sin \theta - mg =& ma_y \\
  \end{align*}
\item{Forces (real things) go on the left side; acceleration goes on the right}
\item{Put in things you know about the acceleration}
\item{Different objects : different acceleration variables (are they related?)}
  \BI
\item{Sometimes $a=0$}
\item{Circular motion: $a_r = \omega^2 r = \frac{v^2}{r}$ toward the center}
  \EI
\EI
  \bigskip\pause

\item{\Large 3. Algebra (``Math'')}
\BI
\item{Put in the stuff you have, solve for the stuff you need}
\item{Need at least as many equations as unknowns}
\item{``Systems of equations'': solve by substitution}
  \EI
  \EI
}

\frame{\frametitle{\textbf{Sample problems: elevator}}
\Large
A 100 kg person stands in an elevator. What is the normal force if the elevator is accelerating upward at 3 $\rm m/\rm s^2$?

\bigskip
\bigskip
\pause

\BI
\item{Sum of forces goes on the left, acceleration goes on the right}
\item{We know the acceleration; we don't know one of the forces $\rightarrow$ solve for it!}
  \EI
}

\frame{\frametitle{\textbf{Sample problems: Mass on a string}}
  \Large
  A 2 kg mass hangs on a string 1m long, which is being spun in a vertical circle once per second. What is the tension force at the bottom
  of the arc?

  \bigskip
  \bigskip
  \pause

  \BI
\item{This is the same idea as the last problem; we just know the acceleration in an indirect way}
    \EI
  }

  \frame{\frametitle{\textbf{Sample problems: Mass on a string}}
  \Large
  A 2 kg mass hangs on a string 1m long, which is being spun in a vertical circle once per second. What is the tension force at the top
  of the arc?

  \bigskip
  \bigskip
  \pause

  \BI
\item{Remember, the acceleration goes toward the center of the circle: think about your signs!}
    \EI
  }

  \frame{\frametitle{\textbf{Sample problems: Mass on a ramp}}
    \Large
    A block sits on a ramp inclined at an angle of 40 degrees. The coefficient of kinetic friction is 0.3. What is its acceleration?

     \bigskip
       \bigskip
         \pause

           \BI
         \item{Tilted coordinate axes}
         \item{Break gravity into components (remember how this goes!!)}
           \EI
         }


  \frame{\frametitle{\textbf{Sample problems: Atwood's machine}}
    \Large
  Two masses of 1 kg and 1.1 kg hang from either side of a massless pulley. What is their acceleration?

  \bigskip
  \bigskip
  \pause

  \BI
\item{Separate force diagram for each object}
\item{How do the accelerations relate?}
\EI
}

\frame{\frametitle{\textbf{Sample problems: Train problem}}
    \Large
    A toy locomotive of mass 4 kg tows two toy train cars of mass 2 kg each, accelerating at 2 $\rm m/\rm s^2$. Find the two tension forces and the force of static friction on the locomotive's wheels.

     \bigskip
       \bigskip
         \pause

           \BI
         \item{Static friction = traction force}
         \item{Separate force diagram for each object}
         \item{Only have forces that directly connect -- don't ``guess the answer'', ask the question}
           \EI
         }


\frame{\frametitle{\textbf{Sample problems: your request!}}}

\frame{\frametitle{\textbf{Mid-semester evaluations}}}

    \end{document}
