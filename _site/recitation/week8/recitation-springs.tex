
\documentclass[12pt]{article}
\setlength\parindent{0pt}
\usepackage{fullpage}
\usepackage{amsmath}
\usepackage[margin=1in]{geometry}
\usepackage{graphicx}
\setlength{\parskip}{4mm}
\def\LL{\left\langle}   % left angle bracket
\def\RR{\right\rangle}  % right angle bracket
\def\LP{\left(}         % left parenthesis
\def\RP{\right)}        % right parenthesis
\def\LB{\left\{}        % left curly bracket
\def\RB{\right\}}       % right curly bracket
\def\PAR#1#2{ {{\partial #1}\over{\partial #2}} }
\def\PARTWO#1#2{ {{\partial^2 #1}\over{\partial #2}^2} }
\def\PARTWOMIX#1#2#3{ {{\partial^2 #1}\over{\partial #2 \partial #3}} }
\newcommand{\BE}{\begin{displaymath}}
\newcommand{\EE}{\end{displaymath}}
\newcommand{\BNE}{\begin{equation}}
\newcommand{\ENE}{\end{equation}}
\newcommand{\BEA}{\begin{eqnarray}}
\newcommand{\EEA}{\nonumber\end{eqnarray}}
\newcommand{\EL}{\nonumber\\}
\newcommand{\la}[1]{\label{#1}}
\newcommand{\ie}{{\em i.e.\ }}
\newcommand{\eg}{{\em e.\,g.\ }}
\newcommand{\cf}{cf.\ }
\newcommand{\etc}{etc.\ }
\newcommand{\Tr}{{\rm tr}}
\newcommand{\etal}{{\it et al.}}
\newcommand{\OL}[1]{\overline{#1}\ } % overline
\newcommand{\OLL}[1]{\overline{\overline{#1}}\ } % double overline
\newcommand{\OON}{\frac{1}{N}} % "one over N"
\newcommand{\OOX}[1]{\frac{1}{#1}} % "one over X"

\pagenumbering{gobble}

\begin{document}
\Large
\centerline{\sc{Recitation Exercises}}
\normalsize
\centerline{\sc{Week 10, Day 1}}



{A rock climber of mass 70 kg is climbing a cliff face when she slips and falls. She is two meters above her last anchor, so she will undergo free fall for 4 meters before the rope begins to arrest her fall. If the stiffness in her rope is 1400 N/m, then:}
\begin{enumerate}
	\item{How far will she fall in total?}
	\vspace{2in}
	
	\item{What is the maximum force that her rope will exert on her as it arrests her fall?}
	\vspace{1.5in}
	
	\item When would it be desirable for a rock climber to use a rope with a large spring constant? What about a smaller spring constant? You'll need to think about 
	the engineering reasons for climbers to use ropes at all: the goal is to minimize the forces involved in arresting a climber's fall.
\end{enumerate}

\newpage
%
%A slingshot is a small stone-throwing device often used as a toy by children; it consists of a forked stick with the ends of the fork connected by a rubber band. The user puts a stone against the rubber band, draws it back,
%and then releases it, propelling the stone forward.
%
%Suppose uses a slingshot to propel a $m=200$ g stone at an angle of $45^\circ$ above the horizontal. (Suppose they are lying on the ground, so you can ignore their height.) If the stone flies 40 meters before landing back on the ground, find the spring constant $k$ of the rubber band.
%
%{\it Hint: There are two stages in the motion here: you will need to understand both how the slingshot propels the rock into the air, and then how the rock flies through the air once it is launched. Think carefully about what technique you will use to understand each stage.}

%
%\newpage
%


% \item{A laptop battery says it has a capacity of 70 ``watt-hours''.}
%   \begin{enumerate}
%     \item{What are the dimensions of this odd unit ``watt-hour'', and what does it measure? What is 70 watt-hours in more familiar units?}
%
%\vspace{2.5in}
%
%     \item{If this battery were used to power an electric motor, how high could it lift the battery? Assume the battery has a mass of 300 grams.}
%\vspace{2.5in}
%   \end{enumerate}
%
%\newpage
%


A cyclist along with their bicycle has a mass of $m=60$ kg. They are capable of generating a power $P=150$ watts for a long time. (This means that they can do work on the pedals of their bicycle at a rate of 150 watts, and thus the bicycle's traction can do work on the bicycle at the same rate.)

\begin{enumerate}
\item There is a hill leading from Jamesville to Manlius that is elevated at an angle of around $6^\circ$. When going up a hill such as this one, gravity is the main force doing negative work on a cyclist; air resistance doesn't matter much (since they are not going that quickly)

How fast could our cyclist ride up the hill?


\vspace{2in}

\item Now, let's imagine this cyclist riding on flat terrain. On flat terrain, the main force doing negative work on the cyclist is air resistance.

The force from air resistance has the form $F_{\rm drag} = \gamma v^2$. $\gamma$ (lowercase Greek letter gamma)is a constant that depends on the cyclist's shape and size.

If the cyclist (with their maximum power output of 150 W) can sustain a top speed of 9 m/s (about 20 mph) on flat ground, what is the value of $\gamma$? {\it (This value, once you find it, is a constant for the cyclist no matter where they go.)}

\vspace{2in}
\newpage
\item Suppose they now travel to a steep mountain road, angled at $\theta=8^\circ$, and coast downhill. (They are not pedaling.) They will travel faster and faster until they reach a speed where the sum of the forces on them is zero and they don't accelerate any more. What top speed will they attain? 

\vspace{3in}

\item This speed can be quite dangerous for a cyclist! Suppose that they don't want to go this fast, and want to use their brakes to slow themselves down and travel at a constant speed of 8 m/s.

Brakes use friction to do negative work on the wheel, slowing the bicycle down. However, since energy is conserved, the kinetic energy removed from the wheel by friction is converted to heat. At what rate will the bicycle's brakes build up heat going down the hill?

\vspace{3in}
\newpage
\item The value of $\gamma$ is roughly the same for racing cyclists of different body types. Cycling races like the Tour de France have both uphill and downhill stages. Cyclists with smaller bodies -- who have much less mass but somewhat lower maximum power output~-- have an advantage on uphill segments of the race. However, more muscular cyclists, with more mass and higher maximum power output, have an advantage on downhill segments. 

Discuss with your group why this might be.

\end{enumerate}

\vspace{3in}

\begin{center}\underline{\hspace{3in}}\end{center}







{\it If you finish all of the above, take a look at Problem 4 on your homework with your group. It's a good one for you to work on together, since it should provoke some discussion!
 \end{document}
