\documentclass[12pt]{article}
\setlength\parindent{0pt}
\usepackage{fullpage}
\usepackage{amsmath}
\usepackage{graphicx}
\setlength{\parskip}{4mm}
\usepackage[left=2cm, right=2cm, top=1.5cm, bottom=1cm]{geometry}
\def\LL{\left\langle}   % left angle bracket
\def\RR{\right\rangle}  % right angle bracket
\def\LP{\left(}         % left parenthesis
\def\RP{\right)}        % right parenthesis
\def\LB{\left\{}        % left curly bracket
\def\RB{\right\}}       % right curly bracket
\def\PAR#1#2{ {{\partial #1}\over{\partial #2}} }
\def\PARTWO#1#2{ {{\partial^2 #1}\over{\partial #2}^2} }
\def\PARTWOMIX#1#2#3{ {{\partial^2 #1}\over{\partial #2 \partial #3}} }
\newcommand{\BI}{\begin{itemize}}
\newcommand{\EI}{\end{itemize}}
\newcommand{\BE}{\begin{displaymath}}
\newcommand{\EE}{\end{displaymath}}
\newcommand{\BNE}{\begin{equation}}
\newcommand{\ENE}{\end{equation}}
\newcommand{\BEA}{\begin{eqnarray}}
\newcommand{\EEA}{\nonumber\end{eqnarray}}
\newcommand{\EL}{\nonumber\\}
\newcommand{\la}[1]{\label{#1}}
\newcommand{\ie}{{\em i.e.\ }}
\newcommand{\eg}{{\em e.\,g.\ }}
\newcommand{\cf}{cf.\ }
\newcommand{\etc}{etc.\ }
\newcommand{\Tr}{{\rm tr}}
\newcommand{\etal}{{\it et al.}}
\newcommand{\OL}[1]{\overline{#1}\ } % overline
\newcommand{\OLL}[1]{\overline{\overline{#1}}\ } % double overline
\newcommand{\OON}{\frac{1}{N}} % "one over N"
\newcommand{\OOX}[1]{\frac{1}{#1}} % "one over X"

\def\BS{\bigskip}

\begin{document}
\pagenumbering{gobble}
\Large
\centerline{\sc{Recitation Questions}}
\normalsize
\centerline{\sc{15 January}}

\medskip

For these problems, we want you to practice doing math with units like physicists do. This means:

\BI
\item Are there any quantities that you will need to estimate? Estimate their order of magnitude, along with physical units. (For instance, the mass of a person is about $10^2$ kg.)
\item Think about the dimensions of your answer, and the units that it is measured in.
\item What sort of math will you need to do to model your answer?
\item While doing that math, write the units by {\it every} quantity that has units, and manipulate those units like you would anything else in algebra.
\item Does your answer have the correct units? If it doesn't, think more carefully about the math that you did. 
\EI

\centerline{\Large Question 1: Calculating with units}

The distance to the Moon is about 400,000 km. The speed of light is very nearly 300 million ($3 \times 10^8$) meters per second.

If you watch the original news footage of the 1969 Moon landing, you'll notice a substantial delay between Mission Control and the astronauts. Estimate how long that delay is.

In your solution, make sure you retain units with every numerical quantity.

\vspace{3in}

Show your solution to your TA or coach when you're done.

Then, if you represent the distance to the moon as $s$, the time delay as $t$, and the speed of light as $v$, write an algebraic expression that illustrates the math you did above.

\vspace{2in}

\newpage

\centerline{\Large Question 2: Fermi problems}

Now, let's practice some problems involving estimation. For these problems, make sure you follow the steps above. Let me demonstrate an example.

Q: ``How many tons of snow fall on the Quad every winter?''

A: To answer this, I'll need to estimate a few things:

\begin{itemize}
\item How big is the Quad? It's about the length and width of a football pitch: $10^2$ meters on a side.
\item How much snow does Syracuse get? Over a winter, we get about 5 meters of snow. 
\item How much snow does it take to make a ton? A ton of water is about one cubic meter. But snow is lighter than water. Suppose it's five times lighter; then the density of snow is $\frac{1 \rm ton}{5 \rm m^3}$.
\end{itemize}

Now, I need to think about the math I should do. Let me be guided by the units, here. I know that my answer should be a number of tons. The density of water is measured in tons per cubic meter, so if I 
multiply something measured in cubic meters by this, I will get an answer in tons.

What could that be? Well, the Quad is a square, so I could square its width to get its surface area (measured in meters squared). Then, I can multiply by the depth of snow in meters 
to get the volume of snow (measured in meters cubed). I have all those things, so let's compute it:

$$(\text{mass of snow}) = (\text{10^2 m})^2 \times (\text{5 m}) \times (\frac{1}{5} \frac{\text{ton}}{m^3}) = 10^4\,\text{tons}$$

Now, you try some of these problems!

\begin{enumerate}

\item How much food does a person eat in a week in kilograms? What about in grams?

\item How many white lines are there in the middle of the highway from Syracuse to Rochester?

\item How long would it take you to empty all the water out of an Olympic swimming pool if you were using a cup?

\end{enumerate}


\centerline{\Large Question 3: 
