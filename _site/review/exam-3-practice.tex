\documentclass[12pt]{article}
\setlength\parindent{0pt}
\usepackage{fullpage}
\usepackage{amsmath}
\usepackage{array}
\usepackage{hyperref}
\usepackage{graphicx}
\setlength{\parskip}{4mm}
\def\LL{\left\langle}   % left angle bracket
\def\RR{\right\rangle}  % right angle bracket
\def\LP{\left(}         % left parenthesis
\def\RP{\right)}        % right parenthesis
\def\LB{\left\{}        % left curly bracket
\def\RB{\right\}}       % right curly bracket
\def\PAR#1#2{ {{\partial #1}\over{\partial #2}} }
\def\PARTWO#1#2{ {{\partial^2 #1}\over{\partial #2}^2} }
\def\PARTWOMIX#1#2#3{ {{\partial^2 #1}\over{\partial #2 \partial #3}} }
\newcommand{\vs}[1]{\vspace{#1}}
\newcommand{\BC}{\begin{center}}
\newcommand{\EC}{\end{center}}
\newcommand{\BI}{\begin{itemize}}
\newcommand{\EI}{\end{itemize}}
\newcommand{\BE}{\begin{enumerate}}
\newcommand{\EE}{\end{enumerate}}
\newcommand{\BNE}{\begin{equation}}
\newcommand{\ENE}{\end{equation}}
\newcommand{\BEA}{\begin{eqnarray}}
\newcommand{\EEA}{\nonumber\end{eqnarray}}
\newcommand{\EL}{\nonumber\\}
\newcommand{\la}[1]{\label{#1}}
\newcommand{\ie}{{\em i.e.\ }}
\newcommand{\eg}{{\em e.\,g.\ }}
\newcommand{\cf}{cf.\ }
\newcommand{\etc}{etc.\ }
\newcommand{\Tr}{{\rm tr}}
\newcommand{\etal}{{\it et al.}}
\newcommand{\OL}[1]{\overline{#1}\ } % overline
\newcommand{\OLL}[1]{\overline{\overline{#1}}\ } % double overline
\newcommand{\OON}{\frac{1}{N}} % "one over N"
\newcommand{\OOX}[1]{\frac{1}{#1}} % "one over X"


\pagenumbering{gobble}
\begin{document}

All the exercises here should be done {\it in groups}. If you are not sitting where you can talk to at least two other people, fix that now.

These exercises are designed to help with ``big picture'' issues. We will have coaches here who can come individually and answer questions if you have them, so don't hesitate to ask!

When you do the ``comparisons'' with your group, head to \url{https://bit.ly/3LZKsYj} and type short notes to share!


\section{Exercise 1}

\subsection{Practice Exam 3 \#7 (spring and block on table)}

With your group, look at Question 7 on Practice Exam 3. Part (c) asks you to write down the work-energy theorem as the block slides down the table. Do this with your group and agree on the answer; if you can't agree, call a coach over.

\subsection{Homework 6 \#4 (sliding penguin)}

With your group, look at HW6 \#4. Write down the work-energy theorem for the penguin as it goes from the top of the hill to the place where it stops sliding. If you don't agree with your group, call one of the assistants.

\subsection{Week 7 Friday Recitation, \#2 (tire swing in wind)}

With your group, look at this question. Agree with your group on the form of the work-energy theorem you can use to find $\phi$.

\subsection{Comparisons}

With your group, discuss:

\begin{itemize}
	\item What did your approach to all of these questions have in common?
	\item What was unique about each of the three of them?
\end{itemize}

\newpage
\section{Exercise 2}

\subsection{Week 8 Friday Recitation, \#1 (astronaut emergency system)}

With your group, look at this question and discuss how you approached it.

\subsection{Homework 7 \#2 (person being knocked down by dog)}

With your group, look at HW7 \#2 and discuss how you approached it. 

\subsection{Comparisons}

Discuss with your group:

\begin{itemize}
	\item What did your approach to these three questions have in common?
	\item What extra feature is present in \#2 that is not present in \#1? How does it change the way you approach it?
	\item You are solving for different things here: for an unknown mass in one case, for a starting velocity in the other. Does this affect your approach?
\end{itemize}


\subsection{Homework 7 \#3 (astronaut jumping between spacecraft)}

With your group, look at HW7 \#3 and discuss how you approached it. 

How does your approach here relate to HW7 \#2? What ``new'' complication is added here?

\newpage

\section{Exercise 3}

\subsection{Practice Exam 3 \#1 (dog and frisbee)}

Look at this question and discuss it with your group. You will need to use two different techniques in series; discuss what they are, how you know, and why.

\subsection{Week 8 Wednesday Recitation \#2 (slingshot)}

Look at this question and discuss it with your group. You will need to use two different techniques in series; discuss what they are, how you know, and why.

\subsection{Comparisons}

With your group, discuss:

\begin{itemize}
	\item What did your approach to each of these questions have in common?
	\item What extra technique did you have to use in \#1?
	\item Why could you not {\it only} use energy methods in each of these questions?
\end{itemize}


\newpage

\section{Exercise 4}

\subsection{Homework 7 \#8 (exploding a boulder)}

Look at this question and discuss it with your group. Here again you will need to use two different techniques in series. Discuss what they are, and how you know.

\subsection{Homework 7 \#4 (forensics lab measuring bullet speeds)}

Same deal, discuss this with your group.


\subsection{Comparisons}

With your group, discuss:

\begin{itemize}
	\item What did your approach to each of these questions have in common?
	\item How did you handle the unusual pattern of ``what you know and what you don't'' in the first question?
	\item In the first example, you were solving for a ``final'' quantity (how far things slid), but in the second, you were solving for an ``initial'' one. How does this affect your approach and how you think about the solution?
\end{itemize}

\newpage

\section{Exercise 5}

\subsection{Practice Exam 3, \#6 (pirate ship and cannon)}

\subsection{Week 8 Friday Recitation \#3 (throwing things at boxes)}

\subsection{Comparisons}

With your group, discuss:

\begin{itemize}
	\item What did your approach to each of these questions have in common?
	\item In the first question, one object exploded into two; in the second question, two objects collided (and maybe stuck together). How does this affect your approach?
\end{itemize}

\newpage

\section{Exercise 6}

\subsection{Week 8 Wednesday Recitation \#3 (cyclist riding uphill)}

\subsection{Homework 6 \#6 (electric car battery capacity)}

\subsection{Homework 7 \#5 (truck going downhill)}

\subsection{Comparisons}

\begin{itemize}
	\item In all three exercises, compare how you related the power applied by a force to the vehicle's speed.
	\item Did you have to write down a detailed equation for the work-energy theorem in any of these questions? Why or why not?
	\item In the second and third exercises, sometimes you calculated a total amount of {\it energy} and sometimes you looked at the {\it power} applied by a force. Under what situations would you use each idea?
\end{itemize}

\end{document}
