\documentclass[12pt]{article}
\setlength\parindent{0pt}
\usepackage{fullpage}
\usepackage{amsmath}
\usepackage{graphicx}
\setlength{\parskip}{4mm}
\def\LL{\left\langle}   % left angle bracket
\def\RR{\right\rangle}  % right angle bracket
\def\LP{\left(}         % left parenthesis
\def\RP{\right)}        % right parenthesis
\def\LB{\left\{}        % left curly bracket
\def\RB{\right\}}       % right curly bracket
\def\PAR#1#2{ {{\partial #1}\over{\partial #2}} }
\def\PARTWO#1#2{ {{\partial^2 #1}\over{\partial #2}^2} }
\def\PARTWOMIX#1#2#3{ {{\partial^2 #1}\over{\partial #2 \partial #3}} }
\newcommand{\BE}{\begin{displaymath}}
\newcommand{\EE}{\end{displaymath}}
\newcommand{\BNE}{\begin{equation}}
\newcommand{\ENE}{\end{equation}}
\newcommand{\BEA}{\begin{eqnarray}}
\newcommand{\EEA}{\nonumber\end{eqnarray}}
\newcommand{\EL}{\nonumber\\}
\newcommand{\la}[1]{\label{#1}}
\newcommand{\ie}{{\em i.e.\ }}
\newcommand{\eg}{{\em e.\,g.\ }}
\newcommand{\cf}{cf.\ }
\newcommand{\etc}{etc.\ }
\newcommand{\Tr}{{\rm tr}}
\newcommand{\etal}{{\it et al.}}
\newcommand{\OL}[1]{\overline{#1}\ } % overline
\newcommand{\OLL}[1]{\overline{\overline{#1}}\ } % double overline
\newcommand{\OON}{\frac{1}{N}} % "one over N"
\newcommand{\OOX}[1]{\frac{1}{#1}} % "one over X"



\begin{document}
\Large
\centerline{\sc{Recitation Questions}}
\normalsize
\centerline{\sc{Week of April 4 -- on energy}}
\small

\medskip

\centerline{\large Question 1 -- work and energy in one dimension}

\medskip

Someone drops a penny of mass 2.5g off of the Empire State Building (height 380 m). It strikes the ground traveling at 50 m/s, having been slowed somewhat by air resistance.


\vspace{1in}

a) With what velocity would it have struck the ground if there were no air resistance? 

\vspace{2in}

b) What was the work done by the drag force?

\vspace{2in}

c) This penny strikes the sidewalk and penetrates the surface, digging a hole 2 cm deep. What was the upward force
exerted on the penny by the pavement? 

\newpage

\centerline{\large Question 2 -- on elasticity}

\medskip

A rock climber of mass 70 kg is climbing a cliff face when she slips and falls. 
There is 4m of slack in her climbing rope, so she undergoes free fall for 4 meters before the 
rope begins to arrest her fall. If the spring constant of her rope is 1400 N/m, then:

\vspace{1in}

a) How far will she fall in total?

\vspace{2in}

b) What is the maximum force that her rope will exert on her as it arrests her fall? Convert this force to either pounds or
kilogram-weights, as appropriate for your background. Is this a large force?

\vspace{2in}

c) Under what conditions would a climber want to use a rope with a lower spring constant? What about a larger spring constant?

\newpage

\centerline{\large Questions 3 and 4 -- on elasticity, again}

A ball of mass $m$ is connected to one end of a rubber band and swung in a circle. The rubber band has spring constant $k$, and
the 



\end{document}
