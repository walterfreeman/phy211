% DPF 09 talk on strangeness in nucleon

\documentclass[10pt]{beamer}
\usefonttheme{professionalfonts} % using non standard fonts for beamer
\usefonttheme{serif} % default family is serif
\usepackage{amsmath}
\usepackage{mathtools}
%\documentclass[12pt]{beamerthemeSam.sty}
\usepackage{epsf}
%\usepackage{pstricks}
%\usepackage[orientation=portrait,size=A4]{beamerposter}
\geometry{paperwidth=160mm,paperheight=120mm}
%DT favorite definitions
\def\LL{\left\langle}	% left angle bracket
\def\RR{\right\rangle}	% right angle bracket
\def\LP{\left(}		% left parenthesis
\def\RP{\right)}	% right parenthesis
\def\LB{\left\{}	% left curly bracket
\def\RB{\right\}}	% right curly bracket
\def\PAR#1#2{ {{\partial #1}\over{\partial #2}} }
\def\PARTWO#1#2{ {{\partial^2 #1}\over{\partial #2}^2} }
\def\PARTWOMIX#1#2#3{ {{\partial^2 #1}\over{\partial #2 \partial #3}} }

\def\rightpartial{{\overrightarrow\partial}}
\def\leftpartial{{\overleftarrow\partial}}
\def\diffpartial{\buildrel\leftrightarrow\over\partial}

\def\BI{\begin{itemize}}
\def\EI{\end{itemize}}
\def\BE{\begin{displaymath}}
\def\EE{\end{displaymath}}
\def\BEA{\begin{eqnarray*}}
\def\EEA{\end{eqnarray*}}
\def\BNEA{\begin{eqnarray}}
\def\ENEA{\end{eqnarray}}
\def\EL{\nonumber\\}


\newcommand{\map}[1]{\frame{\frametitle{\textbf{Course map}}
\centerline{\includegraphics[height=0.86\paperheight]{../../map/#1.png}}}}
\newcommand{\wmap}[1]{\frame{\frametitle{\textbf{Course map}}
\centerline{\includegraphics[width=0.96\paperwidth]{../../map/#1.png}}}}

\newcommand{\etal}{{\it et al.}}
\newcommand{\gbeta}{6/g^2}
\newcommand{\la}[1]{\label{#1}}
\newcommand{\ie}{{\em i.e.\ }}
\newcommand{\eg}{{\em e.\,g.\ }}
\newcommand{\cf}{cf.\ }
\newcommand{\etc}{etc.\ }
\newcommand{\atantwo}{{\rm atan2}}
\newcommand{\Tr}{{\rm Tr}}
\newcommand{\dt}{\Delta t}
\newcommand{\op}{{\cal O}}
\newcommand{\msbar}{{\overline{\rm MS}}}
\def\chpt{\raise0.4ex\hbox{$\chi$}PT}
\def\schpt{S\raise0.4ex\hbox{$\chi$}PT}
\def\MeV{{\rm Me\!V}}
\def\GeV{{\rm Ge\!V}}

%AB: my color definitions
%\definecolor{mygarnet}{rgb}{0.445,0.184,0.215}
%\definecolor{mygold}{rgb}{0.848,0.848,0.098}
%\definecolor{myg2g}{rgb}{0.647,0.316,0.157}
\definecolor{abtitlecolor}{rgb}{0.0,0.255,0.494}
\definecolor{absecondarycolor}{rgb}{0.0,0.416,0.804}
\definecolor{abprimarycolor}{rgb}{1.0,0.686,0.0}
\definecolor{Red}           {cmyk}{0,1,1,0}
\definecolor{Grey}           {cmyk}{.7,.7,.7,0}
\definecolor{Lg}           {cmyk}{.4,.4,.4,0}
\definecolor{Blue}          {cmyk}{1,1,0,0}
\definecolor{Green}         {cmyk}{1,0,1,0}
\definecolor{Brown}         {cmyk}{0,0.81,1,0.60}
\definecolor{Black}         {cmyk}{0,0,0,1}

\usetheme{Madrid}


%AB: redefinition of beamer colors
%\setbeamercolor{palette tertiary}{fg=white,bg=mygarnet}
%\setbeamercolor{palette secondary}{fg=white,bg=myg2g}
%\setbeamercolor{palette primary}{fg=black,bg=mygold}
\setbeamercolor{title}{fg=abtitlecolor}
\setbeamercolor{frametitle}{fg=abtitlecolor}
\setbeamercolor{palette tertiary}{fg=white,bg=abtitlecolor}
\setbeamercolor{palette secondary}{fg=white,bg=absecondarycolor}
\setbeamercolor{palette primary}{fg=black,bg=abprimarycolor}
\setbeamercolor{structure}{fg=abtitlecolor}

\setbeamerfont{section in toc}{series=\bfseries}

%AB: remove navigation icons
\beamertemplatenavigationsymbolsempty
\title[Work and potential energy -- problem solving]{
  \textbf {Energy methods -- problem solving}\\
%\centerline{}
%\centering
%\vspace{-0.0in}
%\includegraphics[width=0.3\textwidth]{propvalues_0093.pdf}
%\vspace{-0.3in}\\
%\label{intrograph}
}

\author[W. Freeman] {Physics 211\\Syracuse University, Physics 211 Spring 2016\\Walter Freeman}

\date{\today}

\begin{document}

\frame{\titlepage}

\frame{\frametitle{\textbf{Announcements}}
\BI
\item{Office hours today: 1:30-3:30}
\item{I can't stay after 3:30, but TA's/coaches will be there 5PM-9PM to help with homework}
\item{Review session: Sunday, 6:30-9:30 PM}
\item{Exam 2 retake next Tuesday}
\EI
}

\frame{\frametitle{\textbf{Where we've been, where we're going}}
  \BI
  \large
\item{Last time: we saw that ``potential energy'' is both a statement about nature and a bookkeeping trick to keep track of work}
\BI
\item{Potential energy only applies to conservative forces (gravity, springs)}
\item{Lets us account for the work done by these forces with no integrals required}
\item{Potential energy due to Earth's gravity: $U_g = mgy$}
\item{Potential energy in a spring: $U_e = \frac{1}{2}k(\Delta x)^2$}
  \EI
\item{This time: we'll introduce the idea of {\bf power}, a rate of doing work}
  \pause
\item{... and see how energy in rotational motion works} 
  \EI
}

\frame{\frametitle{\textbf{Power}}
\large
\BI
\item{We've been concerned with quite a few ``rates`` in this class:}
  \BI
\item{Velocity: the rate of changing position, measured in meters per second}
\item{Angular velocity: the rate of changing angle, measured in radians per second}
\EI
\item{What about the rate of transfer of energy? What units would it be measured in?}
\pause
\item{This quantity is called {\bf power}}
\item{It's measured in joules per second: 1 J/s = 1 watt}
\item{This is the same unit you're familiar with on lightbulbs and hairdryers}
  \EI
}

\frame{\frametitle{\textbf{Power: applications}}
  \large
  When does this idea of a rate of transferring energy or doing work come up?

  \bigskip
  \bigskip
  
  \BI
\item{Rates of energy transfer: ``Sunlight delivers about 1000 watts per square meter to the ground''}
\item{Rates of energy ``consumption'': ``My laptop uses about 15 watts of power''}
\item{Rates of doing work: ``A human can sustain a power output of about 200 watts, generating 800W of waste heat''}
\EI
  
\bigskip
  \bigskip

Most of our ideas here are stepping stones to understanding something else.

The idea of power is more of a standalone concept: a useful application.

Many of our machines are limited by the {\bf rate} that they can convert energy from one form to another.
}

\frame{\frametitle{\textbf{Power: rate of doing work}}
\large
A bit of mathematics that will be useful to you:

\bigskip

{\bf ``An object moves at a constant speed $\vec v$, subject to some force $\vec F$; at what rate does that force do work on the object?''}

\bigskip

An example: an airplane flies at v=1000 m/s, and its engines exert F=300 kN of thrust. What is the rate at which the engines do work (power)?

\bigskip

\centerline{Work = force $\times$ distance}
\pause
\centerline{Power = work / time}
\pause
\centerline{Power = force $\times$ distance / time}
\pause
\centerline{Power = force $\times$ (distance / time)}
\pause
\centerline{Power = force $\times$ velocity}
\pause
\centerline{$P = \vec F \cdot \vec v = {\color{Red}300 MW}$}

\bigskip

\BI
\item{The engines output 300 MW of power: this is around 10 liters per second of fuel even at 100\% efficiency!}
  \item{Some of that 300 MW of energy dissipated by drag heats up the airplane... (real numbers for a SR-71 Blackbird)}
    \EI
  }

  \frame{\frametitle{\textbf{Sample problems}}
    \Large
  A truck pulling a heavy load with mass $m=4000$ kg wants to drive up a hill at a $30^o$ grade.

  \bigskip
  \bigskip

  If the truck's engine can produce 100 kW of power (134 hp), how fast can the truck go? (Neglect drag.)

}

\frame{\frametitle{\textbf{Sample problems}}
    \Large
A 1000 kg car has an engine that produces up to P=100 kW of power. If it accelerates as hard as it can, at what speed
does its acceleration become limited by the engine? 

\bigskip
\bigskip
\pause

(What else would limit its acceleration?)

\bigskip
\bigskip
\pause

At low speeds: static friction limits acceleration\\
At high speeds: engine power limits acceleration}



\frame{\frametitle{\textbf{Sample problems}}
  \includegraphics[width=0.7\textwidth]{elevator.png}
}

\frame{\frametitle{\textbf{What about rotational energy}}
\Large
Things roll with very little friction, so let's roll things down a ramp...
\pause
\large
\BI
\item{Different objects reach the bottom at different speeds}
\pause
\item{... {\it none} of them reach the bottom with the ``correct'' speed given by $mgh = \frac{1}{2}mv_f^2$}
\item{... what's wrong?}
\EI
}

\frame{\frametitle{\textbf{Some energy is diverted to rotational kinetic energy}}
\Large
We can associate a kinetic energy with a spinning object, too...

\bigskip

\Huge

\centerline{$KE_{\rm{trans}} = \frac{1}{2}mv^2$}
\centerline{$KE_{\rm{rot}} = \frac{1}{2}I\omega^2$}

\large
}

  \frame{\frametitle{\textbf{A description of rolling motion without slipping}}
    \large
    \centerline{Rolling motion combines translation of the center of mass with rotation around it}
\bigskip
\centerline{\includegraphics[width=0.6\textwidth]{rolling.png}}

  \bigskip
  \bigskip


\BI
\item{The key idea: tangential velocity of the rim about the center is equal to the speed of the axle}
  \pause
\item{In symbols:}
  \EI

  \bigskip
  \bigskip

  \Large

  \centerline{$v_{\rm{com}} = v_T = \omega r$}
}

\frame{\frametitle{\textbf{Now we can finish our problem}}
\large
An object with $I=\lambda mr^2$ rolls down a hill of height $h$; how fast is it going at the bottom?

\bigskip

Same idea as before: 

\begin{align*}
KE_i + U_{g,i} =& KE_f + U_{g_f} \\
mgh =& \frac{1}{2}mv_f^2 + \frac{1}{2}I\omega_f^2
\end{align*}

But $v_f = r \omega_f \rightarrow \omega_f = v_f/r$ and $I=\lambda mr^2$, so we have

\begin{align*}
mgh =& \small \frac{1}{2} \large mv_f^2 + \frac{1}{2} \lambda mr^2 \left(\frac{v_f^2}{r^2}\right)\\
mgh =& \frac{1}{2}(m + \lambda m)v_f^2\\
v_f =& \sqrt{\frac{2gh}{1 + \lambda}} 
\end{align*}

\bigskip

... the higher the value of $\lambda$, the slower it rolls!
}




\frame{\frametitle{\textbf{Sample problems}}
  \includegraphics[width=0.7\textwidth]{student-on-spring.png}
}

\frame{\frametitle{\textbf{Sample problems}}
  \includegraphics[width=0.7\textwidth]{runaway.png}
}





\end{document}
