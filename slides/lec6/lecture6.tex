\documentclass[10pt]{beamer}
\usepackage{amsmath}
\usepackage{mathtools}
%\documentclass[12pt]{beamerthemeSam.sty}
\usepackage{epsf}
\usefonttheme{professionalfonts} % using non standard fonts for beamer
\usefonttheme{serif} % default family is serif


%\usepackage{pstricks}
%\usepackage[orientation=portrait,size=A4]{beamerposter}
\geometry{paperwidth=160mm,paperheight=120mm}
%DT favorite definitions
\def\LL{\left\langle}	% left angle bracket
\def\RR{\right\rangle}	% right angle bracket
\def\LP{\left(}		% left parenthesis
\def\RP{\right)}	% right parenthesis
\def\LB{\left\{}	% left curly bracket
\def\RB{\right\}}	% right curly bracket
\def\PAR#1#2{ {{\partial #1}\over{\partial #2}} }
\def\PARTWO#1#2{ {{\partial^2 #1}\over{\partial #2}^2} }
\def\PARTWOMIX#1#2#3{ {{\partial^2 #1}\over{\partial #2 \partial #3}} }

\def\rightpartial{{\overrightarrow\partial}}
\def\leftpartial{{\overleftarrow\partial}}
\def\diffpartial{\buildrel\leftrightarrow\over\partial}

\def\BC{\begin{center}}
\def\EC{\end{center}}
\def\BI{\begin{itemize}}
\def\EI{\end{itemize}}
\def\BE{\begin{displaymath}}
\def\EE{\end{displaymath}}
\def\BEA{\begin{eqnarray*}}
\def\EEA{\end{eqnarray*}}
\def\BNEA{\begin{eqnarray}}
\def\ENEA{\end{eqnarray}}
\def\EL{\nonumber\\}


\newcommand{\map}[1]{\frame{\frametitle{\textbf{Course map}}
\centerline{\includegraphics[height=0.86\paperheight]{../../map/#1.png}}}}
\newcommand{\wmap}[1]{\frame{\frametitle{\textbf{Course map}}
\centerline{\includegraphics[width=0.96\paperwidth]{../../map/#1.png}}}}

\newcommand{\etal}{{\it et al.}}
\newcommand{\gbeta}{6/g^2}
\newcommand{\la}[1]{\label{#1}}
\newcommand{\ie}{{\em i.e.\ }}
\newcommand{\eg}{{\em e.\,g.\ }}
\newcommand{\cf}{cf.\ }
\newcommand{\etc}{etc.\ }
\newcommand{\atantwo}{{\rm atan2}}
\newcommand{\Tr}{{\rm Tr}}
\newcommand{\dt}{\Delta t}
\newcommand{\op}{{\cal O}}
\newcommand{\msbar}{{\overline{\rm MS}}}
\def\chpt{\raise0.4ex\hbox{$\chi$}PT}
\def\schpt{S\raise0.4ex\hbox{$\chi$}PT}
\def\MeV{{\rm Me\!V}}
\def\GeV{{\rm Ge\!V}}

%AB: my color definitions
%\definecolor{mygarnet}{rgb}{0.445,0.184,0.215}
%\definecolor{mygold}{rgb}{0.848,0.848,0.098}
%\definecolor{myg2g}{rgb}{0.647,0.316,0.157}
\definecolor{abtitlecolor}{rgb}{0.0,0.255,0.494}
\definecolor{absecondarycolor}{rgb}{0.0,0.416,0.804}
\definecolor{abprimarycolor}{rgb}{1.0,0.686,0.0}
\definecolor{Red}           {cmyk}{0,1,1,0}
\definecolor{Grey}           {cmyk}{.7,.7,.7,0}
\definecolor{Blue}          {cmyk}{1,1,0,0}
\definecolor{Green}         {cmyk}{1,0,1,0}
\definecolor{Brown}         {cmyk}{0,0.81,1,0.60}
\definecolor{Black}         {cmyk}{0,0,0,1}
\definecolor{A}{rgb}{0.8,0.0,0.0}
\definecolor{B}{rgb}{0.0,0.6,0.0}
\definecolor{C}{rgb}{0.4,0.4,0.0}
\definecolor{D}{rgb}{0.0,0.0,0.5}
\definecolor{E}{rgb}{0.4,0.4,0.4}
\usetheme{Madrid}


%AB: redefinition of beamer colors
%\setbeamercolor{palette tertiary}{fg=white,bg=mygarnet}
%\setbeamercolor{palette secondary}{fg=white,bg=myg2g}
%\setbeamercolor{palette primary}{fg=black,bg=mygold}
\setbeamercolor{title}{fg=abtitlecolor}
\setbeamercolor{frametitle}{fg=abtitlecolor}
\setbeamercolor{palette tertiary}{fg=white,bg=abtitlecolor}
\setbeamercolor{palette secondary}{fg=white,bg=absecondarycolor}
\setbeamercolor{palette primary}{fg=black,bg=abprimarycolor}
\setbeamercolor{structure}{fg=abtitlecolor}

\setbeamerfont{section in toc}{series=\bfseries}

%AB: remove navigation icons
\beamertemplatenavigationsymbolsempty
\title[Problem solving: kinematics]{
  \textbf {Problem solving: kinematics (II)}\\
%\centerline{}
%\centering
%\vspace{-0.0in}
%\includegraphics[width=0.3\textwidth]{propvalues_0093.pdf}
%\vspace{-0.3in}\\
%\label{intrograph}
}

\author[W. Freeman] {Physics 211\\Syracuse University, Physics 211 Spring 2023\\Walter Freeman}

\date{\today}

\begin{document}

\frame{\titlepage}

\frame{\frametitle{\textbf{Announcements}}
\large
\BI
\item{Homework 2 due date is {\bf tomorrow}}
\item{Exam 1 is next Tuesday}
  \BI
\item{No homework due next week}
\item{HW2 problems are similar to those on Exam 1}
\item{Next recitation is your group practice exam}
\item{If you must miss, notify your TA in advance and explain to them why you will be absent}
\item If your absence is for a justified reason (see the syllabus), we will give you your grade on Exam 1 as a replacement for Group Exam 1
\BI
\item{Saturday: Exam review in Stolkin, 5-8 PM} 
  \EI
\EI
\EI
}

\frame{\frametitle{\textbf{Exam 1}}
 \BI
 \item{The exam covers kinematics in one and two dimensions}
 \item{Kinematics: how are an object's position, velocity, and acceleration related?}
   \pause
 \item{\color{Red}The exam will be somewhat easier than the homework.}
   \pause

  \item{There will be one problem where you need the quadratic formula}
  \BI
  \item{... this means interpreting the two values it spits out}
  \EI
  \item{There will be at least one instance where you need to interpret or sketch position, velocity, and acceleration graphs}
  \item There will be one problem where you need to do vector addition 
  \item{You will {\it not} need to compute derivatives or integrals algebraically}
  \item{The exam will be four or five problems}
  
\EI
}

\frame{\frametitle{\textbf{Exam 1, protocol}}
	\BI
	\item You will have assigned seats; we will post a seating chart before the exam
	\item {\color{Red}If you are left-handed, please send me an email so we can get you a left-handed desk}
	
	\item{You are allowed to bring one page of notes that {\it you handwrite yourself}}
	\BI
	\item No typed notes unless you have a disability that prevents you from writing
	\item Your friend can't write it
	\item You can't photocopy stuff from the book
	\item It won't help you as much anyway 
	\EI
	\EI
}


  \frame{\frametitle{\textbf{Problem solving: 2D kinematics, constant acceleration}}
    \large
    \begin{enumerate}
      \item{1. If you have vectors in the ``angle and magnitude'' form ($\vec a, \vec v, \vec s$), convert them to components}
      \item{2. Write down the kinematics relations, separately for $x$ and $y$}
        \begin{itemize}
            \large
          \item{Many terms will usually be zero}
          \item{Freefall: $a_x = 0$, $a_y = -g$ (with conventional choice of axes)}
        \end{itemize}
      \item{3. Understand what instant in time you want to know about: ask the right question}
      \item{4. Put in what you know; solve for what you don't (using substitution, if necessary)}
      \item{5. Think about the physical meaning of your solution}
    \end{enumerate}
  }

  \frame{\frametitle{\textbf{Problem solving: 2D kinematics, constant acceleration}}
    \large
    \begin{enumerate}
      \item{1. If you have vectors in the ``angle and magnitude'' form ($\vec a, \vec v, \vec s$), convert them to components}
      \item{2. Write down the kinematics relations, separately for $x$ and $y$}
        \begin{itemize}
            \large
          \item{Many terms will usually be zero}
          \item{Freefall: $a_x = 0$, $a_y = -g$ (with conventional choice of axes)}
        \end{itemize}
      \item{\color{Red}3. Understand what instant in time you want to know about: ask the right question}
      \item{4. Put in what you know; solve for what you don't (using substitution, if necessary)}
      \item{5. Think about the physical meaning of your solution}
    \end{enumerate}
  }

\frame{

\begin{center}
	
	\Large
	Do you have any questions from homework or recitation this week?
\end{center}

}



\frame{\frametitle{\textbf{A demonstration: independence of $x$ and $y$}}
	
	\begin{center}
		\Large	Will the ball land back in the cart?
		\bigskip
		\bigskip
		
		
		Can we tell without doing any mathematics?
	\end{center}
	
}



\frame{\frametitle{\textbf{Throwing a rock off a cliff}}
\Large A hiker throws a rock horizontally off of a $h=40$ m tall cliff at $v_0 = 20$ m/s. 

\bigskip
\begin{itemize}
\item Where does it land?\pause
\item  How fast is it going when it hit the ground? \pause
\item What changes if they throw it at an angle instead?\pause
\item What changes if the ground is sloped?\pause
  \end{itemize}
}



  
\frame{\frametitle{\textbf{A rocket}}
 \Large
  A rocket is launched from rest on level ground. While its motor burns, it accelerates at 10 m/s at an angle $30^\circ$
  below the vertical. After $\tau=10$ s its motor burns out and it experiences freefall until it hits the ground.

  \bigskip

 \begin{itemize}
 	\item How far does it go?
 	\item Graph acceleration, velocity, and position in both $x-$ and $y-$
 \end{itemize}
 


\bigskip

}


      \end{document}
