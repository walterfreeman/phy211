\documentclass[10pt]{article}
\setlength\parindent{0pt}
\usepackage{fullpage}
\usepackage[margin=2cm]{geometry}

\usepackage{amsmath}
\setlength{\parskip}{4mm}

\begin{document}


\begin{center}
	\sc \Large Review of Rotational Motion
	
\end{center}

\section{Introduction}

Most of the ideas you have seen before have close analogues in rotational motion. I'll discuss them one by one.

\section{Angular kinematics}

When thinking about rotation, instead of thinking about objects changing position, we are concerned with them turning through some angle.

Thus, instead of objects having a position vector $s$, they instead have an angle $\theta$ (measured with respect to some fixed axis).

Just as the derivatives of position (velocity and acceleration) are interesting, so too are the derivatives of angle.

\begin{align*}
\frac{d\theta}{dt} &\equiv \omega \hspace{1in} &\text{(angular velocity)} \\
\frac{d^2\theta}{dt^2} &\equiv \frac{d\omega}{dt} \equiv \alpha \hspace{1in} &\text{(angular acceleration)} 
\end{align*}

Since calculus works the same regardless of the functions we apply it to, the constant-acceleration kinematics formulae for translational motion work equally well for rotation:

\begin{align*}
&\omega(t) = \alpha t + \omega_0 \hspace{1in}& \text{equivalent to} \hspace {1in} & \vec v(t) = \vec a t + \vec v_0 \\
&\theta(t) = \frac{1}{2}\alpha t^2 + \omega_0 t + \theta_0 \hspace{1in}& \text{equivalent to} \hspace{1in}&\vec s(t) = \frac{1}{2}\vec at^2 + \vec v_0 t + \vec s_0 
\end{align*}

Note that as angle is a scalar (in two dimensions), not a vector, there is never a concern with components.

Conventionally, counterclockwise is taken to be positive, but you can change this for a problem if you state clearly that you are doing so.

\section{Moment of inertia: the rotational equivalent to mass}

The moment of inertia of an object rotating around an axis describes how much torque to change its angular velocity and how much kinetic energy its rotation has.

It is the rotational analogue to mass.

The moment of inertia of an object is given by 

$$I = m\langle r^2 \rangle$$

where the angle brackets around $\langle r^2 \rangle$ indicate an average over the object.

For objects where all of the mass is the same distance from the axis (for instance, a hollow cylinder or hoop spinning around its axis), the moment of inertia is just $mr^2$, since the average is trivial.

For other objects, taking the average distance of the mass from the axis requires doing multidimensional integrals. We are not going to make you do those. In general, the moment of inertia will take the form $I = \lambda mr^2$ where $\lambda$ is some fraction.

Here are some examples:

\begin{tabular}{|c | c|}
	\hline
Solid sphere & $I=\frac{2}{5}mr^2$\\	\hline
Hollow sphere & $I=\frac{2}{3}mr^2$\\	\hline
Disk or cylinder & $I=\frac{1}{2}mr^2$\\	\hline
Rod (around middle) & $I=\frac{1}{12}mL^2$\\	\hline
Rod (around edge) & $I=\frac{1}{3}mL^2$\\	\hline
\end{tabular}

\subsection{The parallel axis theorem}

Often you know an object's moment of inertia through an axis that passes through its center, but need its moment of inertia around another axis.

Something called the ``parallel axis theorem'' lets you find this. It says that

$$\text{(moment of inertia about any axis)} = \text{(moment of inertia through the center of mass parallel to that axis)} + mR^2$$

where $R$ is the distance between those axes.

There is actually an example in the table above. To find the moment of inertia of a rod around an axis through its edge, you can use the Parallel Axis Theorem to shift the axis by $L/2$. This means that 

\begin{align*}
I_{\text {through edge}} &= I_{\text{through center}} + m\left(\frac{L}{2}\right)^2\\
I_{\text {through edge}} &= \frac{1}{12}mL^2 + \frac{1}{4}mL^2 \\
I_{\text {through edge}} &= \frac{1}{3}mL^2
\end{align*}

which is what's in the table above.

\section{Rotational kinetic energy}

Since moment of inertia is the rotational analogue of mass and angular velocity is the rotational analogue of velocity, you might guess that 

$$KE_{\text{rotational}} = \frac{1}{2}I \omega^2$$

and you'd be right.

Note that if an object is rolling, or in some other fashion both translating and rotating, you should add together rotational and translational kinetic energy in using the conservation of energy.

\subsection{Rotational work and power}

A torque $\tau$ applied to an object that rotates through an angle $\Delta \theta$ does work 

$$W_{\rm rot} = \tau \Delta \theta$$

on it. This causes a change in rotational kinetic energy:

$$\frac{1}{2}I\omega_i^2 + W_{\rm rot} = \frac{1}{2}I\omega_f^2$$

This is analogous to the translational work-energy theorem

$$\frac{1}{2}mv_i^2 + W = \frac{1}{2}mv_f^2.$$

Similarly, a torque $\tau$ applied to an object rotating at angular velocity $\omega$ applies a power 

$$P_{\rm rot} = \tau \omega$$

which is analogous to the relation for translational motion

$$P=\vec F \cdot \vec v.$$

\section{Rolling without slipping, and other constrained rotations}

Often an object rolls without slipping. This means that its rotational motion (meaning: $\omega$ and $\alpha$) is linked to its translational motion (meaning: $v$ and $a$).

If an object rolls without slipping, it must rotate such that its {\it tangential velocity} $v_T = \omega r$ is equal in magnitude to its {\it translational velocity} $v$.

This means that 

$$ v = \pm \omega r \hspace{1in} \text{(rolling-without-slipping constraint)} $$

If the idea is to use conservation of energy with this, note that $(\pm)^2 = 1$, so $v^2 = \omega^2 r^2$.

{\bf Hint:} Since rotational kinetic energy has the form $KE_{\text{rot}} = \frac{1}{2}I\omega^2 = \frac{1}{2}\lambda mr^2 \omega^2$, if you are tracking 
rotational energy for an object that rolls without slipping, you may substitute $v^2 = \omega^2 r^2$ directly, giving $\frac{1}{2}\lambda mv^2$.

Be careful, however: note that the variable $r$ above represents the {\it radius that the object is rolling on}. If there are multiple radii in the problem (the problem on hw8, for instance), you will have to think carefully.

If you have two objects linked together such that one object's translation is linked to another object's rotation, the same logic applies.

\section{Angular momentum}

The rotational analogue of momentum is {\it angular momentum}. While rotational kinetic energy is just another form of energy, angular momentum $L$ is a separate thing, conserved separately for an object or system that does not have external torques.

Since moment of inertia is the rotational analogue of mass and angular velocity is the rotational analogue of velocity, you might guess that 

$$L = I \omega$$

and you'd be right.

The conservation of linear momentum is mostly useful for collisions and explosions. Conservation of angular momentum is similarly useful for collisions and explosions involving rotating objects. 

It is useful in one other case, too. A moving object is unable to change its mass, but a rotating object {\it can} change its moment of inertia (the famous spinning ice skater). In that case you can use conservation of angular momentum to relate the initial and final angular velocities and the initial and final moments of inertia.

Sometimes you have a small object traveling in a straight line and need to calculate its angular momentum relative to a certain rotating object. This can happen, for instance, if a projectile strikes a rotating object and sticks to it, making it rotate.

In this case the angular momentum of the small object is 

$$L = m \vec v \times \vec r$$

where $\times$ indicates the cross product. This is equivalent to either:

\begin{align*}
L=mv_\perp r, &\hspace{1in} \text{or}\\
L=mvr_\perp, &\hspace{1in} \text{or}\\
L=mvr \sin \theta
\end{align*}

where $v_\perp$ means the component of the velocity perpendicular to the radius vector, $r_\perp$ means the component of the radius vector perpendicular to the velocity, and $\theta$ is the angle between the two. (All of these are equivalent ways of computing the cross product.)

\section{Torque, the rotational equivalent to force}

We know that the controlling principle of dynamics is Newton's second law of motion $\vec F = m\vec a$. To find the equivalent for rotation, we need to know the rotational analogue of force.

That is {\it torque}, represented by $\tau$. Forces applying a force off center to an object apply a torque; that torque then causes an angular acceleration by ``Newton's second law for rotation''

$$\tau = I \alpha.$$

\subsection{Extended free-body diagrams}

Forces cause translational accelerations of objects regardless of where they are applied to those objects. This is why we can represent those objects by a dot when drawing free-body diagrams.

However, forces will cause an object to {\it rotate} differently depending on where they act. Thus, if you will be calculating torques, you need to draw an {\it extended free-body diagram}. This is a cartoon of the object that labels forces at the points where they are applied to the object.

Note that gravity always acts at the center of mass an object (if it is uniform, this is just its center).

\subsection{Calculating torque}

Unlike forces, torques are defined in reference to a {\it particular pivot point}. Define the radius vector as the vector from the pivot to the point where a force acts on an object. Then the torque applied by that force can be found by any of:

\begin{itemize}
	\item The magnitude of the radius vector times the component of the force perpendicular to it
	\item The component of the radius vector perpendicular to the force, times the magnitude of the force
	\item The magnitude of the force times the magnitude of the radius times the sine of the angle between them
	\end{itemize}

Or, in symbols:

$$|\tau| = F_\perp r = F r_\perp = Fr \sin \theta$$

where $\theta$ is the angle between them.

The above only allows you to calculate the magnitude of the torque. You will also need to think about its direction. By convention counterclockwise is chosen to be positive and clockwise is chosen to be negative; you may reverse this convention if you are consistent within a problem and you indicate that you've doing that.

\subsection{Choosing a pivot wisely}

The above is true in general: the torque about {\it any} pivot tells you how the object will rotate about that pivot.

The general rule is: Choose a pivot that makes the torque from forces that you {\it do not care about} and/or {\it do not know about} zero.

In many cases, particularly for objects with $\alpha \neq 0$, the most logical pivot is the center of the object. This means that gravity will apply no torque (since it acts at the center too), and often normal forces will not either (since they will act antiparallel to the radius vector). Also, things like rotating wheels and pulleys are supported by an axle through their center; that axle applies a large force to support the wheel and keep it in place, and if you choose the pivot at that point, this force will apply no torque.

In other problems, you have to figure out what arrangement of forces will stop an object from moving. (This is called {\it static equilibrium.}) In this case you know $\alpha=0$. As before, choose a pivot that makes the torque from forces that you {\it do not care about} and/or {\it do not know about} zero. This will often be at the position of a hinge, axle, or support.


\section{Correspondences between translational and rotational motion}


\begin{tabular}{| c | c | c | c |}
	\hline
	Position & $s$ & Angle & $\theta$  \\
	Velocity & $\vec v$ & Angular velocity & $\omega$  \\
	Acceleration & $\vec a$ & Angular acceleration & $\alpha$  \\
	\hline
	& $v(t) = v_0 + at$ & & $\omega(t) = \omega_0 + \alpha t$ \\
	& $x(t) = x_0 + v_0 t + \frac{1}{2} at^2$ & & $\theta(t) = \theta_0 + \omega_0 t + \frac{1}{2} \alpha t^2$ \\
	& $v_f^2 - v_0^2 = 2a \Delta x$ & & $\omega_f^2 - \omega_0^2 = 2 \alpha \Delta \theta$ \\
	\hline
	Mass & $m$ & Moment of inertia & $I$ \\
	\hline
	Force & $F$ & Torque & $\tau = F_\perp r = F r_\perp$ \\
	\hline
	Newton's second law & $\vec F = m \vec a$ & ``Newton's second law for rotation'' & $\tau = I \alpha$ \\
	\hline
	Kinetic energy & $\frac{1}{2} mv^2$ & Kinetic energy & $\frac{1}{2}I\omega^2$ \\
	\hline
	Momentum & $\vec p = m \vec v$ & Angular momentum & $L = I \omega$ \\
	\hline
	Work-energy theorem & $\Delta \frac{1}{2}mv^2 = W = \vec F \cdot \Delta \vec S$ & Work-energy theorem for rotation & $\Delta \frac{1}{2}I\omega^2 = W_{\rm rot} = \tau \Delta \theta$ \\
	\hline
	Translational power & $P = \frac{dW}{dt} = \vec F \cdot \vec v$ & Rotational power & $P = \frac{dW}{dt} = \tau \omega$ \\
	\hline
\end{tabular}

\bigskip
\bigskip
\bigskip

Arc length $s=\theta r$ \\
Tangential velocity $v=\omega r$ \\
Tangential acceleration $a=\alpha r$


\end{document}


