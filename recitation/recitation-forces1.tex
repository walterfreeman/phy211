\documentclass[12pt]{article}
\setlength\parindent{0pt}
\usepackage{fullpage}
\usepackage{amsmath}
\usepackage{graphicx}
\setlength{\parskip}{4mm}
\def\LL{\left\langle}   % left angle bracket
\def\RR{\right\rangle}  % right angle bracket
\def\LP{\left(}         % left parenthesis
\def\RP{\right)}        % right parenthesis
\def\LB{\left\{}        % left curly bracket
\def\RB{\right\}}       % right curly bracket
\def\PAR#1#2{ {{\partial #1}\over{\partial #2}} }
\def\PARTWO#1#2{ {{\partial^2 #1}\over{\partial #2}^2} }
\def\PARTWOMIX#1#2#3{ {{\partial^2 #1}\over{\partial #2 \partial #3}} }
\newcommand{\BE}{\begin{displaymath}}
\newcommand{\EE}{\end{displaymath}}
\newcommand{\BNE}{\begin{equation}}
\newcommand{\ENE}{\end{equation}}
\newcommand{\BEA}{\begin{eqnarray}}
\newcommand{\EEA}{\nonumber\end{eqnarray}}
\newcommand{\EL}{\nonumber\\}
\newcommand{\la}[1]{\label{#1}}
\newcommand{\ie}{{\em i.e.\ }}
\newcommand{\eg}{{\em e.\,g.\ }}
\newcommand{\cf}{cf.\ }
\newcommand{\etc}{etc.\ }
\newcommand{\Tr}{{\rm tr}}
\newcommand{\etal}{{\it et al.}}
\newcommand{\OL}[1]{\overline{#1}\ } % overline
\newcommand{\OLL}[1]{\overline{\overline{#1}}\ } % double overline
\newcommand{\OON}{\frac{1}{N}} % "one over N"
\newcommand{\OOX}[1]{\frac{1}{#1}} % "one over X"



\begin{document}
\Large
\centerline{\sc{Recitation Questions}}
\normalsize
\centerline{\sc{15 February}}

\begin{enumerate}

\item Draw force diagrams for the following, labeling the forces in question, where the length of your arrows reflects the relative magnitude of the forces.
 Make sure that the vector sum $\sum \vec F$ points in the direction of the object's acceleration.

\begin{enumerate}

\item A person stands in a subway car that is moving at a constant speed.
\vspace{2in}
\item A person stands in a subway car that is accelerating forward; the person is holding onto a bar.

\vspace{2in}
\item A car drives on a straight highway at a constant speed. (The force that propels the car forward is the traction of the wheels.)

\vspace{2in}
\item A car drives on a straight highway and is accelerating.

\vspace{2in}
\item A rock hangs from the roof of a car by a string. The car is driving at a constant speed.

\vspace{2in}
\item A rock hangs from the roof of a car by a string; the car is accelerating forward.

\end{enumerate}

\newpage
\item {\it (This problem also appears on your homework. Work on it in your groups; this work will translate directly to your homework that you will turn in Friday.)}

Three books, each of mass $m$, sit on a table. Call them, from the bottom
to the top, A, B, and C.

\begin{enumerate}
\item Draw force diagrams for all the books. Each of you should first do this
separately, without consultation with your groupmates. Then compare your 
diagrams, and discuss what the correct diagram is. 
\vspace{4in}

\item Do any of the forces that you drew in your diagrams in step 1 comprise
a Newton's-third-law pair? If so, note that they are equal when you perform the algebra.

\vspace{1in}

\item Write down a statement of Newton's law ($\sum \vec F = m \vec a$) for 
each of the three books.

\vspace{2in}

\item This will give you three equations. Between these three equations and the information from Newton's third law, you can calculate the size of all of the forces.
Do that.

\vspace{4in}

\item Does this match your intuitive idea of what the solution ought to be?
\vspace{2in}
\end{enumerate}
\end{enumerate}
\end{document}

