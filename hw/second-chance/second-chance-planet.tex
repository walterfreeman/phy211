\documentclass[12pt]{article}
\setlength\parindent{0pt}
%\usepackage{fullpage}
\usepackage{amsmath}
\usepackage{hyperref}
\usepackage[margin=2cm]{geometry}
\usepackage{graphicx}
\setlength{\parskip}{4mm}
\def\LL{\left\langle}   % left angle bracket
\def\RR{\right\rangle}  % right angle bracket
\def\LP{\left(}         % left parenthesis
\def\RP{\right)}        % right parenthesis
\def\LB{\left\{}        % left curly bracket
\def\RB{\right\}}       % right curly bracket
\def\PAR#1#2{ {{\partial #1}\over{\partial #2}} }
\def\PARTWO#1#2{ {{\partial^2 #1}\over{\partial #2}^2} }
\def\PARTWOMIX#1#2#3{ {{\partial^2 #1}\over{\partial #2 \partial #3}} }
\newcommand{\BE}{\begin{displaymath}}
\newcommand{\EE}{\end{displaymath}}
\newcommand{\BNE}{\begin{equation}}
\newcommand{\ENE}{\end{equation}}
\newcommand{\BEA}{\begin{eqnarray}}
\newcommand{\EEA}{\nonumber\end{eqnarray}}
\newcommand{\EL}{\nonumber\\}
\newcommand{\la}[1]{\label{#1}}
\newcommand{\ie}{{\em i.e.\ }}
\newcommand{\eg}{{\em e.\,g.\ }}
\newcommand{\cf}{cf.\ }
\newcommand{\etc}{etc.\ }
\newcommand{\Tr}{{\rm tr}}
\newcommand{\etal}{{\it et al.}}
\newcommand{\OL}[1]{\overline{#1}\ } % overline
\newcommand{\OLL}[1]{\overline{\overline{#1}}\ } % double overline
\newcommand{\OON}{\frac{1}{N}} % "one over N"
\newcommand{\OOX}[1]{\frac{1}{#1}} % "one over X"



\begin{document}
\begin{center}
\Large
\sc Second Chance Exercise - Circular Motion and Accelerating Frames / Exam 2 Planet Problem \rm

\pagenumbering{gobble}




\normalsize
This second chance exercise is for people who want to review the dynamics of uniform circular motion and the apparent effects of being in an accelerating place.

If one of your lowest three exam problem grades was on the ``planet problem'', you may complete it and turn it in to earn an extra homework grade, which we will help you earn 100\% on. This will help your end-of-term average.

If you do this, it is due on the last day of recitation.

\end{center}

\vspace{1.5in}

This exam problem tested your ability to analyze the forces required to make an object move in uniform circular motion and your ability to relate the experience of someone standing on an accelerating surface to the forces required to make them move in the same way as the surface.

\begin{enumerate}
	
		\item What force (on your force diagram for the person) determines ``how heavy they feel''?
	
		\item Compare this exam problem to Homework 3 questions 1 and 2, in which a person is in an elevator as it accelerates up and down. How are those problems the same as the ``planet problem'', and how are they different?
		
	\item Our planet doesn't rotate nearly this fast, but {\it does} this effect apply on Earth with its current rate of rotation? Discuss the relation of questions 2 and 3 from Week 6, Day 1's recitation (at \url{https://walterfreeman.github.io/phy211/recitation/week6/recitation-gravity.pdf}) to this problem

	\item Identify two problems in Practice Exam 2 (at \url{https://walterfreeman.github.io/phy211/practice-exam-2-2023.pdf}) where you need to analyze how a person in an accelerating place (on the surface of an accelerating planet, in an accelerating box, etc.) feels and how that relates to the forces on them. How is each of these situations the same as the rapidly rotating planet, and how is each of them different?
	

	
    \item Write a similar problem where you have to analyze the forces required to make an object move in uniform circular motion, then relate those forces to the experience of a person who experiences them. Then solve it. {\it (Particularly good problems here will earn extra credit!)}
\end{enumerate}



\end{document}
