\documentclass[12pt]{article}
\setlength\parindent{0pt}
\usepackage{fullpage}
\usepackage{amsmath}
\usepackage{array}
\usepackage{hyperref}
\usepackage{graphicx}
\setlength{\parskip}{4mm}
\def\LL{\left\langle}   % left angle bracket
\def\RR{\right\rangle}  % right angle bracket
\def\LP{\left(}         % left parenthesis
\def\RP{\right)}        % right parenthesis
\def\LB{\left\{}        % left curly bracket
\def\RB{\right\}}       % right curly bracket
\def\PAR#1#2{ {{\partial #1}\over{\partial #2}} }
\def\PARTWO#1#2{ {{\partial^2 #1}\over{\partial #2}^2} }
\def\PARTWOMIX#1#2#3{ {{\partial^2 #1}\over{\partial #2 \partial #3}} }
\newcommand{\vs}[1]{\vspace{#1}}
\newcommand{\BC}{\begin{center}}
\newcommand{\EC}{\end{center}}
\newcommand{\BI}{\begin{itemize}}
\newcommand{\EI}{\end{itemize}}
\newcommand{\BE}{\begin{enumerate}}
\newcommand{\EE}{\end{enumerate}}
\newcommand{\BNE}{\begin{equation}}
\newcommand{\ENE}{\end{equation}}
\newcommand{\BEA}{\begin{eqnarray}}
\newcommand{\EEA}{\nonumber\end{eqnarray}}
\newcommand{\EL}{\nonumber\\}
\newcommand{\la}[1]{\label{#1}}
\newcommand{\ie}{{\em i.e.\ }}
\newcommand{\eg}{{\em e.\,g.\ }}
\newcommand{\cf}{cf.\ }
\newcommand{\etc}{etc.\ }
\newcommand{\Tr}{{\rm tr}}
\newcommand{\etal}{{\it et al.}}
\newcommand{\OL}[1]{\overline{#1}\ } % overline
\newcommand{\OLL}[1]{\overline{\overline{#1}}\ } % double overline
\newcommand{\OON}{\frac{1}{N}} % "one over N"
\newcommand{\OOX}[1]{\frac{1}{#1}} % "one over X"


\pagenumbering{gobble}
\begin{document}
\Large
\centerline{\sc{Recitation Problems}}
\large
\BC
\sc
Wednesday, January 25 -- Kinematics in One Dimension
\EC
\normalsize

A car travels down the road at $v_0=30$ m/s when it applies its brakes suddenly. The car's brakes cause it to
decelerate at a rate of $\rm 5\, m/\rm s^2$. In this problem, you'll find
how much time it takes for the car to stop, and how far it travels before it 
does.




\begin{enumerate}
\item {\bf Step 1:} Write expressions for the position $x(t)$ and the velocity $v(t)$ in terms
of $v_0$, $t$, and the acceleration $a$. 

\vs{1in}

\item {\bf Step 2:} Consider the question ``How much time does it take for the
car to stop?'' Write a sentence that tells you how to answer this question
in terms of your physical variables; it will have the form ``What is the 
value of $\rule{1cm}{0.15mm}$ when $\rule{1cm}{0.15mm}$ equals $\rule{1cm}{0.15mm}$''. Which of the expressions you wrote in step 1 will be helpful in 
answering this question?

\vs{1.5in}

\item {\bf Step 3:} Do the algebra indicated by your sentence to find an
algebraic expression for the time required.

\vs{1.5in}

\item {\bf Step 4:} Substitute in the numeric quantities to get an answer. Does it make sense?

\vs{1in}

\end{enumerate}

You can follow these steps for {\it all} kinematics problems. Now, apply them
to the other part of the question:

\begin{enumerate}
\item Consider the question ``How far does the car travel before it stops?'' 
 Write a sentence that tells you how to answer this question
in terms of your physical variables; it will have the form ``What is the
value of $\rule{1cm}{0.15mm}$ at the time when $\rule{1cm}{0.15mm}$
equals $\rule{1cm}{0.15mm}$?

\vs{1.5in}

\item Do the algebra your sentence suggests that you do. (You might want to 
refer to the first part of this question.)

\vs{1.5in}

\item Substitute in the numeric quantities to get an answer.

\vs{1in}

\item If the car is driving twice as fast, by what factor does its stopping
distance increase? {\bf You should be able to answer this question without
a calculator.}
\end{enumerate}

\newpage

A person throws a baseball straight upward at a velocity $v_0 = 20$ m/s
to a friend standing on top of a building that is $h=15$ m tall. 
 
Her friend isn't quite ready to catch it; it goes over his head, up to the 
top of its flight, and then comes back down; he catches it on the way back down.

\begin{enumerate}
\item
How long is the ball in the air? To answer this question, follow the steps
of the previous problem. (These steps can be used for {\it any} kinematics
problem.) Your sentence and algebra will be different, of course.

\item
You have to use the quadratic formula for this problem. How do you know
which solution from the quadratic formula is the one you want?

\item 
Suppose the height of the building was $h'=25$ m. What is the time now?
How do you interpret what the quadratic formula tells you?
\end{enumerate}

\newpage

A rocket is fired straight up. Its motor burns for ten seconds. While the rocket's motor burns, it
accelerates upward at 15 $\rm m/\rm s^2$; 
after it burns out, the rocket is in freefall.

Note: As you go through this problem, sketch position vs. time, velocity vs. time, and acceleration
vs. time graphs for the rocket, as you gather the bits of information that enable you to make
them. The part of the problem that asks you to graph them is listed last, but you shouldn't wait to
the end to do it; you should do it as you go. Don't worry about putting numbers
in until the end, but you can still make a sketch which will guide your thinking.


\begin{enumerate}
\item
Since the rocket's acceleration changes in flight, you can't use the constant-acceleration kinematics
formulae we've learned to understand the whole flight at once. How can you use constant-acceleration
kinematics to understand this problem?

\item On a separate piece of paper, make a table of all of the variables you will
use for this problem, and what they mean. You will want to introduce a variable, for 
instance, for ``how fast the rocket is going when its motor burns out''. 

\item
How high above the ground is the rocket once its motor burns out?

\vs{2in}

\item
How fast is the rocket traveling once its motor burns out?

\vs{2in}

\item
How fast is the rocket traveling when it reaches its maximum height?

\vs{2in}

\item What is that maximum height?

\vs{2in}

\item How long does it take for the rocket to land back on the ground?

\vs{2in}

\item Make position vs. time, velocity vs. time, and acceleration vs. time
graphs for the rocket. Which of these is most accessible to make first?
\end{enumerate}
\end{document}
