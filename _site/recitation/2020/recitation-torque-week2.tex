\documentclass[12pt]{article}
\setlength\parindent{0pt}
\usepackage{fullpage}
\usepackage{amsmath}
\usepackage{graphicx}
\usepackage[margin=0.5in]{geometry}
\setlength{\parskip}{4mm}
\def\LL{\left\langle}   % left angle bracket
\def\RR{\right\rangle}  % right angle bracket
\def\LP{\left(}         % left parenthesis
\def\RP{\right)}        % right parenthesis
\def\LB{\left\{}        % left curly bracket
\def\RB{\right\}}       % right curly bracket
\def\PAR#1#2{ {{\partial #1}\over{\partial #2}} }
\def\PARTWO#1#2{ {{\partial^2 #1}\over{\partial #2}^2} }
\def\PARTWOMIX#1#2#3{ {{\partial^2 #1}\over{\partial #2 \partial #3}} }
\newcommand{\BI}{\begin{itemize}}
\newcommand{\EI}{\end{itemize}}
\newcommand{\BE}{\begin{displaymath}}
\newcommand{\EE}{\end{displaymath}}
\newcommand{\BNE}{\begin{equation}}
\newcommand{\ENE}{\end{equation}}
\newcommand{\BEA}{\begin{eqnarray}}
\newcommand{\EEA}{\nonumber\end{eqnarray}}
\newcommand{\EL}{\nonumber\\}
\newcommand{\la}[1]{\label{#1}}
\newcommand{\ie}{{\em i.e.\ }}
\newcommand{\eg}{{\em e.\,g.\ }}
\newcommand{\cf}{cf.\ }
\newcommand{\etc}{etc.\ }
\newcommand{\Tr}{{\rm tr}}
\newcommand{\etal}{{\it et al.}}
\newcommand{\OL}[1]{\overline{#1}\ } % overline
\newcommand{\OLL}[1]{\overline{\overline{#1}}\ } % double overline
\newcommand{\OON}{\frac{1}{N}} % "one over N"
\newcommand{\OOX}[1]{\frac{1}{#1}} % "one over X"

\pagenumbering{gobble}

\begin{document}
\Large
\centerline{\sc{Recitation Exercises}}
\normalsize
\centerline{\sc{24 April}}

\begin{center}
	\Large Exercise 1: on combining rotation and translation
\end{center}

A Yo-Yo consists of a cylinder of radius $R$ with a thin slit cut in it. Inside the slit is a smaller inner cylinder of radius $r$ with a string attached to it and then wound around the cylinder. Note that the moment of inertia of a cylinder of radius $R$ is $I=\frac{1}{2}mR^2$.

If a person holds the end of the string and drops the Yo-Yo, it will begin to spin as it falls, unwinding the string as it does. 


a) Suppose that you have a red Yo-Yo with $r=0.1 R$ (that is, with a very small inner cylinder) and a blue Yo-Yo with $r=0.4 R$ (with a thicker inner cylinder). Predict which one will fall faster when it is dropped, and describe why it will do so. \textit{(You shouldn't do any calculations here.)}

\vspace{2in}

b) Now, you'll calculate the downward acceleration of the Yo-Yo. In this case, the Yo-Yo both {\it translates} and {\it rotates} as it does so

Start by drawing an extended force diagram for the Yo-Yo, showing all the forces acting on it {\it and where they act}.

\vspace{2in}

\newpage

c) Since it both translates and rotates, you will need both $\vec F = m \vec a$ to relate the forces on it to its translational acceleration and $\tau = I \alpha$ to relate the torques on it to its linear acceleration. Construct both of these equations, using the forces that appear on your force diagram. \textit{(Hint: The tension in the string both applies a torque to the Yo-Yo and affects its translational acceleration.)}

\vspace{2.5in}

d) In the above two equations, you will have three unknowns: the tension in the string, the translational acceleration, and the angular acceleration $\alpha$. However, you can relate two of them to each other. What is that relation? \textit{(Hint: Think carefully about minus signs here!)}}

\vspace{1in}

e) Now you should have enough information to solve for $a$ in terms of $g$, $r$, and $R$. Once you have a value for your acceleration, call your GTA or coach over and have them check your work. Discuss with them whether the red or blue Yo-Yo in part (a) would fall faster. 

\newpage

\begin{center}
	\Large Exercise 2
\end{center}

Consider the demonstration you saw in class yesterday. A person stands on top of a platform that is free to rotate.

a) Estimate the moment of inertia of the person around their center. You will need to figure out which of our simple shapes best approximates a person, then estimate the person's radius and mass. \textit{(The table of moments of inertia is at the end.)}

\vspace{1in}

b) Someone else standing on the ground carries a bicycle wheel filled with concrete of mass $m=5$ kg with radius 30 cm. They make the bicycle wheel spin at an angular velocity $\omega = 20$ radians/sec, turn it so that it is spinning clockwise when viewed from above, then hand it to the person standing on the platform. 

When the person standing on the platform grabbed the wheel to stop it from spinning, they began to rotate slowly. Determine which direction and how fast they begin to rotate after they do this.

\vspace{3in}
\newpage

c) Imagine now that instead of grabbing the wheel, they turned it upside down, so it would be spinning counterclockwise rather than clockwise when seen from above. Without doing any mathematics, predict what should happen. Call your TA or coach over to discuss with your group.

\vspace{2in}

d) Now, calculate which direction they will rotate and how fast when they rotate the wheel upside down. Does the result of your calculation agree with your prediction?
\vspace{3in}

\begin{center}
	\includegraphics[width=3in]{moment-table.png}
\end{center}
% A bucket of mass $m$ hangs from a string wound around a pulley 
%(a solid cylinder) with mass $M$ and radius $r$. When the bucket is
%released, it falls, unwinding the string.
%
%\begin{enumerate}
%
%\item Draw force diagrams for the bucket and the pulley. Note that since the pulley rotates, you will need
%to draw an extended force diagram for it, drawing the object and labeling where each force acts.
%
%\vspace{3in}
%
%\item In terms of the forces in your force diagrams, write an expression for the net torque on the pulley.
%
%\vspace{1in}
%
%\item Write down Newton's laws of motion -- $\sum \vec F = m \vec a$ for translation, and $\sum \tau = I \alpha$
%-- for each object. (One object moves, and the other turns...)
%
%\vspace{2in}
%
%
%\newpage
%
%\item What is the relationship between the angular acceleration $\alpha$ of the pulley and the linear acceleration
%$a$ of the bucket? (The answer may be different depending on how you have drawn your pictures and your choice of
%coordinate system.)
%
%\vspace{1in}
%
%\item Calculate the acceleration of the bucket in terms of $m$ and $M$.
%
%\vspace{4in}
%
%\item Suppose that the pulley were a hollow cylinder with the same mass. How would this acceleration change?

%\newpage
%\end{enumerate}

A simplified model of a car or truck can be thought of as shown below. Consider
the body of the vehicle to be a uniform plate of mass $M$, supported by the
normal force of two axles, each located a distance $L/6$ from each end.
The engine, also of mass $M$, is located a distance $L/6$ from the front. (Don't worry about 
distinguishing the left wheels from the right ones; you can treat ``front wheels'' and ``back wheels''
as single forces.) Note that the car is pointed to the left here (since the engine is in the front).

Some terminology, for those who are not familiar:

\BI
\item ``Front wheel drive'' means that the engine is coupled only to the front wheels, and is only able to use 
them to provide forward traction. This means that the maximum traction is $\mu_s F_{N, \rm front}$.
\item ``Rear wheel drive'' means that the engine is coupled only to the back wheels, and is only able to use 
them to provide forward traction. This means that the maximum traction is $\mu_s F_{N, \rm back}$.
\item ``All-wheel drive'' or ``four-wheel drive'' mean that the engine is coupled to both sets of wheels,
and may use the static friction of both of them to provide traction.
\EI

\begin{center}\includegraphics[width=0.5\textwidth]{car-crop.pdf}\end{center}
\begin{enumerate}

\item Without doing any mathematics, do you expect the normal force from the front wheels 
or from the rear wheels to be larger? Why?

\vspace{1in}

\item Draw an extended force diagram for the vehicle.

\newpage

\item Calculate the normal forces $F_{N1}$ and $F_{N2}$ exerted by each axle.

\vspace{3in}

\item Suppose that the coefficient of friction between the wheels and the ground is $\mu_s$. What is the maximum
traction force that the car can apply if it is front wheel drive? (Most small cars
are front wheel drive.)

\vspace{3.5in}

\item What is the maximum traction force that the vehicle can apply if it is rear wheel drive? (Most trucks are rear
wheel drive.)

\vspace{3.5in}

\item Often people with two-wheel-drive pickup trucks will pile snow in the back of the truck during the Syracuse winter.
Why do they do this?

As a hint, the thing that determines whether you will get stuck or not in slippery conditions is often the ratio between the traction force and the total mass of the vehicle.



\end{enumerate}
\newpage

\newpage


\Large
\centerline{\sc{Recitation Questions}}
\normalsize
\centerline{\sc{26 April}}



%Consider a Ping-Pong ball resting on a smooth table. (Since this is a hollow shell, its moment of inertia is 
%$I=\frac{2}{3}mr^2$.) The coefficient of static friction between the ball and the table is $\mu_s$, 
%the coefficient of kinetic friction is $\mu_k$, and the 
%ball has a mass $m$ and radius $r$.
%
%A gentle breeze begins to blow, exerting a force $F_w$ on the ball that is much less than $mg$. (Since this force is spread out uniformly over the ball,
%you may treat it as acting at the center of the ball.)
%
%\begin{enumerate}
%
%\item Draw an extended force diagram for the ball.
%
%\vspace{4in}
%
%\newpage
%
%\item How large will the frictional force between the ball and the table be? (Note that $\mu_s F_N$ is only the 
%{\it maximum} force of static friction.)
%
%\vspace{3in}
%
%\item Now, suppose that a gust of wind strikes the ball, so that $F_w$ becomes larger. How large can the wind force
%be so that the ball rolls without slipping, instead of skidding?
%
%\vspace{2in}
%
%\item Suppose that the wind force is larger than this, so that the ball {\it skids}. Calculate both its angular
%acceleration and its translational acceleration.
%\end{enumerate}

\newpage

\begin{minipage}{0.5\textwidth}
A table has a mass $m$, and its height is $2/3$ of its width. The legs of the table are very light; all of the mass is in the top.
The legs of the table are located at the ends, and a rope is tied to one side; a tension $T$ is applied to the rope. Assume that the coefficient of friction between
the legs and the ground is very large, so that the table does not slide.
\end{minipage}
\begin{minipage}{0.5\textwidth}
\begin{center}
\includegraphics[width=0.7\textwidth]{table-crop.pdf}
\end{center}
\end{minipage}

In this problem, you will calculate the required tension $T$ to tip the table.

\begin{enumerate}

\item Draw a force diagram for the table. Indicate your choice of pivot.

\newpage

\item What tension force $T$ is required to tip the table (so that the back legs come off the ground)?
(Hint: What is true about the normal forces on the table when it begins to tip?)

\vspace {4in}

\item What coefficient of static friction between the legs and the floor is
required so that the table tips, rather than sliding?

\end{enumerate}
 \end{document}
