% DPF 09 talk on strangeness in nucleon

\documentclass[10pt]{beamer}
\usepackage{amsmath}
\usepackage{mathtools}
%\documentclass[12pt]{beamerthemeSam.sty}
\usepackage{epsf}
%\usepackage{pstricks}
%\usepackage[orientation=portrait,size=A4]{beamerposter}
\geometry{paperwidth=160mm,paperheight=120mm}
%DT favorite definitions
\def\LL{\left\langle}	% left angle bracket
\def\RR{\right\rangle}	% right angle bracket
\def\LP{\left(}		% left parenthesis
\def\RP{\right)}	% right parenthesis
\def\LB{\left\{}	% left curly bracket
\def\RB{\right\}}	% right curly bracket
\def\PAR#1#2{ {{\partial #1}\over{\partial #2}} }
\def\PARTWO#1#2{ {{\partial^2 #1}\over{\partial #2}^2} }
\def\PARTWOMIX#1#2#3{ {{\partial^2 #1}\over{\partial #2 \partial #3}} }

\def\rightpartial{{\overrightarrow\partial}}
\def\leftpartial{{\overleftarrow\partial}}
\def\diffpartial{\buildrel\leftrightarrow\over\partial}

\def\BI{\begin{itemize}}
\def\EI{\end{itemize}}
\def\BE{\begin{displaymath}}
\def\EE{\end{displaymath}}
\def\BEA{\begin{eqnarray*}}
\def\EEA{\end{eqnarray*}}
\def\BNEA{\begin{eqnarray}}
\def\ENEA{\end{eqnarray}}
\def\EL{\nonumber\\}


\newcommand{\map}[1]{\frame{\frametitle{\textbf{Course map}}
\centerline{\includegraphics[height=0.86\paperheight]{../../map/#1.png}}}}
\newcommand{\wmap}[1]{\frame{\frametitle{\textbf{Course map}}
\centerline{\includegraphics[width=0.96\paperwidth]{../../map/#1.png}}}}

\newcommand{\etal}{{\it et al.}}
\newcommand{\gbeta}{6/g^2}
\newcommand{\la}[1]{\label{#1}}
\newcommand{\ie}{{\em i.e.\ }}
\newcommand{\eg}{{\em e.\,g.\ }}
\newcommand{\cf}{cf.\ }
\newcommand{\etc}{etc.\ }
\newcommand{\atantwo}{{\rm atan2}}
\newcommand{\Tr}{{\rm Tr}}
\newcommand{\dt}{\Delta t}
\newcommand{\op}{{\cal O}}
\newcommand{\msbar}{{\overline{\rm MS}}}
\def\chpt{\raise0.4ex\hbox{$\chi$}PT}
\def\schpt{S\raise0.4ex\hbox{$\chi$}PT}
\def\MeV{{\rm Me\!V}}
\def\GeV{{\rm Ge\!V}}

%AB: my color definitions
%\definecolor{mygarnet}{rgb}{0.445,0.184,0.215}
%\definecolor{mygold}{rgb}{0.848,0.848,0.098}
%\definecolor{myg2g}{rgb}{0.647,0.316,0.157}
\definecolor{abtitlecolor}{rgb}{0.0,0.255,0.494}
\definecolor{absecondarycolor}{rgb}{0.0,0.416,0.804}
\definecolor{abprimarycolor}{rgb}{1.0,0.686,0.0}
\definecolor{Red}           {cmyk}{0,1,1,0}
\definecolor{Grey}           {cmyk}{.7,.7,.7,0}
\definecolor{Lg}           {cmyk}{.4,.4,.4,0}
\definecolor{Blue}          {cmyk}{1,1,0,0}
\definecolor{Green}         {cmyk}{1,0,1,0}
\definecolor{Brown}         {cmyk}{0,0.81,1,0.60}
\definecolor{Black}         {cmyk}{0,0,0,1}

\usetheme{Madrid}


%AB: redefinition of beamer colors
%\setbeamercolor{palette tertiary}{fg=white,bg=mygarnet}
%\setbeamercolor{palette secondary}{fg=white,bg=myg2g}
%\setbeamercolor{palette primary}{fg=black,bg=mygold}
\setbeamercolor{title}{fg=abtitlecolor}
\setbeamercolor{frametitle}{fg=abtitlecolor}
\setbeamercolor{palette tertiary}{fg=white,bg=abtitlecolor}
\setbeamercolor{palette secondary}{fg=white,bg=absecondarycolor}
\setbeamercolor{palette primary}{fg=black,bg=abprimarycolor}
\setbeamercolor{structure}{fg=abtitlecolor}

\setbeamerfont{section in toc}{series=\bfseries}

%AB: remove navigation icons
\beamertemplatenavigationsymbolsempty
\title[Work and potential energy]{
  \textbf {Work and potential energy}\\
%\centerline{}
%\centering
%\vspace{-0.0in}
%\includegraphics[width=0.3\textwidth]{propvalues_0093.pdf}
%\vspace{-0.3in}\\
%\label{intrograph}
}

\author[W. Freeman] {Physics 211\\Syracuse University, Physics 211 Spring 2015\\Walter Freeman}

\date{\today}

\begin{document}

\frame{\titlepage}

\frame{\frametitle{\textbf{Announcements}}
\BI
\item{Homework and Mastering Physics assignments due Friday}
\item{Your next homework assignment will be longer and due two weeks from tomorrow. Start early!}
\item{Those of you who didn't pick up your exams last Friday can do so on Wednesday}
\item{If your exam was misgraded, grade appeals will be handled the same way as before (and faster!)}
\EI
}

\frame{\frametitle{\textbf{Exam 2 recap}}
  \centerline{(Statistics from only half of the TA's)}
  \centerline{\includegraphics[width=0.5\textwidth]{grades-crop.pdf}}
    \Large
    \centerline{Mean grade was 68.3; median grade was 71.5}
    \centerline{This exam was quite hard, and had no extra credit}
    \centerline{Your performance was above the historical average}
    \centerline{\color{Red}Be proud of yourselves (again!)}
  }

\frame{\frametitle{\textbf{Where we've been and where we're going}}
  \BI
  \large
\item{Last time: kinetic energy and the work-energy theorem}
\item{This time: the idea of potential energy and conservation of energy}
\BI
\item{Potential energy: ``the most meaningful bookkeeping trick in physics''}
\item{Lets us understand many phenomena without difficult mathematics}
\item{Conservation of energy: there's always the same amount of energy, and it just changes forms}
\EI
\EI
}

\frame{\frametitle{\textbf{Review: kinetic energy}}
\large
We will see that things are often simpler when we look at something called ``energy''
\BI
\item{Basic idea: don't treat $\vec a$ and $\vec v$ as the most interesting things any more}
\item{Treat $v^2$ as fundamental: $\frac{1}{2}mv^2$ called ``kinetic energy''}
  \EI
  \bigskip
  \begin{columns}
    \column{0.5\textwidth}
    \color{Blue}
    \centerline{Previous methods:}
    \BI
    \color{Blue}
  \item{Velocity is fundamental}
  \item{Force: causes velocities to change over time}
  \item{Intimately concerned with vector quantities}
  \EI
    \column{0.5\textwidth}
    \color{Red}
    \centerline{Energy methods:}
    \BI
    \color{Red}
  \item{$v^2$ (related to kinetic energy) is fundamental}
  \item{Force: causes KE to change over distance}
  \item{Energy is a {\it scalar}}
    \EI
  \end{columns}
  
  \bigskip
  \bigskip
  
  \centerline{Energy methods: useful when you don't know and don't care about time}
}

\frame{\frametitle{\textbf{The work-energy theorem in 1D}}
Last time we saw the ``work-energy theorem'' was a consequence of simple kinematics:
\begin{equation*}
  \frac{1}{2} mv_f^2 - \frac{1}{2} mv_0^2 = F \Delta x
\end{equation*}

\bigskip
\bigskip
\bigskip
\pause

Or in more than one dimension:
\begin{equation*}
  \frac{1}{2} mv_f^2 - \frac{1}{2} mv_0^2 = \vec F \cdot \Delta \vec s = (F_\parallel)(\Delta s) = (F)(\Delta s_\parallel)
\end{equation*}


\bigskip
\bigskip
\bigskip
\pause
Or if the force is not constant:

\begin{equation*}
    \frac{1}{2} mv_f^2 - \frac{1}{2} mv_0^2 = \int \vec F \cdot d\vec s
  \end{equation*}

\pause

Some new terminology:
\BI
\item{$\frac{1}{2}mv^2$ called the ``kinetic energy'' (positive only!)}
\item{$\vec F \cdot \Delta \vec s$ called the ``work'' (negative or positive!)}
\item{\color{Red}``Work is the change in kinetic energy''}
\EI
}

\frame{\frametitle{\textbf{Sample problem: a roller coaster}}
\large
(on document camera)

\pause
\bigskip
\bigskip
\bigskip

Strategy: compute the work done by all the forces and equate that to the change in KE.

\bigskip

Work done by normal force = {\bf zero}!

Work done by gravity = $(F)(\Delta s)_\parallel = mg\Delta y = mg(y_0 - y_f)$

\bigskip
\bigskip

\begin{align*}
  KE_f &- KE_i &=& W_g \\
  \frac{1}{2} mv_f^2 &- 0 &=& mg (y_0 - y_f) \\ 
\end{align*}
  
\pause
\bigskip

$\rightarrow v_f = \sqrt{2g(y_0 - y_f)}$

\bigskip
\pause
\bigskip

\centerline{\color{Red}No detailed knowledge of the motion required!}
}

\frame{\frametitle{\textbf{Potential energy: an accounting trick}}
  \BI
\item{Notice that the work done by gravity depends {\it only} on the change in height.}
\item{Some other forces are like this as well: the work done depends only on initial and final position}
  \BI
\item{These are called {\it conservative forces}}
\item{Soon we'll see that the elastic force is like this too}
\EI
\item{Write the work-energy theorem in terms of the heights:}
\begin{equation*}
\frac{1}{2} mv_f^2 - \frac{1}{2} mv_0^2 = mg (y_0 - y_f) + {\color{Grey}W_{\rm{other}}}
\end{equation*}

\pause
\item{Collect all the ``initial'' things on the left and the ``final'' things on the right:}
  \begin{align*}
    \frac{1}{2} mv_0^2& + mgy_0& + W_{\rm{other}} &=& \frac{1}{2}mv_f^2 &+ mgy_f\\
    KE_0& + GPE_0& + W_{\rm{other}}& =& KE_f &+ GPE_f 
  \end{align*}
\item{Identify $mgy$ as ``gravitational potential energy'': how much work will gravity do if something falls?}
  \EI

\bigskip

\centerline{\color{Red} Potential energy lets us easily calculate the work done by conservative forces}
}

\frame{\frametitle{\textbf{Potential energy: more than accounting!}}
  \BI
\item{Another way to look at the roller coaster: {\color{Red} gravitational potential energy being converted to kinetic energy.}}
\item{This perspective is universal: {\color{Blue}all} forces just convert energy from one sort into another}
\item{Some of these types are beyond the scope of this class, but we should be aware of them!}
\EI
\bigskip
\bigskip
A short history of energy conversion:
\small
\begin{columns}
  \column{0.5\textwidth}  
\BI
\item{Hydrogen in the sun fuses into helium}
\item{Hot gas emits light}
\item{Light shines on the ocean, heating it}
\item{Seawater evaporates and rises, then falls as rain}
\item{Rivers run downhill}
\item{Falling water turns a turbine}
\item{Turbine turns coils of wire in generator}
\item{Electric current ionizes gas}
\item{Recombination of gas ions emits light}
  \EI
  \column{0.5\textwidth}
\BI
\item{Nuclear energy $\rightarrow$ thermal energy}
\item{Thermal energy $\rightarrow$ light}
\item{Light $\rightarrow$ thermal energy}
\item{Thermal energy $\rightarrow$ gravitational potential energy}
\item{Gravitational PE $\rightarrow$ kinetic energy and sound} 
\item{Kinetic energy in water $\rightarrow$ kinetic energy in turbine}
\item{Kinetic energy $\rightarrow$ electric energy}
\item{Electric energy $\rightarrow$ chemical potential energy}
\item{Chemical PE $\rightarrow$ light}
\EI
\end{columns}
}

\frame{\frametitle{\textbf{Potential energy: more than accounting!}}
\BI
\item{This class is just the study of motion: we can't treat light or nuclear energy here.}
\item{... but in physics as a whole, the {\it conservation of energy} -- that processes just change energy from one form to another -- is universal!}
\item{Conservation of energy is one of the most tested, ironclad ideas in science}
\item{Nuclear and chemical potential energy: nuclear forces do mechanical work on particles, much like gravity}
\item{Light, and others: kinetic energy of little particles called ``photons''}
\item{Heat: kinetic energy of atoms in random motion}
\item{Sound: kinetic energy of atoms in coordinated motion}
\item{Food: Just chemical potential energy...}
\item{... so all of these things aren't as far removed from mechanics after all!}
\item{Einstein: ``Mass is just another form of energy''}
  \EI
}



\frame{\frametitle{\textbf{A new force: elasticity and Hooke's law}}
To best see how this can be useful, let's introduce a new force: elasticity.

\begin{itemize}

  \item{Springs have a particular length that they like to be: ``equilibrium length'' $L_0$}
  \item{A spring stretched to be longer than this pulls inward to shorten itself}
  \item{A spring compressed to be shorter than this pushes outward to lengthen itself}
  \item{Flexible things like strings and ropes only pull; they kink instead of compressing}
  \item{The force is proportional to the deviation from the optimum length}
\EI
\centerline{    \includegraphics[width=0.4\textwidth]{hooke.png}}
\Large
\centerline{$F_{\rm{elastic}} = -k (L - L_0) = -k \Delta x$ (Hooke's law)}

\bigskip
\pause
\large
$k$ is called the ``spring constant'':

\BI
\item{Measures the stiffness of the spring/rope}
\item{Units of newtons per meter: ``restoring force of $k$ newtons per meter of stretch''}
  \EI
}


\frame{\frametitle{\textbf{A simple spring problem: done with the work-energy theorem}}
\Large
A person of mass $m=100 kg$ falls from a height of $h=3$m onto a trampoline. If the person makes an impression $d=40$ cm deep on the trampoline when he lands, what is the spring constant?

\pause
\bigskip
\bigskip
\bigskip

\normalsize

\BI
\item{Initial kinetic energy + work done by spring + work done by gravity = final kinetic energy}
  \BI
\item{Need to use the integral form of the work-energy theorem since the force isn't constant}
  \EI
\item{The person begins and ends at rest, so we know the initial and final kinetic energy is zero}
\item{The trampoline begins at its equilibrium position}
  \pause
\item{\color{Red}$W_{\rm{grav}} = (mg)(h+d)$}
  \pause
\item{\color{Blue}$W_{\rm{elas}} = \int_0^{-d} \, kx \, dx = -\frac{1}{2} kd^2$}
  \pause
\item{$KE_0 + {\color{Red}W_{\rm{grav}}} + {\color{Blue}W_{\rm{elas}}} = KE_f$}
\item{$0 + \color{Red}(mg)(h+d) \color{Blue}-\frac{1}{2} kd^2 \color{Black}= 0$}
\item{$k = \frac{mg(h+d)}{2d^2}$}
  \EI
}
  

\frame{\frametitle{\textbf{Potential energy stored in a spring}}
\Large 
We saw that an object at height $h$ has gravitational potential energy $mgh$. Can we do something similar for springs?

\large

\bigskip
\bigskip
\pause
\BI
\item{Potential energy, remember, is the work done by some force as it returns to some ``zero'' position.}
\item{A natural choice is $\Delta x=0$, the equilibrium position of the spring.}
  \EI

\Large

``How much work is done by a spring as it goes from $\Delta x=a$ to $\Delta x=0$?

\bigskip
\bigskip

\centerline{$U_{\rm{elastic}} = W_{a \rightarrow 0} = \int_a^0 \, -kx \, dx = \int_0^a \, kx \, dx = \frac{1}{2}ka^2$}

\pause
\bigskip
\bigskip

Now that we have this, we never have to do this integral again!

\bigskip

\centerline{$U_{\rm{elastic}} = \frac{1}{2} kx^2$, where $x$ is the distance from equilibrium}

}


  
  
   
  \frame{\frametitle{\textbf{A simple spring problem: done with potential energy}}
\Large
A person of mass $m=100$ kg falls from a height of $h=3$m onto a trampoline. If the person makes an impression $d=40$ cm deep on the trampoline when he lands, what is the spring constant?

\pause
\bigskip
\bigskip
\bigskip

\normalsize

\BI
\item{Initial total energy + work done by other forces = final total energy}
\item{We have no ``other forces'': we're accounting for gravity and elasticity using potential energy}
\item{The person begins and ends at rest, so we know the initial and final kinetic energy is zero}
\item{Put $y=0$ at the surface of the trampoline}
  \pause
\item{$U_{\rm{grav},0} = mgh$}
\item{$U_{\rm{elas},0} = 0$ (trampoline starts at equilibrium)}
\item{$U_{\rm{grav},f} = -mgd$ (the person falls below $y=0$; PE can be negative!)}
\item{$U_{\rm{elas},f} = \frac{1}{2}kd^2$ (see last slide)}
  \pause
\item{${\color{Lg}KE_0} + U_{\rm{grav},0} + {\color{Lg}U_{\rm{elas},0}} = {\color{Lg}KE_f} + U_{\rm{grav},f} + U_{\rm{elas},f}$}
  \pause
\item{$0 + mgh + 0 = 0 + (-mgd) + \frac{1}{2}kd^2$ (Same terms, maybe on different side)}
  \pause
\item{$k = \frac{mg(h+d)}{2d^2}$}
  \EI
}
  \frame{\frametitle{\textbf{That spring problem: a recap}}
     \large
     \centerline{We don't care about time $\rightarrow$ energy methods}

     \bigskip
     \normalsize

     \begin{columns}
     \column{0.5\textwidth}
     \centerline{Work-energy theorem}
     \column{0.5\textwidth}
     \centerline{Potential energy treatment}
   \end{columns}
   \begin{columns}
     \column{0.5\textwidth}

     \BI
   \item{Initial KE + all work done = final KE}
   \item{Need to compute work done by gravity: easy}
   \item{Need to compute work done by spring: harder\\(need to integrate Hooke's law)}
   \EI
   \column{0.5\textwidth}
   \BI
 \item{Initial KE + initial PE + other work = final KE + final PE}
 \item{No ``other work'' in this problem; all forces have a PE associated}
 \item{Need to know the expressions for PE:}
   \BI
 \item{$U_{\rm{grav}} = mgy$}
 \item{$U_{\rm{elas}} = \frac{1}{2}kx^2$ (x is the distance from the equilibrium point)}
   \EI
 \item{No integrals required (they're baked into the above)}
   \EI
 \end{columns}
 }

 \frame{\frametitle{\textbf{Potential energy with other forces}}
   \large
   What about associating a potential energy with other forces?\normalsize
   \BI
 \item{Friction is a no-go: the work done by friction depends on the path, not just where you start and stop}
 \item{``Ephemeral'' forces like tension and normal force are easiest to deal with by computing work directly}
 \item{The other force we've studied that is easily associated with a potential energy is {\bf universal gravitation}}
   \BI
 \item{Need to choose a point to set $U=0$; here we choose $r = \infty$}
 \item{$U_G$ = ``work done by gravity on $m_1$ when it moves infinitely far from $m_2$}
   \EI
\EI

\bigskip
\bigskip

\Large

   $F_G = \frac{Gm_1 m_2}{r^2}$
   
\bigskip

   $W_G = \int_R^\infty\, -\frac{Gm_1m_2}{r^2}\, dr = -\frac{Gm_1 m_2}{R}$

   \pause
   \bigskip

   $\rightarrow$ Gravitational potential energy between two objects separated by a distance $r$ is $-\frac{Gm_1m_2}{r}$.
 }

 \frame{\frametitle{\textbf{The Earth's ``gravity well''}}
   \large
    \BI
  \item{With this choice of the zero point at $r=\infty$, gravitational potential energy is always negative}
  \item{We have to {\it add energy} to get something away from Earth}
\EI  
\bigskip

\centerline{\includegraphics[width=0.6\textwidth]{ug-crop.pdf}}

\bigskip

This region of large negative potential energy is often called a ``gravity well''.
}

 \frame{\frametitle{\textbf{Summary}}
\large
   \BI
 \item{Potential energy is two things:}
   \BI
 \item{An accounting device that makes it easier to keep track of work done}
 \item{Part of {\it conservation of total energy}, a powerful statement about nature}
   \EI
 \item{Gravitational potential energy (on Earth): $U_g = mgy$}
   \pause
 \item{We learned about a new force: {\bf elasticity}}
   \BI
 \item{Restoring force in a stretched or compressed spring, or a stretched string: }
   \large   \centerline{$F = -k(x-x_0)$ ($x_0$ is the equilibrium length)} \normalsize
 \item{$k$ is the spring constant, measured in force per distance, that gauges stiffness}
 \item{Elastic potential energy: $U_{\rm{elas}}=\frac{1}{2}k(x-x_0)^2$}
   \EI
\pause
\item{Gravitational potential energy in general: $U_G = -\frac{Gm_1 m_2}{r}$}
\EI
 }
   \end{document}
