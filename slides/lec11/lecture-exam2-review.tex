\documentclass[10pt]{beamer}
\usefonttheme{professionalfonts}
\usefonttheme{serif}
\usepackage{amsmath}
\usepackage{mathtools}
%\documentclass[12pt]{beamerthemeSam.sty}
\usepackage{epsf}
%\usepackage{pstricks}
%\usepackage[orientation=portrait,size=A4]{beamerposter}
\geometry{paperwidth=160mm,paperheight=120mm}
%DT favorite definitions
\def\LL{\left\langle}	% left angle bracket
\def\RR{\right\rangle}	% right angle bracket
\def\LP{\left(}		% left parenthesis
\def\RP{\right)}	% right parenthesis
\def\LB{\left\{}	% left curly bracket
\def\RB{\right\}}	% right curly bracket
\def\PAR#1#2{ {{\partial #1}\over{\partial #2}} }
\def\PARTWO#1#2{ {{\partial^2 #1}\over{\partial #2}^2} }
\def\PARTWOMIX#1#2#3{ {{\partial^2 #1}\over{\partial #2 \partial #3}} }

\def\rightpartial{{\overrightarrow\partial}}
\def\leftpartial{{\overleftarrow\partial}}
\def\diffpartial{\buildrel\leftrightarrow\over\partial}

\def\BI{\begin{itemize}}
\def\EI{\end{itemize}}
\def\BE{\begin{displaymath}}
\def\EE{\end{displaymath}}
\def\BEA{\begin{eqnarray*}}
\def\EEA{\end{eqnarray*}}
\def\BNEA{\begin{eqnarray}}
\def\ENEA{\end{eqnarray}}
\def\EL{\nonumber\\}
\def\BS{\bigskip}

\newcommand{\map}[1]{\frame{\frametitle{\textbf{Course map}}
\centerline{\includegraphics[height=0.86\paperheight]{../../map/#1.png}}}}
\newcommand{\wmap}[1]{\frame{\frametitle{\textbf{Course map}}
\centerline{\includegraphics[width=0.96\paperwidth]{../../map/#1.png}}}}

\newcommand{\etal}{{\it et al.}}
\newcommand{\gbeta}{6/g^2}
\newcommand{\la}[1]{\label{#1}}
\newcommand{\ie}{{\em i.e.\ }}
\newcommand{\eg}{{\em e.\,g.\ }}
\newcommand{\cf}{cf.\ }
\newcommand{\etc}{etc.\ }
\newcommand{\atantwo}{{\rm atan2}}
\newcommand{\Tr}{{\rm Tr}}
\newcommand{\dt}{\Delta t}
\newcommand{\op}{{\cal O}}
\newcommand{\msbar}{{\overline{\rm MS}}}
\def\chpt{\raise0.4ex\hbox{$\chi$}PT}
\def\schpt{S\raise0.4ex\hbox{$\chi$}PT}
\def\MeV{{\rm Me\!V}}
\def\GeV{{\rm Ge\!V}}

%AB: my color definitions
%\definecolor{mygarnet}{rgb}{0.445,0.184,0.215}
%\definecolor{mygold}{rgb}{0.848,0.848,0.098}
%\definecolor{myg2g}{rgb}{0.647,0.316,0.157}
\definecolor{abtitlecolor}{rgb}{0.0,0.255,0.494}
\definecolor{absecondarycolor}{rgb}{0.0,0.416,0.804}
\definecolor{abprimarycolor}{rgb}{1.0,0.686,0.0}
\definecolor{Red}           {cmyk}{0,1,1,0}
\definecolor{Grey}           {cmyk}{.7,.7,.7,0}
\definecolor{Blue}          {cmyk}{1,1,0,0}
\definecolor{Green}         {cmyk}{1,0,1,0}
\definecolor{Brown}         {cmyk}{0,0.81,1,0.60}
\definecolor{Black}         {cmyk}{0,0,0,1}

\usetheme{Madrid}


%AB: redefinition of beamer colors
%\setbeamercolor{palette tertiary}{fg=white,bg=mygarnet}
%\setbeamercolor{palette secondary}{fg=white,bg=myg2g}
%\setbeamercolor{palette primary}{fg=black,bg=mygold}
\setbeamercolor{title}{fg=abtitlecolor}
\setbeamercolor{frametitle}{fg=abtitlecolor}
\setbeamercolor{palette tertiary}{fg=white,bg=abtitlecolor}
\setbeamercolor{palette secondary}{fg=white,bg=absecondarycolor}
\setbeamercolor{palette primary}{fg=black,bg=abprimarycolor}
\setbeamercolor{structure}{fg=abtitlecolor}

\setbeamerfont{section in toc}{series=\bfseries}

%AB: remove navigation icons
\beamertemplatenavigationsymbolsempty
\title[Review for Exam 2]{
  \textbf {Review for Exam 2}\\
%\centerline{}
%\centering
%\vspace{-0.0in}
%\includegraphics[width=0.3\textwidth]{propvalues_0093.pdf}
%\vspace{-0.3in}\\
%\label{intrograph}
}

\author[W. Freeman] {Physics 211\\Syracuse University, Physics 211 Spring 2023\\Walter Freeman}

\date{\today}

\begin{document}

\frame{\titlepage}

\frame{\frametitle{\textbf{Announcements}}
	\Large
	Help hours today: I am still sick and will infect people if I go to the Physics Clinic.
\bigskip



Other people will be there to assist you during the rest of the day. In particular, Brendan will be taking my place from 12:45-4:45 (roughly).
}


\frame{\frametitle{\textbf{Group Exam 2}}
	\large
	
	Your second group exam is in your next recitation.
	
\BS\BS

Exam review: Sunday, 2:30-5:30 (the auditorium)

}

	
	\frame{\frametitle{\textbf{Exam 2}}

	
	This exam will be just like Exam 1. A few reminders:
	
	\BI
	\item You may bring a page of notes
	\item You may bring a calculator (not one that does algebra)
	\item There will be assigned seats (different than before)
	\item Taking your exam at CDR? They'll have a copy for you. 
	\item Need other accommodations? Let me know.
	\EI
	
	\BS\BS
	
	What will be on it? 
	{\color{Red} \bf Relating the forces on objects to their motion with $\vec F = m \vec a$:}
	
	\BI
	\item Drawing force diagrams
	\item Dealing with inclines 
	\item Dealing with multiple objects
	\item Dealing with unknown tension/normal forces
	\item Dealing with friction
	\item Dealing with circular motion
	\item Interpreting things like ``why doesn't the frog fall out of the bucket?''
	\EI
}	

\frame{

	
	{\bf Drawing force diagrams:}
	
	\BI
	\item Each object gets its own force diagram
	\item Only forces acting {\it directly on that object} go on the diagram
	\item Let physics take care of indirect things for you (three book problem)
	\item Forces are real tangible things (plus gravity)
	\item Label each force with the symbol you'll use for it in algebra
	\item Draw your diagrams {\bf large} -- you may need to do trig, etc.
	\EI
	
	\BS\BS
	
	{\bf Dealing with inclines:}
	
	
	\BI
	\item Tilt your coordinate system so it aligns with the (possible) acceleration
	\item $a_y$ will generally be zero
	\item You'll need to decompose the weight force into components
	\EI
	
	\BS\BS
	
	{\bf Dealing with multiple objects:}
	
	\BI
	\item Each object gets its own force diagram
	\item Only draw the forces acting on each object on its diagram
	\item Different objects may have different $\vec a$:
	\BI
	\item Use $a_{1,x}$, $a_{2,y}$, etc. -- then think how they relate
	\item You'll have multiple equations -- that's okay
	\EI	\EI
	
}

\frame{
	
	
	{\bf Dealing with unknown tension/normal forces:}
	
	\BI
	\item Just because they're unknown doesn't make them scary
	\item Normal forces are however big they need to be to stop two objects from moving through one another
	\item Tension is however big it needs to be to keep ropes from stretching
	\item Leave $F_N$ or $T$ as unknowns in your system of equations -- you'll solve for them
	\EI
	
	\BS\BS
	
	{\bf Dealing with inclines:}
	
	
	\BI
	\item Tilt your coordinate system so it aligns with the (possible) acceleration
	\item $a_y$ will generally be zero
	\item You'll need to decompose the weight force into components
	\EI
	
	\BS\BS
	
	{\bf Dealing with friction:}
	
\BI
\item Friction opposes the relative motion of two things
\item For passive objects this is simple
\item ``Traction'' -- static friction between propelled vehicle/person/animal and ground
\BI
\item It points whatever direction the driver wants it to
\EI
\item Friction requires you to deal with two dimensions -- first find $F_N$, then substitute into $F_{\rm fric} = \mu F_N$
\EI
	
}
\frame{
	
	
	{\bf Dealing with circular motion:}
	
	\BI
	\item If an object is going in a circle, that just tells you its acceleration
	\item $a = \omega^2 r$ or $v^2/r$ toward the center
	\item Use the first one if you know/care about $\omega$ and the second if you know/care about $v$
	\item {\color{Green} Do not overcomplicate this!}
	\EI
	
	\BS\BS
	
	{\bf Interpreting motion in an accelerating frame:} (guaranteed question on exam)
	
	
	\BI
	\item Newton's laws are {\it not valid} in an accelerating ``box''
	\BI
	\item Accelerating/turning car
	\item A room rotating in a circle
	\EI
	\item Think about what it looks like from the {\it outside}
	\item Bus slams on brakes $\rightarrow$ bus accelerates backwards, passengers don't
	\item Car turns left $\rightarrow$ car accelerates left, passengers keep going straight 
	\item Bucket accelerates toward center of circle $\rightarrow$ bucket must push on frog to make it accelerate with it
	\EI
	
}



\frame{\frametitle{\textbf{Exam 2 -- makeup for Exam 1}}

Each student will have one question on Exam 2 on the material from Exam 1 that they got the lowest score on.

\BS\BS

If you do better on this question, it will replace your grade on that question from Exam 1.

\BS\BS

{\bf Note:} I will send out a Google form over the weekend asking students if they plan to take the exam at CDR.

\bigskip

If you want to take the exam at CDR, you {\it must} tell me so I can bring a personalized exam to them for you.


}


\frame{\frametitle{\textbf{Review - the ``Atwood machine''}}
	\Large
	In terms of $m_1$ and $m_2$, what is the acceleration of the masses?
	
	\BS\BS\pause
	\Large
	
	Key ideas:
	
	\BI
	\item The accelerations are not necessarily equal
	\item The tension force is equal on both objects
	\EI
}

\frame{\frametitle{\textbf{Review - multiple pulleys}}
	\Large
	If the masses are $m_1$ = 1100 g and $m_2$ = 1 kg, what are their accelerations?
		\BS\BS\pause
		
	\Large
	
	Key ideas:
	
	\BI
	\item The accelerations are again not necessarily equal
	\item The tension force is equal on both objects
	\EI
}

\frame{\frametitle{\textbf{Review - circular motion question from recitation}}
\Large
	In terms of $\omega$, $m$, $\theta$, and $g$, what is the tension in the strings?
	\BS\BS\pause
	
	\Large
	
	Key ideas:
	
	\BI
	\item Circular motion $\rightarrow$ $a = \omega^2 r$ toward the center
	\item Forces add like vectors -- do $F = ma$ in both $x$ and $y$
	\EI
}

\frame{\frametitle{\textbf{Review - a horse towing a load uphill}}	
	
	\large
	
	A horse of mass $m_1$ wants to pull a sled uphill. The rope between the horse's harness and the sled is parallel to the ground. If the slope is angled at $\theta$, the coefficient of static friction between the horse's hooves and the snow is $\mu_s$, and the coefficient of kinetic friction between the sled's runners and the snow is $\mu_k$, what's the heaviest load the horse can pull?
\BS\BS\pause
		
	Key ideas:
	
	\BI
	\item Draw one force diagram for each object
	\item ``Passive'' friction opposes the sliding (the sled)
	\item Traction between the horse's hooves and the ground points whichever direction the horse wants (uphill)
	\item $a=0$ here
	\EI
	
}

\end{document}



