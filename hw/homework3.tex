\documentclass[12pt]{article}
\setlength\parindent{0pt}
\usepackage{fullpage}
\usepackage{amsmath}
\setlength{\parskip}{4mm}
\def\LL{\left\langle}   % left angle bracket
\def\RR{\right\rangle}  % right angle bracket
\def\LP{\left(}         % left parenthesis
\def\RP{\right)}        % right parenthesis
\def\LB{\left\{}        % left curly bracket
\def\RB{\right\}}       % right curly bracket
\def\PAR#1#2{ {{\partial #1}\over{\partial #2}} }
\def\PARTWO#1#2{ {{\partial^2 #1}\over{\partial #2}^2} }
\def\PARTWOMIX#1#2#3{ {{\partial^2 #1}\over{\partial #2 \partial #3}} }
\newcommand{\BE}{\begin{displaymath}}
\newcommand{\EE}{\end{displaymath}}
\newcommand{\BNE}{\begin{equation}}
\newcommand{\ENE}{\end{equation}}
\newcommand{\BEA}{\begin{eqnarray}}
\newcommand{\EEA}{\nonumber\end{eqnarray}}
\newcommand{\EL}{\nonumber\\}
\newcommand{\la}[1]{\label{#1}}
\newcommand{\ie}{{\em i.e.\ }}
\newcommand{\eg}{{\em e.\,g.\ }}
\newcommand{\cf}{cf.\ }
\newcommand{\etc}{etc.\ }
\newcommand{\Tr}{{\rm tr}}
\newcommand{\etal}{{\it et al.}}
\newcommand{\OL}[1]{\overline{#1}\ } % overline
\newcommand{\OLL}[1]{\overline{\overline{#1}}\ } % double overline
\newcommand{\OON}{\frac{1}{N}} % "one over N"
\newcommand{\OOX}[1]{\frac{1}{#1}} % "one over X"



\begin{document}
\Large
\centerline{\sc{Homework 3}}
\normalsize
\centerline{\sc{Due Friday, 17 February}}

\begin{enumerate}

 	\item{A person with a mass of 60 kg stands in an elevator. Draw a force diagram for the
 	person, and indicate the magnitude of each of the forces acting on the person, in each
 	of the following situations:}
\begin{enumerate}
 	\item{The elevator is at rest;}
 	\item{the elevator is accelerating upward at 5 $\rm m/\rm s^2$} 
 	\item{the elevator is accelerating downward at 5 m/s2.}
\end{enumerate}
 	\item{Go visit an elevator and ride it up and down. When it’s accelerating upward, do you
 	feel lighter or heavier? What about when it’s accelerating downward? Based on your
 	observations and your answers to problem 1, is there any connection between how
 	heavy you feel and any of the forces that you drew in your force diagrams?}
 	
\item{A 1 kg book sits on a horizontal frictionless table in outer space, where there is no
 	gravity. A 9.8 N force acts diagonally downward on the book; there is a 30 degree
 	angle between that force and the vertical.}

 	\begin{enumerate}
\item{Draw a force diagram for the book. }
\item{What are the components of the external force parallel to and perpendicular to
 	the surface? Draw a right triangle on your force diagram whose hypotenuse is the
 	force and the legs are the components, as we usually do.}
\item{What magnitude must the normal force have? Remember, the normal force has
 	whatever magnitude that it must have to stop the book from moving ``through''
 	the table.}
\item{What is the acceleration of the book?}
\end{enumerate}

\item{Now, back to Earth, where there is gravity. A 1 kg book sits on a frictionless inclined
 	plane, tilted at an angle of $30^\circ$ above the horizontal. Hint: If you have trouble with
 	this problem, look at the force diagram you drew for the last problem and rotate it by
 	thirty degrees. }
\begin{enumerate}
\item{Draw a force diagram for the book. }
\item{What are the components of the gravitational force parallel to and perpendicular
 	to the ramp? Draw a right triangle on your force diagram whose hypotenuse is
 	the book’s weight and the legs are the components, as we usually do. Note that
 	the components will both be diagonal relative to the horizontal/vertical axes. }
\item{What magnitude must the normal force have? Remember, the normal force has
 	whatever magnitude that it must have to stop the book from moving ``through''
 	the ramp. }
 	
\item{What is the acceleration of the book down the ramp? Hint: If you have done the
 	previous problem, this one should be easy; you do not need to show mathematics
 	if you can explain how your answers relate!}
\end{enumerate}
 
 \item{A 1500 kg car is driving at 20 m/s. The driver wishes to stop over a distance of 30 m. }
\begin{enumerate}
 	\item{Draw a force diagram for the car. }
 	\item{What force must the brakes apply to the car to do this? }
\end{enumerate}

 	\item{A stack of three books, each weighing 10 N, sits on a table. }
 	\begin{enumerate}
\item{Draw a force diagram for each of the three books. }
\item{Now, compute the size of each of the forces you have labelled. You might find it
 	easiest to start with the topmost book.}
\end{enumerate}

\item A sled sits on the snow. Three people are pulling on it with ropes. The first rope points $45^\circ$ north of east, and a person pulls it with a force of 200 N.
Another rope points southward, and someone pulls it with a force of 300 N. The third rope points $30^\circ$ north of west, and the last person pulls it with a force of
100 N.

You would like to stop the sled from moving. In which direction, and with what force, should you pull the fourth rope?

\item You have seen the previous problem before -- with very similar numbers. Where have you seen it before? How is this problem ``the same'' as the other one?
 

\end{enumerate}

\end{document}
