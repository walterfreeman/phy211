
\documentclass[12pt]{article}
\setlength\parindent{0pt}
\usepackage{fullpage}
\usepackage{amsmath}
\usepackage{graphicx}
\setlength{\parskip}{4mm}
\def\LL{\left\langle}   % left angle bracket
\def\RR{\right\rangle}  % right angle bracket
\def\LP{\left(}         % left parenthesis
\def\RP{\right)}        % right parenthesis
\def\LB{\left\{}        % left curly bracket
\def\RB{\right\}}       % right curly bracket
\def\PAR#1#2{ {{\partial #1}\over{\partial #2}} }
\def\PARTWO#1#2{ {{\partial^2 #1}\over{\partial #2}^2} }
\def\PARTWOMIX#1#2#3{ {{\partial^2 #1}\over{\partial #2 \partial #3}} }
\newcommand{\BE}{\begin{displaymath}}
\newcommand{\EE}{\end{displaymath}}
\newcommand{\BNE}{\begin{equation}}
\newcommand{\ENE}{\end{equation}}
\newcommand{\BEA}{\begin{eqnarray}}
\newcommand{\EEA}{\nonumber\end{eqnarray}}
\newcommand{\EL}{\nonumber\\}
\newcommand{\la}[1]{\label{#1}}
\newcommand{\ie}{{\em i.e.\ }}
\newcommand{\eg}{{\em e.\,g.\ }}
\newcommand{\cf}{cf.\ }
\newcommand{\etc}{etc.\ }
\newcommand{\Tr}{{\rm tr}}
\newcommand{\etal}{{\it et al.}}
\newcommand{\OL}[1]{\overline{#1}\ } % overline
\newcommand{\OLL}[1]{\overline{\overline{#1}}\ } % double overline
\newcommand{\OON}{\frac{1}{N}} % "one over N"
\newcommand{\OOX}[1]{\frac{1}{#1}} % "one over X"



\begin{document}
\Large
\centerline{\sc{Recitation Questions}}
\normalsize
\centerline{\sc{6 April}}

\begin{enumerate}

\item A bucket of mass $m$ hangs from a string wound around a pulley 
(a solid cylinder) with mass $M$ and radius $r$. When the bucket is
released, it falls, unwinding the string.

\begin{enumerate}

\item Draw force diagrams for the bucket and the pulley. Note that since the pulley rotates, you will need
to draw an extended force diagram for it, drawing the object and labeling where each force acts.

\vspace{3in}

\item In terms of the forces in your force diagrams, write an expression for the net torque on the pulley.

\vspace{1in}

\item Write down Newton's laws of motion -- $\sum \vec F = m \vec a$ for translation, and $\sum \tau = I \alpha$
-- for each object. (One object moves, and the other turns...)

\vspace{2in}


\newpage

\item What is the relationship between the angular acceleration $\alpha$ of the pulley and the linear acceleration
$a$ of the bucket? (The answer may be different depending on how you have drawn your pictures and your choice of
coordinate system.)

\vspace{1in}

\item Calculate the acceleration of the bucket in terms of $m$ and $M$.

\vspace{3in}

\item Suppose that the pulley were a hollow cylinder with the same mass. How would this acceleration change?

\newpage
\end{enumerate}

   \item{A Yo-Yo consists of a cylinder of mass $m$ and radius $r$. A slot is cut in the middle of the cylinder such that the inner radius is only $0.4r$, and a string is wound around the middle. (If you don't know what a Yo-Yo is, there is an animation on Wikipedia.)
     A person holds the string and allows the Yo-Yo to fall. As it falls, it has both a linear acceleration (moving downward) and an angular acceleration (spinning faster and faster).}
     \begin{enumerate}
       \item{What is the relation between the linear velocity $v$ of the Yo-Yo (moving downward) and its angular velocity $\omega$?}

\vspace{1in}
       \item{What is the relation between the linear acceleration $a$ of the Yo-Yo (moving downward) and its angular acceleration $\alpha$? (Note: this is trivial once you've done the previous part...)}

\vspace{1in}

       \item{Draw a force diagram for the Yo-Yo, indicating the location where all forces act as well as their magnitude.}

\vspace{3in}

       \item{Write down Newton's second law ($F=ma$) for the Yo-Yo, putting in expressions for the various forces.}

\vspace{2in}

       \item{Write down ``Newton's second law for rotation'' $\tau = I \alpha$, putting in expressions for the net torque and the moment of inertia.}

\vspace{2in}
       \item{What is the acceleration of the Yo-Yo and the tension in the string?}

\vspace{3in}
       \item{Will the Yo-Yo accelerate faster or slower if the inner radius is changed to $0.2r$?}


     \end{enumerate}
\end{enumerate}
\newpage
\centerline{\sc{Reference Material - Rotational Motion}}


Moments of Inertia:
\begin{itemize}
  \item{Disk or cylinder, rotating about center: $I = \frac{1}{2}MR^2$}
  \item{Sphere, rotating about center: $I = \frac{2}{5}MR^2$}
  \item{Ring or hollow cylinder, rotating about center: $I = MR^2$}
\end{itemize}

\bigskip
\bigskip
\bigskip

Correspondence between linear dynamics and rotational dynamics: 
  \scriptsize

\begin{tabular}{| c | c | c | c |}
  \hline
  Position & $s$ & Angle & $\theta$  \\
  Velocity & $\vec v$ & Angular velocity & $\omega$  \\
  Acceleration & $\vec a$ & Angular acceleration & $\alpha$  \\
  \hline
                                   & $v(t) = v_0 + at$ & & $\omega(t) = \omega_0 + \alpha t$ \\
                                   & $x(t) = x_0 + v_0 t + \frac{1}{2} at^2$ & & $\theta(t) = \theta_0 + \omega_0 t + \frac{1}{2} \alpha t^2$ \\
                                   & $v_f^2 - v_0^2 = 2a \Delta x$ & & $\omega_f^2 - \omega_0^2 = 2 \alpha \Delta \theta$ \\
  \hline
  Mass & $m$ & Moment of inertia & $I$ \\
  \hline
  Force & $F$ & Torque & $\tau = F_\perp r = F r_\perp$ \\
  \hline
  Newton's second law & $\vec F = m \vec a$ & ``Newton's second law for rotation'' & $\tau = I \alpha$ \\
  \hline
  Kinetic energy & $\frac{1}{2} mv^2$ & Kinetic energy & $\frac{1}{2}I\omega^2$ \\
  \hline
  Momentum & $\vec p = m \vec v$ & Angular momentum & $L = I \omega$ \\
  \hline
\end{tabular}

\bigskip
\bigskip
\bigskip

Arc length $s=\theta r$ \\
Tangential velocity $v=\omega r$ \\
Tangential acceleration $a=\alpha r$




 \end{document}
