\documentclass[12pt]{article}
\setlength\parindent{0pt}
\usepackage{fullpage}
\usepackage[margin=0.5in, paperwidth=13.5in, paperheight=8.4375in]{geometry}
\usepackage{amsmath}
\usepackage{graphicx}
\newcommand{\BI}{\begin{itemize}}
\newcommand{\EI}{\end{itemize}}
\newcommand{\BE}{\begin{enumerate}}
\newcommand{\EE}{\end{enumerate}}
\setlength{\parskip}{4mm}

\def\PAR#1#2{ {{\partial #1}\over{\partial #2}} }

\pagenumbering{gobble}

\begin{document}
	\begin{center}\sc \Large Recitation -- February 10 \end{center}
	
	In recitation, you will work in groups of up to three to practice the skills you are learning in our course. You will be assisted by three types of folks:
	
	\BE
	\item {\bf TA's} are PhD candidates in physics who teach classes as part of their training. They are here to help you learn physics and assist with grading. There may be several TA's in your recitation section, but one in particular will be assigned to primarily grade your work. If you have questions about your grade, that's the person to ask.
	
	\item {\bf Coaches} are students like yourself who took this class in the past and now have jobs helping us teach it. They are here to help you learn physics, and only have a very minor role in determining grades.
	
	\item {\bf Your fellow students} in this course will be an invaluable resource for you. In recitation you will be working very closely with your group of three, but you can also ask questions of anyone else in the recitation section or the broader class. 
	\EE
	
\vspace{0.5in}
	
	{\bf Goals for recitation today:}
	
	
	\BE
	\item Meet your teaching staff and your group
	\item Get familiar with the technology we will use
	\item Try out one of our recitation assessment methods
	\item Practice reasoning with dimensions and units
	\item Gain experience doing algebra with units
	\item Apply this way of thinking to a few calculations involving motion
	\EE
	
	\bigskip
	
	\begin{center}
		\bfseries
		Ten minutes before the recitation period is over, a Blackboard ``test'' will appear, asking a brief question about what you did today.\\ This is an experimental way of doing recitation evaluation.
		
		So, toward the end of class, click on ``Assignments'' in Blackboard, and you should see a short quiz. It should take you 3-5 minutes to do. \\You may discuss the answers with your group or instructors, but everyone should submit their own answer.
		
		If it is five minutes before the end of class, stop working wherever you are and go do this assignment on Blackboard.
	\end{center}

\newpage

	The first thing you should do is get used to the tools you'll be using to work with your colleagues and instructors during recitation.
	
	Your TA's will talk to the whole class for a few minutes and introduce themselves. Get to know them; learn your TAs' names and which one will be responsible for grading for you. Say hi to everyone in chat and introduce yourself to the class!	
	
    Then, the TA's will then assign people to ``breakout rooms'' in groups of three. Once you're in your groups of three:
	
	\BE
	\item Get to know each other! Learn your groupmates' names and something about them; get used to talking to each other. Sort out any audio/video issues you have talking to each other.
	
	\item Share {\it this document} in Collaborate. Practice using the Collaborate drawing tools. If members of your group have a stylus, they can write directly on the screen of their iPad or computer. Draw some stick figures!
	
	\EE

\newpage
	
	{\bf When working on recitation materials, it will be helpful for you to write out your thoughts and answers on paper as well as writing them here. That way, you can refer to them again later. One issue with Blackboard Collaborate's shared whiteboard feature is that, if you go to a new page, you will lose everything you've written. So you should use the screen for scratch-work in sharing things with your group, but write your own copy of stuff on paper. 
		
	If your group has a different clever technical solution to this -- another sort of shared document -- please use it, and share it with the rest of us so we can suggest it to other students who have the same technology you do.}
	
	This material is designed to get you familiar with the way that physicists think about {\it units of measure}
	and how to do mathematics with them. There are three principles you need to know:
	
	\BI
	\item Some quantities bear {\it dimensions} -- various types of things that you measure. One example, for instance, is {\it time},
	which can be measured in many different units (seconds, minutes, hours...)
	
	\item Any numeric value that describes something with dimensions must always be attached in the units that it is measured in.
	
	\item {\bf You may multiply and divide by units just like any other variable when doing algebra.}
	For instance, we know that
	
	$$
	1 \,{\rm minute} = 60 \,{\rm seconds}.
	$$
	\EI
	
	\begin{enumerate}
		
		\item What do you get if you divide both sides of this equation by ``1 minute''?
		
		
		
		\vspace{1in}
		
		\item How far is it from Syracuse to Albany along I-90? {\it If you haven't physically been to Syracuse yet, use Google Maps or similar to look at Central New York.} Give your answer in as many different ways as you can think of
		(at least three). It's okay to give an informal answer; for instance, how do drivers talk about distances between cities?
		If you're having trouble thinking of different ways to answer the question, you might ask:
		
		\BI
		\item an American student (if you are international)
		\item an international student (if you are American)
		\item someone who has driven from Syracuse to Albany
		\EI
		
		\vspace{3in}
		
		
	\newpage
	
	

	
	\item Drivers measure distances in ``hours'' frequently. For instance, I might say that it is six hours' drive
	from Syracuse to Baltimore. However, this is not a dimensionally-correct way to describe distance, since an hour
	is a measure of time, not distance. What other piece of information do you need for this to make sense?
	
	\vspace{2in}
	
	\item What are the dimensions of this piece of information, and what units would you measure it in? Write your answer both in
	words and as a fraction. (What mathematical operation does ``per'' suggest?)
	
	\vspace{2in}
	
	\item Use a satellite navigation app on a smartphone to get the distance to Albany, plus the the time that it will take.
	From this, calculate how fast the app thinks you will be driving on average. Show your work to a TA or coach when you are done.
	
	\vspace{2in}
	
	\item In physics we generally measure this quantity in meters per second instead.
	Convert your result to meters per second from the units you have measured it in. As you do your conversion, write out all of the
	steps; the point is to ensure that you get comfortable doing algebra with units and numbers together. Show your work to a TA
	or coach when you are done.
	
	\vspace{2.5in}
	
	
	\item People who make fast cars often brag about how quickly they can accelerate. For instance, Tesla Motors claims that
	their upcoming Roadster can accelerate from 0 to 60 miles per hour in 1.9 seconds.
	
	You will study acceleration in more detail soon, but for now, just know that
	
	$$
	{\rm acceleration} = \frac{\rm change\, in\, velocity}{\rm time}.
	$$
	
	What is the average acceleration of this (ludicrously fast) car as it goes from zero to 60 miles per hour? Write out all of the
	steps in your calculation, making sure to treat units (like hours, seconds, and miles) like variables when manipulating fractions.
	What are the units of your answer? Do they have the right dimensions?
	
	\vspace{2in}
	\newpage
	\item In physics, we like to measure distances in meters and time in seconds. Convert your answer to the previous into these
	units, and simplify as much as possible. Call a TA or coach over to check your work when you are done.
	
	\vspace{2in}
	
	\item What does it mean that acceleration can be measured in ``meters per second squared''? I've never seen a ``squared second'';
	does this make sense?
	
	\vspace{1in}
	
	\item The volume of a cube 10 cm on a side is equal to one liter. How many cubic centimeters (${\rm cm}^3$) are in a liter?
	Make sure you doublecheck your answer with your common sense!
	
	
	\end{enumerate}
	\vfill
	
{\bf \begin{center}
	When you are done, check Blackboard under ``Assignments'' for the assessment question for today.
	\end{center}}

	
	
\end{document}