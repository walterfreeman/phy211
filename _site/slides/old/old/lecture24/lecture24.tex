% DPF 09 talk on strangeness in nucleon

\documentclass[10pt]{beamer}
\usefonttheme{professionalfonts} % using non standard fonts for beamer
\usefonttheme{serif} % default family is serif
\usepackage{amsmath}
\usepackage{mathtools}
\usepackage{mwe}
%\documentclass[12pt]{beamerthemeSam.sty}
\usepackage{epsf}
%\usepackage{pstricks}
%\usepackage[orientation=portrait,size=A4]{beamerposter}
\geometry{paperwidth=160mm,paperheight=120mm}
%DT favorite definitions
\def\LL{\left\langle}	% left angle bracket
\def\RR{\right\rangle}	% right angle bracket
\def\LP{\left(}		% left parenthesis
\def\RP{\right)}	% right parenthesis
\def\LB{\left\{}	% left curly bracket
\def\RB{\right\}}	% right curly bracket
\def\PAR#1#2{ {{\partial #1}\over{\partial #2}} }
\def\PARTWO#1#2{ {{\partial^2 #1}\over{\partial #2}^2} }
\def\PARTWOMIX#1#2#3{ {{\partial^2 #1}\over{\partial #2 \partial #3}} }

\def\rightpartial{{\overrightarrow\partial}}
\def\leftpartial{{\overleftarrow\partial}}
\def\diffpartial{\buildrel\leftrightarrow\over\partial}

\def\BI{\begin{itemize}}
\def\EI{\end{itemize}}
\def\BE{\begin{displaymath}}
\def\EE{\end{displaymath}}
\def\BEA{\begin{eqnarray*}}
\def\EEA{\end{eqnarray*}}
\def\BNEA{\begin{eqnarray}}
\def\ENEA{\end{eqnarray}}
\def\EL{\nonumber\\}


\newcommand{\map}[1]{\frame{\frametitle{\textbf{Course map}}
\centerline{\includegraphics[height=0.86\paperheight]{../../map/#1.png}}}}
\newcommand{\wmap}[1]{\frame{\frametitle{\textbf{Course map}}
\centerline{\includegraphics[width=0.96\paperwidth]{../../map/#1.png}}}}

\newcommand{\etal}{{\it et al.}}
\newcommand{\gbeta}{6/g^2}
\newcommand{\la}[1]{\label{#1}}
\newcommand{\ie}{{\em i.e.\ }}
\newcommand{\eg}{{\em e.\,g.\ }}
\newcommand{\cf}{cf.\ }
\newcommand{\etc}{etc.\ }
\newcommand{\atantwo}{{\rm atan2}}
\newcommand{\Tr}{{\rm Tr}}
\newcommand{\dt}{\Delta t}
\newcommand{\op}{{\cal O}}
\newcommand{\msbar}{{\overline{\rm MS}}}
\def\chpt{\raise0.4ex\hbox{$\chi$}PT}
\def\schpt{S\raise0.4ex\hbox{$\chi$}PT}
\def\MeV{{\rm Me\!V}}
\def\GeV{{\rm Ge\!V}}

%AB: my color definitions
%\definecolor{mygarnet}{rgb}{0.445,0.184,0.215}
%\definecolor{mygold}{rgb}{0.848,0.848,0.098}
%\definecolor{myg2g}{rgb}{0.647,0.316,0.157}
\definecolor{abtitlecolor}{rgb}{0.0,0.255,0.494}
\definecolor{absecondarycolor}{rgb}{0.0,0.416,0.804}
\definecolor{abprimarycolor}{rgb}{1.0,0.686,0.0}
\definecolor{Red}           {cmyk}{0,1,1,0}
\definecolor{Grey}           {cmyk}{.7,.7,.7,0}
\definecolor{Lg}           {cmyk}{.4,.4,.4,0}
\definecolor{Blue}          {cmyk}{1,1,0,0}
\definecolor{Green}         {cmyk}{1,0,1,0}
\definecolor{Brown}         {cmyk}{0,0.81,1,0.60}
\definecolor{Black}         {cmyk}{0,0,0,1}

\usetheme{Madrid}
\newcommand{\vcenteredinclude}[1]{\begingroup
  \setbox0=\hbox{\includegraphics[width=3in]{#1}}%
\parbox{\wd0}{\box0}\endgroup}

%AB: redefinition of beamer colors
%\setbeamercolor{palette tertiary}{fg=white,bg=mygarnet}
%\setbeamercolor{palette secondary}{fg=white,bg=myg2g}
%\setbeamercolor{palette primary}{fg=black,bg=mygold}
\setbeamercolor{title}{fg=abtitlecolor}
\setbeamercolor{frametitle}{fg=abtitlecolor}
\setbeamercolor{palette tertiary}{fg=white,bg=abtitlecolor}
\setbeamercolor{palette secondary}{fg=white,bg=absecondarycolor}
\setbeamercolor{palette primary}{fg=black,bg=abprimarycolor}
\setbeamercolor{structure}{fg=abtitlecolor}

\setbeamerfont{section in toc}{series=\bfseries}

%AB: remove navigation icons
\beamertemplatenavigationsymbolsempty
\title{
  \textbf {Final course review}\\
%\centerline{}
%\centering
%\vspace{-0.0in}
%\includegraphics[width=0.3\textwidth]{propvalues_0093.pdf}
%\vspace{-0.3in}\\
%\label{intrograph}
}

\author[W. Freeman] {Physics 211\\Syracuse University, Physics 211 Spring 2015\\Walter Freeman}

\date{\today}

\begin{document}

\frame{\titlepage}

\frame{\frametitle{\textbf{Announcements}}
  \Large
\BI
\item{No recitation tomorrow due to my review}
\item{Turn in your homework/extra credit in your TA's mailbox}
\item{Final review schedule:}
  \BI
\item{TODAY, 2-5, Sims 337}
\item{Wednesday, 10-4, Physics 208}
\item{Thursday, 9-3, Physics 202}
\item{Weekend TBA?}
  \EI
\EI
}

\frame{\frametitle{\textbf{Final exam format}}
  \Large
\BI
\item{More questions, but less time-consuming}
\item{Expect questions like:}
  \BI
\item{``What concept could you use to solve this problem?''}
\item{``Write but do not solve a system of two equations that will let you find $a$ and $T$''}
\item{``Which of these equations would be useful here?''}
  \EI
\item{You may make {\bf your own reference sheets}}
\item{One standard sheet of paper, double-sided, {\bf handwritten}}
  \EI
}

\frame{\frametitle{\textbf{First, a pitch}}
  \Large
  \centerline{Want to get paid to be a Physics 211 coach next year?}
  \bigskip
  \bigskip
  \centerline{Like to teach your peers?}
  \bigskip
  \bigskip
  \centerline{Earned at least a B+ in this course, or a B with my recommendation?}
  \bigskip
  \bigskip
  \centerline{Prof. Lisa Manning is teaching a class on peer coaching next fall}
  \centerline{Ask either of us for details,  or see the flyer on the course website}
}

\frame{\frametitle{\textbf{Kinematics concepts}}
  \large
  \BI
\item{First derivative of position is velocity; second derivative is acceleration}
\item{Kinematics lets us connect acceleration, velocity, position, and time}
\item{If $\vec a$ is constant:}

  \begin{align*}
    s(t) =& s_0 + v_0 t + \frac{1}{2} a t^2 \\
    v(t) =& v_0 + at \\
    v_f^2 - v_0^2 =& 2a\Delta x
  \end{align*}

\item{These relations hold separately and independently in $x$ and $y$}
\item{Acceleration is $g$ downwards {\bf if and only if} an object is in freefall}
\EI
}

\frame{\frametitle{\textbf{Kinematics sample problem: the frog-and-table problem}}
  \Large

  A frog jumps horizontally off of a table of height $h$ and lands a horizontal distance $d$ away. What was its initial speed?

}

\frame{\frametitle{\textbf{Use kinematics when:}}
    \Large
    \BI
  \item{You need to connect some combination of position, velocity, acceleration, and time}
  \EI

}
  \frame{\frametitle{\textbf{Force concepts and Newton's second law}}
    \large
    \BI
  \item{Newton's second law relates the net force $\sum \vec F$ to the acceleration $\vec a$ of the center of mass of an object}
  \item{If an object both rotates and moves, $\vec F=m\vec a$ gives you $\vec a$ of the center of mass}
  \item{Newton's third law: forces come in pairs}
  \item{Some forces you should know about:}
    \BI
  \item{Normal forces: as big as they need to be}
  \item{Friction: $F_{\rm{fric, static, max}} = \mu_s F_N$, $F_{\rm{fric, kinetic}}=\mu_k F_N$}
  \item{Elastic: $F=-k\Delta x$}
  \item{Gravity (Earth): $F=mg$ downward}
  \item{Gravity (general): $F=\frac{Gm_1m_2}{r^2}$}
  \item{Tension: A rope pulls on both ends}
    \EI
    \EI
  }

  \frame{\frametitle{\textbf{Force diagrams}}
    \Large
    \BI
  \item{Draw all forces acting on the object, as vectors}
  \item{If you're going to care about torque, draw them where they act}
  \item{Gravity acts at the center of mass}
  \item{Draw these diagrams big enough that you can read them clearly and do trig}
    \EI
  }

\frame{\frametitle{\textbf{Uniform circular motion}}
  \Large
  \BI
\item{If an object is traveling in a circle, you know its acceleration is $a_c = \omega^2 r = \frac{v_T^2}{r}$ toward the center}
\item{Often this will ``give you'' the right side of $F=ma$, and let you conclude something about the left}
  \EI
}


  \frame{\frametitle{\textbf{Use Newton's second law when:}}
    \Large
    \BI
  \item{You need to connect the forces on an object to its acceleration}
  \item{If you don't need $\vec a$ directly, and don't care about time, maybe use energy methods instead?}
    \EI
  }


  \frame{\frametitle{\textbf{Sample problem: forces}}
    \Large
    A book rests on a slope at angle $\theta$. What coefficient of static friction is required to make it stick?

  }


  \frame{\frametitle{\textbf{The work-energy theorem and conservation of energy}}
    \large
    \BI
  \item{Work-energy theorem comes from the third kinematics relation}
  \item{Two formulations, one with potential energy and one without:}
    \BI
  \item{$KE_i + W_{\rm{all}} = KE_f$}
  \item{$KE_i + PE_i + W_{\rm{other}} = KE_f + PE_f$}
    \EI
  \item{Draw {\it clear} before and after snapshots}
  \item{Figure out work done in going from one to the other}
  \item{Work = $\vec F \cdot d\vec s$}
    \EI
  }

  \frame{\frametitle{\textbf{Use energy methods when:}}
      \Large
      \BI
    \item{You don't know and don't care about time}
    \item{You can account for the work done by all forces involved}
    \item{This is {\bf not} true at the instant of a collision -- use momentum instead}
      \EI
    }

    \frame{\frametitle{\textbf{Sample problem: energy}}
      \Large
      A pendulum of length $L$ is drawn back to an angle $\theta$ and released. How fast is it going when it reaches the bottom?
    }

    \frame{\frametitle{\textbf{Conservation of momentum}}
      \BI
    \item{In the absence of external forces, $\vec p = m\vec v$ is conserved}
    \item{This is a consequence of Newton's third law}
    \item{Collisions and explosions are short enough that external forces are small}
    \item{Momentum is a vector and is conserved separately in $x$ and $y$}
      \EI
    }
      \frame{\frametitle{\textbf{Sample problem: momentum}}
        \Large
        The hobbits-on-Onondaga problem from Exam 3:
\bigskip
\bigskip
\bigskip

        Merry and Pippin are sledding on the frozen surface of Lake Onondaga, which is quite close to a frictionless surface. Each of them plus his sled has a mass of 30 kg, but Merry also carries a stone that weighs 5 kg.
        They are both traveling east at 2 m/s, right next to each other; Merry is north of Pippin. Merry throws his stone to Pippin, who catches it; the initial velocity of the stone is 5 m/s due south. Treat north as the positive $y-$axis and east as the
        positive $x-$axis.

      }



      \frame{\frametitle{\textbf{Use conservation of momentum when:}}
        \Large
        \BI
      \item{You have a collision or explosion and need to connect the velocities before to the velocities after}
        \EI
      }

        \frame{\frametitle{\textbf{Rotation}}
          \Large
          Many ideas here, most analogous to translational motion:
\large
\BI
\item{Torque plays the role of force: $\tau = F_\perp r = F r_\perp$}
\item{Moment of inertia plays the role of mass: $I = \lambda mr^2$}
\item{$\vec F = m \vec a \rightarrow \tau = I \alpha$: ``Newton's second law for rotation''}
\item{Rolling motion is translation plus rotation: $v = \pm \omega r$, $a = \pm \alpha r$}
\item{\bf You must think about the signs here}
\item{Rotational kinetic energy: $KE_{\rm{rot}} = \frac{1}{2} I \omega^2$}
\item{Angular momentum: $L = I \omega$}
  \EI
}

\frame{\frametitle{\textbf{Static equilibrium problems}}
  \Large
  \BI
\item{Net torque is zero about any pivot}
\item{Net force is zero (you may not need this)}
\item{Torque due to any force applied {\bf at} the pivot is zero}
\EI
}


\frame{\frametitle{\textbf{Standing waves in strings/tubes:}}
  \Large
  \vcenteredinclude{mode1-crop.pdf} Fundamental: $f_1 = \frac{c}{2L}$\\
 \bigskip 
  \vcenteredinclude{mode2-crop.pdf} 2nd harmonic: $f_2 = 2 f_1 $\\
 \bigskip 
  \vcenteredinclude{mode3-crop.pdf} 3rd harmonic: $f_3 = 3 f_1$ \\
  
 \bigskip 
  \vcenteredinclude{mode4-crop.pdf} 4th harmonic: $f_4 = 4 f_1$ \\
  
 \bigskip 
}

\frame{\frametitle{\textbf{Final reminders}}
  \large
  \BI
\item{Huge amounts of extra review available (20-ish hours); use it}
\item{Get some rest during finals week and take care of yourselves}
\item{Practice exam solution sets will be posted after the extra credit is in}
\item{If you're affected by the Calc 2/Physics exam scheduling nonsense, tell SU!}
\EI
}

\frame{\frametitle{\textbf{Beyond mechanics}}
  \Large
What else is there in physics?
\large
\BI
\item{Electrodynamics: electricity, magnetism, light}
\item{Thermodynamics: heat, pressure, gases, temperature, phase changes}
\item{Condensed matter: crystals, the structure of matter}
\item{Quantum mechanics: atoms, very small things, chemistry}
\item{Astrophysics and cosmology: what is the Universe and where is it going?}
\item{Biophysics}
  \pause
\item{Computational physics (my course for Fall 2015)}
  \EI
}

\frame{\frametitle{\textbf{And, finally...}}
  \Large
... thank you all; you've made my first semester at SU a great one.

\pause
\bigskip
\bigskip
\bigskip
\bigskip

\centerline{``Science is the belief in the ignorance of experts.'' (R. Feynman)}

\pause

\bigskip
\bigskip
\bigskip
\bigskip

\centerline{``All science is either physics or stamp collecting.'' (E. Rutherford)}
}

\end{document}
