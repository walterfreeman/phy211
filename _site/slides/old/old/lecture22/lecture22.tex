% DPF 09 talk on strangeness in nucleon

\documentclass[10pt]{beamer}
\usefonttheme{professionalfonts} % using non standard fonts for beamer
\usefonttheme{serif} % default family is serif
\usepackage{amsmath}
\usepackage{mathtools}
\usepackage{mwe}
%\documentclass[12pt]{beamerthemeSam.sty}
\usepackage{epsf}
%\usepackage{pstricks}
%\usepackage[orientation=portrait,size=A4]{beamerposter}
\geometry{paperwidth=160mm,paperheight=120mm}
%DT favorite definitions
\def\LL{\left\langle}	% left angle bracket
\def\RR{\right\rangle}	% right angle bracket
\def\LP{\left(}		% left parenthesis
\def\RP{\right)}	% right parenthesis
\def\LB{\left\{}	% left curly bracket
\def\RB{\right\}}	% right curly bracket
\def\PAR#1#2{ {{\partial #1}\over{\partial #2}} }
\def\PARTWO#1#2{ {{\partial^2 #1}\over{\partial #2}^2} }
\def\PARTWOMIX#1#2#3{ {{\partial^2 #1}\over{\partial #2 \partial #3}} }

\def\rightpartial{{\overrightarrow\partial}}
\def\leftpartial{{\overleftarrow\partial}}
\def\diffpartial{\buildrel\leftrightarrow\over\partial}

\def\BI{\begin{itemize}}
\def\EI{\end{itemize}}
\def\BE{\begin{displaymath}}
\def\EE{\end{displaymath}}
\def\BEA{\begin{eqnarray*}}
\def\EEA{\end{eqnarray*}}
\def\BNEA{\begin{eqnarray}}
\def\ENEA{\end{eqnarray}}
\def\EL{\nonumber\\}


\newcommand{\map}[1]{\frame{\frametitle{\textbf{Course map}}
\centerline{\includegraphics[height=0.86\paperheight]{../../map/#1.png}}}}
\newcommand{\wmap}[1]{\frame{\frametitle{\textbf{Course map}}
\centerline{\includegraphics[width=0.96\paperwidth]{../../map/#1.png}}}}

\newcommand{\etal}{{\it et al.}}
\newcommand{\gbeta}{6/g^2}
\newcommand{\la}[1]{\label{#1}}
\newcommand{\ie}{{\em i.e.\ }}
\newcommand{\eg}{{\em e.\,g.\ }}
\newcommand{\cf}{cf.\ }
\newcommand{\etc}{etc.\ }
\newcommand{\atantwo}{{\rm atan2}}
\newcommand{\Tr}{{\rm Tr}}
\newcommand{\dt}{\Delta t}
\newcommand{\op}{{\cal O}}
\newcommand{\msbar}{{\overline{\rm MS}}}
\def\chpt{\raise0.4ex\hbox{$\chi$}PT}
\def\schpt{S\raise0.4ex\hbox{$\chi$}PT}
\def\MeV{{\rm Me\!V}}
\def\GeV{{\rm Ge\!V}}

%AB: my color definitions
%\definecolor{mygarnet}{rgb}{0.445,0.184,0.215}
%\definecolor{mygold}{rgb}{0.848,0.848,0.098}
%\definecolor{myg2g}{rgb}{0.647,0.316,0.157}
\definecolor{abtitlecolor}{rgb}{0.0,0.255,0.494}
\definecolor{absecondarycolor}{rgb}{0.0,0.416,0.804}
\definecolor{abprimarycolor}{rgb}{1.0,0.686,0.0}
\definecolor{Red}           {cmyk}{0,1,1,0}
\definecolor{Grey}           {cmyk}{.7,.7,.7,0}
\definecolor{Lg}           {cmyk}{.4,.4,.4,0}
\definecolor{Blue}          {cmyk}{1,1,0,0}
\definecolor{Green}         {cmyk}{1,0,1,0}
\definecolor{Brown}         {cmyk}{0,0.81,1,0.60}
\definecolor{Black}         {cmyk}{0,0,0,1}

\usetheme{Madrid}
\newcommand{\vcenteredinclude}[1]{\begingroup
  \setbox0=\hbox{\includegraphics[width=3in]{#1}}%
\parbox{\wd0}{\box0}\endgroup}

%AB: redefinition of beamer colors
%\setbeamercolor{palette tertiary}{fg=white,bg=mygarnet}
%\setbeamercolor{palette secondary}{fg=white,bg=myg2g}
%\setbeamercolor{palette primary}{fg=black,bg=mygold}
\setbeamercolor{title}{fg=abtitlecolor}
\setbeamercolor{frametitle}{fg=abtitlecolor}
\setbeamercolor{palette tertiary}{fg=white,bg=abtitlecolor}
\setbeamercolor{palette secondary}{fg=white,bg=absecondarycolor}
\setbeamercolor{palette primary}{fg=black,bg=abprimarycolor}
\setbeamercolor{structure}{fg=abtitlecolor}

\setbeamerfont{section in toc}{series=\bfseries}

%AB: remove navigation icons
\beamertemplatenavigationsymbolsempty
\title{
  \textbf {Musical instruments: strings and pipes}\\
%\centerline{}
%\centering
%\vspace{-0.0in}
%\includegraphics[width=0.3\textwidth]{propvalues_0093.pdf}
%\vspace{-0.3in}\\
%\label{intrograph}
}

\author[W. Freeman] {Physics 211\\Syracuse University, Physics 211 Spring 2015\\Walter Freeman}

\date{\today}

\begin{document}

\frame{\titlepage}

\frame{\frametitle{\textbf{Announcements}}
  \Large
\BI
\item{``Optional'' recitation section next Wednesday: more review for the final on Reading Day}
\item{Homework 9 due next Wednesday}
\item{Practice exam posted; 8 questions on it can be submitted for extra credit (also Wednesday)}
\item{Daily practice questions and discussion on the Facebook group}
\item{Tentative final exam review schedule will be emailed out tomorrow}
\EI
}

\frame{\frametitle{\textbf{Final exam ``makeup opportunity''}}
  \Large
  \BI
\item{If you do substantially better on the final than one of your other exams...}
  \BI
\item{That exam counts less}
\item{The final counts more}
\item{This can only help you}
\item{Questions about grades: I am not curving numerical grades}
\item{... the cutoffs between A/B/C/D/F are just different, and will be determined at the end}
\item{Over 50\% of grades will be A's and B's}
  \EI
\EI
}

\frame{\frametitle{\textbf{Grade appeals for Exam 3}}
  \Large
  \BI
\item{To appeal your grade, you must submit a correct solution with your appeal form}
  \BI
\item{(Remember recitation attendance counts toward your grade anyway!)}
  \EI
\item{Same procedure as before}
 \EI
 }



\frame{\frametitle{\textbf{Standing waves, a reminder}}
  \large
  \BI
\item{Only certain wavelengths can persist as standing waves in a ``one-dimensional cavity''}
\item{1D cavity: waves on a string, sound waves in a pipe... things we make musical instruments out of!}
\item{Waves are {\it linear} -- multiple standing waves of different wavelengths can coexist}
  \EI
}


\frame{\frametitle{\textbf{Sine waves}}
  \BI
\item{We're particularly concerned with waves that look like sines and cosines}
\item{These waves have two new properties: {\bf wavelength $\lambda$} and {\bf frequency $f$}}
\BI
\item{Wavelength: distance from crest to crest}
\item{Frequency: how many crests go by per second, equal to $1/T$ ($T$ = period)}
  \pause
\item{Speed = distance $\times$ time}
\EI
\EI
\Large
  $$c=\lambda f$$
  \BI
\pause
\item{What kind of sine and cosine waves can we put on our string?}
\item{Not any wavelengths will do, since the ends have to be fixed}
  \EI
}

\frame{\frametitle{\textbf{Standing waves, in more detail}}
  \Large
  \vcenteredinclude{mode1-crop.pdf} Fundamental: $\lambda = \frac{2L}{1}$\\
 \bigskip 
  \vcenteredinclude{mode2-crop.pdf} 2nd harmonic: $\lambda = \frac{2L}{2}$\\
 \bigskip 
  
  \vcenteredinclude{mode3-crop.pdf} 3rd harmonic: $\lambda = \frac{2L}{3}$\\
 \bigskip 
  
  \vcenteredinclude{mode4-crop.pdf} 4th harmonic: $\lambda = \frac{2L}{4}$\\
 \bigskip 

  \bigskip
  \centerline{Can we write these wavelengths in terms of $f$ using $c=f\lambda$?}

}

\frame{\frametitle{\textbf{Standing waves, in more detail}}
  \Large
  \vcenteredinclude{mode1-crop.pdf} Fundamental: $f_1 = \frac{c}{2L}$\\
 \bigskip 
  \vcenteredinclude{mode2-crop.pdf} 2nd harmonic: $f_2 = 2 f_1 $\\
 \bigskip 
  \vcenteredinclude{mode3-crop.pdf} 3rd harmonic: $f_3 = 3 f_1$ \\
  
 \bigskip 
  \vcenteredinclude{mode4-crop.pdf} 4th harmonic: $f_4 = 4 f_1$ \\
  
 \bigskip 

}


\frame{\frametitle{\textbf{Musical instruments: in general}}
  \large
  \BI
\item{Vibrating strings or columns of air inside tubes all can support all of these modes}
  \BI
\item{You can select for particular ones, however: what happens if you pluck a string in its center?}
\pause
\item{Only odd-numbered modes are excited: the even ones have a node there}
  \EI
\item{In general, when you excite a string or air column, you produce them all}
\item{Often we choose to excite strings in ways that prefer some modes over others}
\item{The unique sound of each instrument comes mostly from the relative strengths}
  \EI
}

\frame{\frametitle{\textbf{Controlling pitch}}
  \large
  \BI
\item{Any instrument needs a way of changing $f_1$ to play different notes}
\item{Modern piano: $f_1$ from 28.5 Hz to 4 kHz}
\item{Human voice: $f_1$ from 65 Hz to 1 kHz}
\item{Human hearing: sensitive from 20 Hz to 20 kHz (roughly)}
  \EI
}

\frame{\frametitle{\textbf{Stringed instruments}}
  \large
\BI
\item{Make a string vibrate, its vibrations cause sound waves (not very efficient)}
\item{Make a string vibrate, couple it mechanically to something bigger which makes the air vibrate: better!}
\item{Three ways to control the fundamental frequency of sound in a string:}
  \BI
  \normalsize
\item{Speed of sound on a stretched string: $c = \sqrt{T/\lambda}$}
\item{$T$ is the tension, $\lambda$ is the linear mass density (kg per meter)}
\item{If $c=f\lambda$, then $f_1 = \frac{\sqrt{T}}{2L\sqrt{\lambda}}$}
  \EI
\item{More tension makes the frequency go up (how these instruments are tuned}
\item{A longer string makes the frequency go down (bass vs. violin)}
\item{A thicker string makes the frequency go down (wound strings)}
\EI
}

\frame{\frametitle{\textbf{Wind instruments}}
\BI
\item{Same idea, except we have a column of air instead of a string}
\item{Here the wave speed $c$ is just the speed of sound in air}
\item{Classic example: the pipe organ}
  \BI
\item{Each pipe only sounds one note}
\item{Pipes up to 32 feet long $\rightarrow f_1 = 17$ Hz!}
\EI
\item{Others: use one pipe to sound multiple notes by opening and closing holes}
\item{Excite vibrations with either a reed or something akin to a whistle (on a flute)}
  \EI
}

\frame{\frametitle{\textbf{Brass instruments}}
  \large
  \BI
\item{Don't be fooled by the funny shapes: they (mostly) act like straight pipes}
\item{Here there are two tricks for controlling pitch: change the length of the tube...}
  \BI
\item{Trombone: physically make the tube longer}
\item{Trumpet etc.: Add/subtract lengths of tubing}
  \EI
\item{... or match the buzzing of the player's lips to frequencies other than $f_1$}
\item{How does a trombonist play a scale?}
  \EI
}


\end{document}
