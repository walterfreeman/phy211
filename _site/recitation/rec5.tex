\documentclass[12pt]{article}
\setlength\parindent{0pt}
\usepackage{fullpage}
\usepackage{amsmath}
\usepackage{array}
\usepackage{hyperref}
\usepackage{graphicx}
\setlength{\parskip}{4mm}
\def\LL{\left\langle}   % left angle bracket
\def\RR{\right\rangle}  % right angle bracket
\def\LP{\left(}         % left parenthesis
\def\RP{\right)}        % right parenthesis
\def\LB{\left\{}        % left curly bracket
\def\RB{\right\}}       % right curly bracket
\def\PAR#1#2{ {{\partial #1}\over{\partial #2}} }
\def\PARTWO#1#2{ {{\partial^2 #1}\over{\partial #2}^2} }
\def\PARTWOMIX#1#2#3{ {{\partial^2 #1}\over{\partial #2 \partial #3}} }
\newcommand{\vs}[1]{\vspace{#1}}
\newcommand{\BC}{\begin{center}}
\newcommand{\EC}{\end{center}}
\newcommand{\BI}{\begin{itemize}}
\newcommand{\EI}{\end{itemize}}
\newcommand{\BE}{\begin{enumerate}}
\newcommand{\EE}{\end{enumerate}}
\newcommand{\BNE}{\begin{equation}}
\newcommand{\ENE}{\end{equation}}
\newcommand{\BEA}{\begin{eqnarray}}
\newcommand{\EEA}{\nonumber\end{eqnarray}}
\newcommand{\EL}{\nonumber\\}
\newcommand{\la}[1]{\label{#1}}
\newcommand{\ie}{{\em i.e.\ }}
\newcommand{\eg}{{\em e.\,g.\ }}
\newcommand{\cf}{cf.\ }
\newcommand{\etc}{etc.\ }
\newcommand{\Tr}{{\rm tr}}
\newcommand{\etal}{{\it et al.}}
\newcommand{\OL}[1]{\overline{#1}\ } % overline
\newcommand{\OLL}[1]{\overline{\overline{#1}}\ } % double overline
\newcommand{\OON}{\frac{1}{N}} % "one over N"
\newcommand{\OOX}[1]{\frac{1}{#1}} % "one over X"


\pagenumbering{gobble}
\begin{document}
\Large
\centerline{\sc{Recitation Problems}}
\large
\BC
\sc
Wednesday, February 1 -- Kinematics in Two Dimensions
\EC
\normalsize

{\large \bf Problem 1:}

A ball rolls off a shelf of height $h$ at speed $v$. 

\begin{itemize} 
\item (a) How far from the base of the 
shelf will it land?
\item (b) How fast will it be traveling when it lands?
\item (c) In what direction will it be moving when it lands?
\end{itemize}

{\large \bf Problem 2:}

In class yesterday, you saw us do a demonstration, as follows:

The barrel of a gun that shot ball bearings was pointed directly at a small target.
Once the ball bearing left the barrel of the gun, its path of flight curved 
downward, due to the influence of gravity. 

At the instant that the ball bearing left the barrel, the target was dropped from its
support, and was struck as it fell by the ball bearing.

This happens regardless of the angle of the gun and the initial velocity of the ball bearing.

Argue in (a) words, (b) algebra, and (c) diagrams why this must be the case.

{\it You will receive 5 points extra credit on your group practice exam on Friday
for each separate argument that your group is able to explain to your TA.}

\newpage

{\large \bf Problem 3:}

\bigskip

In a certain baseball stadium, the outer wall is a distance $D=120$ m from 
home plate, and has a height $h=10$ m.

Assume that the batter hits the ball at an angle $\theta=30^\circ$ above the
horizontal. How fast must the batter hit the ball for it to (a) land at the base
of the outer wall, and (b) travel over the wall?

Roadmap for doing this problem:

\begin{enumerate}
\item Write down expressions for $x(t)$, $y(t)$, and (if you need them) $v_x(t)$ and $v_y(t)$.
\begin{itemize}
\item Convert the initial velocity vector $\vec v_0$ into the component representation ($v_{x,0}$ and $v_{y,0}$)
\end{itemize}
\item Write down a sentence in terms of your algebraic variables that can be used
to answer the question. (This is the hardest aspect of these problems; don't neglect
it!)
\item Perform the algebra that your sentence indicates that you do
\begin{itemize}
\item If you have to use the quadratic formula, determine the physical
meaning of each answer
\end{itemize}
\item Substitute numerical values and interpret your answer
\end{enumerate}

\end{document}
