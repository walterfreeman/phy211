\documentclass[12pt]{article}
\setlength\parindent{0pt}
\usepackage{fullpage}
\usepackage[margin=0.5in]{geometry}
\usepackage{amsmath}
\usepackage{graphicx}
\setlength{\parskip}{4mm}
\def\LL{\left\langle}   % left angle bracket
\def\RR{\right\rangle}  % right angle bracket
\def\LP{\left(}         % left parenthesis
\def\RP{\right)}        % right parenthesis
\def\LB{\left\{}        % left curly bracket
\def\RB{\right\}}       % right curly bracket
\def\PAR#1#2{ {{\partial #1}\over{\partial #2}} }
\def\PARTWO#1#2{ {{\partial^2 #1}\over{\partial #2}^2} }
\def\PARTWOMIX#1#2#3{ {{\partial^2 #1}\over{\partial #2 \partial #3}} }
\newcommand{\BE}{\begin{displaymath}}
\newcommand{\EE}{\end{displaymath}}
\newcommand{\BNE}{\begin{equation}}
\newcommand{\ENE}{\end{equation}}
\newcommand{\BEA}{\begin{eqnarray}}
\newcommand{\EEA}{\nonumber\end{eqnarray}}
\newcommand{\EL}{\nonumber\\}
\newcommand{\la}[1]{\label{#1}}
\newcommand{\ie}{{\em i.e.\ }}
\newcommand{\eg}{{\em e.\,g.\ }}
\newcommand{\cf}{cf.\ }
\newcommand{\etc}{etc.\ }
\newcommand{\Tr}{{\rm tr}}
\newcommand{\etal}{{\it et al.}}
\newcommand{\OL}[1]{\overline{#1}\ } % overline
\newcommand{\OLL}[1]{\overline{\overline{#1}}\ } % double overline
\newcommand{\OON}{\frac{1}{N}} % "one over N"
\newcommand{\OOX}[1]{\frac{1}{#1}} % "one over X"



\begin{document}
\pagenumbering{gobble}
\Large
\centerline{\sc{Recitation Questions}}

\normalsize
\centerline{\sc{April 3}}

\bigskip
\bigskip
\begin{enumerate}

\item A firecracker is launched straight upward. It is traveling upward at 30 m/s when it explodes, fracturing into two pieces. The larger piece is twice as massive as the smaller piece; after the explosion, it is traveling horizontally at 50 m/s.
What is the velocity of the smaller piece after the explosion? Give its magnitude and direction.

\newpage

\item A person is sitting on a tree swing, consisting of a light platform suspended by two ropes of length $L$. His exceptionally friendly dog, who is half his mass and far too big to be a lap-dog but doesn't know it, runs toward him 
and then jumps in his lap.\footnote{I'm envisioning Farmer Maggot and his dogs Grip, Fang, and Wolf from {\it The Fellowship of the Ring} here, but you can substitute any person and big ol' doggo you want.}

The impact of the dog causes him to recoil backwards, and the swing swings upward to an angle $\theta$ before coming back down.

\begin{enumerate}

\item Draw cartoons of three relevant moments in time: (i) right before his dog jumps in his lap, (ii) right after his dog jumps in his lap, and (iii) when the swing is at its highest point. Label relevant things, in particular the velocities
of moving objects.

\item What physical principle can you use to relate the velocity of the dog before she jumps in his lap in cartoon (i) to the velocity of the person plus dog afterwards in cartoon (ii)?

\vspace{1in}

\item What physical principle can you use to relate the velocity of the person and dog in cartoon (ii) to the swing angle $\theta$ in cartoon (iii)?

\vspace{1in}

\item What is the work done by gravity during the motion here? You will need to think carefully about geometry.

\vspace{3in}

\item How fast was the dog running when she jumped in his lap? Find an answer in terms of $L$, $\theta$, and $g$.

\end{enumerate}

\newpage

\item A Yo-Yo consists of a cylinder of mass $m$ and radius $r$. A thin slot is cut in the middle of the cylinder such that the inner radius is only $0.4r$, and a string is wound around the middle. (If you don't know what a Yo-Yo is, there is an animation on Wikipedia.) The moment of inertia of a cylinder is $I=\frac{1}{2}mr^2$. Suppose that a Yo-Yo has a string of length $L=1.2$ m, and its handler winds up the string, holds one end, and then drops the Yo-Yo. It will fall, unwinding its string,
until it falls 120 cm and runs out of string.

In this problem, you'll use the conservation of energy to find how fast the Yo-Yo is moving when it reaches the bottom.

\begin{enumerate}

\item Write down the work-energy theorem for the Yo-Yo as it falls. Note that the tension here does no work; there are several explanations for why, but the one based on the things that you know so far is that the {\it string is stationary}. 
The point of contact between the string and the Yo-Yo doesn't slip; you can think of the spool ``rolling down'' the string, much like a ball rolls down a hill.

\vspace{3.5in}

\item What is the relation between the angular velocity $\omega$ and the translational velocity $v$ of the Yo-Yo? Call your TA/coach over to check your work on this part.

\vspace{3.5in}

\newpage

\item Find its velocity when it reaches the bottom. Compare this to the velocity of an object that has been simply {\it dropped} the same distance.

\vspace{3.5in}

\item Does the slot in the middle of the Yo-Yo simply keep the string from sliding out, or does it serve some other purpose? Explain this to your TA/coach.

\end{enumerate}
\end{enumerate}

\end{document}
 


