\documentclass[12pt]{article}
\setlength\parindent{0pt}
\usepackage{fullpage}
\usepackage{amsmath}
\usepackage{graphicx}
\setlength{\parskip}{4mm}
\usepackage[left=2cm, right=2cm, top=1.5cm, bottom=1cm]{geometry}

\def\LL{\left\langle}   % left angle bracket
\def\RR{\right\rangle}  % right angle bracket
\def\LP{\left(}         % left parenthesis
\def\RP{\right)}        % right parenthesis
\def\LB{\left\{}        % left curly bracket
\def\RB{\right\}}       % right curly bracket
\def\PAR#1#2{ {{\partial #1}\over{\partial #2}} }
\def\PARTWO#1#2{ {{\partial^2 #1}\over{\partial #2}^2} }
\def\PARTWOMIX#1#2#3{ {{\partial^2 #1}\over{\partial #2 \partial #3}} }
\newcommand{\BI}{\begin{itemize}}
\newcommand{\EI}{\end{itemize}}
\newcommand{\BE}{\begin{displaymath}}
\newcommand{\EE}{\end{displaymath}}
\newcommand{\BNE}{\begin{equation}}
\newcommand{\ENE}{\end{equation}}
\newcommand{\BEA}{\begin{eqnarray}}
\newcommand{\EEA}{\nonumber\end{eqnarray}}
\newcommand{\EL}{\nonumber\\}
\newcommand{\la}[1]{\label{#1}}
\newcommand{\ie}{{\em i.e.\ }}
\newcommand{\eg}{{\em e.\,g.\ }}
\newcommand{\cf}{cf.\ }
\newcommand{\etc}{etc.\ }
\newcommand{\Tr}{{\rm tr}}
\newcommand{\etal}{{\it et al.}}
\newcommand{\OL}[1]{\overline{#1}\ } % overline
\newcommand{\OLL}[1]{\overline{\overline{#1}}\ } % double overline
\newcommand{\OON}{\frac{1}{N}} % "one over N"
\newcommand{\OOX}[1]{\frac{1}{#1}} % "one over X"

\def\BS{\bigskip}

\begin{document}
\pagenumbering{gobble}
\Large
\centerline{\sc{Recitation Questions}}
\normalsize
\centerline{\sc{26 February}}

\it These recitation problems are designed to build a bridge between the things we have learned so far and things we will learn later in the semester.

\rm


Your recitation evaluation today will be a quiz question on Blackboard at the end of the class period about Homework 2. If you are doing recitation asynchronously, you may either:

\BI
\item Attend any synchronous recitation section and complete the Blackboard quiz for that section
\item Attend any PHY211 TA's office hours on Monday or Tuesday and discuss the recitation activity with them

\newpage

\rm 

\newpage

\begin{center}
\Large
Question 1: the ``third kinematics equation''
\end{center}

Sometimes we are not interested in the {\it time} motion takes. For instance, consider the following problem:

\begin{center}
	\it A car is driving at 30 m/s. The driver wants to slow down to 20 m/s. How far will the car travel before it slows down to 20 m/s?
\end{center}

Our current process for solving this problem is the following:

\BI
\item Interpret the question as a question about algebraic variables: ``What is the value of position at the time when the velocity is equal to 20 m/s?''
\item Use the velocity relation $v(t) = v_0 + at$ to solve for the time when $v = 20 \rm m/\rm s$
\item Use the position relation $x(t) = x_0 + v_0 t + \frac{1}{2}at^2$ to solve for the position at that time
\EI

Note that we had to calculate the time this motion took as an intermediate step, even though we didn't really care about it in the end. Is there a simpler way to do this?

It turns out there is! The best way to do this is to solve this problem {\it without numbers} in the most general sense, then look at the result.

To do this:

\begin{enumerate}
\item Write down the constant-acceleration kinematics relations above (using $x_f$)
\item Solve one 

\end{enumerate}

\end{document}
