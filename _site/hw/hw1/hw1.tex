\documentclass[12pt]{article}
\setlength\parindent{0pt}
\usepackage{fullpage}
\usepackage{amsmath}
\usepackage{hyperref}
\usepackage[margin=2cm]{geometry}
\usepackage{graphicx}
\setlength{\parskip}{4mm}
\def\LL{\left\langle}   % left angle bracket
\def\RR{\right\rangle}  % right angle bracket
\def\LP{\left(}         % left parenthesis
\def\RP{\right)}        % right parenthesis
\def\LB{\left\{}        % left curly bracket
\def\RB{\right\}}       % right curly bracket
\def\PAR#1#2{ {{\partial #1}\over{\partial #2}} }
\def\PARTWO#1#2{ {{\partial^2 #1}\over{\partial #2}^2} }
\def\PARTWOMIX#1#2#3{ {{\partial^2 #1}\over{\partial #2 \partial #3}} }
\newcommand{\BE}{\begin{displaymath}}
\newcommand{\EE}{\end{displaymath}}
\newcommand{\BNE}{\begin{equation}}
\newcommand{\ENE}{\end{equation}}
\newcommand{\BEA}{\begin{eqnarray}}
\newcommand{\EEA}{\nonumber\end{eqnarray}}
\newcommand{\EL}{\nonumber\\}
\newcommand{\la}[1]{\label{#1}}
\newcommand{\ie}{{\em i.e.\ }}
\newcommand{\eg}{{\em e.\,g.\ }}
\newcommand{\cf}{cf.\ }
\newcommand{\etc}{etc.\ }
\newcommand{\Tr}{{\rm tr}}
\newcommand{\etal}{{\it et al.}}
\newcommand{\OL}[1]{\overline{#1}\ } % overline
\newcommand{\OLL}[1]{\overline{\overline{#1}}\ } % double overline
\newcommand{\OON}{\frac{1}{N}} % "one over N"
\newcommand{\OOX}[1]{\frac{1}{#1}} % "one over X"



\begin{document}
\begin{center}
\Large
\sc Homework 1 \rm

\pagenumbering{gobble}

\normalsize

Due Friday, 19 February, by the beginning of your recitation section
\end{center}
{\bf Instructions:} Answer the following on your own paper, or using
a tablet and stylus. Use a separate side of paper for each question.

Remember that you must explain your logic clearly. Good physics involves a mixture of words, pictures, symbols, and numbers as needed.


Solutions that involve
only mathematics, particularly mathematics that does not explain itself clearly, may not receive full credit even if they arrive at the correct final answer.

Also remember that you should not substitute in numbers prematurely. Do not substitute numbers in for algebraic quantities until you are ready to draw some physical insight from the numerical result.

When you are finished, scan your work as a PDF using a cellphone app such as CamScanner, and submit it to Blackboard.

\begin{center}
	
	\underline{\hspace{3in}}
\end{center}


\normalsize


\begin{enumerate}

\item A car driving from Tucson to Nogales is distracted by some physics students launching model rockets in the desert, veers off the road, and crashes into a cactus. The driver is luckily wearing his seatbelt, which causes him to decelerate from 120 km/hour to a dead stop in 50 milliseconds.

\medskip

In this problem, you will calculate the acceleration of the driver as his seatbelt brings him to a stop.

\bigskip
\bigskip

\begin{enumerate}
\item{There are some numerical quantities given to you in the problem: 120 km/hour and 50 milliseconds. 
When in the problem should you introduce those numerical values? What algebraic variables will you use to 
represent them?}
\item{Write an expression that gives the velocity of the car $v(t)$ as a function of time $t$, in terms of the 
acceleration of the car $a$ and the initial velocity $v_0$. What physical moment corresponds to the time $t=0$?}
\item{As we discussed in class, the key to solving kinematic problems is to rephrase them as a question in terms
of your algebraic variables. This problem's question has the following form: ``What is the value of \underline{\hspace{1cm}} such that \underline{\hspace{1cm}} is equal to zero at time \underline{\hspace{1cm}}?'' Fill in
the blanks.}
\item{Write an algebraic statement that corresponds to the sentence you wrote, and solve it for the acceleration.}
\item{Substitute in the given numerical values and find the acceleration. When you do this, make sure you
retain {\it all} units: the initial velocity is not ``120'', but ``120 km/hour''.}
\item{In this problem, you will have encountered some minus signs in your algebra. What is their physical significance?}
\end{enumerate}


\bigskip
\bigskip

\item In the previous problem, you solved for the acceleration of the driver. How many times greater is this
acceleration than that of an object in free-fall (9.8 $\rm m/\rm s^2$)? Does this result surprise you? When making your comparison, remember to retain all units;
you can manipulate them exactly as you do algebraic quantities.

\bigskip
\bigskip

\item In Tolkien's Fellowship of the Ring, Pippin foolishly throws a stone down a well in Moria, only to hear it splash seconds later as it hits the bottom.
Tolkien doesn't tell us how long this takes, but Peter Jackson's film interpretation is on Youtube:
\url{https://youtu.be/5cZ4ABUo6TU}

Here the bucket slides off of the ledge at 0:30, and the last of the clattering sounds (presumably the bucket striking the bottom) happens at 0:44.

You may assume that Middle-Earth has the same gravity as Earth. 

\begin{enumerate}
\item Ignoring air resistance and the travel time of sound, how deep is the well?
\item Did you encounter any minus signs in your solution to this problem? If so, what is their physical meaning?
If not, did you make any particular choices that kept all quantities positive?
\end{enumerate}

Hint: Follow the same schematic as the first problem. Write down equations of motion for the bucket's position
and/or velocity as a function of time; think of a sentence in terms of your 
algebraic variables that asks the correct question; then, apply that sentence to your equations of motion 
to determine the depth.

\bigskip
\bigskip

\item A bicyclist starts from a stop, accelerates at 1 $\rm m/\rm s^2$ for 6 seconds, travels at a constant velocity for 5 seconds, and then applies her brakes, decelerating at 2 $\rm m/\rm s^2$ until she comes to a stop again.

\begin{enumerate}
\item{Make acceleration vs. time and velocity vs. time graphs for the cyclist. Be precise!}
\item{How far does she travel?}
\item{Now make a position vs. time graph for her. (Again, be precise!)}
\end{enumerate}

Hint: You know some relations for constant-acceleration kinematics, but the acceleration is only constant during three separate intervals, not the entire problem. How can you deal with that? What are the initial position and velocity for the different intervals?

\bigskip
\bigskip

\item A driver is driving once again from Nogales to Tucson on I-19. Suddenly a roadrunner darts out into the middle of the road. It doesn't see her at first, because it is focused on a tasty lizard it's trying to catch. She is initially traveling at 110 km/hour, 
the roadrunner is 30m in front of her when she applies her brakes, and the brakes decelerate the car at 9 $\rm m/\rm s^2$.

She slams on her brakes, and the sound of squealing brakes alerts the bird to the onrushing car. Being an agile critter, it hops out of the way, squawking irately that its meal has been disturbed.

    Hint: You will need the quadratic formula for this problem.
\begin{enumerate}
\item    Graph the position vs. time of the car, and label the position of the roadrunner on the graph.
\item    How long does the roadrunner have to get out of the car's way?
\item    The quadratic formula gives you two solutions to the quadratic. Which one is the physically meaningful answer? What does the other solution mean, if anything?
\end{enumerate}

Hint for part c: You wrote down an expression for position vs. time for the car. What assumption did you make
about the car's acceleration when you wrote down that expression? Is that assumption always valid? If not,
in what interval of time is it valid?

\bigskip
\bigskip

\item We studied objects in motion with constant acceleration in class, and deduced some formulae for 
constant-acceleration motion that you have been using for the previous problems. If the acceleration is not
constant, they do not apply, but we can go back to the same procedure we did in class to find other formulae that
{\it are} valid.

 Suppose, that an object moves with a time-varying acceleration that starts at zero but that increases linearly with time, i.e. $a = kt$. 

\begin{enumerate}
\item What must the dimensions of $k$ be? What units might it be measured in?
\item What is the displacement of this object from its initial position (i.e. how far has it moved) as a function of time?
\end{enumerate}

Hint: Recall that the area under the curve, also called the integral, of $ct^n$ is $\frac{1}{n+1}ct^{n+1}$.

\item How many kilograms of snow fall on the SU Quad every winter? Explain how you estimate this, what assumptions you make, and how you arrive at your result. In doing any mathematics that you need to do, remember to retain all units, and cancel them just like
you would any other algebraic quantity.

\end{enumerate}


\end{document}
