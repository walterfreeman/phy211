% DPF 09 talk on strangeness in nucleon

\documentclass[10pt]{beamer}
\usepackage{amsmath}
\usefonttheme{professionalfonts} % using non standard fonts for beamer
\usefonttheme{serif} % default family is serif\
\usepackage{mathtools}
%\documentclass[12pt]{beamerthemeSam.sty}
\usepackage{epsf}
\usepackage{ulem}
\usepackage{array}
%\usepackage{pstricks}
%\usepackage[orientation=portrait,size=A4]{beamerposter}
\geometry{paperwidth=160mm,paperheight=120mm}
%DT favorite definitions
\def\LL{\left\langle}	% left angle bracket
\def\RR{\right\rangle}	% right angle bracket
\def\LP{\left(}		% left parenthesis
\def\RP{\right)}	% right parenthesis
\def\LB{\left\{}	% left curly bracket
\def\RB{\right\}}	% right curly bracket
\def\PAR#1#2{ {{\partial #1}\over{\partial #2}} }
\def\PARTWO#1#2{ {{\partial^2 #1}\over{\partial #2}^2} }
\def\PARTWOMIX#1#2#3{ {{\partial^2 #1}\over{\partial #2 \partial #3}} }

\def\rightpartial{{\overrightarrow\partial}}
\def\leftpartial{{\overleftarrow\partial}}
\def\diffpartial{\buildrel\leftrightarrow\over\partial}

\def\BI{\begin{itemize}}
\def\EI{\end{itemize}}
\def\BE{\begin{displaymath}}
\def\EE{\end{displaymath}}
\def\BEA{\begin{eqnarray*}}
\def\EEA{\end{eqnarray*}}
\def\BNEA{\begin{eqnarray}}
\def\ENEA{\end{eqnarray}}
\def\EL{\nonumber\\}
\def\BS{\bigskip}
\def\BC{\begin{center}}
\def\EC{\end{center}}
\def\BCC{\begin{columns}}
\def\ECC{\end{columns}}
\def\HC{\column{0.5\textwidth}}
\newcommand{\map}[1]{\frame{\frametitle{\textbf{Course map}}
\centerline{\includegraphics[height=0.86\paperheight]{../../map/#1.png}}}}
\newcommand{\wmap}[1]{\frame{\frametitle{\textbf{Course map}}
\centerline{\includegraphics[width=0.96\paperwidth]{../../map/#1.png}}}}

\newcommand{\etal}{{\it et al.}}
\newcommand{\gbeta}{6/g^2}
\newcommand{\la}[1]{\label{#1}}
\newcommand{\ie}{{\em i.e.\ }}
\newcommand{\eg}{{\em e.\,g.\ }}
\newcommand{\cf}{cf.\ }
\newcommand{\etc}{etc.\ }
\newcommand{\atantwo}{{\rm atan2}}
\newcommand{\Tr}{{\rm Tr}}
\newcommand{\dt}{\Delta t}
\newcommand{\op}{{\cal O}}
\newcommand{\msbar}{{\overline{\rm MS}}}
\def\chpt{\raise0.4ex\hbox{$\chi$}PT}
\def\schpt{S\raise0.4ex\hbox{$\chi$}PT}
\def\MeV{{\rm Me\!V}}
\def\GeV{{\rm Ge\!V}}

%AB: my color definitions
%\definecolor{mygarnet}{rgb}{0.445,0.184,0.215}
%\definecolor{mygold}{rgb}{0.848,0.848,0.098}
%\definecolor{myg2g}{rgb}{0.647,0.316,0.157}
\definecolor{abtitlecolor}{rgb}{0.0,0.255,0.494}
\definecolor{absecondarycolor}{rgb}{0.0,0.416,0.804}
\definecolor{abprimarycolor}{rgb}{1.0,0.686,0.0}
\definecolor{Red}           {cmyk}{0,1,1,0}
\definecolor{Grey}           {cmyk}{.7,.7,.7,0}
\definecolor{Lg}           {cmyk}{.4,.4,.4,0}
\definecolor{Blue}          {cmyk}{1,1,0,0}
\definecolor{Green}         {cmyk}{1,0,1,0}
\definecolor{Brown}         {cmyk}{0,0.81,1,0.60}
\definecolor{Black}         {cmyk}{0,0,0,1}

\definecolor{A} {rgb}{1,0.3,0.3}
\definecolor{B} {rgb}{0,0.8,0.0}
\definecolor{C} {rgb}{0.8,0.8,0.0}
\definecolor{D} {rgb}{0.3,0.3,1}
\definecolor{E} {rgb}{0.5,0.5,0.5}
\usetheme{Madrid}


%AB: redefinition of beamer colors
%\setbeamercolor{palette tertiary}{fg=white,bg=mygarnet}
%\setbeamercolor{palette secondary}{fg=white,bg=myg2g}
%\setbeamercolor{palette primary}{fg=black,bg=mygold}
\setbeamercolor{title}{fg=abtitlecolor}
\setbeamercolor{frametitle}{fg=abtitlecolor}
\setbeamercolor{palette tertiary}{fg=white,bg=abtitlecolor}
\setbeamercolor{palette secondary}{fg=white,bg=absecondarycolor}
\setbeamercolor{palette primary}{fg=black,bg=abprimarycolor}
\setbeamercolor{structure}{fg=abtitlecolor}

\setbeamerfont{section in toc}{series=\bfseries}

%AB: remove navigation icons
\beamertemplatenavigationsymbolsempty
\title{
  \textbf {Rotational motion}\\
%\centerline{}
%\centering
%\vspace{-0.0in}
%\includegraphics[width=0.3\textwidth]{propvalues_0093.pdf}
%\vspace{-0.3in}\\
%\label{intrograph}
}

\author[W. Freeman] {Physics 211\\Syracuse University, Physics 211 Spring 2022\\Walter Freeman}

\date{\today}

\begin{document}

\frame{\titlepage}

\frame{\frametitle{\textbf{Announcements}}
	\large
	
\BI
\item{Next homework is due next Wednesday}
\item{Upcoming help hours: }
	\BI 
	\item Today 9:45-10:45ish (between classes) and 1:30-3:30
	\item Tomorrow 1:00-3:00
	\item {\bf Monday 2:00-5:00}
	\EI
	\BS
\item I am working with some study groups already, as are our coaches. Want to form one? Let me know when/where and I or a coach will come help.

\BS

\item Recitation tomorrow:
\BI
\item I know there's a block party\pause
\item We still have recitation\pause
\item We'll leave it to you to decide what your priorities are (you're adults)\pause
\item If you complete the exercise tomorrow in recitation, we'll award you five points extra credit
\EI
\EI
}



\frame{\frametitle{\textbf{A review: Angular momentum}}
	\begin{columns}
		\column{0.5\textwidth}
		\color{Grey}
		\Large
		\centerline{Translational motion}
		\normalsize
		\BI
		{\color{Red}
			\item{Moving objects have momentum}
			\item{$\vec p = m \vec v$}
			\item{Momentum conserved if there are no external forces}}
		\EI
		\column{0.5\textwidth}
		\color{Grey}
		\Large
		\centerline{Rotational motion}
		\normalsize
		\BI
		{\color{Blue}
			\item{Spinning objects have angular momentum $L$}
			\item{$L = I \omega$}
			\item{Angular momentum conserved if no external torques}}
		\EI
	\end{columns}
	
	\bigskip
	\bigskip
	
	\Large
	\color{Red}
	
	$\rightarrow$ $L = I \omega = $ constant; analogue to conservation of momentum
	
}

\frame{\frametitle{\textbf{Conservation of angular momentum}}
	\large
	We saw that the conservation of momentum was valuable mostly in two sorts of situations:
	
	\BI
	\item{Collisions: two objects strike each other}
	\item{Explosions: one object separates into two}
	\EI
	
	There is a third common case for conservation of angular momentum:
	
	\BI
	\item{Collisions: a child runs and jumps on a merry-go-round (your homework)}
	\item{Explosions: throwing a ball off-center (our example)}
	\item{{\color{Red}A spinning object changes its moment of inertia}}
	\EI
	
	\pause
	\bigskip
	
}
\frame{\frametitle{\textbf{Example: the ice-skater}}
	\large
	
	If I am standing on my spinning platform rotating at $\omega_i$ = 1 rad/sec while holding the dumbbells at arm's length, 
	how fast do I start spinning when I pull them inward?
	
	\BS\BS\pause
	
	\begin{center}
		$$L_i = L_f$$
		
		\pause
		
		$$I_i \omega_i = I_f \omega_f$$
		
		\pause
		
		$$\omega_f = \omega_i \frac{I_i}{I_f}$$
	\end{center}
	
}
\frame{\frametitle{\textbf{Example: the ice-skater}}
	\large
	
	If I am standing on my spinning platform rotating at $\omega_i$ = 1 rad/sec while holding the dumbbells at arm's length, 
	how fast do I start spinning when I pull them inward?
	
	\BS\BS\pause
	
	\begin{center}
		$$L_i = L_f$$
		
		
		$$I_i \omega_i = I_f \omega_f$$
		
		$$\omega_f = \omega_i {\color{Red}\frac{I_i}{I_f}}$$
	\end{center}
	
	\pause\BS
	
	Now we just have to estimate the moments of inertia...
}

%\frame{\frametitle{\textbf{Angular momentum of a single object}}
%	
%	\large
%	
%	A single object moving in a straight line also has angular momentum.
%	
%	\Huge
%	$$L = mv_\perp r = mvr_\perp$$
%	\large
%	
%	\BS\BS
%	
%	Example: I am standing on my spinning platform when I throw a baseball at $10$ m/s with my arm held out sideways.
%	
%	\BS
%	
%	How fast do I start spinning?
%	
%}


%
%\frame{\frametitle{\textbf{Rotational kinetic energy, again}}
%
%\large You already know that rotational ideas correspond to translational ones:
%\normalsize
%\BS
%
%\begin{center}
%\begin{tabular}{l | l}
%
% \multicolumn{1}{c|}{\Large Translation} & \multicolumn{1}{c}{\Large Rotation} \\
% \\
%\hline
%\hline
% & \\
%Position $\vec s$ & Angle $\theta$ \\
%Velocity $\vec v$ & Angular velocity $\omega$ \\
%Acceleration $\vec a$ & Angular acceleration $\alpha$ \\
% & \\
%\hline
%\hline
% & \\
%Kinematics: $\vec s(t)\frac{1}{2}\vec at^2 + \vec v_0 t + \vec s_0$ & $\theta(t) = \frac{1}{2}\alpha t^2 + \omega_0 t + \theta_0$ \\
% & \\
%\hline
%\hline
%
% & \\
%Force $\vec F$ & Torque $\vec \tau = \vec r \times \vec F$ \\
%Mass $m$ & Moment of inertia $I$ \\
%Newton's second law $\vec F_{\rm tot} = m \vec a$ & Newton's second law for rotation $\tau_{\rm tot} = I \alpha$ \\
% & \\
%
%\hline
%\hline
%\end{tabular}
%\BS
%\large
%
%\pause
%
%You've also studied {\color{Red}kinetic energy} along with the {\color{Red}work-energy
%theorem}. They have
%rotational analogues as well.
%\EC
%}


\frame{\frametitle{\textbf {Rotational kinetic energy, again}}
\begin{center}
\begin{tabular}{l | l}

 \multicolumn{1}{c|}{\Large Translation} & \multicolumn{1}{c}{\Large Rotation} \\
 \\
\hline
\hline
Force $\vec F$ & Torque $\vec \tau = \vec r \times \vec F$ \\
Mass $m$ & Moment of inertia $I$ \\
Newton's second law $\vec F_{\rm tot} = m \vec a$ & Newton's second law for rotation $\tau_{\rm tot} = I \alpha$ \\

 & \\

\hline
\hline

 & \\
\color{Red}Kinetic energy $KE=\frac{1}{2}mv^2$ & \color{Red}Kinetic energy $KE=\frac{1}{2}I\omega^2$ \\
\color{Red}Work $W = \vec F \cdot \Delta \vec s$ & \color{Red}Work $W = \tau \Delta \theta$ \\
\color{Red}Power $P = \vec F \cdot \vec v$ & \color{Red}Power $P = \tau \omega$ \\
 & \\

\hline
\hline

 & \\

\hline
\end{tabular}
\BS

\large

Rotational kinetic energy and the rotational work-energy theorem
work like their translational counterparts.

\end{center}
}



\frame{\frametitle{\textbf{Rotational kinetic energy, again}}

\Large

Our previous study of rotational kinetic energy went like this:

\normalsize

\BI
\item When an object is spinning, it has kinetic energy $KE_{\rm{rot}}=\frac{1}{2}I\omega^2$
\BS
\item When an object rolls down a hill (for instance), we write 

\begin{align*}
{\color{Red}\text{(initial grav. potential energy)}} &= {\color{Green}\text{(final translational KE)}} + {\color{Blue}\text{(final rotational KE)}}\\
{\color{Red}mgh} &= {\color{Green}\frac{1}{2}mv^2} + {\color{Blue}\frac{1}{2}I\omega^2}
\end{align*}
\EI

\BS\pause

We ignored questions like:

\BI
\item Doesn't friction do work on the rolling object?
\item If there are strings involved, what about tension?
\item What about the rotational work-energy theorem?
\item What about rotational power? (Fast cars! Bicycles!
\EI

Let's go back and understand those things now.

\BS\BS

We'll use as our example the Yo-Yo from recitation yesterday -- but, this time, we'll analyze it with {\it energy} instead of just force and acceleration.


}
%
%
%\frame{\frametitle{\textbf{Rotational kinetic energy}}
%
%\Large
%
%There is also kinetic energy associated with rotation, too! (The pipe
%problem from HW6...)
%
%$$KE_{\rm rot} = \frac{1}{2}I\omega^2$$
%
%\large
%
%This is what we would expect, based on $KE_{\rm trans} = \frac{1}{2}mv^2$:
%
%\BI
%\item Moment of inertia $I$ is the rotational analogue of mass
%\item Angular velocity $\omega$ is the rotational analogue of velocity
%\EI
%}
%


\frame{

\Large
\BCC
\HC
Suppose I release a Yo-Yo whose string has a length $h$. How fast will its center be moving
when it runs out of string?
\HC
\BC
\includegraphics[width=0.4\textwidth]{yoyo-diagram-crop.pdf}
\EC
\ECC
\BS
\BS

\color{A}A: $v_f < \sqrt{2gh}$, because the tension in the string slows it down \\
\color{B}B: $v_f < \sqrt{2gh}$, because part of the GPE is required to make the Yo-Yo spin \\
\color{C}C: $v_f = \sqrt{2gh}$, by the conservation of energy \\
\color{D}D: $v_f > \sqrt{2gh}$, because the spinning disk speeds it up \\
}

\frame{
\large
{\color{C} Answer C} is what we get if there is no string. (We already know how to do that.)

\pause
\BCC
\HC
{\color{A} Answer A} is related to what you did in recitation yesterday; in a force diagram for the Yo-Yo, the tension
means that the net downward force is less than $mg$. 
\HC
\BC
\includegraphics[width=0.4\textwidth]{yoyo-diagram-forces-crop.pdf}
\EC
\ECC

\pause
\BS

{\color{B} Answer B} makes sense as well, though: if the Yo-Yo spins as it falls,
then {\bf some energy is required to make it spin}, leaving less available
energy for translational kinetic energy.

\BS
}


\frame{
	\large

	
	\BCC
	\HC
		If the Yo-Yo spins as it falls,
	then {\bf some energy is required to make it spin}, leaving less available
	energy for translational kinetic energy.
	
	\BS
	
		Before we learned to analyze this with conservation of energy:
	
	\HC
	\BC
	\includegraphics[width=0.3\textwidth]{yoyo-diagram-forces-crop.pdf}
	\EC
	\ECC
	
	
	\begin{align*}
		{\color{Red}\text{(initial grav. potential energy)}} &= {\color{Green}\text{(final translational KE)}} + {\color{Blue}\text{(final rotational KE)}}\\
		{\color{Red}mgh} &= {\color{Green}\frac{1}{2}mv^2} + {\color{Blue}\frac{1}{2}I\omega^2} \\
		{\color{Red}mgh} &= {\color{Green}\frac{1}{2}mv^2} + {\color{Blue}\frac{1}{2}mR^2\omega^2} \\
		{\color{Red}mgh} &= {\color{Green}\frac{1}{2}mv^2} + {\color{Blue}\frac{1}{2}m\frac{R^2}{r^2}v^2} \\
		v &= \sqrt{\frac{\color{Red}2gh}{{\color{Green}1} + {\color{Blue}\frac{1}{2}\frac{R^2}{r^2}}}}
	\end{align*}
}





\frame{\frametitle{\textbf{The work done by tension}}
\large
We know the work-energy theorem for translational motion (for constant $\vec F$):

$$W_{\rm trans} \equiv \Delta \frac{1}{2}mv^2 = \vec F \cdot \Delta \vec s$$

\BS

Replacing $m$, $\vec F$, $\vec s$, and $v^2$ with their rotational counterparts, we get:

\BS
$$W_{\rm rot} \equiv \Delta \frac{1}{2}I\omega^2 = \tau \Delta \theta $$

\BS

This is the {\it rotational work-energy theorem.}
}

\frame{\frametitle{\textbf{The work done by tension}}

\Large
Which is true regarding the work done by tension here?
\BS
\Huge

\color{A}A: $W_{\rm total} = 0$ \\
\color{B}B: $W_{\rm trans} > 0, W_{\rm rot} > 0$ \\
\color{C}C: $W_{\rm trans} < 0, W_{\rm rot} > 0$ \\
\color{D}D: $W_{\rm trans} > 0, W_{\rm rot} < 0$ \\
\color{E}E: $W_{\rm trans} < 0, W_{\rm rot} < 0$ \\
\pause

\BS
\Large
\color{Black}
The string makes the Yo-Yo fall more slowly (negative translational work),
but makes it spin (positive rotational work). That means {\color{C}Answer C}
is correct. What about {\color{A}Answer A}?
}

\frame{\frametitle{\textbf{We forgot something: what about the work done by tension?}}

\large


\BS\BS \pause

Rotational work: $W_{\rm rot} = \tau \Delta \theta$

\pause\BS

If the Yo-Yo falls a distance $h$, it turns through a (positive!) angle 
given by $\Delta \theta = h/r $.

\BS

The torque applied by the tension is $\tau = Tr$ (positive!).

\pause

\BS\BS
{\color{Red} Rotational work: $W_{\rm rot} = \tau \Delta \theta = Tr(h/r) = Th$.}\\
{\color{Blue}Translational work: $W_{\rm trans} = \vec F \cdot \Delta \vec s =-Th$.}
\BS
\pause

$\rightarrow$ The {\color{Red} total work done by tension here is zero.} (We could have guessed that!)
}

\frame{\frametitle{\textbf{Conservation of energy, including rotation}}

\Large
$$ {\rm PE_i} + \frac{1}{2}mv_i^2 + \frac{1}{2}I\omega_i^2 + W_{\rm NC} = 
{\rm PE_f} + \frac{1}{2}mv_f^2 + \frac{1}{2}I\omega_f^2 $$

\Large\BS

Which expression will let us find the velocity of the Yo-Yo at the bottom?

\begin{align*}
&\color{A}\rm A:& \color{A}mgh - Th &\color{A}= \color{A}\frac{1}{2}mv_f^2 + \frac{1}{2} I\omega_f^2 \\\BS
&\color{B}\rm B:&\color{B} mgh + \frac{1}{2}mv_i^2 &\color{B}= \frac{1}{2}mv_f^2 + \frac{1}{2} I\omega_f^2 \\\BS
&\color{C}\rm C:&\color{C} mgh &\color{C}= \frac{1}{2}mv_f^2  \\\BS
 &\color{D}\rm D:&\color{D} mgh &\color{D}= \frac{1}{2}mv_f^2 + \frac{1}{2} I\omega_f^2 \\\BS
\end{align*}
}

\frame{\frametitle{\textbf{What about rolling objects?}}

\large
In the Yo-Yo problem, we saw that:

\BI
\item Tension did positive rotational work (it made the Yo-Yo spin faster)
\item Tension did negative translational work (it made the Yo-Yo move more slowly)
\item ... the {\bf net work done by tension was zero}.
\EI

This happened because the string was stationary, and thus enforced $a=\pm \alpha r$.
This is also true in {\color{Red}rolling motion}.
}


\frame{\frametitle{\textbf{An object rolling down a hill}}
\Large

\BC
Consider first a ball sliding down a hill without friction.
\EC
\BCC
\HC
\BC\includegraphics[width=0.8\textwidth]{hill-crop.pdf}\EC
\HC
Which of these forces applies a torque to the ball?

\BS
\color{A}A: Just the normal force \\
\color{B}B: Just gravity \\
\color{C}C: Both of them \\
\color{D}D: Neither of them \\
\ECC

\BS

\BC
\pause
Friction is required to make the ball spin!
\EC
}

\frame{\frametitle{\textbf{An object rolling down a hill}}
\Large

\BC
If the ball {\it rolls without slipping...}
\EC
\BCC
\HC
\BC\includegraphics[width=0.8\textwidth]{hill-crop.pdf}\EC
\HC
\large
What is true about the frictional force?

\BS
\color{A}A: Static friction points down the ramp \\
\color{B}B: Static friction points up the ramp \\
\color{C}C: Kinetic friction points down the ramp \\
\color{D}D: Kinetic friction points up the ramp \\
\color{E}E: There is no friction
\ECC

\BS

}


\frame{\frametitle{\textbf{An object rolling down a hill}}
\Large

\BC
If the ball {\it rolls without slipping...}
\EC
\BCC
\HC
\BC\includegraphics[width=0.8\textwidth]{hill-2-crop.pdf}\EC
\HC
\large
What is true about the frictional force?

\BS
\color{A}A: Static friction points down the ramp \\
\color{B}B: Static friction points up the ramp \\
\color{C}C: Kinetic friction points down the ramp \\
\color{D}D: Kinetic friction points up the ramp \\
\color{E}E: There is no friction
\ECC

\BS

\BC
The point of contact would {\color{Red} slide downward} without friction, so friction 
points {\color{Red} back up the ramp}. This is static friction since the ball 
doesn't slide.
\EC




}

\frame{\frametitle{\textbf{Energy rolling down a hill}}

\large

Static friction {\color{Red} does no total work} on the ball:
\BI
\item it reduces the translational kinetic energy $\frac{1}{2}mv^2$
\item it increases the rotational kinetic energy $\frac{1}{2}I\omega^2$
\item ... but it leaves the sum $\frac{1}{2}mv^2 + \frac{1}{2}I\omega^2$ unchanged
\EI

\BS

\BCC
\HC
\BC
\includegraphics[width=0.8\textwidth]{mu-table.png}

\scriptsize (From {\it Physics for Scientists and Engineers}, Knight, 3rd ed.)
\EC
\HC
This is not {\it quite} true -- rolling friction does exist. There is 
a little bit of overall negative work done as tires flex and so on, but it is small.
\ECC
}



\frame{
\Large

This means that we can use our standard expression for conservation of energy
for rolling objects, {\it ignoring} the force of static friction required to keep them
from slipping:

$$ {\rm PE}_i + \frac{1}{2}mv_i^2 + \frac{1}{2}I\omega_i^2 +W_{\text{other}} = 
{\rm PE}_f + \frac{1}{2}mv_f^2 + \frac{1}{2}I\omega_f^2 $$

\BS\BS

How fast will each object ($I = \lambda mr^2$) be traveling at the bottom
of the ramp?}

\frame{
\Large

How high must I start the ball for it to make it around the loop?

}



\frame{\frametitle{\textbf{Rotational dynamics and power}}

\Large

If $W = \tau \Delta \theta$, then $P = \tau \omega$.

\BS

If I want to supply a power $P$, I can either exert a large torque at a 
small angular velocity, or a small torque at a large angular velocity.

\BS
\BS

$\rightarrow$ bicycle demonstration!
}


\frame{\frametitle{\textbf{Gears}}
	
	\Large
	
	{\bf Note for recitation tomorrow:}
	
	\BS\BS
	
	Whether gears are linked by a chain or directly:
	
	
	\BI
	\item The {\it forces} between them are the same (Newton's 3rd Law)
	\item The {\it tangential velocity} of the gears is the same
	\item The angular velocities are {\it not} the same
	\item The torques are {\it not} the same
	\EI
}

\end{document}
