\documentclass[12pt]{article}
\setlength\parindent{0pt}
\usepackage{fullpage}
\usepackage{graphicx}
\usepackage{amsmath}
\setlength{\parskip}{4mm}
\def\LL{\left\langle}   % left angle bracket
\def\RR{\right\rangle}  % right angle bracket
\def\LP{\left(}         % left parenthesis
\def\RP{\right)}        % right parenthesis
\def\LB{\left\{}        % left curly bracket
\def\RB{\right\}}       % right curly bracket
\def\PAR#1#2{ {{\partial #1}\over{\partial #2}} }
\def\PARTWO#1#2{ {{\partial^2 #1}\over{\partial #2}^2} }
\def\PARTWOMIX#1#2#3{ {{\partial^2 #1}\over{\partial #2 \partial #3}} }
\newcommand{\BE}{\begin{displaymath}}
\newcommand{\EE}{\end{displaymath}}
\newcommand{\BNE}{\begin{equation}}
\newcommand{\ENE}{\end{equation}}
\newcommand{\BEA}{\begin{eqnarray}}
\newcommand{\EEA}{\nonumber\end{eqnarray}}
\newcommand{\EL}{\nonumber\\}
\newcommand{\la}[1]{\label{#1}}
\newcommand{\ie}{{\em i.e.\ }}
\newcommand{\eg}{{\em e.\,g.\ }}
\newcommand{\cf}{cf.\ }
\newcommand{\etc}{etc.\ }
\newcommand{\Tr}{{\rm tr}}
\newcommand{\etal}{{\it et al.}}
\newcommand{\OL}[1]{\overline{#1}\ } % overline
\newcommand{\OLL}[1]{\overline{\overline{#1}}\ } % double overline
\newcommand{\OON}{\frac{1}{N}} % "one over N"
\newcommand{\OOX}[1]{\frac{1}{#1}} % "one over X"



\begin{document}
\Large
\centerline{\sc{Homework 9}}
\normalsize
\centerline{\sc{Due Tuesday, 30 April}}

This homework set contains only two problems, both somewhat involved. Only {\it one} of them will be graded.

\begin{enumerate}

\item Consider an electric motor that turns a driveshaft counter-clockwise. This driveshaft is connected to a machine; the motor applies a counter-clockwise torque to the driveshaft, which is connected to a machine connected to the other end that applies an equal and opposite clockwise torque. Thus, the motor delivers power to the machine. For this problem, the motor will always be spinning at a constant angular velocity, so $\sum \tau = 0$.

Suppose that the motor can apply a torque $\tau_m=100~\rm N \cdot \rm m$ to the driveshaft, but it has a maximum angular velocity $\omega_{\rm max} = 50~\rm s^{-1}$.

\begin{enumerate}
	\item What is the maximum power that the motor can deliver to the machine? (Hint: You know that for translational motion, $P = \vec F \cdot \vec v$. What is the analogous formula for rotation?)
	\item However, in general, machines need to run at different speeds; for instance, cars can drive at many different speeds. Suppose that the operator of the machine wants to run it at low speed -- say, at 10~$\rm s^{-1}$. Can the motor still deliver its full power in this case? If not, how much power can it deliver?

\end{enumerate}




\item The drive components of a two-speed bicycle are constructed as follows:
\begin{itemize}
\item The rear wheel has a radius of 50 cm
\item The front gear has a radius of 10 cm, and is connected to the pedals
\item There are two rear gears: one with a radius of 5 cm (``high gear''), and one with a radius of 10 cm (``low gear'')
\item A chain connects the front gear to one of the rear gears; the cyclist can choose which one (``shift gears'')
\item None of the other components of the bicycle matter.
\end{itemize}


Suppose that our cyclist wants to ride at a constant $v=10$ m/s, and that there is a constant 
drag force of 15 N.  Determine each of the following for both low gear
and for high gear. (Some of these will be the same for both.)

It will be helpful for you to draw an extended force diagram for the rear wheel + gear, and another for the front gear,
since you will need to think about the torques applied to each from traction and the chain (rear), and the chain and pedals (front).

\begin{enumerate}
\item The traction force on the rear wheel
\item The angular velocity of the rear wheel (and the gear attached to it)
\item The tangential velocity of the rear gear, which is equal to the velocity of the chain and thus to the tangential velocity of the front gear
\item The tension in the chain; note that only the top portion of the chain bears tension
\item The angular velocity of the front gear (and thus the pedals)
\item The torque the cyclist must apply to the pedals
\item The {\it power} the cyclist must supply to the pedals
\end{enumerate}

Some of your values will be different for low gear and high gear; some of them will be the same. In particular, comment on
the values you get for the angular velocity of the pedals, the torque applied to the pedals, and the overall power supplied 
by the cyclist.

Note that the bike is going at a constant rate, so the net torque on everything is zero.
\newpage
\item A ball ($I=\lambda mr^2$) of radius $r$ and mass $m$ rolls down a slope of length $L$ elevated at an angle $\theta$ without slipping. The coefficient of static friction between the ball and the slope is $\mu_s$.

In this problem, you'll calculate the forces on it, and its acceleration. Take the following steps:

\begin{enumerate}
\item Draw an extended force diagram for the bowling ball. Choose a tilted coordinate system, as usual.
\item Is the frictional force on the bowling ball equal to $\mu_s F_N$, or might it be some other value? (If it's some other value, introduce a symbol for it.)
\item What is the relationship between the linear acceleration $a$ of the bowling ball and its angular acceleration $\alpha$? (Note that counterclockwise is positive; the sign in this relationship will depend on which way your ramp is facing.)
\item Construct both $\sum \vec F = m \vec a$ and $\sum \tau = I \alpha$ for the ball. This should produce a system of 
two equations with three unknowns. Combined with your result from (c), you can solve for the acceleration and for your 
other unknown quantity in terms of $m$, $\theta$, $\lambda$, and $g$.
\item Sanity-check the acceleration you get for your bowling ball. Does it make sense?
\item What is the steepest slope that your ball can roll down without slipping?
\end{enumerate}
\end{enumerate}

\end{document}
