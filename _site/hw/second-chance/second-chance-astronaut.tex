\documentclass[12pt]{article}
\setlength\parindent{0pt}
%\usepackage{fullpage}
\usepackage{amsmath}
\usepackage{hyperref}
\usepackage[margin=2cm]{geometry}
\usepackage{graphicx}
\setlength{\parskip}{4mm}
\def\LL{\left\langle}   % left angle bracket
\def\RR{\right\rangle}  % right angle bracket
\def\LP{\left(}         % left parenthesis
\def\RP{\right)}        % right parenthesis
\def\LB{\left\{}        % left curly bracket
\def\RB{\right\}}       % right curly bracket
\def\PAR#1#2{ {{\partial #1}\over{\partial #2}} }
\def\PARTWO#1#2{ {{\partial^2 #1}\over{\partial #2}^2} }
\def\PARTWOMIX#1#2#3{ {{\partial^2 #1}\over{\partial #2 \partial #3}} }
\newcommand{\BE}{\begin{displaymath}}
\newcommand{\EE}{\end{displaymath}}
\newcommand{\BNE}{\begin{equation}}
\newcommand{\ENE}{\end{equation}}
\newcommand{\BEA}{\begin{eqnarray}}
\newcommand{\EEA}{\nonumber\end{eqnarray}}
\newcommand{\EL}{\nonumber\\}
\newcommand{\la}[1]{\label{#1}}
\newcommand{\ie}{{\em i.e.\ }}
\newcommand{\eg}{{\em e.\,g.\ }}
\newcommand{\cf}{cf.\ }
\newcommand{\etc}{etc.\ }
\newcommand{\Tr}{{\rm tr}}
\newcommand{\etal}{{\it et al.}}
\newcommand{\OL}[1]{\overline{#1}\ } % overline
\newcommand{\OLL}[1]{\overline{\overline{#1}}\ } % double overline
\newcommand{\OON}{\frac{1}{N}} % "one over N"
\newcommand{\OOX}[1]{\frac{1}{#1}} % "one over X"



\begin{document}
\begin{center}
\Large
\sc Second Chance Exercise - Techniques in Combination / Astronaut Problem \rm

\pagenumbering{gobble}




\normalsize
This second chance exercise is for people who want to review the conservation of momentum in two dimensions.

If one of your lowest three exam problem grades was on the ``astronaut problem'' from Exam 3, you may complete it and turn it in to earn an extra homework grade, which we will help you earn 100\% on. This will help your end-of-term average.

If you do this, it is due on the last day of recitation.

\end{center}

\vspace{1.5in}

This exam problem tested your ability to use conservation of momentum to analyze a collision where objects move in two dimensions.

\begin{enumerate}
	
		\item How did you handle the fact that objects moved in two dimensions in this problem? Compare your use of vectors in this problem (using the conservation of momentum) to your application of vectors to other ideas like force and kinematics.
		
		\item Compare this exam problem to the ``jumping astronaut problem'' on Homework 5 Question 1 (\url{https://walterfreeman.github.io/phy211/hw/hw5/hw5-momentum-2023.pdf}). How are these two problems similar, and how are they different? (What did each problem require you to do the other one didn't?)

\item Identify two problems in Practice Exam 3 (at \url{https://walterfreeman.github.io/phy211/practice-exam-3-2023.pdf} and \url{https://walterfreeman.github.io/phy211/practice-exam-3-2023-part2.pdf}) where you also needed to think about the conservation of momentum in two dimensions. How are these problems similar to this one, and how are the different?
	
    \item Write a similar problem where you need to analyze a collision or explosion in which objects move in two dimensions. Then solve it. {\it (Particularly good problems here will earn extra credit!)}
\end{enumerate}



\end{document}
