\documentclass[12pt]{article}
\setlength\parindent{0pt}
\usepackage{fullpage}
\usepackage{amsmath}
\usepackage{hyperref}
\usepackage{graphicx}
\setlength{\parskip}{4mm}
\def\LL{\left\langle}   % left angle bracket
\def\RR{\right\rangle}  % right angle bracket
\def\LP{\left(}         % left parenthesis
\def\RP{\right)}        % right parenthesis
\def\LB{\left\{}        % left curly bracket
\def\RB{\right\}}       % right curly bracket
\def\PAR#1#2{ {{\partial #1}\over{\partial #2}} }
\def\PARTWO#1#2{ {{\partial^2 #1}\over{\partial #2}^2} }
\def\PARTWOMIX#1#2#3{ {{\partial^2 #1}\over{\partial #2 \partial #3}} }
\newcommand{\BE}{\begin{displaymath}}
\newcommand{\EE}{\end{displaymath}}
\newcommand{\BNE}{\begin{equation}}
\newcommand{\ENE}{\end{equation}}
\newcommand{\BEA}{\begin{eqnarray}}
\newcommand{\EEA}{\nonumber\end{eqnarray}}
\newcommand{\EL}{\nonumber\\}
\newcommand{\la}[1]{\label{#1}}
\newcommand{\ie}{{\em i.e.\ }}
\newcommand{\eg}{{\em e.\,g.\ }}
\newcommand{\cf}{cf.\ }
\newcommand{\etc}{etc.\ }
\newcommand{\Tr}{{\rm tr}}
\newcommand{\etal}{{\it et al.}}
\newcommand{\OL}[1]{\overline{#1}\ } % overline
\newcommand{\OLL}[1]{\overline{\overline{#1}}\ } % double overline
\newcommand{\OON}{\frac{1}{N}} % "one over N"
\newcommand{\OOX}[1]{\frac{1}{#1}} % "one over X"



\begin{document}
\Large
\centerline{\sc{Homework 2, due Friday, 3 February}}
\normalsize

\begin{enumerate}

\item Part 4 of Question 2 asks you to compute the acceleration of an object, given the change in position, the initial velocity, and the final velocity. To do this, you will need to use both $x(t)$ and $v(t)$ kinematics equations, but ultimately you will eliminate the variable $t$.

Often in mechanics we aren't particularly concerned about time -- we're only concerned with the change in position, initial velocity, final velocit, and acceleration.
These are related by the ``third kinematics equation'', 

$$
v_f^2 - v_0^2 = 2a(x_f - x_0)
$$

Show that this equation is simply a consequence of the other two.

Algebra hint: Starting from the $x(t)$ and $v(t)$ formulae for constant acceleration, solve one equation for $t$ and substitute it back into the other one.


\item During the siege of Constantinople that led to its conquest by the Ottomans in 1453, the Hungarian engineer Orban built a set of bombards (primitive cannon) to throw enormous stones at the city to breach its walls. The largest of these could throw a 300 kg stone a distance $x_f=2$ km. Assume that the stone was launched at an angle of $\theta=45^\circ$ above the horizontal; in the absence of air resistance, this gives the largest range.

\begin{enumerate}

\item What speed did the stone have to be launched at to achieve this range?
\item How long was the ball in the air?
\item How fast was the ball traveling at the apex of its flight?
\item Orban's cannon was 8m long. What was the average acceleration of the stone as it was launched down the bore of the cannon? {\it Hint: Note that during its movement down the bore of the cannon, it accelerated from $v=0$ to the velocity you found as your solution to the first part of this problem.}
\end{enumerate}

\item A car drives off of a cliff 60 meters above sea level and splashes into the ocean 150 meters out to sea.

\begin{enumerate}
\item What was the car's speed when it left the cliff?
\item What was the car's speed when it struck the water?
\item In what direction was the car traveling when it struck the water? (Give your answer in a physically meaningful way: "X degrees below the horizontal" or similar.
\end{enumerate}

\item A particularly athletic frog can push herself off of the ground with enough velocity to jump 1m high. Note that the initial velocity of the frog, once you find it, is a property of the frog: no matter what we do with the frog, the initial velocity (which determines how high she can jump) is the same.

\begin{enumerate}
\item With what velocity does the frog leave the ground?
\item How long will the frog be in the air before she lands?
\end{enumerate}

\item Suppose we now put the frog in an elevator, accelerating upward at $\alpha=2\, \rm m/\rm s^2$. There is a tasty spider at a height $h=85$ cm above the floor of the elevator. This is the same frog, so it is capable of jumping with an initial velocity equal to the value you found in the previous problem.

\begin{enumerate}
\item If the frog jumps as high as she can, will she catch the spider? How far above the elevator floor will the frog make it?
\item How long will the frog be in the air?
\end{enumerate}

{\it Hint: Think very carefully about your coordinate system, and all of the consequences of the accelerating elevator. You may need the quadratic formula for this problem. If you are still stuck, draw position vs. time graphs for the frog and the spider.}

\item Our same famous frog is now taken to the planet Twilo, which is quite Earthlike except for its value of $g=11.8\, \rm m/\rm s^2.$

\begin{enumerate}
\item How high can our spacefaring frog jump on Twilo?
\item Based on a comparison of the two jumping-frog problems, can you make any statements regarding gravity and acceleration?
\end{enumerate}

\item An object's position is given by the vector 

$$\vec s(t) = (3\,\rm m/\rm s^3) t^3 \hat i + (4\, \rm m/\rm s^2) t^2 \hat j + (8\, \rm m/\rm s) t\hat k.$$

\begin{enumerate}
\item What is its velocity as a function of time?
\item What is its acceleration as a function of time?
\item Where is it at $t=10$ s?
\end{enumerate}

\item An object's position is given by the vector 

$$\vec s(t) = (2\, \rm m) \cos (\omega t) \hat i + (2\, \rm m) \sin (\omega t) \hat j$$

where $\omega = 1$ radian per second.

\begin{enumerate}
\item How would you describe this object's motion? 
\item Hint: if you are stuck, compute and graph its position in the Cartesian plane at a variety of values of t (say, for integers 0-6).
\end{enumerate}
\end{enumerate}
\end{document}

