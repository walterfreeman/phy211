\documentclass[12pt]{article}
\setlength\parindent{0pt}
\usepackage{fullpage}
\usepackage[margin=0.5in, paperwidth=13.5in, paperheight=8.4in]{geometry}
\usepackage{amsmath}
\usepackage{graphicx}
\setlength{\parskip}{4mm}
\def\LL{\left\langle}   % left angle bracket
\def\RR{\right\rangle}  % right angle bracket
\def\LP{\left(}         % left parenthesis
\def\RP{\right)}        % right parenthesis
\def\LB{\left\{}        % left curly bracket
\def\RB{\right\}}       % right curly bracket
\def\PAR#1#2{ {{\partial #1}\over{\partial #2}} }
\def\PARTWO#1#2{ {{\partial^2 #1}\over{\partial #2}^2} }
\def\PARTWOMIX#1#2#3{ {{\partial^2 #1}\over{\partial #2 \partial #3}} }
\newcommand{\BE}{\begin{displaymath}}
\newcommand{\EE}{\end{displaymath}}
\newcommand{\BNE}{\begin{equation}}
\newcommand{\ENE}{\end{equation}}
\newcommand{\BEA}{\begin{eqnarray}}
\newcommand{\EEA}{\nonumber\end{eqnarray}}
\newcommand{\EL}{\nonumber\\}
\newcommand{\la}[1]{\label{#1}}
\newcommand{\ie}{{\em i.e.\ }}
\newcommand{\eg}{{\em e.\,g.\ }}
\newcommand{\cf}{cf.\ }
\newcommand{\etc}{etc.\ }
\newcommand{\Tr}{{\rm tr}}
\newcommand{\etal}{{\it et al.}}
\newcommand{\OL}[1]{\overline{#1}\ } % overline
\newcommand{\OLL}[1]{\overline{\overline{#1}}\ } % double overline
\newcommand{\OON}{\frac{1}{N}} % "one over N"
\newcommand{\OOX}[1]{\frac{1}{#1}} % "one over X"



\begin{document}
\Large
\centerline{\sc{Recitation Questions -- April 1}}
\normalsize

\pagenumbering{gobble}

\begin{enumerate}
	
\item Consider an electric automobile (such as a Tesla Model 3) of mass $m$ whose fully-charged battery has electric potential energy $U$ when it is fully charged. The car's electric motor can convert this energy to kinetic energy with a maximum power $P_e$.

Recall that if a force $\vec F$ acts on an object moving with velocity $\vec v$, that force does work on the object with a power $P = \vec F \cdot \vec v$.

\begin{enumerate}
	\item First, let's determine the maximum speed this car can sustain on flat ground.
	
	At high speeds, the most significant force slowing the car down is air drag. This has the form $F_{\rm drag} = \gamma v^2$.
	
		If the driver of this car wants to drive at a speed $v$, find the rate that air drag does work (that is, the power) on the car. Is this power positive or negative?
	
	\vspace{2in}
	
	\item In order to sustain this speed, the car's motor must do positive work on the car at the same rate. If the engine's maximum power is $P_e$, find the top speed of this car in terms of $\gamma$ and $P_e$.
	
	\newpage
	
	\item For a Tesla Model 3, the values are approximately:
	
	\begin{itemize}
	\item $P_e = 200$ kW
	\item $U = 270$ MJ ($2.7 \times 10^8$ J)
	\item $\gamma = 0.8$ kg/m
	\end{itemize}
	
	Based on these values, estimate the top speed of a Model 3. (At this high speed, air drag is the dominant force slowing the car down.) Convert your value into km/hr or miles per hour; is it reasonable?
	
	\vspace{1.5in}
	
	\item How far could a Model 3 drive at this speed before its battery is depleted?
		\vspace{1.5in}
	\item Suppose that the driver only went half as fast. How many times further could the driver travel on a charge? (\it Hint: Look back at your result for (a). You should be able to figure this out without much math.)\rm
	
	
\end{enumerate}	

\newpage

\item Suppose that a block of mass $m$ is resting against a spring of spring constant $k$ right before the base of a 
ramp elevated at an angle $\theta$ above the horizontal. The coefficient of kinetic friction between the block and the ramp is $\mu$. The spring is compressed by an amount $d$. 

When the spring is released, it propels the block up the ramp. (There is no friction until the block reaches the ramp; this will simplify your algebra.) It slides up the ramp a distance $L$,
then slides back down and compresses the spring again by a maximum amount $b$. (You know $m$, $k$, $\mu$, $\theta$, and $d$; you don't know $L$ or $b$, but will find them later.)

\begin{enumerate}
	\item Will $b$ (the distance it compresses the spring when it comes back down) be larger than, the same as, or smaller than $d$ (the distance the spring was compressed in the beginning)? You should be able to make a logical argument here without doing any mathematics.
	\newpage
	
	\item Determine $L$, the distance the block travels up the ramp, in terms of $k$, $m$, $d$, $\mu$, $g$, and $\theta$.
	
	\vspace{3.3in}
	
	\item Determine $b$, the distance the block compresses the spring when it comes back down. Since you have found a formula for $L$ previously, you may use $L$ in your answer here. (This is to save you writing, since substituting in for it is not all that enlightening.)
	
\end{enumerate}
\end{enumerate}





	









\end{document}
