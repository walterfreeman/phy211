
\documentclass[12pt]{article}
\setlength\parindent{0pt}
\usepackage{fullpage}
\usepackage[margin=0.5in, paperwidth=13.5in, paperheight=8.4in]{geometry}
\usepackage{amsmath}
\usepackage{graphicx}
\setlength{\parskip}{4mm}
\def\LL{\left\langle}   % left angle bracket
\def\RR{\right\rangle}  % right angle bracket
\def\LP{\left(}         % left parenthesis
\def\RP{\right)}        % right parenthesis
\def\LB{\left\{}        % left curly bracket
\def\RB{\right\}}       % right curly bracket
\def\PAR#1#2{ {{\partial #1}\over{\partial #2}} }
\def\PARTWO#1#2{ {{\partial^2 #1}\over{\partial #2}^2} }
\def\PARTWOMIX#1#2#3{ {{\partial^2 #1}\over{\partial #2 \partial #3}} }
\newcommand{\BE}{\begin{displaymath}}
\newcommand{\EE}{\end{displaymath}}
\newcommand{\BNE}{\begin{equation}}
\newcommand{\ENE}{\end{equation}}
\newcommand{\BEA}{\begin{eqnarray}}
\newcommand{\EEA}{\nonumber\end{eqnarray}}
\newcommand{\EL}{\nonumber\\}
\newcommand{\la}[1]{\label{#1}}
\newcommand{\ie}{{\em i.e.\ }}
\newcommand{\eg}{{\em e.\,g.\ }}
\newcommand{\cf}{cf.\ }
\newcommand{\etc}{etc.\ }
\newcommand{\Tr}{{\rm tr}}
\newcommand{\etal}{{\it et al.}}
\newcommand{\OL}[1]{\overline{#1}\ } % overline
\newcommand{\OLL}[1]{\overline{\overline{#1}}\ } % double overline
\newcommand{\OON}{\frac{1}{N}} % "one over N"
\newcommand{\OOX}[1]{\frac{1}{#1}} % "one over X"



\begin{document}
\pagenumbering{gobble}
\Large
\centerline{\sc{Recitation Questions}}

\normalsize
\centerline{\sc{March 11}}

\begin{enumerate}

%\item A train car of mass $m_1$ is traveling along a railway at a speed of $v_0$ when it encounters another train car of mass $m_2$. The two cars couple together when
%they collide. How fast are they traveling after the collision?



\item Explain how the conservation of momentum is a consequence of Newton's second and third laws. Call your TA or coach over when you have an argument, and give them your explanation. 

\newpage

  \item{The driver of a Mini Cooper (mass 1200 kg) is traveling at 10 m/s westward when he runs a stop sign and collides with a Toyota Camry (mass 2000 kg), traveling at 15 m/s northward. The two cars stick together after the collision.} 
  
  \vspace{2in}
  
  
      \begin{enumerate}
    \item What is the total momentum before the collision? (Will your answer be one value or two? Why?)

\vspace{2.5in}

    \item What is the total momentum after the collision?

\newpage

\vspace{2.5in}
    \item{What are the speed and direction of the cars after the collision?}

\vspace{3in}

    \item{If the coefficient of kinetic friction between the cars' tires and the pavement is 0.6, how far do they skid before coming to rest?}

\vspace{4in}

  \end{enumerate}



\item{A 5 kg box is sitting on a table; the coefficient of kinetic friction between the box and the table is 0.5.
  Two people throw things at it: a lump of clay and a rubber ball. Both objects have a mass of 500 g and strike the box at a speed of 4 m/s. The lump of clay collides inelastically (sticking to the box), while the ball bounces back at a speed of 2 m/s.}
  \begin{enumerate}
    \item{Without doing any mathematics, which object will knock the box further? How do you know? Hint: In the collision, the impulse delivered to the box is equal and opposite to the impulse delivered to the object thrown at it.}
\vspace{2in}
    \item{Calculate how far each object knocks the box.}
  \end{enumerate}
  \end{enumerate}

   \end{document}
