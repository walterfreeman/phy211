\documentclass[12pt]{article}
\setlength\parindent{0pt}
%\usepackage{fullpage}
\usepackage{amsmath}
\usepackage{hyperref}
\usepackage[margin=2cm]{geometry}
\usepackage{graphicx}
\setlength{\parskip}{4mm}
\def\LL{\left\langle}   % left angle bracket
\def\RR{\right\rangle}  % right angle bracket
\def\LP{\left(}         % left parenthesis
\def\RP{\right)}        % right parenthesis
\def\LB{\left\{}        % left curly bracket
\def\RB{\right\}}       % right curly bracket
\def\PAR#1#2{ {{\partial #1}\over{\partial #2}} }
\def\PARTWO#1#2{ {{\partial^2 #1}\over{\partial #2}^2} }
\def\PARTWOMIX#1#2#3{ {{\partial^2 #1}\over{\partial #2 \partial #3}} }
\newcommand{\BE}{\begin{displaymath}}
\newcommand{\EE}{\end{displaymath}}
\newcommand{\BNE}{\begin{equation}}
\newcommand{\ENE}{\end{equation}}
\newcommand{\BEA}{\begin{eqnarray}}
\newcommand{\EEA}{\nonumber\end{eqnarray}}
\newcommand{\EL}{\nonumber\\}
\newcommand{\la}[1]{\label{#1}}
\newcommand{\ie}{{\em i.e.\ }}
\newcommand{\eg}{{\em e.\,g.\ }}
\newcommand{\cf}{cf.\ }
\newcommand{\etc}{etc.\ }
\newcommand{\Tr}{{\rm tr}}
\newcommand{\etal}{{\it et al.}}
\newcommand{\OL}[1]{\overline{#1}\ } % overline
\newcommand{\OLL}[1]{\overline{\overline{#1}}\ } % double overline
\newcommand{\OON}{\frac{1}{N}} % "one over N"
\newcommand{\OOX}[1]{\frac{1}{#1}} % "one over X"



\begin{document}
\begin{center}
\Large
\sc Second Chance Exercise - The Work-Energy Theorem / Skateboarder Problem \rm

\pagenumbering{gobble}




\normalsize
This second chance exercise is for people who want to review the work-energy theorem.

If one of your lowest three exam problem grades was on the ``skateboarder problem'' from Exam 3, you may complete it and turn it in to earn an extra homework grade, which we will help you earn 100\% on. This will help your end-of-term average.

If you do this, it is due on the last day of recitation.

\end{center}

\vspace{1.5in}

This exam problem tested your ability to use the work-energy theorem in a detailed way to analyze the motion of an object, and to think about energy as a conserved quantity and its conversion from one form to another.

\begin{enumerate}
	
		\item Write down the expression of the work-energy theorem that you used to solve part (a). Then, under each term, label its meaning in words.
		
		\item Compare this exam problem to the first problem on Week 9 Day 2's recitation at {\url {https://walterfreeman.github.io/phy211/recitation/week9/recitation-potential-energy.pdf}}. How are these two problems similar, and how are they different?
		
		\item Compare this exam problem to the ``tire swing problem'' on Week 9 Day 2's recitation at {\url {https://walterfreeman.github.io/phy211/recitation/week9/recitation-potential-energy.pdf}}. How are these two problems similar, and how are they different?
		
		\item Think about the last part of this problem, where you need to determine how many times the skateboard slides back and forth over the sand before running out of energy. Gravity is a conservative force; friction is a nonconservative force. How does this affect how they affect the motion of the skateboard as it goes back and forth?

\item Identify two problems in Practice Exam 3 (at \url{https://walterfreeman.github.io/phy211/practice-exam-3-2023.pdf} and \url{https://walterfreeman.github.io/phy211/practice-exam-3-2023-part2.pdf}) where you need to use the work-energy theorem to understand the motion of an object. How are these problems similar to this one, and how are the different?
	
    \item Write a similar problem where you need to use the work-energy theorem and think carefully about the work done by different forces. Then solve it. {\it (Particularly good problems here will earn extra credit!)}
\end{enumerate}



\end{document}
